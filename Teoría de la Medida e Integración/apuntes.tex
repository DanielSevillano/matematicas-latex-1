\documentclass{book}
\usepackage[utf8]{inputenc}
\usepackage{amsmath,amsfonts}
\usepackage[mathscr]{euscript}
\usepackage[spanish]{babel}
\usepackage{mathrsfs}
\usepackage{fancyhdr}
\usepackage{graphicx}
\usepackage{caption}
\usepackage{subcaption}
\usepackage{amssymb}
\usepackage{vmargin}
\usepackage{hyperref}
\usepackage{amsthm}
\usepackage{mdframed}
\usepackage{xcolor}

\pagestyle{fancy}
\fancyhf{}
\lhead[\leftmark]{}
\rhead[]{\rightmark}
\fancyfoot[LE]{\thepage}
\fancyfoot[RE]{\includegraphics[width=0.045\textwidth]{insta.png} \text{@jorgeroddom}}
\fancyfoot[RO]{\thepage}
\fancyfoot[LO]{\includegraphics[width=0.045\textwidth]{insta.png} \text{@jorgeroddom}}

\title{\Huge \textbf{Teoría de la Medida e Integración}}
\author{Jorge Rodríguez Domínguez}
\date{}

\definecolor{light-gray}{gray}{0.85}

\newmdtheoremenv[backgroundcolor=light-gray]{teo}{Teorema}[section]
% \newtheorem{teo}{Teorema}[section]
\newmdtheoremenv[backgroundcolor=light-gray]{cor}[teo]{Corolario}
%\newtheorem{cor}[teo]{Corolario}
\newmdtheoremenv[backgroundcolor=light-gray]{lema}[teo]{Lema}
%\newtheorem{lema}[teo]{Lema}

%\newtheorem{defi}[teo]{Definición}
\newmdtheoremenv[backgroundcolor=light-gray]{prop}[teo]{Proposición}
% \newtheorem{prop}[teo]{Proposición}

\theoremstyle{definition}
\newmdtheoremenv[backgroundcolor=light-gray]{defi}[teo]{Definición}
\newtheorem{ejemplo}[teo]{Ejemplo}
\newtheorem{obs}[teo]{Observación}
% \newtheorem{obs}[teo]{Observación}
\newcommand{\rcom}{\overline{\mathbb{R}}}

\begin{document}
% =========================================
\frontmatter
 \begin{titlepage}
    \centering
    {\bfseries\LARGE \ \par}
    \vspace{1cm}
    {\scshape\Large \ \par}
    \vspace{3cm}
    \rule{\linewidth}{0.5mm}
    {\scshape\Huge Análisis Numérico \par}
    \rule{\linewidth}{0.5mm} \par
    \vspace{3cm}
    {\itshape\Large Basado en las clases de Carlos María Parés Madroñal \par}
    \vfill
    {\Large Autor: \par}
    {\Large Jorge Rodríguez Domínguez \par}
    \vfill
\end{titlepage}
\tableofcontents


% =========================================

\mainmatter
\chapter{Los números complejos}
$(\mathbb{R}^2, +, \cdot)$ con las siguientes operaciones
\begin{itemize}
    \item Suma : $(a,b) + (c,d) = (a + b, c + d)$.
    \item Producto: $(a,b) \cdot (c,d) = (ac - bd, ad + bc)$
\end{itemize}
es un cuerpo, lo llamaremos $\mathbb{C}$ y sus elementos se llaman números complejos.
\\
\newline
Observamos que 
\begin{align*}
    E = \{ (a,0) : a \in \mathbb{R} \} \subset \mathbb{C}
\end{align*}
es subcuerpo de $\mathbb{C}$, pues 
\begin{itemize}
    \item $(a,0) + (c,0) = (a+c,0) \in E$.
    \item $(a,0) \cdot (c,0) = (ac,0) \in E$.
    \item El opuesto de $(a,0)$ es $(-a,0) \in E$.
    \item El inverso de $(a,0) \not = (0,0)$ es $\left(\frac{1}{a},0\right) \in E$.
\end{itemize}
Esto nos dice que $E$ es subcuerpo de $\mathbb{C}$. Además $E$ es isomorfo a $\mathbb{R}$ (en sentido de cuerpos) mediante la siguiente identificación
\begin{align*}
    (a,0) \in E \longleftrightarrow a \in \mathbb{R}
\end{align*}
\section{Terminología y nomenclatura}
\begin{enumerate}
    \item[1)] Los elementos de $\mathbb{C} \longleftrightarrow \mathbb{R}^2$ se llaman números complejos.
    \item[2)] Si $(a,b) \in \mathbb{C}$, su parte real es $a$ y su parte imaginaria es $b$.
    \item[3)] $(1,0) \equiv 1$.
    \item[4)] $(0,1) \equiv i$.
\end{enumerate}
Mediante la identificación $E \longleftrightarrow \mathbb{R}$, tenemos que para $x,y \in \mathbb{R}$
\begin{itemize}
    \item $x \cdot 1 = x$.
    \item $y \cdot i = (0,y)$.
    \item $(x,y) = (x,0) + (0,y) = x + iy$.
\end{itemize}
De esta manera
\begin{align*}
    \mathbb{C} = \{ x + iy : x,y \in \mathbb{R} \}.
\end{align*}
Los números complejos se representan en $\mathbb{R}^2$ de la siguiente manera:

\begin{align*}
\includegraphics[width=0.4\textwidth]{imagenes/complejos.png}
\end{align*}

Si $z = x + iy \in \mathbb{C}$, entonces $\re(z) = x$ e $\im(z) = y$.
\begin{itemize}
    \item $\mathbb{C}$ no tiene orden ($\mathbb{R}$ sí).
    \item $i^2 = (0,1) \cdot (0,1) = (-1,0) = -1$.
\end{itemize}

\begin{defi}
Si $z = a + ib \in \com$, definimos su conjugado como $\overline{z} = a - ib$.
\end{defi}

Esta operación de conjugación se puede ver en $\mathbb{R}^2$ como la siguiente aplicación lineal

\begin{align*}
    \mathbb{R}^2 &\longrightarrow \mathbb{R}^2 \\
    \begin{pmatrix}
            x \\
            y 
        \end{pmatrix} &\longmapsto \begin{pmatrix}
            1 & 0 \\
            0 & -1 
        \end{pmatrix} \begin{pmatrix}
            x \\
            y 
        \end{pmatrix}
\end{align*}

Algunas propiedades innmediatas son
\begin{enumerate}
    \item[1)] $\overline{0} = 0$.
    \item[2)] $\overline{1} = 1$.
    \item[3)] $\overline{z + w} = \overline{z} + \overline{w}$, $z,w \in \com$.
    \item[4)] $\overline{z \cdot w} = \overline{z} \cdot \overline{w}$, $z,w \in \com$.
    \item[5)] Involución : $\overline{\overline{z}} = z$, $z \in \com$.
    \item[6)] Si $z = a + ib \in \com$,
    \begin{align*}
        \re(z) = a = \frac{z + \overline{z}}{2} \ \ \ \text{y} \ \ \ \im(z) = b = \frac{z - \overline{z}}{2i}.
    \end{align*}
\end{enumerate}

\section{$\mathbb{C}$ como espacio vectorial}
Al estar $\mathbb{C}$ identificado con $\mathbb{R}^2$, tenemos que $\mathbb{C}$ es un espacio vectorial de dimensión 2. La base canónica es $\{1,i\}$. Pero como $\mathbb{C}$ es un cuerpo, tenemos que es un espacio vectorial complejo de dimensión y tiene como base canónica $\{1\}$.
\\
\newline
Veamos como son las aplicaciones lineales de $\mathbb{C}$ en $\mathbb{C}$.
\begin{itemize}
    \item \textbf{Punto de vista real}.
    \begin{align*}
        L : \com &\longrightarrow \com \\
        \begin{pmatrix}
            x \\
            y 
        \end{pmatrix} &\longmapsto L\begin{pmatrix}
            x \\
            y 
        \end{pmatrix} = xL(1) + yL(i)
    \end{align*}
    En términos de números complejos, $z = x + iy$. Entonces
    \begin{align*}
        L(z) &= L \begin{pmatrix}
            x \\
            y 
        \end{pmatrix} = \begin{pmatrix}
            a_{11} & a_{12} \\
            a_{21} & a_{22} 
        \end{pmatrix} \begin{pmatrix}
            x \\
            y 
        \end{pmatrix} \\
        &= \\
        & \ \vdots \\
        &= \frac{(a_{11} + a_{22}) + i(-a_{12} + a_{21})}{2}z + \frac{(a_{11} - a_{22}) +i(a_{12} + a_{21})}{2}\overline{z} \\
        & = \alpha z + \beta \overline{z}.
    \end{align*}
    \item \textbf{Punto de vista complejo}
    \begin{align*}
         L : \com &\longrightarrow \com \\
         z &\longmapsto zL(1)
    \end{align*}
\end{itemize}

Veamos como son las \textbf{rectas de números complejos}. En $\mathbb{R}$ una recta es de la forma
\begin{align*}
    Ax + By + C = 0, \ \ \ A,B,C \in \mathbb{R}, |A| + |B| > 0.
\end{align*}
En términos de números complejos
\begin{align*}
    0 &= A \frac{z + \overline{z}}{2} + B \frac{z - \overline{z}}{2i} + C = \frac{A - iB}{2}z + \frac{A +iB}{2}\overline{z} + C \\
    &= \beta z + \overline{\beta}\overline{z} + \gamma \ \ \text{donde } \beta = \frac{A - iB}{2}, \gamma = C.
\end{align*}
Nos queda que la ecuación de una recta en el plano complejo es
\begin{align*}
\boxed{
    (E) \ \beta z + \overline{\beta}\overline{z} + \gamma = 0, \ \ \ \beta \in \com, \beta \not = 0, \gamma \in \mathbb{R}
    }
\end{align*}

\begin{defi}
Definimos el módulo o valor absoluto de un número complejo como la aplicación
\begin{align*}
    |.| : \com &\longrightarrow\mathbb{R}^+ \\
    z &\longmapsto \sqrt{\re(z)^2 + \im(z)^2} = \sqrt{z \cdot \overline{z}}.
\end{align*}
\end{defi}
Veamos que el módulo es, efectivamente, una norma.
\begin{proof}
\begin{enumerate}
    \item[1)] $|z| \ge 0$.
    \item[2)] $|z| = 0 \Longleftrightarrow z = 0$.
    \item[3)] Desigualdad triangular : $|z + w| \leq |z| + |w|$. Veamoslo. Sean $z = x + iy$, $w = u +iv$ donde $x,y,u,v \in \mathbb{R}$.
    \\
    \newline
    \textit{Cuentas previas}
    \begin{enumerate}
        \item[(i)] 
        \begin{align*}
            \re(z) &= x \leq |x| = \sqrt{x^2} \leq \sqrt{x^2 + y^2} = |z| \\
            \im(z) &= y \leq |y| = \sqrt{y^2} \leq \sqrt{x^2 + y^2} = |z|.
        \end{align*}
        \item[(ii)] $|z \cdot w| = \sqrt{(zw) \cdot \overline{(zw)}} = \sqrt{z\overline{z}w\overline{w}} = \sqrt{|z|^2|w|^2} = |z|\cdot |w|$.
        \item[(iii)] $|\overline{z}| = |z|$.
    \end{enumerate}
    Ahora si, pasamos a probar la desigualdad triangular.
    \begin{align*}
        |z+w|^2 &= (z+w)\overline{(zw)} = z\overline{z} + w\overline{w} + z\overline{w} + \overline{z}w \\
        &= |z|^2 + |w|^2 + 2\re(z\overline{w}) \\
        & \leq |z|^2 + |w|^2 + 2|z\overline{w}| = |z|^2 + |w|^2 + 2|z||\overline{w}| \\
        &= |z|^2 + |w|^2 + 2|z||w|\\
        &= (|z| + |w|)^2
    \end{align*}
    Luego $|z+w| \leq |z| + |w|$.
    \item[4)] Compatibilidad de la norma con el producto por escalares.
    \begin{align*}
        |\lambda z| = |\lambda| |z|, \ \ \ \lambda, z \in \com.
    \end{align*}
\end{enumerate}
\end{proof}
El hecho de tener definida la multiplicación en $\com$ y la propiedad $|zw| = |z||w|$ nos dice que $\com$ es un álgebra (real o compleja) conmutativa (por ser la multiplicación conmutativa). La norma que hemos defiindo en $\com$ viene del siguiente producto escalar complejo
\begin{align*}
    <.,.> : \com \times \com &\longrightarrow \com \\
    (z,w) &\longmapsto <z,w>_{\com} = z\overline{w}
\end{align*}
Veamos que es, efectivamente, un producto escalar complejo
\begin{proof}
\begin{enumerate}
    \item[1)] Sesguilinealidad (lineal por la izquierda y lineal conjugado por la derecha). Dados $\lambda_1, \lambda_2 \in \com$, $z,z_2,z_2,w,w_1,w_2 \in \com$ entonces
    \begin{align*}
        <\lambda_1z_2 + \lambda_2z_2,w>_{\com} &= \lambda_1<z_1,w>_{\com} + \lambda_2<z_2,w>_{\com} \\
        <z,\lambda_2w_2 + \lambda_2w_2>_{\com} &= \overline{\lambda_1}<z,w_1>_{\com} + \overline{\lambda_2}<z,w_2>_{\com}
    \end{align*}
    \item[2)] Hermeticidad (simetría conjugada). Dados $z,w \in \com$ entonces
    \begin{align*}
        <z,w>_{\com} = z\overline{w} = \overline{\overline{z\overline{w}}} = \overline{\overline{z}w} = \overline{w\overline{z}} = \overline{<w,z>_{\com}}
    \end{align*}
    \item[3)] Definido positivo. Dado $z \in \com$. Entonces
    \begin{align*}
        &<z,z>_{\com} = z\overline{z} = |z|^2 \ge 0 \ \ \ \text{y} \\
        &<z,z>_{\com} = 0 \Longleftrightarrow |z|^2 = 0 \Longleftrightarrow z = 0.
    \end{align*}
\end{enumerate}
\end{proof}
Podemos ver este producto escalar en $\mathbb{R}^2$ en términos complejos. Sean 
\begin{align*}
    z_1 = x_1 + iy_1 \text{ \ \ \ y \ \ \ } z_2 = x_2 + iy_2,
\end{align*}
$x_1,y_1,x_2,y_2 \in \mathbb{R}$. Entonces
\begin{align*}
    <(x_1,y_1), (x_2,y_2)>_{\mathbb{R}^2} = x_1x_2 + y_1y_2 = \re(z_1\overline{z_2}).
\end{align*}
Veamos como es una \texbf{circunferencia de números complejos} de centro $z_0 = x_0 +iy_0$ y radio $r > 0$.
\begin{align*}
 (E) \ \{ z \in \com : |z - z_0| = r \}  
\end{align*}
\begin{align*}
    0 &= |z-z_0|^2 - r^2 = ... = |z|^2 + |z_0|^2 - 2\re(z\overline{z_0}) - r^2 \\
    &= |z|^2 - \overline{z_0}z - z_0\overline{z} + |z_0|^2 - r^2.
\end{align*}
Multiplicando por $\alpha \in \mathbb{R}$, $\alpha \not = 0$
\begin{align*}
    0 &= \alpha |z|^2 - \alpha \overline{z_0}z - \alpha z_0\overline{z} + \alpha (|z_0|^2 - r^2) \\
\end{align*}
Llamando $\beta = -\alpha\overline{z_0}z$ y $\gamma =  \alpha (|z_0|^2 - r^2)$ nos queda que
\begin{align*}
    \alpha|z|^2 + \beta z + \overline{\beta} \overline{z} + \gamma = 0
\end{align*}
donde $\alpha, \gamma \inn \mathbb{R}$, $\alpha \not = 0$, $\beta \in \com$ y $|\beta|^2 > \alpha \gamma$. Si $\alpha = 0$ tendríamos la ecuación de una recta.
\\
\newline
Nos queda que la ecuación de una circunferencia (o recta) en el plano complejo es
\begin{align*}
\boxed{
    (E) \ \alpha|z|^2 + \beta z + \overline{\beta} \overline{z} + \gamma = 0, \ \ \ \alpha, \gamma \in \mathbb{R}, \beta \in \com, |\beta|^2 > \alpha \gamma
    }
\end{align*}

\section{Topología en $\com$}

La norma $|.|$ en $\com$ genera la siguiente métrica
\begin{align*}
    d(z_2,z_2) = |z_1 - z_2|, \ \ \ z_1,z_2 \in \com.
\end{align*}
Como sabemos, una métrica genera un topología. Las bolas las llamaremos discos.
\begin{defi}
Definimos
\begin{itemize}
    \item Disco abierto de centro $z_0$ y radio $r$
    \begin{align*}
        \Delta(z_0,r) = \mathbb{D}(z_0,r) = \{z \in \com : |z - z_0| < r \}.
    \end{align*}
    \item Circunferencia de centro $z_0$ y radio $r$
    \begin{align*}
        \partial \Delta(z_0,r) = \text{C}(z_0,r) = \{z \in \com : |z - z_0| =  r \}.
    \end{align*}
    \item Disco cerrado de centro $z_0$ y radio $r$
    \begin{align*}
        \overline{\Delta(z_0,r)} = \{z \in \com : |z - z_0| \leq r \}.
    \end{align*}
\end{itemize}
\end{defi}
Con la métrica inducida tenemos el concepto de convergencia de sucesiones y el concepto de continuidad.
\begin{obs}
$(\com, |.|)$ es un espacio vectorial normado completo, es decir, toda sucesión de Cauchy converge. En cuanto a la continuidad de funciones, $f: D \subset \com \longrightarrow \com$ es continua si y solo si
\begin{align*}
    &\re(f) : D \longrightarrow \mathbb{R} \text{ es continua y } \\
    &\im(f) : D \longrightarrow \mathbb{R} \text{ es continua}.
\end{align*}
Tenemos que $(\com, +, \cdot, |.|)$ es un álgebra de Banach conmutativa, luego la teoría de series de potencias tiene sentido completo en $\com$ y, en particular, podemos definir la exponencial de cualquier número complejo de la siguiente manera
\begin{align*}
    e^z = \sum_{n=0}^{\infty}{\frac{z^n}{n!}}, \ \ z \in \com,
\end{align*}
y como el producto es conmutativo,
\begin{align*}
    e^{z_1 + z_2} = e^{z_1} \cdot e^{z_2}, \ \ z_1,z_2 \in \com.
\end{align*}
\end{obs}
Algunas propiedades de la exponencial son
\begin{enumerate}
    \item[1)] $e^0 = 1$.
    \item[2)] $e^x = \sum_{n=0}^{\infty}{\frac{x^n}{n!}}$, $x \in \mathbb{R}$.
    \item[3)] $z = x +iy$, $x,y \in \mathbb{R}$ entonces
    \begin{align*}
        e^z = e^{x+iy} = e^{x}e^{iy} = e^{x}\sum_{n=0}^{\infty}{\frac{(iy)^n}{n!}} = ... = e^{x}(\cos(y) + i\sen(y))
    \end{align*}
    \begin{itemize}
        \item $\re(e^z) = e^x\cos(y) = e^{\re(z)}\cos(\im(z))$.
        \item $\im(e^z) = e^x\sen(y) = e^{\re(z)}\sen(\im(z))$.
        \newline
        Al ser $\re(e^z)$ e $\im(e^z)$ continuas, se tiene que $e^z$ es continua en $\com$.
    \item $|e^z| = \sqrt{(e^x\cos(y))^2 + (e^x\sen(y))^2} = e^x$, donde $z = x + iy \in \com$. En particular $|e^{i\theta}| = 1$, para todo $\theta \in \mathbb{R}$.
    \item $\overline{e^z} = \overline{e^{x +iy}} = \overline{e^x(\cos(y) + i\sen(y))} = e^x(\cos(y) - i\sen(y)) = e^x(\cos(y) + i\sen(-y)) = e^{\overline{z}}$.
    \end{itemize}
    \item[4)] La exponencial compleja es periódica de periodo $2\pi i$.
    \begin{align*}
        e^{z + 2\pi i} &= e^{x + i(y + 2\pi)} = e^x(\cos(y + 2\pi) + i\sen(y + 2\pi)) \\
        &= e^x(\cos(y) + i\sen(y)) = e^z.
    \end{align*}
    En particular, la exponencial no es inyectiva.
    \item[5)] La exponencial no es sobreyectiva, pues $e^z \not = 0$ para todo $z \in \com$. De hecho, el $0$ es el único número omitido por la exponencial, es decir, si $w \in \com \backslash \{0\}$, entonces existe $z \in \com$ tal que $e^z = w$.
\end{enumerate}

\section{Representación polar y exponencial de números complejos}

\begin{defi}
Si $z \in \comz$, definimos el argumento de z como el siguiente conjunto
\begin{align*}
    \arg(z) = \left\{ \theta \in \mathbb{R} : \cos(\theta) = \frac{\re(z)}{|z|}, \sen(\theta) = \frac{\im(z)}{|z|} \right\}
\end{align*}
\end{defi}

\begin{prop}
Si $\theta_0 \in \arg(z)$ entonces $\arg(z) = \{ \theta_0 + 2k\pi : k \in \mathbb{Z} \}$.
\end{prop}

\begin{proof}
Solo hace falta probar que $\arg(z) \supseteq \{ \theta_0 + 2k\pi : k \in \mathbb{Z} \}$ (la otra inclusión es fácil de ver). Sea $\theta_1 \in \arg(z)$. Entonces
\begin{align*}
    \cos(\theta_1) = \frac{\re(z)}{|z|} = \cos(\theta_0) \Longleftrightarrow \left\{ \begin{array}{lcc}
            \theta_1 = \theta_0 + 2k\pi : k \in \mathbb{Z} \\
            \theta_1 = -\theta_0 + 2k\pi : k \in \mathbb{Z}\\
             \end{array}
        \right.
\end{align*}
Si se da el primer caso, entonces se cumple la proposición. Supongamos que se da el segundo caso, entonces también ha de ocurrir que
\begin{align*}
    \sen(\theta_1) = \frac{\im(z)}{|z|} = \sen(\theta_0)
\end{align*}
luego, 
\begin{align*}
    \sen(-\theta_0) = \sen(-\theta_0 + 2k\pi) = \sen(\theta_0) \Longrightarrow 2\sen(\theta_0) = 0 \Longrightarrow \theta_0 = k_1\pi, \  k_1 \in \mathbb{Z}.
\end{align*}
De aquí
\begin{align*}
    \theta_1 &= -\theta_0 + 2k\pi = -k_1\pi + 2k\pi \\
    &= k_1\pi - 2k_1\pi + 2k\pi = k_1\pi + 2(k - k_1)\pi \\
    &= \theta_0 + 2(k - k_1)\pi.
\end{align*}
\end{proof}

\begin{defi}
Si $z \in \comz$, su argumento principal es 
\begin{align*}
    \argp(z) = \arg(z) \cap [-\pi,\pi)
\end{align*}
\end{defi}

\begin{ejemplo}
\begin{itemize}
    \item $\argp(1) = 0$.
    \item $\argp(i) = \frac{\pi}{2}$.
    \item $\argp(-1) = -\pi$.
    \item $\argp(-i) = -\frac{\pi}{2}$.
\end{itemize}
\end{ejemplo}

\begin{obs}
Si $z \in \comz$
\begin{align*}
    \argp(z) = \left\{ \begin{array}{lcc}
            \arccos\left( \frac{\re(z)}{|z|}\right) & si & \im(z) > 0 \\
            -\arccos\left( \frac{\re(z)}{|z|}\right) & si & \im(z) \leq 0\\
             \end{array}
        \right.
\end{align*}
Luego, $\argp : \com \backslash (-\infty,0] \longrightarrow \mathbb{R}$ es continua en $\com \backslash (-\infty,0]$ y no se puede ser extendida de forma continua a $(-\infty,0]$.
\end{obs}

\begin{defi}[Forma polar y exponencial]
Sea $z \in \comz$, entonces
\begin{itemize}
    \item Forma polar: $z = |z|(\cos\theta + i\sen\theta)$.
    \item Forma exponencial: $z = |z|e^{i\theta}$.
\end{itemize}
\end{defi}
La representación exponencial de un número complejo es muy útil, por ejemplo,
\begin{enumerate}
    \item[1)] Multiplicación de números complejos: Dados $z_1 = r_1e^{i\theta_1}$ y $z_1 = r_2e^{i\theta_2}$ tenemos que
    \begin{align*}
        z_1z_2 = r_1e^{i\theta_1}r_2e^{i\theta_2} = r_1r_2e^{i(\theta_1 + \theta_2)} = r_1r_2(\cos(\theta_1 + \theta_2) + i\sen(\theta_1 + \theta_2)) 
    \end{align*}
    \item[2)] Desarrollando el miembro izquierdo de 1) tenemos que
    \begin{align*}
        z_1z_2 &= r_1(\cos\theta_1 + i\sen\theta_1)r_2(\cos\theta_2 + i\sen\theta_2) \\
        &= r_1r_2(\cos\theta_1\cos\theta_2 - \sen\theta_1\sen\theta_2) + i(\cos\theta_1\sen\theta_2 + \sen\theta_1\cos\theta_2).
    \end{align*}
    Con lo que llegamos a la fórmula de la suma de ángulos en el coseno y el seno.
    \item[3)] Fórmula de Moivre.
    \begin{align*}
        (\cos\theta + i\sen\theta)^n = (e^{i\theta})^n = e^{in\theta} = \cos(n\theta) +i\sen(n\theta).
    \end{align*}
    \item[4)] Podemos ver $re^{i\theta}$ como un número complejo en la circunferencia de centro 0 radio $r$ que forma un ángulo $\theta$ con el eje real positivo.
\end{enumerate}

\begin{ejemplo}
\begin{itemize}
    \item $i = e^{i\frac{\pi}{2}}$.
    \item $3 = 3e^{i\cdot 0} = 3$.
    \item $-3i = 3e^{-i\frac{\pi}{2}} = 3e^{i\frac{3\pi}{2}}$.
\end{itemize}
\end{ejemplo}

Volvamos a la exponencial.
\begin{enumerate}
    \item[1)] Sabemos que la exponencial tiene periodo $2\pi i$. Veamos que este es el periodo más pequeño, para ello, hemos de probar que $e^z = e^w$ si y solo si $z - w = 2k\pi i$, $k \in \mathbb{Z}$. Solo tenemos que probar que si $e^z = e^w$ entonces $z - w = 2k\pi i$, $k \in \mathbb{Z}$ (la otra implicación ya la vimos).
    \begin{proof}
    \begin{align*}
        e^z = e^w \Longleftrightarrow e^{z-w} = 1.
    \end{align*}
    Escribimos $z -w = x + iy$, $x,y \in \mathbb{R}$. Entonces
    \begin{align*}
        e^{z-w} = 1 &\Longleftrightarrow e^x(\cos y + i\sen y) = 1 \Longleftrightarrow \left\{ \begin{array}{lcc}
            e^x\cos y = 1 \\
            e^x\sen y = 0\\
             \end{array}
        \right. \underset{e^x > 0}{\Longleftrightarrow} \left\{ \begin{array}{lcc}
            e^x\cos y = 1 \\
            \sen y = 0\\
             \end{array}
        \right. \\
        &\Longleftrightarrow \left\{ \begin{array}{lcc}
            e^x\cos y = 1 \\
            y = k\pi, \ k \in \mathbb{Z}\\
             \end{array}
        \right. \Longleftrightarrow \left\{ \begin{array}{lcc}
            e^x(-1)^k = 1 \\
            y = k\pi, \ k \in \mathbb{Z}\\
             \end{array}
        \right. \Longleftrightarrow \left\{ \begin{array}{lcc}
            e^x \cdot 1 = 1 \\
            y = k\pi, \ k \in \mathbb{Z}, \ k \text{ par}\\
             \end{array}
        \right. \\
        &\Longleftrightarrow \left\{ \begin{array}{lcc}
            x  = 0 \\
            y = k\pi, \ k \in \mathbb{Z}\\
             \end{array}
        \right. \Longleftrightarrow z - w = 2k\pi i, \ k \in \mathbb{Z}
    \end{align*}
    \end{proof}
    \begin{obs}
    Esto nos dice que $f(z) = e^z$ es inyectiva en cualquier banda horizontal de altura $2\pi$, es decir, del tipo
    \begin{align*}
        S_{y_0} = \{ x + iy \in \com : y \in [y_0, y_0 + 2\pi) \}.
    \end{align*}
    \end{obs}
    \item[2)] $0$ es el único valor omitido por $f(z) = e^z$.
    \begin{proof}
    Sea $w \in \comz$. Vamos a encontrar $z \in \com$ tal que $e^z = w$. Tomamos la representación exponencial $w = re^{i\theta}$, $r>0$, $\theta \in \mathbb{R}$. Basta tomar $x = \log r$ e $y = \theta$, así
    \begin{align*}
        e^z = e^{x + iy} = e^xe^{iy} = e^{\log r}e^{i\theta} = re^{i\theta} = w.
    \end{align*}
    \end{proof}
\end{enumerate}

\begin{obs}
Si $k \in \mathbb{Z}$ y $e^z = w$, entonces $e^{z + 2k\pi i} = w$.
\end{obs}

Introducimos ahora la definición de logaritmo complejo.
\begin{defi}
Si $z \in \comz$, definimos el logaritmo complejo de z como
\begin{align*}
    \log(z) &= \{ w \in \com : e^w = z \} \\
    &= \log |z| + i\arg(z).
\end{align*}
Y definimos el logaritmo principal como $\logp(z) = \log|z| + i\argp(z)$.
\end{defi}
Algunas propiedades del logaritmo son
\begin{enumerate}
    \item[1)] Si $z_1,z_2 \in \comz$, $z_1 = r_1e^{i\theta_1}$, $z_2 = r_2e^{i\theta_2}$, $r_1,r_2 > 0$, $\theta_1 \in \arg(z_1)$ y $\theta_2 \in \arg(z_2)$, entonces
    \begin{align*}
        \arg(z_1z_2) = \arg(r_1e^{i\theta_1}r_2e^{i\theta_2}) = \arg(r_1r_2e^{i(\theta_1 + \theta_2)}) = \arg(z_1) + \arg(z_2).
    \end{align*}
    Con esto probado, es claro que $\log(z_1z_2) = \log(z_1) + \log(z_2)$ (como conjuntos).
    \item[2)] $\argp(z_1z_2)$ no es necesariamente $\argp(z_1) + \argp(z_2)$, basta tomar $z_1 = z_2 = i$.
    \begin{align*}
        \argp(i \cdot i) = \argp(-1) = -\pi \not = \pi = \frac{\pi}{2} + \frac{\pi}{2} = \argp(i) + \argp(i)
    \end{align*}
    \item[3)] $\argp$ es continua en $\com \backslash (-\infty,0]$ y no admite extensión continua a un conjunto mayor.
\end{enumerate}

\begin{defi}
\begin{itemize}
    \item Sea $A \subset \comz$, decimos que $\varphi : A \longrightarrow \mathbb{R}$ es una rama (continua) de $\arg(z)$ en $A$, si $\varphi$ es continua en $A$ y $\varphi(z) \in \arg(z)$ para cada $z \in A$.
    \item Sea $A \subset \com$ y sea $f: A \longrightarrow \com$ continua, decimos que $\varphi : A \longrightarrow \mathbb{R}$ es una rama (continua) de $\arg(f)$ en $A$, si $\varphi$ es continua en $A$ y $\varphi(z) \in \arg(f(z))$ para cada $z \in A$.
    \item Sea $A \subset \comz$, decimos que $\psi : A \longrightarrow \mathbb{R}$ es una rama (continua) de $\log(z)$ en $A$, si $\psi$ es continua en $A$ y $\psi(z) \in \log(z)$ para cada $z \in A$, o  equivalentemente, si $e^{\psi(z)} = z$ para todo $z \in A$.
    \item Sea $A \subset \com$ y sea $f: A \longrightarrow \com$ continua, decimos que $\psi : A \longrightarrow \mathbb{R}$ es una rama (continua) de $\arg(f)$ en $A$, si $\psi$ es continua en $A$ y $\psi(z) \in \log(f(z))$ para cada $z \in A$, o  equivalentemente, si $e^{\psi(z)} = f(z)$ para todo $z \in A$.
\end{itemize}
\end{defi}

\begin{ejemplo}
\begin{enumerate}
    \item[1)] $\argp$ es una rama del $\arg(z)$ en $\com \backslash (-\infty,0]$.
    \item[2)] $\logp$ es una rama del $\log(z)$ en $\com \backslash (-\infty,0]$.
    \item[3)] $\varphi_0(z) = \arg(z) \cap [0,2\pi)$, $z \in \comz$ es continua en $\comz$ y por tanto es una rama de $\arg(z)$ en $\com \backslash [0,+\infty)$.
    \begin{proof}
    Observamos que $\varphi_0(z) = \pi + \argp(-z)$.
    \begin{enumerate}
        \item[(i)] $\varphi_0(z) \in \pi + [-\pi,\pi) = [0,2\pi)$.
        \item[(ii)] $\varphi_0(z) \in \arg(z)$ pues
        \begin{align*}
            \cos(\varphi_0(z)) &= \cos(\pi + \argp(-z)) = -\cos(\argp(-z)) = -\frac{\re(-z)}{|z|} = \frac{\re(z)}{|z|} \\
            \sen(\varphi_0(z)) &= \sen(\pi + \argp(-z)) = -\sen(\argp(-z)) = -\frac{\im(-z)}{|z|} = \frac{\im(z)}{|z|}.
        \end{align*}
        \item[(iii)] $\varphi_0$ es continua en $\com \backslash [0,+\infty)$ si y solo si $\pi + \argp(z)$ es continua en $\com \backslash [0,+\infty)$ si y solo si $\argp(-z)$ es continua en $\com \backslash [0+\infty)$ si y solo si $-z \not \in (-\infty,0]$ si y solo si $z \not \in [0,+\infty)$ si y solo si $z \in \com \backslash [0,+\infty)$.
        \item[(iv)] $\varphi_0$ no adite extensión continua a un conjunto mayor (pues $\argp$ no admite extensión continua a un conjunto mayor).
    \end{enumerate}
    \end{proof}
    \item[4)] Si $\theta_0 \in \mathbb{R}$ fijo, la función $\varphi_{\theta_0}(z) = \arg(z) \cap [\theta_0, \theta_0 + 2\pi)$, $z \in \comz$ está bien definida y es continua en $\com \backslash \{re^{i\theta_0} : r \ge 0 \}$. Por tanto, $\varphi_{\theta_0}$ es una rama del $\arg(z)$ en 
    \begin{align*}
        S_{\theta_0} = \com \backslash \{ re^{i\theta_0} : r \ge 0 \}
    \end{align*}
    y además
    \begin{align*}
        \varphi_{\theta_0}(z) = \theta_0 + \pi + e^{-i(\theta_0 + \pi)}z.
    \end{align*}
    \item[(5)] Si $\theta_0 \in \mathbb{R}$ fijo, la función $\psi_{\theta_0}(z) = \log|z| + i\varphi_{\theta_0}(z)$, $z \in \comz$ está bien definida y es continua en $\com \backslash \{re^{i\theta_0} : r \ge 0 \}$. Por tanto, $\psi_{\theta_0}$ es una rama de $\log(z)$ en
     \begin{align*}
        S_{\theta_0} = \com \backslash \{ re^{i\theta_0} : r \ge 0 \}
    \end{align*}
    y además
    \begin{align*}
        e^{\varphi_{\theta_0}(z)} = e^{\log|z| + i\varphi_{\theta_0}(z)} = |z|e^{i\varphi_{\theta_0}(z)} = z.
    \end{align*}
\end{enumerate}
\end{ejemplo}

\begin{obs}
Sea $\Omega \subset \comz$, $\varphi : \Omega \longrightarrow \mathbb{R}$ y $\psi : \Omega \longrightarrow \mathbb{R}$.
\begin{enumerate}
    \item[1)] $\varphi$ es una rama del $\arg(z)$ en $\Omega$ si y solo si $\log|z| + i\varphi(z)$ es una rama del $\log(z)$ en $\Omega$.
    \item[2)] $\psi$ es una rama del $\log(z)$ en $\Omega$ si y solo si $\re(\psi(z))$ ($= \log|z|$) e $\im(\psi(z))$ son ramas del $\arg(z)$ en $\Omega$.
\end{enumerate}
\end{obs}

\begin{prop}
Sea $\Omega \subset \comz$ conexo,
\begin{enumerate}
    \item[1)] Si $\varphi_1, \varphi_2$ son ramas de $\arg(z)$ en $\Omega$, entonces existe $k \in \mathbb{Z}$ tal que $\varphi_2(z) = \varphi_1(z) + 2k\pi$, $z \in \Omega$.
    \item[2)] Si $\psi_1, \psi_2$ son ramas de $\log(z)$ en $\Omega$, entonces existe $k \in \mathbb{Z}$ tal que $\psi_2(z) = \psi_1(z) + 2k\pi i$, $z \in \Omega$.
\end{enumerate}
\end{prop}

\begin{proof}
Probaremos solo 1). Para cada $z \in \Omega$, $\varphi_1(z), \varphi_2(z) \in \arg(z)$, luego existe $k(z) \in \mathbb{Z}$ tal que 
\begin{align*}
    \varphi_2(z) - \varphi_1(z) = 2k(z)\pi, \ \ z \in \Omega.
\end{align*}
Observaos que $\varphi_2 - \varphi_1$ es continua en $\Omega$, que es conexo, luego $(\varphi_2 - \varphi_1)(\Omega)$ es conexo y además es la imagen de un conjunto discreto, por tanto, se tiene que reducir (por continuidad) a un punto, es decir, existe $k \in \mathbb{Z}$ tal que
\begin{align*}
    \varphi_2(z) - \varphi_1(z) = 2k\pi, \ \ z \in \Omega.
\end{align*}
\end{proof}

\begin{prop}
Sea $A$ un conjunto conexo y $ f: A \longrightarrow \com$ continua.
\begin{enumerate}
    \item[1)] Si $\varphi_1,\varphi_2$ son ramas de $\arg(f)$ en $A$, entonces existe $k \in \mathbb{Z}$ tal que $\varphi_2(z) = \varphi_1(z) + 2k\pi$, $z \in A$.
    \item[2)] Si $\psi_1,\psi_2$ son ramas de $\log(f)$ en $A$, entonces existe $k \in \mathbb{Z}$ tal que $\psi_2(z) = \psi_1(z) + 2k\pi i$, $z \in A$.
\end{enumerate}
\end{prop}

\begin{prop}
En el plano complejo
\begin{itemize}
    \item No existe rama de $\arg(z)$ en $\comz$.
    \item No existe rama de $\log(z)$ en $\comz$.
\end{itemize}
De forma general, si $\Omega \subset \comz$ contiene una circunferencia $C$ de centro $0$ y radio $r$, entonces no existe una rama de $\arg(z)$ en $\Omega$ ni una rama de $\log(z)$ en $\Omega$.
\end{prop}

\begin{proof}
Basta demostrar que no hay una rama del $\arg(z)$ en $C$. Supongamos por reducción al absurdo que $\varphi: C \longrightarrow \mathbb{R}$ es una rama de $\arg(z)$ en $C$, entonces $\varphi$ es continua en $C$ y $\varphi(z) \in \arg(z)$ para todo $z \in C$.
\\
\newline
Observamos que $\argp(z)$ es rama del $\arg(z)$ en $C \backslash \{ -r \}$. Por la proposición anterior:
\begin{align*}
    \argp(z) = \varphi(z) + 2k\pi, \ \ z \in C \backslash \{-r\}.
\end{align*}
Esto nos dice que como $\varphi$ adimite extensión continua a $C$, entonces $\argp(z)$ admite extensión continua a $C$, lo que es imposible. Esta contradicción surge de suponer que existe $\varphi$ rama de $\arg(z)$ en $C$, luego no existe una rama de $\arg(z)$ en $C$.
\end{proof}

\begin{obs}
Acabamos de probar que, por ejemplo, la circunferencia unidad $\{ |z| = 1 \}$ no tiene rama de $\arg(z)$. Sin embargo, si consideramos una parametrización de la circunferencia unidad:
\begin{align*}
    \gamma : [-\pi, \pi] &\longrightarrow \com \\
    t &\longmapsto \gamma(t) = e^{it}
\end{align*}
Observamos que $\varphi : [-\pi, \pi] \longrightarrow \mathbb{R}$, $\varphi(t) = t$ es rama del $\arg(\gamma)$ en $[-\pi,\pi]$.
\end{obs}

\begin{teo}
Sean $a,b \in \mathbb{R}$ con $a < b$ y $\gamma : [a,b] \longrightarrow \comz$ una función continua. Entonces existe una rama de $\arg(\gamma)$ en $[a,b]$ y existe una rama del $\log(\gamma)$ en $[a,b]$.
\\
\newline
Estas ramas son únicas salvo adición de múltiplos enteros de $2\pi$ en el caso de $\arg(z)$, y adición de múltiplos enteros de $2\pi i$ en el caso de $\log(z)$.
\end{teo}

\begin{proof}
\underline{Existencia} : Como $\gamma$ es continua en el compacto $[a,b]$, entonces $\gamma([a,b])$ es compacto en $\comz$, luego la distancia de $\gamma([a,b])$ a $0$ es postiva y, en consecuencia, existe $r > 0$ tal que $|\gamma(t)| > r$ para todo $t \in [a,b]$.
\\
\newline
Como $\gamma$ es continua en el compacto $[a,b]$, entonces $\gamma$ es uniformemente continua en $[a,b]$, luego para $\varepsilon = r/2$, existe $\delta > 0$ tal que si $t,s \in [a,b]$ con $|t - s| < \delta$, entonces $|\gamma(t) - \gamma(s)| < r/2$. Sea $N \in \mathbb{N}$ tal que $\frac{b - a}{N} < \delta$. Particionamos el intervalo $[a,b]$ en $N$ trozos de igual longitud, tomando la siguiente partición
\begin{align*}
    \mathcal{P} = \left\{ t_0 = a, \ t_1 = t_0 + \frac{b - a}{N}, \ ..., \ t_j = t_0 + j\frac{b - a}{N}, \ ..., \ t_N = b \right\}.
\end{align*}
Observamos que para $j = 1,...,N$ se tiene que $\gamma([t_{j-1}, t_j]) \subset \Delta(\gamma(t_j), r/2)$. Efectivamente, si $t \in [t_{j-1},t_j]$, entonces 
\begin{align*}
    |t - t_j| \leq t_j - t_{j-1} = \frac{b - a}{N} < \delta,
\end{align*}
por tanto, $|\gamma(t) - \gamma(t_j)| < r/2$, o sea, $\gamma(t) \in \Delta(\gamma(t_j), r/2)$.
\end{proof}
Observamos también que
\begin{align*}
    \Delta(\gamma(t_j), r/2) \subset \com \backslash \{ re^{i(\theta_j - \pi)} : r \ge 0\}
\end{align*}
donde $\theta_j \in \arg(\gamma(t_j))$. Efectivamente, basta ver que $0 \not \in \Delta(\gamma(t_j), r/2)$. Si $z \in \Delta(\gamma(t_j), r/2)$ entonces
\begin{align*}
    |z| &= |z - \gamma(t_j) + \gamma(t_j)| \ge | \ |z - \gamma(t_j)| - |\gamma(t_j)| \ | \\
    & \ge |\gamma(t_j)| - |z - \gamma(t_j)| > r - \frac{r}{2} = \frac{r}{2}.
\end{align*}
Esto nos permite considerar la correspondiente rama del argumento, $\varphi_{\theta_j \pi}$, en $\com \backslash \{ re^{i(\theta_j - \pi)} : r \ge 0\}$, que también es rama del $\arg(z)$ en $\Delta(\gamma(t_j), r/2)$ (válido para cualquier $j \in \{1,...,N\}$). Cualquier otra rama del $\arg(z)$ en $\Delta(\gamma(t_j), r/2)$ se diferenciará de $\varphi_{\theta_j - \pi}$ en un múltiplo entero de $2\pi$.
\\
\newline
Empezamos con las definiciones. Consideramos una rama del $\arg(z)$ en $\Delta(\gamma(t_1), r/2)$ que llamaremos $\varphi_1$. Esto nos dice que $\varphi_1 \circ \gamma$ es rama del $\arg(\gamma)$ en $[t_0, t_1]$, pues $\gamma([t_0,t_1]) \subset \Delta(\gamma(t_1), r/2)$.
\\
\newline
Consideremos ahora $\Delta(\gamma(t_2), r/2)$, que sabemos que contiene a $\gamma([t_1,t_2])$ y además dista de $0$ más de $r/2$, luego es posible encontrar una rama del $\arg(z)$ en $\Delta(\gamma(t_2), r/2)$, $\varphi_2$ y tal que
\begin{align*}
    \varphi_2 (\gamma(t_1)) = \varphi_1(\gamma(t_1)).
\end{align*}
De esta manera, $\varphi_2 \circ \gamma$ es rama del $\arg(\gamma)$ en $[t_1,t_2]$ y además $\varphi_2 \circ \gamma (t_1) = \varphi_1 \circ \gamma (t_1)$.
\\
\newline
Continuamos este proceso hasta que lleguemos a definir una rama del $\arg(z)$, $\varphi_N$, en $\Delta(\gamma(t_N), r/2)$ que contiene a $\gamma([t_{N-1},t_N])$ y de manera que
\begin{align*}
    \varphi_N (\gamma(t_{N-1})) = \varphi_{N-1}(\gamma(t_{N-1})).
\end{align*}
De esta manera, $\varphi_N \circ \gamma$ es rama del $\arg(\gamma)$ en $[t_{N-1},t_N]$ y además $\varphi_N \circ \gamma (t_{N-1}) = \varphi_{N-1} \circ \gamma (t_{N-1})$.
\\
\newline
Esto da pie a que la función
\begin{align*}
    \varphi : [a,b] &\longrightarrow \mathbb{R} \\
    t &\longmapsto \varphi(t) = \varphi_j(\gamma(t)) \text{ si } t \in [t_{j-1},t_j], \ j \in \{1,...,N\},
\end{align*}
está bien definida, es continua y representa una rama del $\arg(z)$ en $[a,b]$.

\begin{obs}
Si $\gamma : [a,b] \longrightarrow \comz$ y $\varphi_1,\varphi_2$ son dos ramas del $\arg(z)$ en $[a,b]$, al ser $[a,b]$ conexo, existe $k \in \mathbb{Z}$ tal que
\begin{align*}
    \varphi_2(t) = \varphi_1(t) + 2k\pi
\end{align*}
de manera que
\begin{align*}
    \varphi_2(b) - \varphi_2(a) = \varphi_1(b) - \varphi_1(a)
\end{align*}
permanece invariante, y dicha constante se llama \textbf{variación del argumento de $\gamma$ en $[a,b]$} y se denota por
\begin{align*}
    \var_{\gamma}(\arg(z)) = \varphi(b) - \varphi(a).
\end{align*}
\end{obs}
\begin{defi}
Sea $n \in \mathbb{N}$, $n \ge 1$. 
\begin{itemize}
    \item Para $\Omega \in \com$, decimos que $h : \Omega \longrightarrow \com$ es rama de $\sqrt[n]{z}$ en $\Omega$, si $h$ es continua en $\Omega$ y $h(z)^n = z$ para todo $z \in \Omega$.
    \item Para $A$ conjunto y $f: A \longrightarrow \com$ continua, decimos que $h : A \longrightarrow \com$ es rama de $\sqrt[n]{f}$, si $h$ es continua en $A$ y $h(z)^n = f(z)$ para todo $z \in A$.
\end{itemize}
\end{defi}

\begin{obs}
\begin{itemize}
    \item $\sqrt[n]{0} = 0$.
    \item Sea $z = re^{i\theta}$, $r > 0$, entonces $z$ tiene exatamente $n$ raíces $n$-éseimas:
    \begin{align*}
        w_0 = r^{1/n}e^{i\frac{\theta}{n}}, \ w_1 = r^{1/n}e^{i\frac{\theta + 2\pi}{n}}, \ ..., \ w_{n-1} = r^{1/n}e^{i\frac{\theta + 2\pi(n-1)}{n}}.
    \end{align*}
    Además, dos raíces $n$-ésimas de $z$ distintas se diferencian en una raíz $n$-ésima de la unidad:
    \begin{align*}
        \frac{w_j}{w_k} = \frac{r^{1/n}e^{i\frac{\theta + 2\pi j}{n}}}{r^{1/n}e^{i\frac{\theta + 2\pi k}{n}}} = e^{i(j - k)\frac{2\pi}{n}}
    \end{align*}
    es raíz de la unidad.
\end{itemize}
\end{obs}

\begin{prop}
Sean $\Omega \subset \comz$ conexo y $n \in \mathbb{N}$, $n \ge 2$. Supongamos que $h_1,h_2$ son ramas de $\sqrt[n]{z}$ en $\Omega$, entonces existe $\xi$, raíz $n$-ésima de la unidad, tal que $h_2(z) = \xi h_1(z)$ para todo $z \in \Omega$.
\end{prop}

\begin{prop}
Sean $\Omega \subset \comz$ conexo y $n \in \mathbb{N}$, $n \ge 2$. Supongamos que $h$ es rama de $\sqrt[n]{z}$ en $\Omega$, entonces hay exactamente $n$ ramas de $\sqrt[n]{z}$ en $\Omega$.
\end{prop}

\begin{prop}
Sea $n \in \mathbb{N}$, $n \ge 2$
\begin{itemize}
    \item Sea $\Omega \subset \comz$ conexo. Supongamos que $\psi$ es una rama del $\log(z)$ en $\Omega$, entonces 
    \begin{align*}
        e^{\frac{1}{n}\psi(z)}, \ z \in \Omega
    \end{align*}
    es una rama de $\sqrt[n]{z}$ en $\Omega$.
    \item Sea $\Omega$ un conjunto y $f : \Omega \longrightarrow \com$ continua. Si $\psi$ es una rama del $\log(f)$ en $A$, entonces
    \begin{align*}
        e^{\frac{1}{n}\psi(z)}, \ z \in \Omega
    \end{align*}
    es una rama de $\sqrt[n]{f}$ en $A$.
\end{itemize}
\end{prop}

\begin{proof}
\begin{itemize}
    \item Como $\psi$ es rama del $\log(z)$ en $\Omega$ entonces
    \begin{align*}
        \left(e^{\frac{1}{n}\psi(z)} \right)^n = e^{\psi(z)} = z, \ z \in \Omega,
    \end{align*}
    es decir, $\psi$ es rama de $\sqrt[n]{z}$ en $\Omega$.
    \item Como $\psi$ es rama del $\log(f)$ en $\Omega$ entonces
    \begin{align*}
        \left(e^{\frac{1}{n}\psi(z)} \right)^n = e^{\psi(z)} = f(z), \ z \in \Omega,
    \end{align*}
    es decir, $\psi$ es rama de $\sqrt[n]{f}$ en $\Omega$. 
\end{itemize}
\end{proof}

\begin{obs}
Pueden existir ramas de $\sqrt[n]{f}$ en $A$ sin que existan ramas del $\log(f)$ en $A$.
\\
\newline
Basta considerar $f: \com \longrightarrow \com$, $f(z) = z^2$. Es claro que $f$ es continua en $\com$ y que $h(z) = z$ es rama de $\sqrt{f}$ en $\com$. Sin embargo no existe rama del $\log(f) = \log(z^2)$ en $\com$, no tiene sentido considerarla pues $0 \in f(\com)$.
\end{obs}

\begin{defi}
Para $\alpha, z \in \com$, $z \not = 0$, definimos el conjunto
\begin{align*}
    z^{\alpha} = e^{\alpha \log(z)} = \{e^{\alpha w} : w \in \log(z) \}.
\end{align*}
El valor principal de $z^{\alpha}$ es $e^{\alpha \logp(z)}$.
\end{defi}

\section{El Teorema Fundamental del Álgebra}

\begin{teo}[Teorema Fundamental del Álgebra]
Todo polinomio no constante con coeficientes complejos tiene al menos una raíz.
\end{teo}

\begin{proof}
Sea $P(z) = a_0 + a_1z + ... + z_nz^n$, $a_n \not = 0$ un polinomio de grado $n$ en la variable $z$, $a_0, a_1,...,a_n \in \com$. Observamos que
\begin{itemize}
    \item $P$ es continua en $\com$.
    \item Si $z \not = 0$
    \begin{align*}
        |P(z)|^n = |z|^n \left| \frac{a_0}{z^n} + \frac{a_1}{z^{n-1}} + ... + \frac{a_{n-1}}{z} + a_n\right| \ \Longrightarrow \ \lim_{z \to \infty}{|P(z)| = 0}.
    \end{align*}
    De aquí deducimos que $|P|$ alcanza un mínimo en $\com$ en cierto $z_0$. Veamos que $|P(z_0)| = 0$ (con lo que habremos terminado).
\end{itemize}
Por reducción al absurdo supongamos que $P(z_0) = \alpha \not = 0$. Consideramos
\begin{align*}
    Q(z) = P(z_0 + z) = a_0 + a_1(z_0 + z) + ... + a_n(z_0 + z)^n
\end{align*}
que vuelve a ser un polinomio de grado $n$ y satisface que $Q(0) = P(z_0) = \alpha \not = 0$ y si $z \in \com$
\begin{align*}
    |Q(z)| = |P(z_0 + z)| \ge |P(z_0)| = |Q(0)|,
\end{align*}
lo que nos dice que $|Q|$ tiene un mínimo en 0.
\\
\newline
Escribamos $Q$ de la siguiente forma
\begin{align*}
    Q(z) = \alpha + \beta z^m + c_{m+1}z^{m+1} + ... + c_n z^n
\end{align*}
donde $\beta \in \comz$, $c_{m+1},..,c_n \in \com$, $c_n \not = 0$ y $m \in \mathbb{N} \cap [1,n]$ es el menor de los exponentes positivos de $z$ que aparece en la expresión de $Q$. Como $\alpha, \beta \in \comz$ y $m \in \mathbb{N}$, $n \ge 1$, existe $\gamma \in \com$ tal que $\gamma^m = -\frac{\alpha}{\beta}$.
\\
\newline
Observamos que
\begin{align*}
    Q(\gamma z) &= \alpha + \beta(\gamma z) + c_{m+1}(\gamma z)^{m+1} + ... + c_n (\gamma z)^n \\
    &= \alpha + \beta \gamma^m z^m + \widetilde{Q}(z) \\
    &= \alpha - \alpha z^m + \widetilde{Q}(z)
\end{align*}
siendo $\widetilde{Q}$ un polinomio de grado $n$ (a no ser que $m = n$ en cuyo caso $Q = 0$) tal que
\begin{align*}
    \left| \frac{\widetilde{Q}(z)}{z^m}\right| \xrightarrow[z \to 0]{} 0.
\end{align*}
En virtud a esto, existe $\delta > 0$ tal que si $0 < |z| < \delta$, entonces
\begin{align*}
    |\widetilde{Q}(z)| < \frac{|\alpha|}{2}|z|^m
\end{align*}
(simplemente hemos aplicado la definición de límite). En particular, para $z = \frac{\delta}{2} < 1$, tenemos que $\gamma = \frac{\delta}{2} \not = 0$ y
\begin{align*}
    \left|Q\left(\gamma \frac{\delta}{2}\right)\right| &= \left| \alpha - \alpha\left(\frac{\delta}{2}\right)^m + \widetilde{Q}\left( \frac{\delta}{2} \right)\right| \leq \left| \alpha\left( 1 - \left(\frac{\delta}{2}\right)\right)^m \right| + \left|\widetilde{Q}\left( \frac{\delta}{2} \right)\right| \\
    &= |\alpha| \left| 1 - \left( \frac{\delta}{2}\right)^m\right| + \frac{|\alpha|}{2}\left|\frac{\delta}{2}\right|^m = |\alpha| \left( 1 - \left( \frac{\delta}{2}\right)^m\right) + \frac{|\alpha|}{2}\left|\frac{\delta}{2}\right|^m \\
    &= |\alpha|\left[ 1 - \left( \frac{\delta}{2}\right)^m + \frac{1}{2}\left( \frac{\delta}{2}\right)^m\right] = |\alpha| \left[ 1 - \frac{1}{2}\left( \frac{\delta}{2}\right)^m\right] \\
    & < \alpha
\end{align*}
Contradiciendo que $|\alpha|$ es el valor mínimo de $|Q|$.
\end{proof}

\section{La esfera de Riemann}
Consideramos lo que se conoce como compactificación de Alexandroff de $\com$. Lo denotaremos por
\begin{align*}
    \com^* = \widehat{\com} = \overline{\com} = \com \cup \{\infty\}.
\end{align*}
La topología de $\com^*$ se caracteriza porque los entornos básicos de puntos finitos siguen siendo los habituales (discos centrados en el punto), mientras que los entornos básicos del $\infty$ son exteriores de discos centrados en $0$.
\\
\newline
Sin embargo, la topología que acabamos de describir se puede enriquecer si le asociamos una métrica. Para llegar a esta métrica es conveniente obtener otra visualización de $\com^*$, mediante la identificación de $\com^*$ con $\mathbb{S}^2 = \{(X,Y,Z) : X^2 + Y^2 + Z^2 = 1\}$. De ahí que $\com^*$ se llame \textit{esfera de Riemann}.
\\
\newline
La identificación se conoce como \textit{proyección estereográfica} de $\mathbb{S}^2$ sobre $\com$ (identificado a su vez con el plano $Z = 0$ de $\mathbb{R}^3$).
\\
\newline
Definimos $\Pi : \mathbb{S}^2 \longrightarrow \com^*$ como la aplicación que lleva un punto $(X_0,Y_0,Y_0)$ de la esfera hacia un punto $x_0 + iy_0$ de $\com^*$ tal que dicho punto está en la recta que pasa por $N$ y $(X_0,Y_0,Z_0)$. Así, tenemos que $\Pi$ es
\begin{align*}
    &\Pi(N) = \infty \\
    &\Pi(X_0,Y_0,Z_0) = \frac{X_0}{1-Z_0} + i\frac{Y_0}{1 - Z_0}, (X_0,Y_0,Z_0) \not = N
\end{align*}
De igual modo, se puede probar que la inversa de $\Pi$, $\Pi^{-1} : \com^* \longrightarrow \mathbb{S}^2$ es
\begin{align*}
    &\Pi^{-1}(\infty) = N\\
    &\Pi(z_0) = \left( \frac{2\re(z_0)}{|z_0|^2 + 1}, \frac{2\im(z_0)}{|z_0|^2 + 1}, \frac{|z_0|^2 - 1}{|z_0|^2 + 1}\right)
\end{align*}
Observamos que $\Pi$ y $\Pi^{-1}$ son continuas, por tanto, $\Pi$ define un homeomorfismo entre $\mathbb{S}^2$ y $\com^*$.
\\
\newline
En $\mathbb{S}^2$, la topología inducida tiene asociada la distancia euclídea heredada de $\mathbb{R}^3$:
\begin{align*}
    d_3((X_1,Y_1,Z_1),(X_2,Y_2,Z_2)) = \sqrt{(X_1 - X_2)^2 + (Y_1 - Y_2)^2 + (Z_1 - Z_2)^2}
\end{align*}
y esta medida puede ser transferida a una métrica en $\com^*$ (que genera la topología) y se llama \textbf{métrica cordal}. Dados $z_1,z_2 \in \com^*$
\begin{align*}
    \rho(z_1,z_2) = d_3\left( \Pi^{-1}(z_1), \Pi^{-1}(z_2)\right) = ... = \frac{2|z_1 - z_2|}{\sqrt{|z_1|^2+1}\sqrt{|z_2|^2+1}}
\end{align*}

\begin{obs}
\begin{itemize}
    \item Bajo la métrica cordal, los polinomios admiten extensión a $\com^*$, de esta forma
    \begin{align*}
        \lim_{z \to \infty}{P(z) = \infty} \ (P(\infty) = \infty).
    \end{align*}
    \item También las funciones racionales $R = \frac{P}{Q}$ admiten extensión a $\com^*$. Si $z_0$ es un cero de $Q$ y no de $P$, entonces $R(z_0) = \infty$.
\end{itemize}
\end{obs}

\begin{prop}
Si $P(z) = a_0 + a_1z + ... + a_nz^n$, $a_n \not = 0$ y $Q(z) = b_0 + b_1z + ... + b_mx^m$, $b_m \not = 0$ son polinomios sin ceros en común. Entonces $R(z) = \frac{P(z)}{Q(z)}$, $z \in \com^*$, toma cada valor de $\com^*$ tantas veces como $\max\{n,m\}$.
\end{prop}

\begin{proof}
\underline{Caso 1} : Supongamos que $n > m$. Entonces $R(\infty) = \infty$ y 
\begin{align*}
    \frac{R(z)}{z}, ..., \frac{R(z)}{z^{n - m - 1}}
\end{align*}
tienen límite $\infty$ en $\infty$, por tanto, $R$ toma el valor $\infty$ $n -m veces$. También $R$ toma el valor $\infty$ en los $m$ ceros de $Q$ (que no son ceros de $P$), por tanto, $R$ toma el valor $\infty$ $(n - m) + m = n$ veces.
\\
\newline
Sea ahora $w \in \com^*$. Consideramos
\begin{align*}
    R(z) - w = \frac{P(z)}{Q(z)} - w = \frac{P(z) - wQ(z)}{Q(z)}
\end{align*}
donde el grado de $P - wQ$ es $n$ y el grado de $Q$ es $m$. Además estos polinomios no tienen ceros en común (porque $P$ y $Q$ no tienen ceros en común). Por tanto, $R - w$ es cero en tantos puntos como $P - wQ$ es cero, es decir, en $n$ puntos, lo que prueba que $R$ toma el valor $w$ $n$ veces.
\\
\newline
\underline{Caso 2}: Supongmos $n < m$. Consideramos $\frac{1}{R} = \frac{Q}{P}$. Aplicando el caso 1, $\frac{1}{R}$ toma cada valor de $\com^*$ $m$ veces y en consecuencia $R$ toma cada valor de $\com^*$ $m$ veces.
\\
\newline
\underline{Caso 3} : Supongamos $n = m$. En este caso $R(\infty) = \frac{a_n}{b_n}$. Consideramos
\begin{align*}
    R(z) - \frac{a_n}{b_n} = \frac{P(z)}{Q(z)} - \frac{a_n}{b_n} = \frac{P(z) - \frac{a_n}{b_n}Q(z)}{Q(z)}
\end{align*}
que es una función racional con grado del numerador menor estricto que el grado del denominador. Por el caso 2, tenemos que $R - \frac{a_n}{b_n}$ toma el valor cero $n$ veces, es decir, $R$ toma el valor $\frac{a_n}{b_n}$ $n$ veces.
\\
\newline
Por otro lado, $R$ toma el valor $\infty$ en los $n$ ceros de $Q$ (que no son ceros de $P$). Sea $w \in \com^* \backslash \{ \frac{a_n}{b_n} \}$, entonces
\begin{align*}
    R(z) - w = \frac{P(z)}{Q(z)} - w = \frac{P(z) - wQ(z)}{Q(z)}
\end{align*}
es una función racioal con grado del numerador igual que el grado del denominador. Además el numerador y el denominador no tienen ceros en común, por tanto, $R - w$ vale cero en los $n$ ceros de $P - wQ$, es decir, $R$ toma el valores $w$ $n$ veces.
\end{proof}
Algunas propiedades mas de $\com^*$
\begin{enumerate}
    \item[1)] Toda función de $\com^*$ en $\com^*$ tiene un equivalente de $\mathbb{S}^2$ en $\mathbb{S}^2$. Es decir, el siguiente diagrama conmuta
    \begin{align*}
        \xymatrix{
        \mathbb{S}^2 \ar[d]_{\Pi} \ar[r]^{\widetilde{T}} & \mathbb{S}^2 \ar[d]^{\Pi} \\
        \com^* \ar[r]_{T} & \com^*
        }
    \end{align*}
    \item[2)] $\Pi$ transforma circuferencias de $\mathbb{S}^2$ en rectas o circunferencias de $\com^*$.
    \item[3)] $\Pi : \mathbb{S}^2 \longrightarrow \com^*$ es una aplicación conforme, en el sentido de que preserva ángulos.
\end{enumerate}
\chapter{Variables aleatorias}

\section{Variable aleatoria. Función indicadora. Variables aleatorias simples}

\begin{defi}
    Sea $(\Omega, \mathcal{A})$ espacio probabilizable. Una aplicación $X: \Omega \longrightarrow \mathbb{R}$ diremos que es variable aleatoria si $X^{-1}(B) \in \mathcal{A}$ para todo $B \in \mathbb{B}_1$.
\end{defi}

\begin{defi}
    Sea $(\Omega, \mathcal{A})$ espacio probabilizable. Una aplicación $X: \Omega \longrightarrow \mathbb{R}$ diremos que es variable aleatoria si $X^{-1}((-\infty,a]) \in \mathcal{A}$ para todo $a \in \mathbb{R}$.
\end{defi}

\begin{obs}
    Estas dos definiciones son equivalentes.
\end{obs}

\begin{defi}
    Sea $(\Omega, \mathcal{A})$ espacio probabilizable y sea $A \in \mathcal{A}$. La función $I_A: \Omega \longrightarrow \mathbb{R}$ definida por
    \begin{align*}
        I_A(\omega) = \left\{ \begin{array}{lcc}
                                  1 & si & \omega \in \mathcal{A}      \\
                                  0 & si & \omega \not \in \mathcal{A} \\
                              \end{array}
        \right.
    \end{align*}
    es una variable aleatoria y se denomina función indicadora o variable indicadora.
\end{defi}

\begin{defi}
    Sea $(\Omega, \mathcal{A})$ espacio probabilizable y sea $\{A_i\}_{i=1}^{n}$ una partición de $\Omega$, $A_i \in \mathcal{A}$. Sean $x_1,...,x_n \in \mathbb{R}$. La aplicación $X: \Omega \longrightarrow \mathbb{R}$ definida por
    \begin{align*}
        X(\omega) = \left\{ \begin{array}{lcc}
                                x_1    & si     & \omega \in A_1  \\
                                x_2    & si     & \omega  \in A_2 \\
                                \vdots & \vdots & \vdots          \\
                                x_n    & si     & \omega  \in A_n
                            \end{array}
        \right.
    \end{align*}
    es una variable aleatoria y se denomina variable aleatoria simple. Nótese que $X(\omega) = \sum_{i=1}^{n}{x_iI_{A_i}(\omega)}$.
\end{defi}

\section{Propiedades y operaciones algebraicas con variables aleatorias}

\begin{teo}
    Sean X e Y variables aleatorias definidas sobre $(\Omega, \mathcal{A})$. Entonces
    \begin{enumerate}
        \item[(i)] $A = \{ \omega \in \Omega : X(\omega) < Y(\omega) \} \in \mathcal{A}$.
        \item[(ii)] $B = \{ \omega \in \Omega : X(\omega) \leq Y(\omega) \} \in \mathcal{A}$.
        \item[(iii)] $C = \{ \omega \in \Omega : X(\omega) = Y(\omega) \} \in \mathcal{A}$.
    \end{enumerate}
\end{teo}

\begin{proof}
    Probemos $(i)$. $A = \{ \omega \in \Omega : X(\omega) < Y(\omega) \}$. Sea $\omega_0 \in A$, entonces $X(\omega_0) < Y(\omega_0)$. Ambos son números reales, por tanto, existe $r_0 \in \mathbb{Q}$ tal que $X(\omega_0) < r_0 < Y(\omega_0)$. Por tanto
    \begin{align*}
        A = \{ \omega \in \Omega : X(\omega) < Y(\omega) \} & = \bigcup_{r \in \mathbb{Q}}{\{ \omega \in \Omega : X(\omega) < r < Y(\omega) \}}                                                \\
                                                            & = \bigcup_{r \in \mathbb{Q}}{\left(\{ \omega \in \Omega : X(\omega) < r \} \cap \{ \omega \in \Omega :  r < Y(\omega) \}\right)} \\
                                                            & = \bigcup_{r \in \mathbb{Q}}{\left( X^{-1}((-\infty, r)) \cap Y^{-1}((r,+\infty))\right)}
    \end{align*}
    Como $X$ e $Y$ son variables aleatorias se tiene que $ X^{-1}((-\infty, r)) \in \mathcal{A}$ e $Y^{-1}((r,+\infty)) \in \mathcal{A}$. Como $A$ es unión numerable de elementos de $\mathcal{A}$ se tiene que $A \in \mathcal{A}$.

    Para probar $(ii)$ basta ver que
    \begin{align*}
        B = \{ \omega \in \Omega : X(\omega) \leq Y(\omega) \} = (A^*)^c
    \end{align*}
    donde $A^* = \{ \omega \in \Omega : X(\omega) > Y(\omega) \}$. Como $A^* \in \mathcal{A}$ entonces $(A^*)^c = B \in \mathcal{A}$.

    Para probar $(iii)$ basta ver que
    \begin{align*}
        C = \{ \omega \in \Omega : X(\omega) = Y(\omega) \} = B \backslash A = B \cap A^c \in \mathcal{A}.
    \end{align*}
\end{proof}

\begin{prop}
    Sean X e Y variables aleatorias definidas sobre $(\Omega, \mathcal{A})$. Entonces
    \begin{enumerate}
        \item[(1)] Dado $a \in \mathbb{R}$ se tiene que Z = aX es variable aleatoria.
        \item[(2)] X + Y es variable aleatoria.
        \item[(3)] $X^2$ es variable aleatoria.
        \item[(4)] $|X|$ es variable aleatoria.
        \item[(5)] $X \cdot Y$ es variable aleatoria.
    \end{enumerate}
\end{prop}

\begin{teo}
    Sea X variable aleatoria definida sobre  $(\Omega, \mathcal{A})$. Sea $g: \mathbb{R} \longrightarrow \mathbb{R}$ una función tal que $g^{-1}(B) \in \mathbb{B}_1$ para todo $B \in \mathbb{B}_1$. Entonces $Y = g(X)$ es también una variable aleatoria definida sobre  $(\Omega, \mathcal{A})$.
\end{teo}

\begin{proof}
    \begin{align*}
        \Omega \xrightarrow[]{X} \mathbb{R} \xrightarrow[]{g} \mathbb{R}
    \end{align*}
    Seea $B \in \mathbb{B}_1$. Entonces
    \begin{align*}
        Y^{-1}(B) = X^{-1}(g^{-1}(B)),
    \end{align*}
    $g^{-1}(B) \in \mathbb{B}_1$ por hipótesis y como $X$ es variable aleatoria, entonces $X^{-1}(g^{-1}(B)) = Y^{-1}(B) \in \mathbb{B}_1$.
\end{proof}

\section{Probabilidad inducida en $\mathbb{R}$ por una variable aleatoria}
\begin{prop}
    Sea $(\Omega, \mathcal{A}), P$ un espacio de probabilidad y sea X una variable aleatoria definida en $\Omega$, esto es, $X: \Omega \longrightarrow \mathbb{R}$. Entonces X induce sobre $(\mathbb{R}, \mathbb{B}_1)$ una medida de probabilidad $P_X : \mathbb{B}_1 \longrightarrow [0,+\infty)$ dada por
    \begin{align*}
        P_X(B) = P[\{\omega \in \Omega : X(\omega) \in B\}] = P(X^{-1}(B))
    \end{align*}
    para todo $B \in \mathbb{B}_1$.
\end{prop}
\chapter{Modelos lineales generalizados}

\section{Introducción}
El modelo lineal generalizado es una generalización de la regresión lineal que permite que lo $y_i$ no sigan una distribución normal. Este modelo tiene tres componentes:
\begin{itemize}
    \item \textbf{Componente aleatoria.}
          Viene dada por la variable $Y$.
          Las $y_i$ pueden seguir varias distribuciones comunes, como la Bernoulli, Binomial, Binomial Negativa, Poisson y Gamma.
          Únicamente veremos el caso en el que $y_i \sim Ber(p)$.
    \item \textbf{Componente sistemática.}
          Viene dada por las variables $X_1, \dots, X_k$, que están relacionadas mediante el predictor lineal $\vec{x}_i'\vec{\beta}$, siendo $\vec{x}_i' = (1, x_{1i}, \dots, x_{ki})$.

    \item \textbf{Función enlace.}
          La función enlace proporciona la relación entre el predictor lineal y la media de la función de distribución.
          \begin{align*}
              E(y_i | x_{1i}, \dots, x_{ki}) = g_i(\vec{x}_i'\vec{\beta}) \Rightarrow \vec{x}_i'\vec{\beta} = g_i^{-1}(E(y_i | x_{1i}, \dots, x_{ki}))
          \end{align*}
          La función $g_i^{-1}$ es la \textbf{función enlace}.
          \begin{obs}
              Si $g_i = g = Id$ para todo $i$, se corresponde con el modelo de regresión lineal múltiple.
          \end{obs}
\end{itemize}

\section{Modelo de regresión con respuesta binaria}
Los modelos de regresión con respuesta binaria son aquellos en los que $y_i \sim Ber(p)$. En estos casos tenemos datos que queremos clasificar en dos poblaciones $A$ y $B$. El conocimiento de una serie de variables nos ayudará a determinar de qué población son.

Tenemos entonces un conjunto de datos $\{x_{1i}, \dots, x_{ki}, y_i\}$, donde $y_i$ es una variable dicotómica.
\begin{align*}
    y_i = \begin{cases}
              1 & \text{si el dato $i$-ésimo procede de $A$} \\
              0 & \text{si el dato $i$-ésimo procede de $B$}
          \end{cases}
\end{align*}
Así que $y_i \sim Ber(p_i)$ con $p_i = P(Y_i = 1)$. Queremos estimar $\widehat{p}_i = \widehat{P(Y_i = 1)}$, es decir, la probabilidad de que el individuo $i$ sea de la población $A$.
\begin{align*}
     & E(y_i | x_{1i}, \dots, x_{ki}) = p_i = g_i(\vec{x}_i'\vec{\beta}) \\
     & \widehat{E}(y_i | x_{1i}, \dots, x_{ki}) = \widehat{p}_i
\end{align*}
Sin embargo, este modelo tiene un problema y es que, queremos estimar $p_i$, que es una probabilidad, por tanto, $p_i \in [0,1]$, y en principio, según este modelo, $p_i$ podría tomar cualquier valor real. Pero este problema, se puede arreglar de la siguiente forma, tomar $p_i = F(\vec{x}_i'\vec{\beta})$, con $F$ función de distribución. Usaremos dos funciones de distribución:
\begin{itemize}
    \item \textbf{Función de distribución logística}.
          \begin{align*}
              F(x) = \frac{1}{1 + e^{-x}}, \quad x \in \mathbb{R}
          \end{align*}
          La función $F^{-1}$ es la función enlace y se conoce como \textbf{función logit}.
          Podemos calcularla:
          \begin{align*}
              p_i = F(\vec{x}_i'\vec{\beta}) = \frac{1}{1 + e^{-\vec{x}_i'\vec{\beta}}} & \Longleftrightarrow p_i + p_ie^{-\vec{x}_i'\vec{\beta}} = 1 \Longleftrightarrow e^{\vec{x}_i'\vec{\beta}} = \frac{p_i}{1 - p_i} \Leftrightarrow \\
                                                                                        & \Longleftrightarrow \vec{x}_i'\vec{\beta} = \log\left(\frac{p_i}{1-p_i}\right)
          \end{align*}
    \item \textbf{Función de distribución normal (estándar)}.
          \begin{align*}
              \Phi(x) = \int_{-\infty}^x \frac{1}{\sqrt{2\pi}} e^{-\frac{1}{2}t^2} dt
          \end{align*}
          La función $\Phi^{-1}$ es la función enlace y se conoce como \textbf{función probit}.
\end{itemize}

\begin{figure}[h]
    \centering
    \includegraphics[width=0.6\textwidth]{imagenes3/logit.png}
\end{figure}

\section{Riesgo, oportunidad, riesgo relativo y odds ratio}
Iluestremos estos conceptos con un ejemplo.
\begin{ejemplo}
    En 1 de cada 200 nacimientos ocurre un parto gemelar. Entonces la probabilidad o \textit{riesgo} de que un embarazo elegido al azar de un parto gemelar es $R_1 = \frac{1}{200}$. Hay un forma de expresar lo mismo en términos de apuestas, de 200 partos, 1 es gemelar y 199 no lo son, es decir $O_1 = \frac{1}{199}$. Nótese que
    \begin{align*}
        O_1 = \frac{R_1}{1-R_1}.
    \end{align*}
    $O_1$ es la \textit{oportunidad}. Se observó que entre 100 muejeres que habían tomado ácido fólico, 3 de cada 200 partos eran gemelar. Ahora,
    \begin{align*}
        R_2 = \frac{3}{200}, \quad O_2 = \frac{3}{197} = \frac{R_2}{1-R_2}.
    \end{align*}
    El aumento del riesgo del embarazo gemeral se puede expresar numéricamente como
    \begin{align*}
        RR :=  \frac{R_2}{R_1} = 3, \quad OR:= \frac{O_2}{O_1} \frac{199 \cdot 3}{197} \approx 3.03.
    \end{align*}
\end{ejemplo}

\begin{defi} \
    \begin{itemize}
        \item El riesgo es la probabilidad de que ocurra un resultado.
        \item La oportunidad es el cociente del número de eventos que producen un resultado entre el número de eventos que no lo producen.
        \item  El riesgo relativo es el cociente de los riesgos de dos grupos de población.
              \begin{align*}
                  RR = \frac{R_1}{R_2}.
              \end{align*}
        \item La razón de oportunidades es el cociente de las oportunidades de dos grupos de población.
              \begin{align*}
                  OR = \frac{O_1}{O_2}.
              \end{align*}
    \end{itemize}
\end{defi}

\begin{obs}
    Existe una relación entre el riesgo y la oportunidad:
    \begin{align*}
        O = \frac{R}{1-R}.
    \end{align*}
\end{obs}

\section{Regresión logística}
El modelo de regresión logística es:
\begin{align*}
    E(y_i | x_{1i}, \dots, x_{ki}) = p_i = F(\vec{x}_i'\vec{\beta}) = \frac{1}{1 + e^{-\vec{x}_i'\vec{\beta}}}
\end{align*}
Además, hemos visto que:
\begin{align*}
    \vec{x}_i'\vec{\beta} = \log\left(\frac{p_i}{1-p_i}\right)
\end{align*}
Luego queremos estimar:
\begin{align*}
    \widehat{p}_i = \frac{1}{1 + e^{-\vec{x}_i'\widehat{\vec{\beta}}}}
\end{align*}
Para encontrar los estimadores $\widehat{\beta}_i$ usamos el método de máxima verosimilitud.
Calculamos la función de verosimilitud:
\begin{align*}
    L(\beta_0, \dots, \beta_k) = \prod_{i=1}^n p_i^{y_i}(1-p_i)^{1-y_i}
\end{align*}
Tomamos logaritmos en ambos miembros de la igualdad:
\begin{align*}
    \log L(\beta_0, \dots, \beta_k) & = \sum_{i=1}^n y_i\log(p_i) + \sum_{i=1}^n (1-y_i)\log(1-p_i)                                                                                                                                                              \\
                                    & = \sum_{i=1}^n y_i\log\left(\frac{p_i}{1-p_i}\right) + \sum_{i=1}^n \log(1-p_i) = \sum_{i=1}^n y_i \vec{x}_i'\vec{\beta} + \sum_{i=1}^n \log\left(\frac{e^{-\vec{x}_i'\vec{\beta}}}{1 + e^{-\vec{x}_i'\vec{\beta}}}\right) \\
                                    & = \sum_{i=1}^n y_i \vec{x}_i'\vec{\beta} + \sum_{i=1}^n \log\left(\frac{1}{e^{\vec{x}_i'\vec{\beta}} + 1}\right) = \sum_{i=1}^n y_i \vec{x}_i'\vec{\beta} - \sum_{i=1}^n \log(1 + e^{\vec{x}_i'\vec{\beta}})
\end{align*}
Así que, derivando tenemos que:
\begin{align*}
    \frac{\partial \log L}{\partial \vec{\beta}} = \sum_{i=1}^n y_i\vec{x}_i - \sum_{i=1}^n \frac{\vec{x}_i e^{\vec{x}_i'\vec{\beta}}}{1 + e^{\vec{x}_i'\vec{\beta}}}
\end{align*}
Para hallar los $\widehat{\beta}_i$ hay que resolver el sistema $\frac{\partial \log L}{\partial \beta_i} = 0$ para todo $i = 0, \dots, k$ numéricamente.

Podemos darles significado a los $\beta_j$. Para ello calculamos la oportunidad:
\begin{align*}
    O(x_{1i}, \dots, x_{ki}) = \frac{p_i}{1-p_i} = \frac{P(Y_i = 1)}{P(Y_i = 0)} = \dfrac{\frac{1}{1+e^{-\vec{x}_i'\vec{\beta}}}}{\frac{e^{-\vec{x}_i'\vec{\beta}}}{1+e^{-\vec{x}_i'\vec{\beta}}}} = e^{\vec{x}_i'\vec{\beta}} = e^{\beta_0 + \beta_1x_{1i} + \dots + \beta_kx_{ki}}
\end{align*}
Calculamos ahora la razón de oportunidades cuando aumenta $x_{ji}$ en una unidad:
\begin{align*}
    OR_j = \frac{O(x_{1i}, \dots, x_{ji}+1, \dots, x_{ki})}{O(x_{1i}, \dots, x_{ji}, \dots, x_{ki})} = e^{\beta_j}
\end{align*}
Así que $e^{\beta_j}$ es lo que varía la oportunidad cuando aumenta la componente $j$-ésima en una unidad. Luego $OR_j$ determina si las variable $j$-ésima es significativa.
\chapter{Construcción de medidas}

\section{Medidas exteriores}

\begin{defi}
Sea X un conjunto. Una medida exterior sobre X es una aplicación $\mu^*: \mathcal{P}(X) \longrightarrow [0,+\infty]$ tal que
\begin{enumerate}
    \item[(a)] $\mu^*(\emptyset) = 0$.
    \item[(b)] Si $A \subset B$ entonces $\mu^*(A) \leq \mu^*(B)$.
    \item[(c)] (Subaditividad finita)
    \begin{align*}
        \mu^*\left(\bigcup_{i=1}^{\infty}{A_i} \right) \leq \sum_{i=1}^{\infty}{\mu^*(A_i)}
    \end{align*}
    para cualquier colección numerable $\{A_i\}_{i=1}^{\infty}$ de subconjuntos de X.
\end{enumerate}
\end{defi}

\subsection{Construcción de medidas exteriores}

\begin{defi}
Sea X un conjunto. Decimos que $\mathcal{E}$ es una familia recubridora de X si
\begin{enumerate}
    \item[(a)] $\mathcal{E} \subset \mathcal{P}(X)$.
    \item[(b)] $\emptyset \in \mathcal{E}$.
    \item[(c)] Existe una colección $\{X_i\}_{i=1}^{\infty}$ de elementos de $\mathcal{E}$ tal que 
    $X = \cup_{i=1}^{\infty}{X_i}$.
\end{enumerate}
\end{defi}

\begin{teo}
Supongamos que X es un conjunto y que $\mathcal{E}$ es una familia recubridora de X. Sea $\rho: \mathcal{E} \longrightarrow [0,+\infty]$ una aplicación tal que $\rho(\emptyset) = 0$
\begin{enumerate}
    \item[(a)] Sea $\mu_{\rho}^*: \mathcal{P}(X) \longrightarrow [0,+\infty]$ definida por
    \begin{align*}
        \mu_{\rho}^*(A) = \inf(H_A)
    \end{align*}
    donde
    \begin{align*}
        H_A = \left\{ \lambda \in [0,+\infty] : \lambda = \sum_{i=1}^{\infty}{\rho(E_i)} : A \subset \bigcup_{i=1}^{\infty}{E_i}, E_i \in \mathcal{E} \right\}.
    \end{align*}
    Entonces $\mu_{\rho}^*$ es una medida exterior y $\mu_{\rho}^*(A) \leq \rho(A)$ para todo $A \in \mathcal{E}$. (Nótese que el ínfimo de $H_A$ está bien definido y es un elemento de $[0,+\infty]$).
    \item[(b)] Supongamos que, además, la aplicación $\rho$ tiene la propiedad siguiente
    \begin{align*}
        E \subset \bigcup_{i=1}^{\infty}{E_i}, \ (E, E_i \in \mathcal{E}) \Longrightarrow \rho(E) \leq \sum_{i=1}^{\infty}{\rho(E_i)}.
    \end{align*}
    Entonces $\mu_{\rho}^*$ es una medida exterior tal que $\mu_{\rho}^*(A) = \rho(A)$ para todo $A \in \mathcal{E}$.
\end{enumerate}
\end{teo}

\begin{proof}
$(a)$ Observamos primero que $\mu_{\rho}^*$ está bien definida porque cada subconjunto $A$ es recubierto por alguna familia numerable de $\maathcal{E}$ ya que $A \subset X = \cup_{i=1}^{\infty}{X_i}$.
\\
\newline
Antes de probar que $\mu_{\rho}^*$ es una medida exterior, veremos que $\mu_{\rho}^*(A) \leq \rho(A)$ para todo $A \in \mathcal{E}$. Fijado dicho $A$, tomamos la sucesión $\{E_i\}_{i=1}^{\infty}$ definida por $E_1 = A$, $E_i = \emptyset$ para todo $i \ge 2$. Es claro que $A \subset \cup_{i=1}^{\infty}{E_i}$ y que $E_i \in \mathcal{E}$ para todo $i$. Luego $\rho(A) = \rho(E_1) = \sum_{i=1}^{\infty}{\rho(E_i)} \in H_A$ y, por lo tanto,
\begin{align*}
    \mu_{\rho}^*(A) = \inf(H_A) \leq \rho(A).
\end{align*}
Ahora probemos que $\mu_{\rho}^*$ es una medida exterior. Por hipótesis, $\rho(\emptyset) = \sum_{i=1}^{\infty}{\emptyset} = 0 \in H_{\emptyset}$, luego $\mu_{\rho}^*(\emptyset) = 0$. Sean $A, B \in \mathcal{E}$ tales que $A \subset B$, entonces $H_B \subset H_A$, por tanto, $\inf(H_A) \leq \inf(H_B)$, es decir, $\mu_{\rho}^*(A) \leq \mu_{\rho}^*(B)$. Para que $\mu_{\rho}^*$ sea medida exterior solo nos queda probar la subaditividad numerable.
\\
\newline
Fijemos $\varepsilon > 0$. Entonces, para cada $i$, consideramos $\mu_{\rho}^*(A_i) + \frac{\varepsilon}{2^i}$. Como $\mu_{\rho}^*(A_i) = \inf(H_{A_i})$, existe $\lambda_i \in H_{A_i}$ tal que
\begin{align*}
    \lambda_i \leq \mu_{\rho}^*(A_i) + \frac{\varepsilon}{2^i}.
\end{align*}
Como $\lambda_i \in H_{A_i}$, existe una familia numerable $\{E_{ij}\}$, $E_{ij} \in \mathcal{E}$, tal que
\begin{align*}
    A_i \subset \bigcup_{j=1}^{\infty}{E_{ij}} \ \ y \ \ \lambda_i = \sum_{j=1}^{\infty}{\rho(E_{ij})}.
\end{align*}
Es claro que $\cup_{i=1}^{\infty}{A_i} \subset \cup_{i,j=1}^{\infty}{E_{ij}}$. Si $\sigma: \mathbb{N} \longrightarrow \mathbb{N} \times \mathbb{N}$ es una biyección, tenemos que $\{E_{\sigma(k)}\}_{k=1}^{\infty}$ es una colección numerable de elementos de $\mathcal{E}$ tal que $\cup_{i=1}^{\infty}{A_i} \subset \cup_{k=1}^{\infty}{E_{\sigma(k)}} = \cup_{i,j=1}^{\infty}{E_{ij}}$. Por lo tanto, por la definición de $\mu_{\rho}^*$,
\begin{align*}
    \mu_{\rho}^*\left( \bigcup_{i=1}^{\infty}{A_i}\right) & \leq \sum_{k=1}^{\infty}{\rho(E_{\sigma(k)})} = \sum_{i=1}^{\infty}{\left( \sum_{j=1}^{\infty}{\rho(E_{ij})}\right)}\\
    & = \sum_{i=1}^{\infty}{\lambda_i} \leq \sum_{i=1}^{\infty}{(\mu_{\rho}^*\left(A_i) + \frac{\varepsilon}{2^i}\right)} = \sum_{i=1}^{\infty}{\mu_{\rho}^*(A_i) + \varepsilon}.
\end{align*}
Haciendo que $\varepsilon \to 0^+$ obtenemos
\begin{align*}
    \mu_{\rho}^*\left( \bigcup_{i=1}^{\infty}{A_i}\right) \leq \sum_{i=1}^{\infty}{\mu_{\rho}^*(A_i)}.
\end{align*}
$(b)$ La propiedad añadida a $\rho$ implica trivialmente que $\rho(A) \leq \lambda$ para todo $\lambda \in H_A$ y, en consecuencia, $\rho(A) \leq \mu_{\rho}^*(A)$ para todo $A \in \mathcal{E}$. Por el apartado $(a)$ tenemos la otra desigualdad, luego $\rho(A) = \mu_{\rho}^*(A)$ para todo $A \in \mathcal{E}$.
\end{proof}

\begin{cor}
La medida exterior de Lebesgue es una medida exterior y $m^*(I) = \mathcal{V}(I)$ para todo intervalo abierto I.
\end{cor}

\subsection{Construcción de medidas: el teorema de Carathéodory}
\begin{teo}
Supongamos que estamos en las condiciones del teorema anterior. Sea $A \subset X$. Las condiciones siguientes son equivalentes:
\begin{enumerate}
    \item[(a)] $\mu_{\rho}^*(A \cap E) + \mu_{\rho}^*(A^c \cap E) = \mu_{\rho}^*(E)$ para todo $E \subset X$. \item[(b)] $\mu_{\rho}^*(A \cap E) + \mu_{\rho}^*(A^c \cap E) = \mu_{\rho}^*(E)$ para todo $E \subset \mathcal{E}$.
\end{enumerate}
\end{teo}

\begin{proof}
Nótese que solo tenemos que probar $(b) \Longrightarrow (a)$. En primer lugar, por ser $\mu_{\rho}^*$ medida exterior, tenemos
\begin{align*}
    \mu_{\rho}^*(E) = \mu_{\rho}^*((A \cap E) \cup (A^c \cap E)) \leq \mu_{\rho}^*(A \cap E) + \mu_{\rho}^*(A^c \cap E)
\end{align*}
para todo $E \subset X$. Probemos la otra desigualdad. Sea $\{E_i\}_{i=1}^{\infty}$ tal que $E \subset \cup_{i=1}^{\infty}{E_i}$ donde $E_i \in \mathcal{E}$. Entonces
\begin{align*}
    A \cap E \subset \cup_{i=1}^{\infty}{(A \cap E_i)} \ \ \ y \ \ \ A^c \cap E \subset \cup_{i=1}^{\infty}{(A^c \cap E_i)}.
\end{align*}
Aplicamos de nuevo que $\mu_{\rho}^*$ es medida exterior y concluimos que
\begin{align*}
    \mu_{\rho}^*(A \cap E) \leq \sum_{i=1}^{\infty}{ \mu_{\rho}^*(A \cap E_i)} \ \ \ y \ \ \ \mu_{\rho}^*(A^c \cap E) \leq \sum_{i=1}^{\infty}{ \mu_{\rho}^*(A^c \cap E_i)}.
\end{align*}
Sumando ambas desigualdades,
\begin{align*}
    \mu_{\rho}^*(A \cap E) + \mu_{\rho}^*(A^c \cap E) \leq \sum_{i=1}^{\infty}{( \mu_{\rho}^*(A \cap E_i) + \mu_{\rho}^*(A^c \cap E_i))}.
\end{align*}
Aplicando $(b)$, el último término es $\sum_{i=1}^{\infty}{\mu_{\rho}^*(E_i)}$. Luego,
\begin{align*}
    \mu_{\rho}^*(A \cap E) + \mu_{\rho}^*(A^c \cap E) \leq \sum_{i=1}^{\infty}{\mu_{\rho}^*(E_i)} \leq \sum_{i=1}^{\infty}{\rho(E_i)},
\end{align*}
es decir,
\begin{align*}
    \mu_{\rho}^*(A \cap E) + \mu_{\rho}^*(A^c \cap E) \leq \lambda \ \ \ \forall \lambda \in H_E,
\end{align*}
de donde,
\begin{align*}
    \mu_{\rho}^*(A \cap E) + \mu_{\rho}^*(A^c \cap E) \leq \inf(H_E) = \mu_{\rho}^*(E)
\end{align*}
como queríamos demostrar.
\end{proof}

\begin{obs}
Si tenemos una medida exterior $\mu^*$ arbitraria entonces la afirmación $(b)$ del teorema anterior no tiene sentido porque no existe la familia $\mathcal{E}$. Sin embargo, $(a)$ tiene sentido completo. Por ello, se introduce la siguiente definición.
\end{obs}

\begin{defi}
Sea $\mu^*$ una medida exterior sobre X. Decimos que $A \subset X$ es un conjunto $\mu^*$-medible (o que A es $mu^*$-medible en el sentido de Carathéodory) si para todo $E \subset X$ se tiene que
\begin{align*}
    \mu^*(A \cap E) + \mu^*(A^c \cap E) = \mu^*(E),
\end{align*}
o, equivalentemente,
\begin{align*}
    \mu_{E}^*(A) + \mu_{E}^*(A^c) \leq \mu_{E}^*(X).
\end{align*}
\end{defi}

\begin{obs}
\begin{enumerate}
    \item[(a)] La desigualdad $\mu^*(A \cap E) + \mu^*(A^c \cap E) \ge \mu^*(E)$ se verifica siempre porque $\mu^*$ es una medida exterior. Luego para probar que $A \subset X$ es un conjunto $\mu^*$-medible basta establecer
    \begin{align*}
        \mu^*(A \cap E) + \mu^*(A^c \cap E) \leq \mu^*(E)
    \end{align*}
    para todo $E$ con $\mu^*(E) < \infty$.
    \item[(b)] Si $\mu^*(A) = 0$ entonces $A$ es un conjunto $\mu^*$-medible.
    \begin{align*}
        &A \cap E \subset A \Longrightarrow \mu^*(A \cap E) \leq \mu^*(A) = 0 \Longrightarrow \mu^*(A \cap E) = 0, \\
        &A^c \cap E \subset E \Longrightarrow \mu^*(A^c \cap E) \leq \mu^*(E).
    \end{align*}
    Sumando estas desigualdades, tenemos que
    \begin{align*}
        \mu^*(A \cap E) + \mu^*(A^c \cap E) = \mu^*(A^c \cap E) \leq \mu^*(E).
    \end{align*}
\end{enumerate}
\end{obs}

\begin{teo}[Teorema de Carathéodory]
Sea $\mu^*$ una medida exterior sobre X y sea
\begin{align*}
    \mathcal{M}^* = \{ A \subset X : A \ es \ \mu^*\text{-medible} \}.
\end{align*}
Entonces
\begin{enumerate}
    \item[(a)] $\mathcal{M}^*$ es una $\sigma$-álgebra (la $\sigma$-álgebra de Carathéodory).
    \item[(b)] $\mu^*|_{\mathcal{M}^*} = \mu$ es una medida.
    \item[(c)] $(X, \mathcal{M}^*, \mu)$ es un espacio de medida completo.
\end{enumerate}
\end{teo}

\begin{proof}
Comenzaremos probando que $\mathcal{M}^*$ es un álgebra.
\\
\newline
Es evidente que $A \in \mathcal{M}^* \Longrightarrow A^c \in \mathcal{M}^*$. Es claro entonces que $X \in \mathcal{M}^*$ puesto que $\emptyset \in \mathcal{M}^*$ ya que $\mu^*(\emptyset) = 0$.
\\
\newline
Supongamos ahora que $A, B \in \mathcal{M}^*$ y demostremos que $A \cup B \in \mathcal{M}^*$. Para ellos, basta usar la subaditividad y la definición de conjunto $\mu^*$-medible (dos veces):
\begin{align*}
    \mu^*((A \cup B) \cap E) + \mu^*((A \cup B)^c \cap E ) &= \mu^*((A \cup B) \cap E) + \mu^*((A^c \cap B^c \cap E )\\
    &= \mu^*((A \cup (B \backslash A)) \cap E) + \mu^*((A^c \cap B^c \cap E )\\
    & \leq \mu^*(A \cap E) + \mu^*(B \cap A^c \cap E) + \mu^*(B^c \cap A^c \cap E)\\
    &= \mu^*(A \cap E) + \mu^*(A^c \cap E) = \mu^*(E).
\end{align*}
Puesto que $\mathcal{M}^*$ es un álgebra, sabemos que si $A, B \in \mathcal{M}^*$ entonces $A^c, A \cup B, A \cap B \in \mathcal{M}^*$.
\\
\newline
Nótese que para todo $E \subset X$, $\mu_E^*(A) = \mu^*(A \cap E)$ es una medida sobre $X$. Veremos que es numerablemente aditiva sobre $\mathcal{M}^*$.
\\
\newline
$(i)$ Comenzamos probando que, para todo $E \subset X$, $\mu_E^*$ es finitamente aditivida sobre $\mathcal{M}^*$, es decir, si $\{A_i\}_{i=1}^{s}$ es una colección finita y disjunta de elementos de $\mathcal{M}^*$, entonces $\mu_E^*(\cup_{i=1}^{s}{A_i}) = \sum_{i=1}^{s}{\mu_E^*(A_i)}$. Por ser $\mathcal{M}^*$ un álgebra, basta verlo para $s = 2$. Sean $A, B \in \mathcal{M}^*$, $A \cap B = \emptyset$.
\\
\newline
Por ser $A \in \mathcal{M}^*$ y $A \cap B = \emptyset$ tenemos que
\begin{align*}
    \mu_E^*(A \cup B) &= \mu^*((A \cup B) \cap E)\\
    &= \mu^*((A \cup B)\cap E \cap A) + \mu^*((A \cup B) \cap E \cap A^c)\\
    &= \mu^*(A \cap E) + \mu^*(B \cap E) = \mu_E^*(A) + \mu_E^*(B).
\end{align*}
$(ii)$ Vamos a probar ahora que, para todo $E \subset X$, $\mu_E^*$ es numerablemente aditiva sobre $\mathcal{M}^*$, es decir,  si $\{A_i\}_{i=1}^{\infty}$ es una colección numerable y disjunta de elementos de $\mathcal{M}^*$, entonces $\mu_E^*(\cup_{i=1}^{\infty}{A_i}) = \sum_{i=1}^{\infty}{\mu_E^*(A_i)}$.
\\
\newline
Por la subaditividad numerable de la medida exterior $ \mu_E^*$,
\begin{align*}
    \mu_E^*(\cup_{i=1}^{\infty}{A_i}) \leq \sum_{i=1}^{\infty}{\mu_E^*(A_i)}.
\end{align*}
Por ser $ \mu_E^*$ finitamente aditiva sobre $\mathcal{M}^*$ y por ser una medida exterior, 
\begin{align*}
    \sum_{i=1}^{N}{\mu_E^*(A_i)} = \mu_E^*(\cup_{i=1}^{N}{A_i}) \leq \mu_E^*(\cup_{i=1}^{\infty}{A_i}).
\end{align*}
Tomando límite $N \to \infty$, llegamos a que
\begin{align*}
    \sum_{i=1}^{\infty}{\mu_E^*(A_i)}  \leq \mu_E^*(\cup_{i=1}^{\infty}{A_i}).
\end{align*}
Por consiguiente, $\mu_E^*(\cup_{i=1}^{\infty}{A_i}) = \sum_{i=1}^{\infty}{\mu_E^*(A_i)}$, como queríamos demostrar.
\\
\newline
$(iii)$ Del punto anterior, tomando $E = X$, tenemos que si $\{A_i\}_{i=1}^{\infty}$ es una colección numerable y disjunta de elementos de $\mathcal{M}^*$, entonces
\begin{align*}
      \mu^*(\cup_{i=1}^{\infty}{A_i}) = \sum_{i=1}^{\infty}{\mu^*(A_i)}.
\end{align*}
Una vez que sabemos que $\mathcal{M}^*$ es un álgebra, estableceremos que es cerrada para uniones numerables y disjuntasm es decir, vamos a probar que
\\
\newline
$(iv)$ si $\{A_i\}_{i=1}^{\infty}$ es una sucesión tal que $A_i \in \mathcal{M}^*$, $A_i \cap A_j = \emptyset$ si $i \not = j$ entonces
\begin{align*}
    B = \cup_{i=1}^{\infty}{A_i} \in \mathcal{M}^*.
\end{align*}
Sea $B_n = \cup_{i=1}^{N}{A_i}$. Por ser $\mathcal{M}^*$ un álgebra, ya sabemos que $B_n \in \mathcal{M}^*$. Además si $E \subset X$
\begin{align*}
    \mu_E^*(B) &= \mu_E^*(\cup_{i=1}^{\infty}{A_i}) = \sum_{i=1}^{\infty}{\mu_E^*(A_i)}\\
    &= \lim_{N \to \infty}{ \sum_{i=1}^{N}{\mu_E^*(A_i)}} = \lim_{N \to \infty}{\mu_E^*(\cup_{i=1}^{N}{A_i})} = \lim_{N \to \infty}{\mu_E^*(B_n)}.
\end{align*}
Así,
\begin{align*}
    \mu_E^*(B) + \mu_E^*(B^c) = \lim_{N \to \infty}{(\mu_E^*(B_n) + \mu_E^*(B^c))}.
\end{align*}
Por otra parte, como $B_n \subset B$ tenemos que $B^c \subset B_n^c$. Luego
\begin{align*}
    \mu_E^*(B^c) \leq \mu_E^*(B_n^c)
\end{align*}
para todo n. Puesto que $\mu_E^*$ es una medida exterior y $B_n \in \mathcal{M}^*$,
\begin{align*}
    \mu_E^*(B_n) + \mu_E^*(B^c) \leq \mu_E^*(B_n) + \mu_E^*(B_n^c) = \mu^*(E).
\end{align*}
Por lo tanto,
\begin{align*}
    \mu_E^*(B) + \mu_E^*(B^c)  = \lim_{N \to \infty}{(\mu_E^*(B_n) + \mu_E^*(B^c))} \leq \mu^*(E).
\end{align*}
lo que prueba que $B \in \mathcal{M}^*$.
\\
\newline
$(v)$ Veamos que $\mathcal{M}^*$ es una $\sigma$-álgebra. Tomemos $\{E_i\}_{i=1}^{\infty}$, $E_i \in \mathcal{M}^*$. Si $A_1 = E_1$ y $A_i = E_i \backslash \cup_{n=1}^{i-1}{E_n}$, $i > 1$, se tiene que $A_i \in \mathcal{M}^*$ (por ser $\mathcal{M}^*$ un álgebra), $cup_{i=1}^{\infty}{E_i} = \cup_{i=1}^{\infty}{A_i}$ y los $A_i$ son disjuntos. Por lo ya probado, se sigue que $\cup_{i=1}^{\infty}{E_i} \in \mathcal{M}^*$.
\\
\newline
$(vi)$ Pasamos a demostrar que $\mu^*|_{\mathcal{M}^*} = \mu$ es una medida. Es obvio que $\mu(\emptyset) = 0$. Por $(iii)$, $\mu^*$ es numerablemente aditivda sobre $\mathcal{M}^*$.
\\
\newline
$(vii)$ Nos queda ver que $\mu$ es completa pero esto es obvio: si $\mu(N) = 0$ y $F \subset N$ entonces $\mu^*(F) \leq \mu^*(N) = \mu(N) = 0$; luego $\mu^*(F) = 0$ y, en consecuencia, $F \in \mathcal{M}^*$.
\end{proof}

\section{La medida de Lebesgue}

Definimos la medida exterior de Lebesgue $m^*: \mathcal{P}(\mathbb{R}^n) \longrightarrow [0,+\infty]$ como
\begin{align*}
    m^*(A) = \inf \left\{ \sum_{i=1}^{\infty}{\mathcal{V}(I_i)} : A \subset \cup_{i=1}^{\infty}{I_i}, I_i \text{ es un intervalo de } \mathbb{R}^n\right\}
\end{align*}
La familia $\mathcal{E}$ de los intervalos abiertos es una familia recubridora de $\mathbb{R}^n$ y la aplicación 
\begin{align*}
    \rho : \mathcal{E} \longrightarrow [0,+\infty], \ \ \ \rho(I) = \mathcal{V}(I),
\end{align*}
cumple que $\rho(\emptyset) = 0$. Por el teorema de construcción de medidas exteriores, $m^*$ es una medida exterior. Además, si $I \subset \cup_{j=1}^{\infty}{I_j}$ donde $I$ y los $I_j$ son intervalos abiertos, se tiene que
\begin{align*}
    \rho(I) = \mathcal{V}(I) \leq \sum_{j=1}^{\infty}{\mathcal{V}(I_j)} = \sum_{j=1}^{\infty}{\rho(I_j)}.
\end{align*}
Luego, por el apartado $(b)$ del mismo teorema, $m^*(I) = \mathcal{V}(I)$ para todo intervalo abierto.
\\
\newline
Obsérvese que $m^*(A) = 0$ si y solo si para todo $\varepsilon > 0$ existe una familia numerable de intervalos abiertos $\{I_i\}_{i=1}^{\infty}$ tal que
\begin{align*}
    A \subset \cup_{i=1}^{\infty}{I_i} \ \ \ y \ \ \ \sum_{j=1}^{\infty}{\mathcal{V}(I_i)} < \varepsilon.
\end{align*}
Recordemos que, por el teorema de Carathéodory, 
\begin{align*}
    \mathcal{L} &= \{A \subset \mathbb{R}^n : A \ es \ m^* \text{-medible} \}\\
    &= \{A \subset \mathbb{R}^n : m^*(A \cap E) + m^*(A^c \cap E) = m^*(E) \text{ para todo } E \subset \mathbb{R}^n \}
\end{align*}
es una $\sigma$-álgebra.
\\
\newline
Como $m^*$ viene definida a través de $\rho = \mathcal{V}$ y la familia elemental de los intervalos abiertos, resulta que
\begin{align*}
    \mathcal{L} = \{A \subset \mathbb{R}^n : m^*(A \cap E) + m^*(A^c \cap E) = m^*(E) \text{ para todo intervalo abierto } I\}
\end{align*}
A $\mathcal{L}$ la llamaremos $\sigma$-álgebra de Lebesgue. A los conjuntos $A \in \mathcal{L}$ los llamaremos conjuntos medibles-Lebesgue. Además, por el teorema de Carathéodory, si $m = m^*|_{\mathcal{L}}$ entonces $(\mathbb{R}^n, \mathcal{L}, m)$ es un espacio de medida completo. A $m$ le llamamos la medida de Lebesgue.

\begin{prop}
Algunas propiedades de $m^*$, $m$ y $\mathcal{L}$.
\begin{enumerate}
    \item[(a)] Si $m^*(A) = 0$ entonces $A \in \mathcal{L}$.
    \item[(b)] La medida exterior de un intervalo I es  su volumen.
    \item[(c)] $m^*(\{x\}) = 0$ para todo $x \in \mathbb{R}^n$.
    \item[(d)] La medida exterior de Lebesgue de un conjunto numerable A es cero, por lo tanto, $A \in \mathcal{L}$.
    \item[(e)] La medida exterior de $\mathbb{R}^n$ es $+\infty$.
    \item[(f)] La medida exterior es invariante frente a traslaciones: si $A \subset \mathbb{R}^n$ y $b \in \mathbb{R}^n$ entonces $m^*(A + b) = m^*(A)$, donde
    \begin{align*}
        A + b = \{ x : x = a + b, a \in A \}.
    \end{align*}
    \item[(g)] Para cada $a \in \mathbb{R}$ y para cada $j \in \{1,..., n \}$ los conjuntos
    \begin{itemize}
        \item $\{x \in \mathbb{R}^n : x = (x_1,...,x_n), x_j > a\}$.
        \item $\{x \in \mathbb{R}^n : x = (x_1,...,x_n), x_j \leq a\}$.
        \item $\{x \in \mathbb{R}^n : x = (x_1,...,x_n), x_j \ge a\}$.
        \item $\{x \in \mathbb{R}^n : x = (x_1,...,x_n), x_j < a\}$.
    \end{itemize}
    son medibles Lebesgue.
    \item[(h)] Los intervalos I son medibles Lebesgue y $m(I) = \mathcal{V}(I)$.
    \item[(i)] $\mathcal{B}_{\mathbb{R}^n} \subset \mathcal{L}$ (los borelianos son medibles Lebesgue).
    \item[(j)] $(\mathbb{R}^n, \mathcal{L}, m)$ es un espacio de medida $\sigma$-finito o, en otras palabras, la medida de Lebesgue es $\sigma$-finita.
\end{enumerate}
\end{prop}

\begin{proof}
Probemos $(g)$. Lo veremos solo para $A = \{x \in \mathbb{R}^n : x = (x_1,...,x_n), x_1 > a\}$. Sea $I = (a_1,b_1)\times ... \times (a_n,b_n)$ intervalo abierto. Tenemos tres posibilidad:
\begin{enumerate}
    \item[(i)] $a_1 \ge a$. Entonces $I \subset A$. Además, $m^*(A \cap I) + m^*(A^c \cap I) = m^*(A \cap I) = m^*(I)$.
    \item[(ii)] $b_j \leq a$. Entonces $I \subset A^c$. Así, $m^*(A \cap I) + m^*(A^c \cap I) = m^*(A \cap I) = m^*(I)$.
    \item[(iii)] $a_j < a < b_j$. Entonces
    \begin{align*}
        &A \cap I = (a_1,b_1) \times ... \times (a_n,b_n) \text{ y } \\
        &A^c \cap I =  (a_1,a] \times ... \times (a_n,b_n).
    \end{align*}
    Como estos conjuntos son intervalos,
    \begin{align*}
        &m^*(A \cap I) = (b_1 - a_1) \dotsb (b_n - a_n) \text{ y } \\
        &m^*(A^c \cap I) = (a - a_1) \dotsb (b_n - a_n).
    \end{align*}
    de donde
    \begin{align*}
        m^*(A \cap I) + m^*(A^c \cap I) = \frac{(b_1 - a_1)\dotsb (b_n - a_n)}{(b_1 - a_1)}(b_1 - a + a - a_1) = \mathcal{V}(I) = m^*(I).
    \end{align*}
\end{enumerate}
\end{proof}

\newpage
\section{Conjuntos medibles Lebesgue y la medida de Lebesgue: caracterizaciones}

\begin{teo}
Sea $A \subset \mathbb{R}^n$. Las afirmaciones siguientes son equivalentes:
\begin{enumerate}
    \item[(a)] A es medible-Lebesgue.
    \item[(b)] Para cada $\varepsilon > 0$ existe un abierto G tal que $A \subset G$ y $m^*(G \backslash A) < \varepsilon$.
    \item[(c)] Existen una familia numerable de conjuntos abiertos $\{G_i\}_{i=1}^{\infty}$ y un conjunto N de medida cero tales que $A = (\cap_{i=1}^{\infty}{G_i}) \backslash N$.
\end{enumerate}
\end{teo}

\begin{proof}
$(a) \Longrightarrow (b)$ Supongamos que $m(A) < +\infty$. Sea $\varepsilon > 0$. Por la definición de $m = m^*|_{\mathcal{L}}$, existe $\{I_j\}_{j=1}^{\infty}$, un recubrimiento de $A$ por intervalos abiertos, con $\sum_{j=1}^{\infty}{\mathcal{V}(I_J)} < m(A) + \varepsilon$. Tomemos $G = \cup_{j=1}^{\infty}{I_j}$. Así, $G$ es abierto y $A \subset G$. Además,
\begin{align*}
    m^*(G \backslash A) &= m(G \backslash A) = m(G) - m(A) = m(\cup_{j=1}^{\infty}{I_j}) -  m(A) \\
    & \leq \sum_{j=1}^{\infty}{m(I_j)} - m(A) = \sum_{j=1}^{\infty}{\mathcal{V}(I_j)} - m(A) < \varepsilon.
\end{align*}
Supongamos que $m(A) = +\infty$. Sea, para cada $k \in \mathbb{N}$, $I_k = [-k,k]^n$. Así,
\begin{align*}
    A = A \cap \mathbb{R}^n = \bigcup_{k=1}^{\infty}{A \cap I_k} = \bigcup_{k=1}^{\infty}{A_k},
\end{align*}
donde $A_k = A \cap I_k$. Para todo $k \in \mathbb{N}$, $A_k$ es medible y con medida finita ($m(A) \leq m(I_k) < +\infty$). Por lo que hemos demostrado anteriormente, para cada $k \in \mathbb{N}$, existe un abierto $G_k$ con $A_k \subset G_k$ y tal que $m^*(G_k \backslash A_k) < \frac{\varepsilon}{2^k}$. Sea $G = \cup_{k=1}^{\infty}{G_k}$. Es claro que $G$ es abierto y $A \subset G$. Además,
\begin{align*}
    G \backslash A = \left( \bigcup_{k=1}^{\infty}{G_k} \backslash \bigcup_{k=1}^{\infty}{A_k} \right) \subset \bigcup_{k=1}^{\infty}{(G_k \backslash A_k)}.
\end{align*}
Aplicando las propiedades de la medida exterior,
\begin{align*}
    m^*(G \backslash A) \leq m^*\left(  \bigcup_{k=1}^{\infty}{(G_k \backslash A_k)} \right) \leq \sum_{k=1}^{\infty}{m^*(G_k \backslash A_k)} \leq \sum_{k=1}^{\infty}{\frac{\varepsilon}{2^k}} = \varepsilon.
\end{align*}
$(b) \Longrightarrow (c)$ Por hipótesis, para todo $k \in \mathbb{N}$, existe un abierto $G_k$ tal que $A \subset G_k$ y con $m^*(G_k \backslash A) < \frac{1}{k}$. Sea $G = \cap_{k=1}^{\infty}{G_k}$. Entonces $A \subset G$. Por lo tanto, $G \backslash A \subset G_k \backslash A$ para todo $k \in \mathbb{N}$, y, consecuentemente,
\begin{align*}
    m^*(G \backslash A) < \frac{1}{k} \ \ \text{para todo } k \in \mathbb{N}.
\end{align*}
Tomando límite, obtenemos que $m^*(G \backslash A) = 0$, lo que implica que $N = G \backslash A$ es medible y $m(N) = 0$. Por otra parte, $G \backslash N = G \backslash (G \backslash A) = A$.
\\
\\
\nenwline
$(c) \Longrightarrow (a)$ Es trivial. Por hipótesis, se tiene que $\cap_{i=1}^{\infty}{G_k} \in \mathcal{L}$ y $N \in \mathcal{L}$, por tanto, $A = (\cap_{i=1}^{\infty}{G_k}) \backslash N \in \mathcal{L}$.
\end{proof}

\begin{teo}
Sea $A \subset \mathbb{R}^n$. Las afirmaciones siguientes son equivalentes:
\begin{enumerate}
    \item[(a)] A es medible-Lebesgue.
    \item[(b)] Para cada $\varepsilon > 0$ existe un cerrado F tal que $F \subset A$ y $m^*(A \backslash F) < \varepsilon$.
    \item[(c)] Existen una familia numerable de cerrados $\{F_i\}_{i=1}^{\infty}$ y un conjunto N de medida cero tales que $A = (\cup_{i=1}^{\infty}{F_i}) \cup N$.
\end{enumerate}
\end{teo}

\begin{proof}
$(a) \Longrightarrow (b)$ Sea $\varepsilon > 0$. Como $A$ es medible, tenemos que $A^c$ es medible. Por el teorema anterior, existe un abierto $G$ tal que $A^c \subset G$ y $m^*(G \backslash A^c) < \varepsilon$. Sea $F = G^c$. Es claro que $F$ es cerrado y $F = G^c \subset A$. Además, $A \backslash F = A \cap F^c = A \cap G = G \backslash A^c$. Por consiguiente,
\begin{align*}
    m^*(A \backslash F) = m^*(G \backslash A^c) < \varepsilon.
\end{align*}
$(b) \Longrightarrow (a)$ Sea $\varepsilon > 0$. Entonces, por hipótesis, existe un cerrado $F \subset A$ tal que $m^*(A \backslash F) < \varepsilon$, de donde, $A^c \subset  F^c$ y $m^*(F^c \backslash A^c) < \varepsilon$ (ya que $F^c \backslash A^c = F^c \cap (A^c)^c = F^c \cap A = A \backslash F$). Tomando $G = F^c$, se tiene que $A^c \subset G$, $G$ es abierto y $m^*(G \backslash A^c) < \varepsilon$. Entonces, por el teorema anterior, $A^c$ es medible-Lebesgue y, consecuentemente, $A$ es medible-Lebesgue.
\\
\newline
$(a) \Longrightarrow (c)$ Supongamos que $A$ es medible-Lebesgue. Entonces $A^c$ es medible. Por el teorema anterior, existe una familia numerable de abiertos $\{G_j\}_{j=1}^{\infty}$ y un conjunto $N$ tales que $m(N) = 0$ y $A^c = (\cap_{j=1}^{\infty}{G_j}) \backslash N$. Entonces
\begin{align*}
    A^c = \left( \bigcap_{j=1}^{\infty}{G_j}\right) \backslash N = \bigcap_{j=1}^{\infty}{G_j \cap N^c}
\end{align*}
o, equivalentemente,
\begin{align*}
    A = \left( \bigcup_{j=1}^{\infty}{G_j^c}\right) \cup N.
\end{align*}
Lo que prueba la implicación ya que los conjuntos $G_j^c$ son cerrados y $m(N) = 0$.
\\
\newline
$(c) \Longrightarrow (a)$ Es trivial. Por hipótesis, se tiene que $\cup_{i=1}^{\infty}{F_k} \in \mathcal{L}$ y $N \in \mathcal{L}$, por tanto, $A = (\cup_{i=1}^{\infty}{F_k} )\cup N \in \mathcal{L}$. 
\end{proof}

\begin{cor}
Un conjunto $A \subset \mathbb{R}^n$ es medible-Lebesgue si y solo si existen un conjunto de Borel B y un conjunto F de medida cero tal que $A = B \cup F$.
\end{cor}

\begin{proof}
$\Longrightarrow$ Sea $N \in \mathcal{B}_{\mathbb{R}^n}$ con $m(N) = 0$. Entonces
\begin{align*}
    m^*(F) \leq m^*(N) = m(N) = 0,
\end{align*}
por tanto, $m^*(F) = 0$, luego, $F$ es medible-Lebesgue.
\\
\newline
$\Longleftarrow$ Sea $F \in \mathcal{L}$ con $m(F) = 0$. Como $F$ es medible-Lebesgue, existen una familia numerable de conjuntos abiertos $\{G_i\}_{i=1}^{\infty}$ y un conjunto N de medida cero tales que $F = (\cap_{i=1}^{\infty}{G_i}) \backslash N$, por tannton $N \subset \cap_{i=1}^{\infty}{G_i}$, luego, $F \subset \cap_{i=1}^{\infty}{G_i} \in \mathcal{B}_{\mathbb{R}^n}$. Sea $B = F \cup N$, entonces $B$ es medida cero, pues es unión de dos conjuntos de medida cero.
\end{proof}

\begin{cor}
El espacio de medida de Lebesgue $(\mathbb{R}^n, \mathcal{L}, m)$ es la completación del espacio de medida $(\mathbb{R}^n, \mathcal{B}_{\mathbb{R}^n}, m|_{\mathcal{B}_{\mathbb{R}^n}})$.
\end{cor}

\begin{proof}
Consideremos el espacio de medida $(\mathbb{R}^n, \mathcal{L}, m)$, donde $\mathcal{L}$ es la $\sigma$-álgebra de Lebesgue y $m$ es la medida de Lebesgue.
\\
\newline
Como $\mathcal{B}_{\mathbb{R}^n} \subset \mathcal{L}$, se tiene que $(\mathbb{R}^n, \mathcal{B}_{\mathbb{R}^n}, \mu)$ es un espacio de medida, donde $\mu =  m|_{\mathcal{B}_{\mathbb{R}^n}}$. Además,
\begin{align*}
    \mathcal{L} &= \{ A \subset \mathbb{R}^n : A = E \cup F, E \in \mathcal{B}_{\mathbb{R}^n}, F \in \mathcal{L}, m(F) = 0\}\\
    & = \{ A \subset \mathbb{R}^n : A = E \cup F, E \in \mathcal{B}_{\mathbb{R}^n}, F \subset N \in \mathcal{B}_{\mathbb{R}^n}, m(N) = 0\} \\
    &= \overline{\mathcal{B}_{\mathbb{R}^n}}.
\end{align*}
Por otra parte, si $A \in \mathcal{L}$, $A = E \cup F$, $E \in \mathcal{B}_{\mathbb{R}^n}$, $F \subset N \in \mathcal{B}_{\mathbb{R}^n}$, $m(N) = 0$, entonces
\begin{align*}
    \overline{\mu}(A) = \mu(E) = m(E). 
\end{align*}
y
\begin{align*}
    m(A) = m(E \cup F) \leq m(E) + m(F) = m(E) \leq m(A).
\end{align*}
Luego, $\overline{\mu}(A) = m(A)$.
\end{proof}

\begin{teo}
Si $A \subset \mathbb{R}^n$ es un conjunto medible entonces
\begin{align*}
    m(A) &= \inf{\{ m(G) : A \subset G, G \ abierto \ de \ \mathbb{R}^n \}} \\
    &= \sup{\{ m(K) : K \subset A, K \ compacto \ de \mathbb{R}^n \}}
\end{align*}
\end{teo}

\begin{teo}[La medida de Lebesgue es invariante frente a traslaciones]
Sean $A \subset \mathbb{R}^n$ y $b \in \mathbb{R}^n$. A es un conjunto medible si y solo si A + b es medible y, en este caso, $m(A + b) = m(A)$.
\end{teo}

\begin{proof}
Definimos $T_b: \mathbb{R}^n \longrightarrow \mathbb{R}^n$ dada por $T_b(x) = x + b$, que es biyectiva. Ya sabemos que $m^*(T_bE) = m^*(E + b) = m^*(E)$ para todo conjunto $E$. Basta demostrar que si $A$ es un conjunto medible-Lebesgue entonces $A + b$ es medible y, en este caso, $m(A + b) = m(A)$. Supongamos que $A$ es medible. Sea $\varepsilon > 0$. Por ser $A$ medible, existe un abierto $G$ tal que $A \subset G$ y $m(G \backslash A) < \varepsilon$. Como $T_b$ es un homeomorfismo, $T_bG$ es un abierto y, además, $T_bA \subset T_bG$. Como $T_bG \backslash T_bA = T_b(G \backslash A)$ y $m^*$ es invariante frente a traslaciones, resulta que
\begin{align*}
    m^*(T_bG \backslash T_bA) = m^*(T_b(G \backslash A)) = m^*(G \backslash A) < \varepsilon.
\end{align*}
Por lo tanto, $T_bA$ es medible y $m(T_bA) = m^*(T_bA) = m(A)$.
\end{proof}

\section{Cubos diádicos y conjuntos abiertos}

\begin{defi}
Un intervalo diádico de longitud $2^{-k}$ en $\mathbb{R}$ es un intervalo de la forma $I = (m2^{-k}, (m+1)2^{-k}]$, con $m,k \in \mathbb{Z}$.
\end{defi}

\begin{obs}
Algunas propiedades inmediatas son
\begin{enumerate}
    \item[1.] Fijado $k \in \mathbb{Z}$, los intevalos $2^{-k}$ constituyen una partición numerable de $\mathbb{R}$.
    \item[2.] Dados dos intervalos diádicos de longitud $2^{-k}$, $k \in \mathbb{Z}$, se tiene que, o son iguales o son disjuntos.
    \item[3.] Dados dos intervalos diádicos cualesquiera, o bien son disjuntos, o bien uno está contenido en el otro.
\end{enumerate}
\end{obs}

\begin{defi}
Un cubo diádico de $\mathbb{R}^n$ de lado $2^{-k}$, con $k \in \mathbb{Z}$, es un producto cartesiano de n intervalos diádicos de $\mathbb{R}$ de longitud $2^{-k}$, es decir, es de la forma
\begin{align*}
    I = (m_12^{-k}, (m_1 + 1)2^{-k}] \times ... \times (m_n2^{-k}, (m_n + 1)2^{-k}], \ \ \ \ m_j \in \mathbb{Z}.
\end{align*}
Denotaremos por $\mathcal{D}_k$ a la familia de los cubos diádicos en $\mathbb{R}^n$ de lado $2^{-k}$, mientras que $\mathcal{D}$ será la unión $\cup_{k \in \mathbb{Z}}{\mathcal{D}_k}$, esto es, $\mathcal{D}$ es la familia de todos los cubos diádicos de $\mathbb{R}^n$.
\end{defi}

\begin{teo}
Sea $G = \emptyset$ un abierto de $\mathbb{R}^n$. Entonces existe una familia numerable $\{I_j\}$ de cubos diádicos disjuntos tal que $G = \cup_{j}{I_j}$.
\end{teo}

\begin{teo}
Sea $E \subset \mathbb{R}^n$ medible y de medida finita. Para todo $\varepsilon > 0$ existe una colección finita $\{Q_j\}_{j=1}^{N}$ de cubos cerrados disjuntos tal que si $F = \cup_{j=1}^{N}{Q_j}$ se tiene que $m(E \bigtriangleup F) < \varepsilon$. ($E \bigtriangleup F = (E \backslash F) \cup (F \backslash E))$.
\end{teo}

\section{Funciones simples en $\mathbb{R}^n$}

\begin{defi}
Una función paso $\varphi$ es una combinación lineal finita de funciones características de rectángulos (intervalos acotados), es decir,
\begin{align*}
    \varphi = \sum_{i=1}^{s}{a_i\mathcal{X}_{R_i}},
\end{align*}
donde $a_i \in \mathbb{R}$ y $R_i$ es un rectángulo.
\end{defi}

\begin{teo}
Si $f: \mathbb{R}^n \longrightarrow \mathbb{R}$ es una función medible, entonces existe una sucesión $\{\varphi_k\}$ de funciones paso que converge a f en casi todo punto de $\mathbb{R}^n$.
\end{teo}

\section{Relación entre las medidas de Lebesgue}

\begin{teo}
Consideramos $(\mathbb{R}^p, \mathcal{L}_p, m_p)$ y $(\mathbb{R}^q, \mathcal{L}_q, m_q)$ los espacios de medida de Lebesgue de $\mathbb{R}^p$ y $\mathbb{R}^q$, respectivamente. Si A es medible en $\mathbb{R}^p$ ($A \in \mathcal{L}_p$) y B es medible en $\mathbb{R}^q$ ($B \in \mathcal{L}_q$) entonces $A \times B$ es medible en $\mathbb{R}^{p + q}$ ($A \times B \in \mathcal{L}_{p + q}$) y $m_{p + q}(A \times B) = m_p(A)m_q(B)$.
\end{teo}

\section{La medida de Lebesgue-Stieltjes en $\mathbb{R}$}

Las funciones crecientes y continuas por la derecha se denominan funciones de distribución. A cada función de distribución $F$ le vamos a asignar una medida de Borel en $\mathbb{R}$, es decir, una medida definida en la $\sigma$-álgebra de Borel tal que la medida del intevalo $(a,b]$ es $F(b) - F(a)$. Es de destacar que solo existe una medida con estas propiedades.

\begin{teo}
Sea $F: \mathbb{R} \longrightarrow \mathbb{R}$ una función creciente continua por la derecha (una función de distribución). Entonces existe una única medida $m_F$ definida sobre $\mathcal{B}_{\mathbb{R}}$ tal que $m_F((a,b]) = F(b) - F(a)$ para todo intervalo $(a,b]$, $a,b \in \mathbb{R}$, $a \leq b$. Dicha medida recibe el nombre de medida de Lebesgue-Stieltjes asociada a $F$. Si $F$ es la función identidad, $F(x) = x$, $m_F$ es la medida de Lebesgue (es la única medida $m$ definida sobre $\mathcal{B}_{\mathbb{R}}$ tal que $m((a,b]) = b - a$).
\end{teo}

\begin{itemize}
    \item \textbf{Existencia}. Sea $\mathcal{E} = \{ (a,b] : a,b \in \mathbb{R}, a \leq b \}$. Definamos $\rho : \mathcal{E} \longrightarrow [0,+\infty)$ como $\rho((a,b]) = F(b) - F(a)$. $\mathcal{E}$ es una familia recubridora de $\mathbb{R}$ pues
    \begin{itemize}
        \item $\emptyset = (a,a] \in \mathcal{E}$ y
        \item $\mathbb{R} = \cup_{n=1}^{\infty}(-n,n]$.
    \end{itemize}
    Además, $\rho(\emptyset) = \rho((a,a]) = F(a) - F(a) = 0$, independientemente del $a$ elegido. Luego, por el teorema de construcción de medidas exteriores, $m_F^*: \mathcal{P}(\mathbb{R}) \longrightarrow [0,+\infty]$ definida por
    \begin{align*}
        m_F^*(A) = \inf \left\{ \sum_{i=1}^{\infty}{\rho(E_i)} : A \subset \cup_{i=1}^{\infty}{(a_i,b_i]}, (a_i,b_i] \in \mathcal{E} \right\}
    \end{align*}
    es una medida exterior. Sea
    \begin{align*}
        \mathcal{M}_F^* &= \{ A \subset \mathbb{R} : m_F^*(A \cap E) + m_F^*(A^c \cap E) = m_F^*(E) \text{ para todo } E \subset \mathbb{R} \} \\
        &= \{ A \subset \mathbb{R} : m_F^*(A \cap (a,b]) + m_F^*(A^c \cap (a,b]) = m_F^*((a,b]) \text{ para todo } (a,b] \in \mathcal{E} \}
    \end{align*}
    la $\sigma$-álgebra de Carathéodory asociada a $m_F^*$. Se tienen las propiedades siguientes
    \begin{itemize}
        \item $m_F^*((a,b]) = F(b) - F(a)$, $a \leq b$, $a,b \in \mathbb{R}$. Para probar esta propiedad, hacemos uso del siguiente lema
        \begin{lema}
            Si $(a,b] \subset \cup_{i=1}^{\infty}{(a_i,b_i]}$ entonces
            \begin{align*}
                \rho((a,b]) = F(b) - F(a) \leq \sum_{i=1}^{\infty}(F(b_i) - F(a_i)).
            \end{align*}
        \end{lema}
        \item Es importante el siguiente lema.
        \begin{lema}
        Todo intervalo semiabierto $(-\infty,c]$ es $m_F^*$-medible.
        \end{lema}
        \begin{proof}
        Sea $(a,b] \in \mathcal{E}$, entonces
        \begin{enumerate}
            \item[(i)] Si $c \leq a$
            \begin{align*}
                &(-\infty,c] \cap (a,b] = \emptyset \\
                &(-\infty,c]^c \cap (a,b] = (c,+\infty) \cap (a, b] = (a,b]
            \end{align*}
            por tanto, $m_F^*((-\infty,c] \cap (a,b]) + m_F^*( (-\infty,c]^c \cap (a,b]) = m_F^*((a,b])$.
            \item[(ii)] Si $c \ge b$
            \begin{align*}
                &(-\infty,c] \cap (a,b] = (a,b]\\ 
                &(-\infty,c.]^c \cao (a,b] = (c,+\infty) \cap (a,b] = \emptyset
            \end{align*}
            por tanto, $m_F^*((-\infty,c] \cap (a,b]) + m_F^*( (-\infty,c]^c \cap (a,b]) = m_F^*((a,b])$.
            \item[(iii)] Si $a < c < b$
            \begin{align*}
                &(-\infty,c] \cap (a,b] = (a,c]\\
                &(-\infty,c]^c \cap (a,b] = (c, +\infty) \cap (a,b] = (c,b]
            \end{align*}
            por tanto, $m_F^*((-\infty,c] \cap (a,b]) + m_F^*( (-\infty,c]^c \cap (a,b]) = m_F^*((a,b])$.
        \end{enumerate}
        \end{proof}
        \item El lema anterior demuestra que $\mathcal{B}_{\mathbb{R}} \subset \mathcal{M}_F^*$ ya que $\mathcal{M}_F^*$ es una $\sigma$-álgebra y los intervalos $(-\infty,c]$ generan la $\sigma$-álgebra de Borel de $\mathbb{R}$.
        \item $m_F = m_F^* |_{\mathcal{M}_F^*}$ es completa (por el Teorema de Carathéodory).
        \item Es claro que $m_F = m_F^* | _{\mathcal{B}_{\mathbb{R}}}$ cumple las condiciones requeridas.
    \end{itemize}
    \item \textbf{Unicidad}: Solo nos interesa saber que es única, la demostración se deja como ejercicio.
    \\
    \newline
    El resultado siguiente es un recíproco de lo que hemos demostrado.
    \begin{teo}
    Si $\mu$ es una medida de Borel sobre $\mathbb{R}$ que es finita para todo intervalo acotado, entonces existe $F: \mathbb{R} \longrightarrow \mathbb{R}$ creciente y continua por la derecha tal que $m_F = \mu$.
    \end{teo}
    \begin{proof}
    Definimos
    \begin{align*}
        F(x) = \left\{ \begin{array}{lcc}
             -\mu((x,0]) &  si  & x \leq 0\\
             \mu((0,x]) &  si  & x > 0\\
             \end{array}
        \right. 
    \end{align*}
    $F$ cumple las condiciones.
    \end{proof}
\end{itemize}

\begin{prop}
Sean F y G funciones de $\mathbb{R}$ en $\mathbb{R}$ crecientes y continuas por la derecha. Entonces $m_F = m_G$ si y solo si existe una constante C tal que $F - G = C$.
\end{prop}

\begin{teo}
Si $F: \mathbb{R} \longrightarrow \mathbb{R}$ es creciente y continua por la derecha, entonces para todo 
$E \in \mathcal{B}_{\mathbb{R}}$
\begin{align*}
    m_F(E) &= \inf \{ m_F(G) : E \subset G, G \ abierto \}\\
    &= \sup \{ m_F(K) : K \subset E, K \ compacto \}.
\end{align*}
\end{teo}

\begin{ejemplo}
\begin{itemize}
    \item Para cada $x \in \mathbb{R}$, calcular $m_F(\{x\})$.
    \begin{align*}
        \{x\} = \bigcap_{n=1}^{\infty}\left( x - \frac{1}{n}, x \right]
    \end{align*}
    Nótese que $( x - \frac{1}{n}, x]$ es una sucesión contractiva, por tanto
    \begin{align*}
        m_F(\{x\}) = \lim_{n \to \infty}{m_F}\left( x - \frac{1}{n}, x \right] = \lim_{n \to \infty}{F(x) - F\left(x -\frac{1}{n}\right)} = F(x) - F(x^-).
    \end{align*}
    \item Determinar $m_F([a,b])$, con $a,b \in \mathbb{R}$.
    \begin{align*}
        m_F([a,b]) &= m_F(\{a\} \cup (a,b]) = m_F(\{a\}) + m_F((a,b])\\
        &= F(a) - F(a^-) + F(b) - F(a) \\
        &= F(b) - F(a^-).
    \end{align*}
\end{itemize}
\end{ejemplo}

\begin{ejemplo}
Sea $F$ la función de distribución dada por
\begin{align*}
    F(x) = \left\{ \begin{array}{lcc}
             0 &  si  & x < 0\\
             1 &  si  & x \ge 0\\
             \end{array}
        \right. 
\end{align*}
Determinar $m_F$.
\begin{itemize}
    \item Calculemos $m_F((-\infty,0))$.
    \begin{align*}
        m_F((-\infty,0)) &= m_F\left( \bigcup_{n=1}^{\infty}\left( -n, -\frac{1}{n}\right)\right) = \lim_{n \to \infty}{m_F}\left( -n, -\frac{1}{n}\right) \\
        &= \lim_{n \to \infty}{F\left(-\frac{1}{n}\right) - F(-n)} = 0 - 0 = 0.
    \end{align*}
    \item Calculemos $m_F((0,+\infty))$.
    \begin{align*}
        m_F((0,+\infty)) &= m_F\left( \bigcup_{n=1}^{\infty}\left(\frac{1}{n}, n\right)\right) = \lim_{n \to \infty}{m_F}\left(\frac{1}{n}, n\right) \\
        &= \lim_{n \to \infty}{F(n) - F\left(\frac{1}{n}\right)} = 1 - 1 = 0 .
    \end{align*}
    \item Calculemos $m_F(\{0\})$.
    \begin{align*}
        m_F(\{0\}) = m_F(0) - m_F(0^-) = 1 - 0 = 1.
    \end{align*}
\end{itemize}
Por tanto $m_F = \delta_0$, es decir, $m_F$ es la delta de Dirac en $a = 0$.
\end{ejemplo}

\subsection{La integral asociada a medidas de Lebesgue-Stieltjes}

Sea $F: \mathbb{R} \longrightarrow \mathbb{R}$ una función de distribución que es derivable con derivada continua. Sea $m_F$ la medida de Lebesgue-Stieltjes asociada a $F$ (recordemos que $m_F$ es la única medida definida sobre la $\sigma$-álgebra de Borel tal que la medida del intervalo $(a,b]$ es $F(b) - F(a)$). Sea $f = F'$, la derivada de $F$ y sea $\nu$ la medida con densidad $f$ definida sobre todos los medibles-Lebesgue, es decir,
\begin{align*}
    \nu(B) = \int_{B}{f(x) \ dx}.
\end{align*}
Si calculamos la medida de los intervalos $(a,b]$ queda
\begin{align*}
    \nu((a,b]) = \int_{(a,b]}{f(x) \ dx} = \int_{[a,b]}{f(x) \ dx} = \int_{a}^{b}{F' \ dx} = F(b) - F(a), 
\end{align*}
donde hemos empleado que los puntos tienen medida de Lebesgue cero, que las funciones integrables-Riemann son integrables-Lebesgue y, por último, hemos aplicado la regla de Barrow. Como vemos,
\begin{align*}
    \nu((a,b]) = m_F((a,b])
\end{align*}
para todo intervalo $(a,b]$. Entonces $\nu$ y $m_F$ coinciden en la $\sigma$-álgebra de Borel. Por lo tanto, aplicando los resultados de la sección anterior, si $g: \mathbb{R} \longrightarrow [0,+\infty]$ es medible-Borel, la integral de Lebesgue-Stieltjes de $g$ respecto de $m_F$ es
\begin{align*}
    \int_{\mathbb{R}}{g \ dm_F} = \int_{\mathbb{R}}{g \ d\nu} = \int_{\mathbb{R}}{g(x)f(x) \ dx}.
\end{align*}
De igual forma, una función medible-Borel $g$ definida sobre $\mathbb{R}$ es integrable si y solo si $gf$ es integrable respecto de la medida de Lebesgue y, en ese caso,
\begin{align*}
    \int_{\mathbb{R}}{g \ dm_F} = \int_{\mathbb{R}}{g(x)f(x) \ dx}.
\end{align*}

La medida de Lebesgue-Stieltjes asociada a la función de distribución $F$ se denota también por $dF$. Con esta notación, tenemos
\begin{align*}
    \int_{B}{g(x) \ dF(x)} = \int_{B}{g(x)F'(x) \ dx}.
\end{align*}

\begin{ejemplo}
Sea $F$ la función de distribución dada por
\begin{align*}
    F(x) = \left\{ \begin{array}{lcc}
             -x^2 &  si  & x \leq 0\\
             2x &  si  &  0 < x < 2\\
             5 &  si  &  2 \leq x < 3\\
             3^x &  si  &  x \ge 3.\\
             \end{array}
        \right. 
\end{align*}
Si $g$ es medible Borel y no negativa, dar una expresión de la integral $\int_{\mathbb{R}}{g \ dm_F}$ en término de integrales respecto de la medida de Lebesgue.

\begin{align*}
    \int_{\mathbb{R}}{g(x) \ dF(x)} &= \int_{(-\infty,0)}{g(x) \ dF(x)} + \int_{\{0\}}{g(x) \ dF(x)} + \int_{(0,2)}{g(x) \ dF(x)} + \int_{\{2\}}{g(x) \ dF(x)} +\\
    & + \int_{(2,3)}{g(x) \ dF(x)} + \int_{\{3\}}{g(x) \ dF(x)} + \int_{(3,+\infty))}{g(x) \ dF(x)} \\
    &= \int_{-\infty}^{2}{g(x)(-2x) \ dx} + g(0)m_F(\{0\}) + \int_{0}^{2}{g(x)2 \ dx} + g(2)m_F(\{2\}) \\
    &+ \int_{2}^{3}{g(x)0 \ dx} + g(3)m_F(\{3\}) + \int_{3}^{+\infty}{g(x)3^x\log(3) \ dx}
\end{align*}
\end{ejemplo}

\begin{ejemplo}[La escalera del diablo]
A continuación se define una sucesión $\{f_n\}$ de funciones sobre el intervalo $[0,1]$ que converge a la función de Cantor.
\\
\newline
Sea $f_1(x) = x$. Para cada $n \in \mathbb{N}$, la siguiente función $f_{n+1}(x)$ se definirá en términos de $f_n(x)$ como sigue:
\begin{align*}
    f_{n+1}(x) = \left\{ \begin{array}{lcc}
             \frac{1}{2}f_n(3x) &  si  & 0 \leq x \leq \frac{1}{3}\\
            \\  \frac{1}{2} &  si  &  \frac{1}{3} < x < \frac{2}{3}\\
            \\  \frac{1}{2} + \frac{1}{2}f_n(3x-2) &  si  &  \frac{2}{3} < x \leq 1.\\
             \end{array}
        \right. 
\end{align*}
Gráficamente:
\begin{align*}
    \includegraphics[width=0.4\textwidth]{diablo.png}
\end{align*}
Nótese que
\begin{itemize}
    \item $f_n$ es creciente.
    \item $f_s$ es continua.
    \item Además
    \begin{align*}
        |f_{n+1}(x) - f_n(x)| \leq \frac{1}{2^n}.
    \end{align*}
    Como $\sum_{n=1}^{\infty}{\frac{1}{2^n}} < +\infty$ entonces $\sum_{n=1}^{\infty}{(f_{n+1}(x) - f_n(x))}$ converge uniformemente. 
    \begin{align*}
        S_N(x) = sum_{n=1}^{N}{(f_{n+1}(x) - f_n(x))} = f_{N+1}(x) - f_1(x),
    \end{align*}
    por tanto, $\{f_n\}$ converge uniformemente es $[0,1]$.
\end{itemize}
Sea $F$ el límite uniformente de $\{f_n\}$, entonces
\begin{itemize}
    \item $F$ es continua.
    \item $F$ es creciente.
    \item $F(1) = 1$ y $F(0) = 0$.
    \item $F'(x) = 0$ para todo $x \in [0,1] \backslash C$, donde $C$ es \textit{el conjunto de Cantor}, por tanto
    \begin{align*}
        \int_{0}^{1}{F'(x) \ dx}  = 0.
    \end{align*}
    y $F(1) - F(0) = 1 - 0 = 1$. Por tanto
    \begin{align*}
        \int_{0}^{1}{F'(x) \ dx}  = 0 \not = F(1) - F(0),
    \end{align*}
    es decir, no se cumple la regla de Barrow.
\end{itemize}
Además
\begin{align*}
    &m_F(\mathbb{R}) = m_F([0,1]) = m_F((0,1]) = F(1) - F(0) = 1 \ \ \text{y} \\
    &m_F([0,1] \backslash C) = 0 \Longrightarrow m_F(C) = 1 \not = m(C) = 0.
\end{align*}
Consideremos ahora $H: [0,1] \longrightarrow \mathbb{R}$ dada por $H(x) = F(x) + x$, es claro que
\begin{itemize}
    \item $H$ es continua.
    \item $H$ es estrictamente creciente.
    \item $H(0) = F(0) + 0 = 0$ y $H(1) = F(1) + 1 = 2$.
    \item $H([0,1]) = [0,2]$.
\end{itemize}
Restringiendo la imagen teneos que $H : [0,1] \longrightarrow [0,2]$ es continua y biyectiva y $H^{-1} : [0,2] \longrightarrow [0,1]$ es continua. Además, $m(H([0,1]\backslash C)) = m([0,1]\backslash C) = 1$. Calculemos ahora $m(H(C))$.
\begin{align*}
    &[0,1] = C \cup ([0,1] \backslash C) \\
    &H([0,1]) = H(c) \cup H([0,1] \backslash C) = [0,2]\\
\end{align*}
por lo que $m(H(C)) = 1$. Como tiene medida positiva, entonces existe $E \subset H(C)$ tal que $E$ no es medible-Lebesgue. Consideremos $H^{-1}(E) \subset C$. Como $m(C) = 0$ entonces $m(H^{-1}(E)) = 0$, por lo que $H^{-1}(E)$ es medible-Lebesgue, pero ¿es medible-Borel? No, porque si lo fuera, $H(H^{-1}(E)) = E$ sería medible-Borel, que no es cierto (puesto que $E$ no es medible-Lebesgue). Luego $H^{-1}(E)$ es medible-Lebesgue pero no medible-Borel.
\\
\newline
Adeás, $\mathcal{X}_E$ no es medible-Lebesgue, ya que
\begin{align*}
    \mathcal{X}_E = \mathcal{X}_{H^{-1}(E)}(H^{-1}(x)) = (\mathcal{X}_{H^{-1}(E)} \circ H^{-1})(x).
\end{align*}
\end{ejemplo}

\chapter{Integración compleja. Versiones simples del teorema de Cauchy}

\section{Primitivas}

\begin{defi}
Sea $\Omega \subset \com$ abierto y $f: \Omega \longrightarrow \com$ una función. Decimos que $F : \Omega \longrightarrow \com$ es primitiva de $f$ en $\Omega$ si
\begin{enumerate}
    \item $F$ es holomorfa en $\Omega$.
    \item $F' = f$ en $\Omega$.
\end{enumerate}
\end{defi}

\begin{ejemplo}
\begin{enumerate}
    \item Una primitiva de $e^z$ en $\com$ es $e^z$.
    \item Una primitiva de $a_0 + a_1z + ... + a_nz^n$ en $\com$ es
    \begin{align*}
        a_0z + a_1\frac{z^2}{2} + ... + a_n \frac{z^{n+1}}{n+1} 
    \end{align*}
    \item Si $f(z) = \sum_{n=0}^{\infty}{a_n(z-a)^n}$ es una serie de potencias con radio de convergencia $R > 0$, entonces
    \begin{align*}
        F(z) =  \sum_{n=0}^{\infty}{\frac{a_n}{n+1}(z-a)^{n+1}}
    \end{align*}
    es una primitiva de $f$ en $\Delta(a,R)$.
    \item Si $F$ es una primitiva de $f$ en $\Omega$, entonces $F + \lambda$ es una primitiva de $f$ en $\Omega$.
    \item Si $D \subset \com$ es dominio y $F_1,F_2$ son primitivas de $f$ en $D$, entonces existe $\lambda \in \com$ tal que $F_2 = F_1 + h$ en $D$.
    \begin{proof}
    $F_2 - F_1$ es holomorfa en $D$ y $(F_2 - F_1)' = F_2' - F_1' = f - f = 0$. Luego, $F_2 - F_1$ es constante en $D$.
     \end{proof}
     \item Una primitiva de $\frac{1}{z}$ en $\com \backslash (-\infty,0]$ es $\logp z$.
\end{enumerate}
\end{ejemplo}

\begin{prop}
Sea $D \in \comz$ dominio. Entonces existe una rama del $\log z$ en $D$ si y solo si $\frac{1}{z}$ tiene primitiva en $D$.
\end{prop}

\begin{proof}
$\boxed{\Longrightarrow}$ Supongamos que $g$ es una rama del $\log z$ en $D$, entonces sabemos que $g$ es derivable en $D$ y que $g'(z) = \frac{1}{z}$, $z \in D$, por tanto, $g$ es primitiva de $\frac{1}{z}$ en $D$.
\\
\newline
$\boxed{\Longleftarrow}$ Supongamos que $g: D \longrightarrow \com$ es primitiva de $\frac{1}{z}$ en $D$. Entonces $g$ es holomorfa en $D$ y $g'(z) = \frac{1}{z}$, $z \in D$.
\\
\newline
Consideremos $G(z) = ze^{-g(z)}$, $z \in D$. Entonces
\begin{enumerate}
    \item[(i)] $G$ es holomorfa en $D$.
    \item[(ii)] Dado $z \in D$
    \begin{align*}
        G'(z) = e^{-g(z)} - ze^{-g(z)}g'(z) = e^{-g(z)} - ze^{-g(z)} \frac{1}{z}= e^{-g(z)} - e^{-g(z)} = 0
    \end{align*}
    Por tanto, $G$ es constante y no nula en $D$. De esta manera, si $\beta$ es un logaritmo de dicha constante, entonces tenemos que
    \begin{align*}
        G(z) = e^{\beta} = ze^{-g(z)} \Longrightarrow  z = e^{g(z) + \beta}, \ \ z \in D
    \end{align*}
    Luego, $g(z) + \beta$ es rama del $\log z$ en $D$.
\end{enumerate}
\end{proof}

\begin{prop}
Sea $D \subset \com$ dominio y $f: D \longrightarrow \com$ holomorfa y nunca nula en $D$. Entonces existe una rama del $\log (f)$ en $D$ si y solo si $\frac{f'}{f}$ tiene primitiva en $D$.
\end{prop}

\section{Integración de funciones complejas sobre intervalos}

\begin{defi}
Sea $[a,b]$ un intervalo real no degenerado. Decimos que $\varphi : [a,b] \longrightarrow \com$ es integrable en $[a,b]$ (Riemann o Lebesgue) si lo son $\re(\varphi)$ e $\im(\varphi)$ y en ese caso
\begin{align*}
    \int_{a}^{b}{\varphi(t) \ dt} = \int_{a}^{b}{\re(\varphi(t)) \ dt} + i\int_{a}^{b}{\im(\varphi(t)) \ dt} 
\end{align*}
\end{defi}

\begin{obs}
\begin{enumerate}
    \item \underline{Linealidad}:
    \begin{align*}
        \int_{a}^{b}{\alpha_1\varphi_1(t) + \alpha_2\varphi_2(t) \ dt} = \alpha_1\int_{a}^{b}{\varphi_1(t) \ dt} + \alpha_2\int_{a}^{b}{\varphi_2(t) \ dt}
    \end{align*}
    \item \underline{Aditividad}: Si $c \in (a,b)$
    \begin{align*}
        \int_{a}^{b}{\varphi(t) \ dt} = \int_{a}^{c}{\varphi(t) \ dt} + \int_{c}^{b}{\varphi(t) \ dt}
    \end{align*}
    \item \underline{Notación}:
    \begin{align*}
        \int_{a}^{b}{\varphi(t) \ dt} = -\int_{b}^{a}{\varphi(t) \ dt} \ \ \ \text{y} \ \ \
        \int_{c}^{c}{\varphi(t) \ dt} = 0
    \end{align*}
    \item \underline{Estimación}:
    \begin{align*}
        \left| \int_{a}^{b}{\varphi(t) \ dt} \right| \leq \int_{a}^{b}{|\varphi(t)| \ dt}
    \end{align*}
    \begin{proof}
    Si $\varphi$ es integrable en $[a,b]$, entonces $|\varphi| = \sqrt{\re(\varphi)^2 + \im(\varphi)^2}$ es integrable en $[a,b]$.
    \begin{itemize}
        \item Si $I = \int_{a}^{b}{\varphi(t) \ dt}$, no hay nada que probar.
        \item Supongamos que $I \not = 0$, entonces $I = |I|e^{i\theta}$, $\theta \in \arg (I)$.
        \begin{align*}
             \left| \int_{a}^{b}{\varphi(t) \ dt} \right| &= |I| = Ie^{-i\theta} = \int_{a}^{b}{e^{-i\theta}\varphi(t) \ dt} \\
             &= \int_{a}^{b}{\re(e^{-i\theta}\varphi(t)) \ dt} + i\int_{a}^{b}{\im(e^{-i\theta}\varphi(t)) \ dt} \\
             &= \int_{a}^{b}{\re(e^{-i\theta}\varphi(t)) \ dt} \leq \int_{a}^{b}{\left|\re(e^{-i\theta}\varphi(t))\right| \ dt} \\
             &= \int_{a}^{b}{|\varphi(t)| \ dt}
        \end{align*}
    \end{itemize}
    \end{proof}
    \item Si $\varphi$ es continua en $[a,b]$, entonces $\varphi$ es integrable en $[a,b]$.
    \item El Teorema Fundamental del Cálculo tenemos que si $\varphi : [a,b] \longrightarrow \com$ es derivable y $\varphi'$ es integrable en $[a,b]$ entonces:
    \begin{align*}
        \int_{a}^{b}{\varphi'(t) \ dt} = \varphi(b) - \varphi(a)
    \end{align*}
    \item \underline{Cambio de variable}: Si $h :[a,b] \longrightarrow \mathbb{R}$ es de clase $\mathscr{C}^1$ y $\varphi : h([a,b]) \longrightarrow \com$ es continua, entonces $\varphi = \varphi \circ h$ son integrables en $h([a,b])$, $[a,b]$ respectivamente y
    \begin{align*}
        \int_{a}^{b}{\varphi \circ h(t)h'(t) \ dt} = \int_{h(a)}^{h(b)}{\varphi(s) \ ds}
    \end{align*}
    \itm \underline{Integración por partes}: $\varphi, \psi : [a,b] \longrightarrow \com$ son de clase $\mathscr{C}^1$ a trozos entonces
    \begin{align*}
        \int_{a}^{b}{\varphi(t)\psi'(t)} = \left\Big[ \varphi(t)\psi(t)\right\Big]_a^b - \int_{a}^{b}{\psi(t)\varphi(t) \ dt}
    \end{align*}
\end{enumerate}
\end{obs}

\section{Curvas y caminos}

\subsection{Curvas}

Sea $\mathscr{C}$ el conjunto de pares $(I,\varphi)$ donde $I$ es intervalo compacto de $\mathbb{R}$ y $\varphi : I \longrightarrow \com$ continua. Definimos la relación de equivalencia
\begin{align*}
    (I,\varphi) \sim (J,\psi) \Longleftrightarrow\text{Existe } h: I \longrightarrow J \text{ homeomorfismo creciente tal que } \varphi = \psi \circ h
\end{align*}

\begin{align*}
    \xymatrix{
    I \ar[r]^{\varphi} & \com  & & J \ar[r]^{\psi} & \com  \\
    & I \ar[r]^{h} \ar@/_1pc/[rr]_{\varphi} & J \ar[r]^{\psi} & \com 
    }
\end{align*}

\begin{defi}
\begin{itemize}
    \item Una curva en $\com$ es un elemento de $\mathscr{C}/\sim$.
    \item Cada representante de una curva $\gamma$ se llama parametrización de $\gamma$.
    \item Cada homeomorfismo creciente que liga dos parametrizaciones se llama cambio de parámetro.
\end{itemize}
\end{defi}

\begin{defi}
Sea $\gamma$ una curva de $\com$ parametrizada por $\varphi : [a,b] \longrightarrow \com$. Definimos:
\begin{itemize}
    \item $origen(\gamma) = \varphi(a)$.
    \item $extremo(\gamma) = \varphi(b)$.
    \item $soporte(\gamma) = sop(\gamma) = \varphi([a,b])$.
\end{itemize}
\end{defi}

\begin{obs}
Estas definiciones son independientes de la parametrización elegida.
\begin{proof}
Sea $\psi : [c,d] \longrightarrow \com$ otra parametrización de $\gamma$, entonces existe $h : [a,b] \longrightarrow [c,d]$ homeomorfismo creciennte tal que $\varphi = \psi \circ h$. Entonces
\begin{itemize}
    \item $origen(\gamma) = \varphi(a) = \varphi(h^{-1}(c)) = \psi(c)$.
    \item $extremo(\gamma) = \varphi(b) = \varphi(h^{-1}(d)) = \psi(d)$.
    \item $sop(\gamma) = \varphi([a,b]) = \psi(h([a,b])) = \psi([c,d])$.
\end{itemize}
\end{proof}
\end{obs}

\begin{defi}
Sea $\gamma$ una curva de $\com$.
\begin{itemize}
    \item Decimos que $\gamma$ es simple si una (todas) parametrización es inyectiva.
    \item Decimos que $\gamma$ es cerrada si $origen(\gamma) = extremo(\gamma)$.
    \item Decimos que $\gamma$ es una curva de Jordan si una (todas) parametrización suya $([a,b], \varphi)$ es cerrada y $\varphi$ es inyectiva en $[a,b)$.
\end{itemize}
\end{defi}

\begin{ejemplo}
\begin{enumerate}
    \item El segmento de origen $z_1$ y extremo $z_2$, denotado por $[z_1,z_2]$, lo podemos parametrizar como
    \begin{align*}
        \varphi : [0,1] &\longrightarrow \com \\
        t & \longmapsto \varphi(t) = z_1 + t(z_2 - z_1)
    \end{align*}
    \item La circunferecia de centro $a$ y radio $r$ recorrida una vez en sentido positivo (horario) empezando por $a+r$ se puede parametrizar como
    \begin{align*}
        \varphi : [0,2\pi) &\longrightarrow \com \\
        t & \longmapsto \varphi(t) = a + re^{i\theta}
    \end{align*}
\end{enumerate}
\end{ejemplo}

\begin{defi}
Sean $([a_1,b_1],\varphi_1)$ y $([a_2,b_2],\varphi_2)$ dos parametrizaciones de curvas $\gamma_1$ y $\gamma_2$ respectivamente con $\varphi_1(b_1) = \varphi_2(a_2)$, entonces
\begin{align*}
    \varphi(t) = \left\{ \begin{array}{lcc}
             \varphi_1(t) &  si  & t \in [a_1,b_1]\\
             \\ \varphi_2(t-b_1+a_2) &  si &t \in [b_1,b_1 + b_2 - a_2] \\
             \end{array}
   \right.
\end{align*}
es una parametrización de una curva $\gamma$, que se llama $\gamma_1 + \gamma_2$.
\end{defi}

\begin{obs}
La definición es independiente de las parametrizaciones elegidas.
\end{obs}

\begin{ejemplo}
La poligonnal de vértices $z_1,...,z_n$, denotada por $[z_1,...,z_n]$ se puede parametrizar como
    \begin{align*}
        [z_1,...,z_n] = [z_1,z_2] + ... + [z_{n-1},z_n]
    \end{align*}
\end{ejemplo}

\begin{defi}
Si $\gamma$ es una curva de $\com$ parametrizada por $\varphi : [a,b] \longrightarrow \com$, entonces su curva opuesta, $-\gamma$, viene parametrizada por
\begin{align*}
    -\gamma : [-b,-a] \longrightarrow \com, \ \ -\gamma(t) = \varphi(-t)
\end{align*}
\end{defi}

\begin{obs}
$\gamma + (-\gamma)$ no es una curva constante.
\end{obs}

\subsection{Funciones de variaciones acotadas}

\begin{defi}
Sea $\varphi : [a,b] \longrightarrow \com$ función.
\begin{itemize}
    \item Para una partición $\Pi = \{ a = t_0 < t_1 < ... < t_n = b \}$ de $[a,b]$, definimos la variación de $\varphi$ respecto de $\Pi$ como
    \begin{align*}
        Var(\varphi, \Pi) = \sum_{j=1}^{n}{|\varphi(t_j) - \varphi(t_{j-1})|}
    \end{align*}
    \item La variación total de $\varphi$ en $[a,b]$ se define como
    \begin{align*}
        Var_{[a,b]}(\varphi) = \sup_{\Pi \in \mathcal{P}([a,b])} Var(\varphi, \Pi)
    \end{align*}
    \item Decimos que $\varphi$ es de variación acotada en $[a,b]$ si $Var_{[a,b]}(\varphi)$ es finita.
\end{itemize}
\end{defi}

\begin{obs}
\begin{enumerate}
    \item $\varphi$ no tiene que ser necesariamente continua.
    \item Si $\varphi$ es continua, entonces $\varphi$ es una parametrización de una curva $\gamma$ y $Var(\varphi,\Pi)$ representa la longitud de una poligonal con vértices en $\gamma$, ordenados en orden creciente de los parámetros.
    \item Si $\Pi_1 \subseteq \Pi_2$ entonces $Var(\varphi,\Pi_1) \leq Var(\varphi,\Pi_2)$.
    \item 
    \begin{enumerate}
        \item Si $\varphi : [a,b] \longrightarrow \com$ es función y $[\alpha,\beta] \subset [a,b]$ entonces
    \begin{align*}
        Var_{[\alpha,\beta]}(\psi) \leq Var_{[a,b]}(\varphi)
    \end{align*}
    siendo $\psi = \varphi |_{[\alpha,\beta]}$.
    \item Si $c \in (a,b)$ entonces
    \begin{align*}
        Var_{[a,b]}(\varphi) = Var_{[a,c]}(\psi_1) + Var_{[c,b]}(\psi_2)
    \end{align*}
     siendo $\psi_1 = \varphi |_{[a,c]}$ y $\psi_2 = \varphi |_{[c,b]}$.
    \end{enumerate}
    \begin{proof}
    $a)$ Basta ver que $Var_{[a,b]}(\varphi)$ es cota superior de $\left\{ Var(\psi,\Pi) : \Pi \in \mathcal{P}([a,b]) \right\}$. Sea $\Pi = \{ \alpha = t_0 < t_1 < ... < t_n = \beta\}$ una partición de $[\alpha,\beta]$. Añadimos a $\Pi$ los extremos $a$ y $b$ si fueran necesarios para obtener una partición de $[a,b]$
    \begin{align*}
        P = \{s_0 = a < s_1 < ... < s_m = b\}
    \end{align*}
    Entonces
    \begin{align*}
        Var(\psi,\Pi) &= \sum_{j=1}^{n}{|\psi(t_j) - \psi(t_{j-1})|} = \sum_{j=1}^{n}{|\varphi(t_j) - \varphi(t_{j-1})|} \\
        & \leq \sum_{k=1}^{m}{|\varphi(s_k) - \varphi(s_{k-1})|} = Var(\varphi, P) \leq Var_{[a,b]}(\varphi)
    \end{align*}
    $b)$ Se deja como ejercicio.
    \end{proof}
    \item Si $\varphi : [a,b] \longrightarrow \com$ es función y $h : [\alpha,\beta] \longrightarrow [a,b]$ es homeomorfismo, entonces
    \begin{align*}
        Var_{[a,b]}(\varphi) = Var_{[\alpha,\beta]}(\varphi \circ h)
    \end{align*}
    \begin{proof}
    veamos primero que $Var_{[a,b]}(\varphi) \leq Var_{[\alpha,\beta]}(\varphi \circ h)$.
    Sea $\Pi$ partición de $[\alpha,\beta]$, $\Pi = \{t_0 = a < t_1 < ... < t_n = \beta\}$. Entonces
    \begin{itemize}
        \item $\Pi^* = \{ a = h(t_0) < ... < b = h(t_n) \}$ es partición de $[a,b]$ si $h$ crece.
        \item $\Pi^* = \{ b = h(t_0) < ... < a = h(t_n) \}$ es partición de $[a,b]$ si $h$ decrece.
    \end{itemize}
    y entonces
    \begin{align*}
        Var(\varphi,\Pi^*) &= \sum_{j=1}^{n}{|\varphi(h(y_j)) - \varphi(h(t_{j-1}))|} = \sum_{j=1}^{n}{|\varphi \circ h (t_j) - \varphi \circ h (t_{j-1})|} \\
        &= Var(\varphi \circ h, \Pi) \leq Var_{[\alpha,\beta]}(\varphi \circ h)
    \end{align*}
    Lo que nos dice que $Var_{[a,b]}(\varphi) \leq Var_{[\alpha,\beta]}(\varphi \circ h)$. \\
    \newline
    Veamos ahora que $Var_{[a,b]}(\varphi) \ge Var_{[\alpha,\beta]}(\varphi \circ h)$. Se hace de forma análoga trabajando con la inversa de $h$ (que existe puesto que $h$ es homeomorfismo y por tanto, su inversa también es homeomorfismo).
    \end{proof}
\end{enumerate}
\end{obs}

\begin{cor}
$\varphi$ es variación acotada si y solo si $\varphi \circ h$ es de variación acotada (cualquiera que sea el homeomorfimos $h$).
\end{cor}

\begin{ejemplo}
\begin{enumerate}
    \item Si $\varphi : [a,b] \longrightarrow \mathbb{R}$ es monótona, entonces $\varphi$ es de variación acotada.
    \item Si $\varphi : [a,b] \longrightarrow \mathbb{R}$ es diferencia de funciones crecientes, entonces $\varphi$ es de variación acotada en $[a,b]$.
    \item Existen funciones continuas que no son de variación acotada, por ejemplo:
    \begin{align*}
        \varphi : \left[ -\frac{2}{\pi},0 \right] &\longrightarrow \com \\
        t &\longmapsto \varphi(t) = t +it\sen\left( \frac{1}{t}\right)
    \end{align*}
    \begin{itemize}
        \item $\varphi$ es continua en $\left[ -\frac{2}{\pi},0 \right]$ ($\varphi(0) = 0$).
        \item La idea de por qué no es de variación acotada es la siguiente. Definimos la partición 
        \begin{align*}
            \Pi_N = \{t_0 < t_1 < ... < t_{2N +1} < t_{\infty} \}, \ \ N \in \mathbb{N}
        \end{align*}
        donde 
        \begin{align*}
            t_j = -\frac{1}{\frac{\pi}{2} + j\pi}, \ \ j \in \mathbb{N}_{0}
        \end{align*}
        Observamos que
        \begin{align*}
            \varphi(t_j) = t_j - it_j\sen\left( \frac{\pi}{2} +j\pi \right) = t_j + i(-1)^jt_j
        \end{align*}
        Y con esto (y desarrollando algunos cálculos) tenemos que
        \begin{align*}
            Var(\varphi,\Pi_N) = ... \ge \frac{1}{\pi} \sum_{k=0}^{N}{\frac{1}{k+1}} \xrightarrow[N \to \infty]{} \infty
        \end{align*}
    \end{itemize}
\end{enumerate}
\end{ejemplo}

\begin{prop}
Si $\varphi : [a,b] \longrightarrow \com$ es de clase $\mathscr{C}^1$ en $[a,b]$, entonces $\varphi$ es de variación acotada en $[a,b]$ y
\begin{align*}
    Var_{[a,b]}(\varphi) = \int_{a}^{b}{\left|\varphi'(t)\right| \ dt}
\end{align*}
\end{prop}

\begin{proof}
Haremos la demostración en dos partes.
\begin{itemize}
    \item Probemos que $\int_{a}^{b}{\left|\varphi'(t)\right| \ dt}$ es cota superior de $\{Var(\varphi, \Pi) : \Pi \in \mathcal{P}([a,b])\}$. Sea $\Pi = \{t_0 = a < t_1 <... < t_n = b\}$ una partición de $[a,b]$. Entonces
    \begin{align*}
        Var(\varphi, \Pi) &= \sum_{j=1}^{n}{|\varphi(t_j) - \varphi(t_{j-1})|} = \sum_{j=1}^{n}{\left| \int_{t_{j-1}}^{t_j}{\varphi'(t)} \ dt \right|} \\
        & \leq \sum_{j=1}^{n}{\int_{t_{j-1}}^{t_j}{|\varphi'(t)|} \ dt } = \int_{a}^{b}{\left|\varphi'(t)\right| \ dt}
    \end{align*}
    \item Probemos que $\int_{a}^{b}{\left|\varphi'(t)\right| \ dt}$ es supremo $\{Var(\varphi, \Pi) : \Pi \in \mathcal{P}([a,b])\}$. Sea $\varepsilon > 0$, queremos encontrar una partición $\Pi$ de $[a,b]$ tal que $Var(\varphi, \Pi) > \int_{a}^{b}{|\varphi'(t)| \ dt} - \varepsilon$. Como $\varphi'$ es continua en $[a,b]$, dado $\varepsilon > 0$, existe $\delta > 0$ tal que si $s,t \in [a,b]$ con $|s-t| < \delta$, entonces $|\varphi'(s) - \varphi'(t)| < \frac{\varepsilon}{2(b-a)}$. Así
    \begin{align*}
        \int_{a}^{b}{\left|\varphi'(t)\right| \ dt} &= \sum_{j=1}^{n}{\int_{t_{j-1}}^{t_j}{|\varphi'(t)|} \ dt } =  \sum_{j=1}^{n}{\int_{t_{j-1}}^{t_j}{|\varphi'(t) -\varphi'(t_j) + \varphi(t_j)|} \ dt } \\
        & \leq \sum_{j=1}^{n}\left({\int_{t_{j-1}}^{t_j}{|\varphi'(t) -\varphi'(t_j)| + |\varphi(t_j)|} \ dt }\right) \\
        &< \sum_{j=1}^{n}{\int_{t_{j-1}}^{t_j}{\frac{\varepsilon}{2(b-a)}}\ dt + \int_{t_{j-1}}^{t_j}{|\varphi'(t_j)| \ dt}} \\
        &= \frac{\varepsilon}{2} + \sum_{j=1}^{n}{|\varphi'(t_j)|(t_j - t_{j-1})} = \frac{\varepsilon}{2} + \sum_{j=1}^{n}{|\varphi'(t_j)(t_j - t_{j-1})|} \\
        &= \frac{\varepsilon}{2} + \sum_{j=1}^{n}{\left| \int_{t_{j-1}}^{t_j}{\varphi'(t_j)} \ dt \right|} = \frac{\varepsilon}{2} + \sum_{j=1}^{n}{\left| \int_{t_{j-1}}^{t_j}{\varphi'(t_j) - \varphi'(t) + \varphi'(t)} \ dt \right|} \\
        & \leq \frac{\varepsilon}{2} + \sum_{j=1}^{n} \left( \left| \int_{t_{j-1}}^{t_j}{\varphi'(t_j) - \varphi'(t) }\right| + \left|\int_{t_{j-1}}^{t_j}{\varphi'(t) \ dt} \right|\right) \\
        & \leq \frac{\varepsilon}{2} + \sum_{j=1}^{n} \left( \ \int_{t_{j-1}}^{t_j}{|\varphi'(t_j) - \varphi'(t) |} \left|\int_{t_{j-1}}^{t_j}{\varphi'(t) \ dt} \right|\right) \\
        & < \frac{\varepsilon}{2} + \frac{\varepsilon}{2} + \sum_{j=1}^{n}{\left|\int_{t_{j-1}}^{t_j}{\varphi'(t) \ dt} \right|} = \varepsilon + \sum_{j=1}^{n}{|\varphi(t_j) - \varphi(t_{j-1})|} = \varepsilon + Var(\varphi, \Pi)
    \end{align*}
\end{itemize}
\end{proof}

\begin{prop}
Si $\varphi : [a,b] \longrightarrow \com$ es de clase $\mathscr{C}^1$ a trozos en $[a,b]$, entonces $\varphi$ es de variación acotada en $[a,b]$ y
\begin{align*}
    Var_{[a,b]}(\varphi) = \int_{a}^{b}{\left|\varphi'(t)\right| \ dt}
\end{align*}
\end{prop}

\begin{defi}
Sea $\gamma$ una curva en $\com$.
\begin{itemize}
    \item  Definimos la longitud de $\gamma$ como
    \begin{align*}
        long(\gamma) := Var_{[a,b]}(\varphi)
    \end{align*}
    donde $\varphi : [a,b] \longrightarrow \com$ es una parametrización cualquiera de $\gamma$.
    \item Decimos que $\gamma$ es rectificable si $long(\gamma) < \infty$.
    \item Decimos que $\gamma$ es un camino si tiene una parametrización de clase $\mathscr{C}^1$ a trozos.
\end{itemize}
\end{defi}

\begin{obs}
\begin{itemize}
    \item Todo camino es rectificable.
    \item $long(\gamma) = long(-\gamma)$.
    \item Si $\gamma_1,\gamma_2$ son curvas tales que $extremo(\gamma_1) = origen(\gamma_2)$, entonces
    \begin{align*}
        long(\gamma_1 + \gama_2) = long(\gamma_1) + long(\gamma_2)
    \end{align*}
\end{itemize}
\end{obs}

\subsection{Integración sobre caminos}

\begin{defi}
Sea $\gamma$ un camino de $\com$ y sea $f$ una función continua sobre $sop(\gamma)$. Definimos la ingral de $f$ sobre $\gamma$ como
\begin{align*}
    \int_{\gamma}{f(z) \ dz} := \int_{a}^{b}{f(\varphi(t))\cdot \varphi'(t) \ dt}
\end{align*}
donde $\varphi : [a,b] \longrightarrow \com$ es una paramemtrización de clase $\mathscr{C}^1$ a trozos en $[a,b]$ de $\gamma$.
\end{defi}

\begin{lema}
Si $\varphi : [a,b] \longrightarrow \com$ es una parametrización de clase $\mathscr{C}^1$ a trozos y $f$ una función continua sobre $\varphi([a,b])$, entonces
\begin{align*}
    \int_{a}^{b}{f(\varphi(z))\cdot \varphi'(t) \ dt} = \lim_{\|P\| \to 0}{S(f,\varphi,P)}
\end{align*}
donde $P \in \mathcal{P}([a,b])$, $P = \{t_0 = a < t_1 < ... < t_n = b\}$ y
\begin{align*}
    S(f,\varphi,P) = \sum_{j=1}^{n}f(\varphi(t_j))(\varphi(t_j) - \varphi(t_{j-1}))
\end{align*}
\end{lema}

\begin{proof}
Como $\varphi$ es de clase $\mathscr{C}^1$ a trozos en $[a,b]$, entonces $\varphi$ es de variación acotada en $[a,b]$ y
\begin{align*}
    Var_{[a,b]}(\varphi) = \int_{a}^{b}{\left|\varphi'(t)\right| \ dt}
\end{align*}
Sea $\varepsilon > 0$ y sea $0 < Var_{[a,b]}(\varphi) < V$. Como $f \circ \varphi$ es continua en $[a,b]$, entonces es uniformemente continua en $[a,b]$. Así, dado $\varepsilon >x 0$, existe $\delta > 0$ tal que si $s,t \in [a,b]$ con $|s-t| < \delta$, entonces $|f \circ \varphi(s) - f \circ \varphi(t)| < \varepsilon/V$. 
\\
\newline
Ahora, si $P = \{t_0 = a < t_1 < ... < t_n = b\}$ es una partición de $[a,b]$ tal que $\|P\| < \delta$, entonces
\begin{align*}
    \left| \int_{a}^{b}{f(\varphi(z))\cdot \varphi'(t) \ dt} - {S(f,\varphi,P)} \right| &= \left| \sum_{j=1}^{n} \left[ \int_{t_{j-1}}^{t_j} f(\varphi(t))\varphi'(t) \ dt - f(\varphi(t_j))(\varphi(t_j) - \varphi(t_{j-1})) \right] \right| \\
    &= \left| \sum_{j=1}^{n} \left[ \int_{t_{j-1}}^{t_j} f(\varphi(t))\varphi'(t) \ dt -f(\varphi(t_j))\int_{t_{j-1}}^{t_j}{\varphi'(t) \ dt} \right] \right| \\
    &= \left| \sum_{j=1}^{n}  \int_{t_{j-1}}^{t_j} \Big[f(\varphi(t)) - f(\varphi(t_j))\Big]\varphi'(t) \ dt   \right| \\
    & \leq  \sum_{j=1}^{n}  \int_{t_{j-1}}^{t_j} |f(\varphi(t)) - f(\varphi(t_j))| \cdot |\varphi'(t)| \ dt  \\
    & <  \sum_{j=1}^{n}{\frac{\varepsilon}{V}\int_{t_{j-1}}^{t_j}|\varphi'(t)| \ dt} = \frac{\varepsilon}{V}\int_{a}^{b}{|\varphi'(t)| \ dt} \\
    &= \frac{\varepsilon}{V}Var_{[a,b]}(\varphi) < \frac{\varepsilon}{V}V = \varepsilon
\end{align*}
\end{proof}

\begin{lema}
Sea $\varphi : [a,b] \longrightarrow \com$ una parametrización de clase $\mathscr{C}^1$ a trozos en $[a,b]$ y sea $f$ una función continua sobre $\varphi([a,b])$. Si $h : [\alpha,\beta] \longrightarrow [a,b]$ es un homeomorfismo, entonces
\begin{align*}
    \int_{a}^{b}{f(\varphi(z))\cdot \varphi'(t) \ dt} = \lim_{\|P\| \to 0}{S(f,\varphi \circ h,P)}
\end{align*}
\end{lema}

\begin{proof}
Como $\lim_{\|P\| \to 0}{S(f,\varphi,P)} = \int_{a}^{b}{f(\varphi(t))\varphi'(t) \ dt}$. 
\\
\newline
Dado $\varepsilon > 0$, existe $\delta > 0$ tal que si $\|P\| < \delta$, entonces $\left|S(f,\varphi,P) - \int_{a}^{b}{f(\varphi(t))\varphi'(t) \ dt}\right| < \varepsilon$. Como $h : [\alpha,\beta] \longrightarrow [a,b]$ es homeomorfismo, dado $\varepsilon > 0$, existe $\delta > 0$ tal que si $s,t \in [\alpha,\beta] < \delta$ entonces $|h(s) - h(t)| < \varepsilon$. Sea $P = \{ t_0 = \alpha < t_1 < ... < t_n = \beta \}$ una partición de $[\alpha,\beta]$ con $\|P\| < \delta$. Definimos $P^h = \{h(t_0) = a < ... < h(t_n) = b \}$, que es una partición de $[a,b]$ con $\|P^h\| = \max_{j}{|h(t_j) - h(t_{j-1})|} < \delta$. Por tanto
\begin{align*}
    \left|S\left(f,\varphi,P^h\right) - \int_{a}^{b}{f(\varphi(t))\varphi'(t) \ dt}\right| < \varepsilon
\end{align*}
De aquí se sigue que
\begin{align*}
    \left|S(f,\varphi \circ h, P) \int_{a}^{b}{f(\varphi(t))\varphi'(t) \ dt}\right|
    &= \left| \sum_{j=1}^{n} f(\varphi \circ h(t_j))\Big[\varphi \circ h(t_j) - \varphi \circ h(t_{j-1})\Big] -\int_{a}^{b}{f \circ \varphi (t) \varphi'(t) \ dt} \right| \\
    &= \left| \sum_{j=1}^{n} f(\varphi( h(t_j)))\Big[\varphi( h(t_j)) - \varphi(h(t_{j-1}))\Big] -\int_{a}^{b}{f \circ \varphi (t) \varphi'(t) \ dt} \right| \\
    &= \left|S\left(f,\varphi,P^h\right) - \int_{a}^{b}{f(\varphi(t))\varphi'(t) \ dt}\right| < \varepsilon
\end{align*}
\end{proof}

\begin{obs}
Algunas propiedades inmediatas son
\begin{enumerate}
    \item \underline{Linealidad}:
    \begin{align*}
        \int_{\gamma}{(\alpha f + \beta g)(z) \ dz} = \alpha\int_{\gamma}{f(z) \ dz} + \beta\int_{\gamma}{g(z) \ dz}
    \end{align*}
    \item 
    \begin{align*}
        \int_{-\gamma}{f(z) \ dz} = -\int_{\gamma}{f(z) \ dz}
    \end{align*}
    \item Dados $\gamma_1,\gamma_2$ caminnos tales que $extremo(\gamma_1) = origen(\gamma_2)$. Si $f$ es continua en $sop(\gamma_1 + \gamma_2)$ entonces
    \begin{align*}
        \int_{\gamma_1 + \gamma_2}{f(z) \ dz} = \int_{\gamma_1}{f(z) \ dz} + \int_{\gamma_2}{f(z) \ dz}
    \end{align*}
    \item
        \begin{align*}
        \int_{\gamma + (-\gamma)}{f(z) \ dz} = \int_{\gamma}{f(z) \ dz} + \int_{-\gamma}{f(z) \ dz} = \int_{\gamma}{f(z) \ dz} - \int_{\gamma}{f(z) \ dz} = 0
    \end{align*}
    \item Si $\gamma$ es camino cerrado  y $f$ es continua en $sop(\gamma)$, entonces $\int_{\gamma}{f(z) \ dz}$ es independiente del $origen(\gamma)$.
\end{enumerate}
\end{obs}

\begin{prop}[Regla de Barrow]
Si $\gamma$ es camino en $\com$ y $f$ es de clase $\mathscr{C}^1$ en un entorno del $sop(\gamma)$, entonces
\begin{align*}
    \int_{\gamma}{f'(z) \ dz} = f(extremo(\gamma)) - f(origen(\gamma))
\end{align*}
\end{prop}

\begin{obs}
\begin{enumerate}
    \item \underline{Acotación de la integral}: Sea $\gamma$ camino de $\com$, $f$ continua en $sop(\gamma)$ y $\varphi : [a,b] \longrightarrow \com$ una paramatrización de clase $\mathscr{C}^1$ a trozos  en $[a,b]$ de $\gamma$, entonces 
    \begin{align*}
        \left| \int_{\gamma}{f(z) \ dz} \right| &= \left| \int_{a}^{b}{f(\varphi(t))\varphi'(t) \ dt} \right| \leq \int_{a}^{b}{\left|f(\varphi(t))\varphi'(t) \right|\ dt} \\
        & \leq \max_{z \in sop(\gamma)} |f(z)|\int_{a}^{b}{\varphi'(t) \ dt} = \max_{z \in sop(\gamma)} |f(z)| \cdot long(\gamma)
    \end{align*}
    \item \underline{Intercambio límite e integral}: Sea $\gamma$ un camino de $\com$, $\{f_n\}$ una sucesión de funciones continua sobre $sop(\gamma)$ que converge uniformemente a una función continua $f$ en $sop(\gamma)$. Entonces
    \begin{align*}
        \lim_{n}{\int_{\gamma}{f_n(z) \ dz}} = \int_{\gamma}{\lim_{n}f_n(z) \ dz} = \int_{\gamma}{f(z) \ dz}
    \end{align*}
    \begin{proof}
    Basta observar que
    \begin{align*}
        \left| \int_{\gamma}{f(z) \ dz} -  \int_{\gamma}{f_n(z) \ dz}\right|= \left| \int_{\gamma}{f(z) - f_n(z) \ dz} \right| \leq \max_{z \in sop(\gamma)} |f_n(z) - f(z)| \cdot long(\gamma)
    \end{align*}
    Como $\lim_n{\max_{z \in sop(\gamma)} |f_n(z) - f(z)| \cdot long(\gamma)} = 0$, pues $\{f_n\}$ converge uniformemente a $f$ en $sop(\gamma)$, entonces 
    \begin{align*}
        \lim_{n}  \left| \int_{\gamma}{f(z) \ dz} -  \int_{\gamma}{f_n(z) \ dz}\right| = 0
    \end{align*}
    \end{proof}
    \item \underline{Intercambio límite y serie}: Sea $\gamma$ un camino en $\com$, $\sum_{n=1}^{\infty}{f_n}$ una serie de funciones continuas sobre $sop(\gamma)$ que converge uniformemente en $sop(\gamma)$, entonces
    \begin{align*}
        \sum_{n=1}^{\infty} \int_{\gamma}{f_n(z) \ dz} = \int_{\gamma}{\sum_{n=1}^{\infty} f_n(z) \ dz}
    \end{align*}
\end{enumerate}
\end{obs}

\begin{defi}
Sea $\gamma$ camino de $\com$ representado por una parametrización $\varphi : [a,b] \longrightarrow \com$ de clase $\mathscr{C}^1$ a trozos en $[a,b]$. Sea $f$ una función continua sobre $sop(\gamma)$. Definimos
\begin{itemize}
    \item Integral de $f$ respecto del elemento de longitud de arco
    \begin{align*}
        \int_{\gamma}{f(z) \ |dz|} := \int_{a}^{b}{f(\varphi(t))|\varphi'(t)| \ dt}
    \end{align*}
    \item Integrales respecto de la parte real e imaginaria de $\gamma$
    \begin{itemize}
        \item 
        \begin{align*}
            \int_{\gamma}{f(z) \ dx} := \int_{a}^{b}{f(\varphi(t)) (\re \ \varphi)'(t)) \ dt}
        \end{align*}
        \item 
        \begin{align*}
            \int_{\gamma}{f(z) \ dy} := \int_{a}^{b}{f(\varphi(t)) (\im \ \varphi)'(t)) \ dt}
        \end{align*}
    \end{itemize}
\end{itemize}
\end{defi}

\begin{obs}
Las definiciones no dependen de la parametrización elegida.
\end{obs}

\begin{defi}
Sea $D$ un dominio en $\com$ y sea $f: D \longrightarrow \com$ continua. Decimos que la integral de $f$ es independiente del camino en $D$ si para todo par de puntos $z_1,z_2 \in D$ y para todo par de caminos $\gamma_1,\gamma_2$ en $D$ con $origen(\gamma_1) = origen(\gamma_2) = z_1$ y $extremo(\gamma_1) = extremo(\gamma_2) = z_2$ se tiene que
\begin{align*}
    \int_{\gamma_1}{f(z) \ dz} = \int_{\gamma_2}{f(z) \ dz}
\end{align*}
\end{defi}

\begin{obs}
Esta definición es equivalente a que $\int_{\gamma}{f(z) \ dz} = 0$ para todo camino cerrado $\gamma$ en $D$.
\end{obs}

\begin{teo}
Sea $D$ un dominio en $\com$ y $f: D \longrightarrow \com$ continua. Son equivalentes:
\begin{enumerate}
    \item[(i)] La integral de $f$ es independiente del camino en $D$.
    \item[(ii)] $f$ tiene primitiva en $D$.
\end{enumerate}
\end{teo}

\begin{proof}
\
\newline
$\boxed{(i) \Longleftarrow (ii)}$ Sea $F$ primitiva de $f$ en $D$, entonces $F$ es holomorfa en $D$ y $F = f'$ en $D$, luego $F$ es de clase $\mathscr{C}^1$ en $D$. Así, si $\gamma$ es un camino cerrado en $D$, por la regla de Barrow
\begin{align*}
    \int_{\gamma}{f(z) \ dz} = \int_{\gamma}{F'(z) \ dz} = F(extremo(\gamma)) - F(origen(\gamma)) = 0
\end{align*}
$\boxed{(i) \Longrightarrow (ii)}$ Busquemos una primitiva de $f$ en $D$. Fijemos $a \in D$. Sea $\gamma_z$ un camino en $D$ de origen $a$ y extremo $z$ (siempre existe al menos uno). Definimos $F(z) = \int_{\gamma_z}{f(\xi) \ d\xi}$, que está bien definida pues la integral de $f$ es independiente del camino en $D$.
\\
\newline
Probemos que $F$ es derivable en $D$ y $F' = f$ en $D$. Fijamos $z_0 \in D$. Sea $\gamma_0$ un camino en $D$ de origen $a$ y extremo $z_0$. Entonces $F(z_0) = \int_{\gamama_{z_0}}{f(\xi) \ d\xi}$. Como $z_0 \in D$ y $D$ e s dominio, entonces existe $r > 0$ tal que $\Delta(z_0,r) \subset D$. Para $z \in \Delta(z_0,r)$, consideramos el segmento $[z_0,z]$ que está en $\Delta(z_0,r)$ (pues un disco es convexo). Observamos que $\gamma_0 + [z_0,z]$ es un camino en $D$ de origen $a$ y extremo $z$, luego
\begin{align*}
    F(z) = \int_{\gamma_0 + [z_0,z]}{f(\xi) \ d\xi}
\end{align*}
Así
\begin{align*}
    \left| \frac{F(z) - F(z_0)}{z - z_0} - f(z_0)\right| &= \left| \frac{1}{z - z_0} \left[ \int_{\gamma_0}{f(\xi) \ d\xi} + \int_{[z_0,z]}{f(\xi) \ d\xi} - \int_{\gamma_0}{f(\xi) \ d\xi} - f(z_0) \right] \right| \\
    &= \left| \frac{1}{z - z_0} \int_{[z_0,z]}{f(\xi) - f(z_0)\ d\xi}\right| \\
    & \leq \frac{1}{z - z_0} \cdot \max_{\xi \in [z_0,z]}|f(\xi) - f(z_0)| \cdot long([z_0,z]) \\
    & \leq \frac{1}{z - z_0} \cdot \max_{\xi \in [z_0,z]}|f(\xi) - f(z_0)| \cdot |z-z_0| \\
    & = \max_{\xi \in [z_0,z]}|f(\xi) - f(z_0)| \xrightarrow[z \to z_0]{} 0
\end{align*}
Lo que prueba que $F$ es derivable en $D$ y que $F' = f$ en $D$.
\end{proof}

\begin{ejemplo}
\begin{enumerate}
    \item $\int_{\gamma}{z^n \ dz} = 0$ para todo $n \in \mathbb{N}_0$.
    \item Si $n \in \mathbb{Z}$, $n < 0$ y $n \not = 1$, entonces $z^n$ es derivada de $\frac{z^{n+1}}{n+1}$ en $\com \backslash \{0\}$, por tanto, mientras $sop(\gamma) \subset \com \backslash \{0\}$,  $\int_{\gamma}{z^n \ dz} = 0$.
    \item En general,  $\int_{\gamma}{P(z) \ dz} = 0$, para todo polinomio $P$.
    \item  $\int_{\gamma}{\sum_{n=0}^{\infty}a_n(z-a)^n \ dz} = 0$ siempre que $sop(\gamma)$ esté en el disco de convergencia de la serie.
    \item $\frac{1}{z}$ no tiene primitiva en $\com \backslash \{0\}$, luego la intergal de $\frac{1}{z}$ no es independiente del camino en $\com \backslash \{0\}$.
\end{enumerate}
\end{ejemplo}

\section{Índice de un punto respecto de un camino cerrado}

\begin{defi}
Sea $\gamma$ un camino cerrado en $\com$ y $z_0 \in \com \backslash sop(\gamma)$. Definimos el índice de $z_0$ respecto de $\gamma$ como
\begin{align*}
    n(\gamma,z_0) = \frac{1}{2\pi i}\int_{\gamma}{\frac{1}{z-z_0} \ dz}
\end{align*}
\end{defi}

\begin{teo}
Sea $\gamma$ un camino cerrado en $\com$. Entonces
\begin{enumerate}
    \item[(i)] $n(\gamma,z) \in \mathbb{Z}$ para cualquier $z \in \com \backslash sop(\gamma)$.
    \item[(ii)] $n(\gamma, \bullet)$ es una función continua en $\com \backslash sop(\gamma)$.
    \item[(iii)] $n(\gamma,z) = 0$ para cada $z$ en la componente conexa de $\com \backslash sop(\gamma)$ no acotada.
\end{enumerate}
\end{teo}

\begin{proof}
\begin{enumerate}
    \item[(i)] Sabemos que si $\gamma$ es un camino cerrado en $\com$ que no pasa por $z_0 \in \com$, entonces el número de vueltas netas que $\gamma$ da alrededor de $z_0$ viene dado por
    \begin{align*}
        n(\gamma,z_0) = \frac{1}{2\pi i}\int_{\gamma}{\frac{1}{z-z_0} \ dz} = \frac{1}{2\pi}\var_{\gamma}(\arg(z-z_0)) \in \mathbb{Z}
    \end{align*}
    \item[(ii)] Sea $z_0 \in \com \backslash sop(\gamma)$. Fijemos $\varepsilon > 0$. Como $\com \backslash sop(\gamma)$ es abierto, existe $r > 0$ tal que $\Delta(z_0,r) \subset \com \backslash sop(\gamma)$ ($|\xi - z_0| \ge r$ para todo $\xi \in sop(\gamma)$). Tomamos $\delta < \min\left\{ \frac{r}{2}, \frac{\varepsilon \pi r^2}{long(\gamma)}\right\}$. Si $z \in \Delta(z_0,\delta)$ y $\xi \in sop(\gamma)$, entonces
    \begin{align*}
        |\xi - z| \ge |\xi - z_0| - |\xi - z| \ge r - \delta > r - \frac{r}{2} = \frac{r}{2}
    \end{align*}
    Y además, si $z \in \com \backslash sop(\gamma)$ y $z \in \Delta(z_0,\delta)$, entonces
    \begin{align*}
        |n(\gamma,z) - n(\gamma,z_0)| &= \left| \frac{1}{2\pi i}\int_{\gamma}{\frac{1}{\xi - z} \ d\xi} - \frac{1}{2\pi i}\int_{\gamma}{\frac{1}{\xi - z_0} \ d\xi} \right| = \left| \frac{1}{2\pi i}\int_{\gamma}{\frac{1}{\xi - z} - \frac{1}{\xi - z_0} \ d\xi}  \right| \\
        &= \frac{1}{2\pi}\left| \int_{\gamma}{\frac{z - z_0}{(\xi - z)(\xi - z_0)} \ d\xi}\right| \leq \frac{1}{2\pi} long(\gamma) \max_{\xi \in sop(\gamma)}\frac{|z-z_0|}{|\xi -z||\xi - z_0|} \\
        & \leq \frac{long(\gamma)}{\pi r^2}\delta < \varepsilon,
    \end{align*}
    lo que prueba que $n(\gamma, \bullet)$ es una función continua en $\com \backslash sop(\gamma)$.
    \item[(iii)] Teenemos que $sop(\gamma)$ es un compacto en $\com$, luego existe $R > 0$ tal que $sop(\gamma) \subset \delta(0,R)$. Sea $z \not \in \overline{\Delta(0,R)}$ ($|z| > R$). Entonces
    \begin{align*}
        |n(\gamma,z)| &= \left| \frac{1}{2\pi i} \int_{\gamma}{\frac{1}{\xi - z} \ d\xi} \right| \leq \frac{long(\gamma)}{2\pi} \max_{\xi \in sop(\gamma)}\frac{1}{|\xi - z|} \\
        & \leq \frac{long(\gamma)}{2\pi} \frac{1}{d(z,sop(\gamma))} \xrightarrow[z \to \infty]{} 0
    \end{align*}
    Esto prueba que $|n(\gamma,z) \xrightarrow[z \to \infty]{} 0$. Por tanto, dado $\varepsilon > 0$, existe $R_0$ tal que si $|z| > R_0$, entonces $|n(\gamma,z)| < \frac{1}{2}$. Pero $n(\gamma,z) \in \mathbb{Z}$, luego $n(\gamma,z) = 0$ si $|z| > R_0$. Como  $n(\gamma, \bullet)$ es una función continua en $\com \backslash sop(\gamma)$, se tiene que $n(\gamma,z) = 0$ para todo $z$ en la componente conexa no acotada de $\com \backslash sop(\gamma)$.
\end{enumerate}
\end{proof}

\section{Teorema de Cauchy para dominios convexos}

\begin{defi}
Decimos que $S \subseteq \com$ es un conjunto convexo si para cualesquiera $z_1,z_2 \in S$ se tiene que $[z_1,z_2] \subset S$.
\end{defi}

\begin{defi}
Definimos
\begin{itemize}
    \item Triángulo $T$ de vértices $z_1,z_2,z_3 \in \com$ como
    \begin{align*}
        T = \overline{co}\{z_1,z_2,z_2\} = \{ t_1z_2 + t_2z_2 + t_3z_3 : t_1,t_2,t_3 \in [0,1], t_1 + t_2 + t_3 = 1\}
    \end{align*}
    \item Frontera del triángulo $T$ de vértices $z_1,z_2,z_3 \in \com$ como
    \begin{align*}
        \partial T = [z_1,z_2,z_3] = [z_1,z_2] + [z_2,z_3] + [z_3,z_1]
    \end{align*}
\end{itemize}
\end{defi}

\begin{teo}[Teorema de Cauchy para triángulos]
Sea $\Omega$ un abierto de $\com$ y sea $T$ un triángulo en $\Omega$. Sea $f: \Omega \longrightarrow \com$ una función continua en $\Omega$ y holomorfa en $\Omega \backslash \{p\}$ siendo $p \in \Omega$. Entonces
\begin{align*}
    \int_{\partial T}{f(z) \ dz} = 0
\end{align*}
\end{teo}

\begin{teo}[Teorema de Cauchy para dominios convexos]
Sea $D$ un dominio convexo en $\com$. Sea $p \in D$ y $f: D \longrightarrow \com$ continua en $D$ y holomorfa en $D \backslash \{p\}$. Entonces
\begin{align*}
    \int_{\gamma}{f(z) \ dz} = 0
\end{align*}
para todo camino cerrado $\gamma$ en $D$.
\end{teo}

\begin{proof}
Basta probar que $f$ tiene primitiva en $D$.
\\
\newline
Fijamos $z_0 \in D$. Como $[z_0,z] \subset D$ (pues $D$ es convexo), definimos
\begin{align*}
    F(z) = \int_{[z_0,z]}{f(\xi) \ d\xi}
\end{align*}
Vamos a probar que $F$ es holomorfa en $D$ y que $F' = f$ en $D$. Para ellos, hemos de probar que fijado $z_1 \in D$ se tiene que
\begin{align*}
    \lim_{z \to z_1} \frac{F(z) - F(z_1)}{z - z_1} - f(z_1) = 0
\end{align*}
Observamos que si $z \in D$, entonces el triángulo $T = \overline{co}\{z_0,z_1,z\}$ está en $D$, luego por el teorema de Cauchy para triángulos
\begin{align*}
    0 = \int_{\partial T}{f(z) \ dz} &= \int_{[z_0,z_1]}{f(\xi) \ d\xi} + \int_{[z_1,z]}{f(\xi) \ d\xi} + \int_{[z,z_0]}{f(\xi) \ d\xi} \\
    &= F(z_1) + \int_{[z_1,z]}{f(\xi) \ d\xi}- F(z)
\end{align*}
Luego
\begin{align*}
    \left| \frac{F(z) - F(z_1)}{z - z_1} - f(z_1) \right| &= \left| \frac{\int_{[z_1,z]}{f(\xi) \ d\xi}}{z-z_1} - \frac{\int_{[z_1,z]}{f(z_1) \ d\xi}}{z-z_1} \right| = \left| \frac{1}{z-z_1}\int_{[z_1,z]}{f(\xi) - f(z_1) \ d\xi}\right| \\
    & \leq \frac{long([z_1,z])}{|z-z_1|} \max_{\xi \in [z_1,z]}(f(\xi) - f(z_1)) = \max_{\xi \in [z_1,z]}(f(\xi) - f(z_1)) \xrightarrow[z \to z_1]{} 0 
\end{align*}
\end{proof}

\begin{obs}
\begin{enumerate}
    \item La conclusión del teorema de Cauchy para dominios convexos también es que $f$ tiene primitiva en $D$.
    \item La hipótesis de que $D$ sea convexo se puede debilitar, por ejemplo, que $D$ sea estrellado con respecto a un punto $z_0 \in D$. En general, el teorema de Cauchy es cierto si $D$ es simplemente conexo.
    \item El teorema de Cauchy no es cierto sobre dominios cualesquiera. Por ejemplo, $f(z) = \frac{1}{z}$ es holomorfa en $\comz$ y $f$ no tiene primitiva en $\comz$.
    \item \underline{Existencia de conjugada armónica en dominios convexos}: Si $D$ es un dominio convexo y $u´: D \longrightarrow \mathbb{R}$ es armónica en $D$, entonces $u$ tiene conjugada armónica en $D$.
    \begin{proof}
    Consideramos $f(z) = u_x(z) - iu_y(z)$. Sabemos que $f$ es holomorfa en $D$ y por el teorema de Cauchy, $f$ tiene primitiva en $D$. Sea $F$ una función holomorfa en $D$ tal que $F' = f$ en $D$. Sea $U = \re(F)$ y $V = \im(F)$. Por Cauchy-Riemann, tenemos que
    \begin{align*}
        \left\{ \begin{array}{lcc}
            U_x = V_y \\
            U_y = -V_x\\
             \end{array}
        \right.
    \end{align*}
    en $D$. Observamos que $F = U + iV$. Por tanto
    \begin{align*}
        U_x - iU_y = F' = f = u_x - iu_y
    \end{align*}
    lo que nos dice que
    \begin{align*}
        \left\{ \begin{array}{lcc}
            U_x = u_x \\
            U_y = u_y\\
             \end{array}
        \right.
    \end{align*}
    en $D$. Por tanto, $U = u + \alpha$ en $D$, $\alpha \in \mathbb{R}$. O sea, $u = U - \alpha = \re(F) - \alpha = \re(F - \alpha)$.
    \end{proof}
\end{enumerate}
\end{obs}

\begin{teo}[Fórmula integral de Cauchy para dominios convexos]
Sea $f$ una función holomorfa en un dominio convexo $D \subseteq \com$. Sea $\gamma$ un camino cerrado en $D$. Entonces
\begin{align*}
    f(z)n(\gamma,z) = \frac{1}{2\pi i}\int_{\gamma}{\frac{f(\xi)}{\xi - z} \ d\xi}
\end{align*}
para todo $z \in D \backslash sop(\gamma)$.
\end{teo}

\begin{proof}
Sea $z \in D \backslash sop(\gamma)$. Consideramos 
\begin{align*}
    g(\xi) = \left\{ \begin{array}{lcc}
             \frac{f(\xi) - f(z)}{\xi - z} &  si  & \xi \in D \backslash \{z\}\\
             f'(z) &  si & \xi = z \\
             \end{array}
   \right.
\end{align*}
Observamos que $g$ es continua en $D$ y holomorfa en $D$ salvo en quizás en $z$. Por el teorema de Cauchy para dominios convexos:
\begin{align*}
    0 = \int_{\gamma}{g(\xi) \ d\xi} = \int_{\gamma}{\frac{f(\xi) - f(z)}{\xi - z} \ d\xi} = \int_{\gamma}{\frac{f(\xi)}{\xi - z} \ d\xi} - \int_{\gamma}{\frac{f(z)}{\xi - z} \ d\xi} 
\end{align*}
de donde deducimos que
\begin{align*}
    \int_{\gamma}{\frac{f(\xi)}{\xi - z} \ d\xi} = f(z)\int_{\gamma}{\frac{1}{\xi - z} \ d\xi} = f(z)n(\gamma,z)2\pi i
\end{align*}
\end{proof}

\begin{teo}[Propiedad del valor medio]
Sea $f$ una función holomorfa en un abierto $\Omega \subseteq \com$ y sean $a \in \Omega$ y $R > 0$ tales que $\Delta(a,R) \subset \Omega$. Entonces:
\begin{enumerate}
    \item[(i)] Propiedad del valor medio para circunferencias: Para cada $0 \leq r < R$ se tiene que
    \begin{align*}
        f(a) = \frac{1}{2\pi}\int_{0}^{2\pi}{f\left(a + re^{it}\right) \ dt}
    \end{align*}
    \item[(ii)] Propiedad del valor medio para discos: Para cada $0 < r < R$ se tiene que
    \begin{align*}
        f(a) = \frac{1}{\pi r^2}\int_{\Delta(a,r)}{f(\xi) \ dA(\xi)}
    \end{align*}
\end{enumerate}
\end{teo}

\begin{obs}
$\xi = x + iy$, entonces $dA(\xi) = dxdy$.
\end{obs}

\begin{cor}[Propiedad del valor medio para funciones armónicas]
Sea $u$ una función armónica en un abierto $\Omega \subseteq \com$ y sean $a \in \Omega$ y $R > 0$ tales que $\Delta(a,R) \subset \Omega$. Entonces:
\begin{enumerate}
    \item[(i)] Propiedad del valor medio para circunferencias: Para cada $0 \leq r < R$ se tiene que
    \begin{align*}
        u(a) = \frac{1}{2\pi}\int_{0}^{2\pi}{u\left(a + re^{it}\right) \ dt}
    \end{align*}
    \item[(ii)] Propiedad del valor medio para discos: Para cada $0 < r < R$ se tiene que
    \begin{align*}
        u(a) = \frac{1}{\pi r^2}\int_{\Delta(a,r)}{u(\xi) \ dA(\xi)}
    \end{align*}
\end{enumerate}
\end{cor}

\begin{teo}[Forma débil del principio del módulo máximo]
Sea $f$ una función holomorfa en un abierto $\Omega \subseteq \com$. Si $|f|$ alcanza un máximo local en $a \in \Omega$, entonces $f$ es constante en un entorno de $a$.
\end{teo}

\begin{proof}
Sea $R > 0$ tal que $\Delta(a,R) \subset \Omega$  y además, tal que $|f(z)| \leq |f(a)|$ para todo $z \ in \Delta(a,R)$. Entonces para cada $r \in (o,R)$, por el teorema del valor medio para discos:
\begin{align*}
    |f(a)| &= \left| \frac{1}{\pi r^2} \int_{\Delta(a,r)}{f(z) \ dA(z)} \right| \leq \frac{1}{\pi r^2} \int_{\Delta(a,r)}{|f(z)| \ dz} \\
    & \leq  \frac{1}{\pi r^2} \int_{\Delta(a,r)}{|f(a)| \ dz} = |f(a)|
\end{align*}
Así, las desigualdades anteriores, son en realidad, igualdades, por tanto, 
\begin{align*}
    |f(a)| = \frac{1}{\pi r^2} \int_{\Delta(a,r)}{|f(z)| \ dz}
\end{align*}
Luego, como $|f|$ es continua, obtenemos que $|f| = |f(a)|$ en $\Delta(a,R)$. Recordemos además que si $f$ es holomorfa y $|f|$ es constante en un entorno de $a$, entonces $f$ es constante en dicho entorno.
\end{proof}

\begin{teo}[Forma débil del principio del módulo mínimo]
Sea $f$ una función holomorfa en un abierto $\Omega \subseteq \com$, y tal que $f(z) \not = 0$ para todo $z \in \Omega$. Si $|f|$ alcanza un mínio local en $a \in \Omega$, entonces $f$ es constante en un entorno de $a$.
\end{teo}

\begin{proof}
Basta observar que $\frac{1}{f}$ es una función holomorfa en $\Omega$ y que $\frac{1}{|f|}$ alcanza un máximo local en $a \in \Omega$. Solo hay que aplicar la forma débil del principio del módulo máximo para obtener el resultado del teorema.
\end{proof}

\begin{teo}[Forma débil del principio del máximo y del mínimo para funciones armónicas]
Sea $u$ una función armónica en un abierto $\Omega \subseteq \com$. Entonces:
\begin{enumerate}
    \item[(i)] Si $u$ alcanza un máximo local en $a \in \Omega$, entonces $u$ es constante en un entorno de $a$.
    \item[(ii)] Si $u$ alcanza un mínimo local en $a \in \Omega$, entonces $u$ es constante en un entorno de $a$.
\end{enumerate}
\end{teo}

\newpage

\section{Analiticidad de las funciones holomorfas}

\begin{teo}[Diferenciación bajo el signo integral]
Sea $\Omega$ un abierto de $\com$ y sea $\gamma$ un camino en $\com$. Supongamos que $h: sop(\gamma) \times \Omega \longrightarrow \com$ es una función tal que:
\begin{enumerate}
    \item[a)] $h$ es continua en $sop(\gamma) \times \Omega$.
    \item[b)] Para cada $\xi \in sop(\gamma)$, la función $h_{\xi}: \Omega \longrightarrow \com$ dada por $h_{\xi}(z) = h(\xi,z)$ es holomorfa en $\Omega$.
    \item[c)] La función $H : sop(\gamma) \times \Omega \longrightarrow \com$ dada por
    \begin{align*}
        H(\xi,z) = (h_{\xi})'(z) = \frac{\partial h}{\partial z}(\xi,z)
    \end{align*}
    es continua en $sop(\gamma) \times \Omega$.
\end{enumerate}
Entonces, la función $F(z) = \int_{\gamma}{h_{\xi}(z) \ d\xi}$, $z \in \Omega$, es holomorfa en $\Omega$ y
\begin{align*}
    F'(z) = \int_{\gamma}{(h_{\xi})'(z) \ d\xi}
\end{align*}
\end{teo}

\begin{teo}[Analiticidad de la integral de Cauchy]
Sea $\gamma$ un camino sobre $\com$ y sea $\varphi$ una función continua en $sop(\gamma)$. Consideremos la función
\begin{align*}
    F : \com \backslash sop(\gamma) \longrightarrow \com, \ \ F(z) = \int_{\gamma}{\frac{\varphi(\xi)}{\xi - z} \ d\xi}
\end{align*}
Entonces $F$, conocida como la integral de Cauchy de $\varphi$ sobre $\gamma$, está bien definida y es análitica en $\com \backslash sop(\gamma)$, o sea, es desarrollable en serie de potencias alrededor de cualquier punto de $\com \backslash sop(\gamma)$. Esto implica en $F$ es infinitamente derivable en $\com \backslash sop(\gamma)$.
\\
\newline
Además, para cada $n \in \mathbb{N}$ se tiene que
\begin{align*}
    F^{(n)}(a) = n! \int_{\gamma}{\frac{\varphi(\xi)}{(\xi - a)^{n+1}} \ d\xi}
\end{align*}
para todo $a \in \com \backslash sop(\gamma)$.
\end{teo}

\begin{proof}
$F$ está bien definida en $\com \backslash sop(\gamma)$. Sea $a \not \in sop(\gama)$. Sea $R > 0$ tal que $\Delta(a,R) \cap sop(\gamma) = \emptyset$. Sea $z \in \Delta(a,R)$ arbitrario, pero fijo. Observamos que si $\left| \frac{z-a}{\xi - a} \right| < 1$ tenemos que
\begin{align*}
    \frac{1}{\xi - z} &= \frac{1}{(\xi - a) - (z - a)} = \frac{1}{\xi - a} \cdot \frac{1}{1 - \frac{z-a}{\xi - a}} \\
    &= \frac{1}{\xi - a} \sum_{n=0}^{\infty}{\left( \frac{z-a}{\xi -a}\right)^n} = \sum_{n=0}^{\infty}{\frac{(z-a)^n}{(\xi -a)^{n+1}}}
\end{align*}
siendo la convergencia de la serie absoluta y uniforme en cada subconjunto compacto de $A = \left\{ \xi \in \com : \left| \frac{z-a}{\xi - a} \right| < 1 \right\}$. En particular, $sop(\gamma) \subset A$ y es compacto, como además $\varphi$ es contina sobre $sop(\gamma)$ tenemos que
\begin{align*}
    F(z) &= \int_{\gamma}{\frac{\varphi(\xi)}{\xi - z} \ d\xi} = \int_{\gamma}{\sum_{n=0}^{\infty}{\frac{\varphi(\xi)}{(\xi - a)^{n}}}(z-a)^{n} \ d\xi} \\
    &= {\sum_{n=0}^{\infty}{\left[\int_{\gamma}\frac{\varphi(\xi)}{(\xi - a)^{n}} \ d\xi \right]}(z-a)^{n} }
\end{align*}
Tomando $\{a_n\} = \{\int_{\gamma}\frac{\varphi(\xi)}{(\xi - a)^{n}} \ d\xi \} $, tenemos una expresión válida para cda $z \in \Delta(a,R)$, por lo que concluimos que $F$ es desarrollable e serie de potencias alrededor de $a$ con radio de convergencia al menos $dist(a,sop(\gamma))$.
\\
\newline
Como esta serie debe coincidir con la serie de taylor de $F$ centrada en $a$, tenemos que
\begin{align*}
    F^{(n)}(a) = n! \cdot a_n = n!\int_{\gamma}\frac{\varphi(\xi)}{(\xi - a)^{n}} \ d\xi 
\end{align*}
\end{proof}

\begin{teo}[Analiticidad de las funciones holomorfas]
Sea $f$ holomorfa en un abierto $\Omega \subseteq \com$. Entonces $f$ es analítica en $\Omega$. Además, para cada $a \in \Omega$, el desarrollo en serie de potencias de $f$ en $a$ tiene radio de convergencia $R = dist(a, \com \backslash \Omega)$.
\end{teo}

\begin{proof}
Sea $a \in \Omega$ y sea $R = dist(a, \com \backslash \Omega)$. Sea $C_r = \{ |\xi - a| = r \}$, para $r \in (0,R)$. Como $f$ es holomorfa en $\Delta(a,R)$, que es convexo, podemos aplicar la fórmula de la integral de Cauchy, con lo que tenemos que:
\begin{align*}
    f(z)n(C_r,z) = \frac{1}{2\pi i}\int_{C_r}{\frac{f(\xi)}{\xi - z} \ d\xi},
\end{align*}
para todo $z \in \Delta(a,R)$. En particular, si $z \in \Delta(a,r)$, tenemos que $n(C_r,z) = 1$ y por tanto
\begin{align*}
    f(z) = \frac{1}{2\pi i}\int_{C_r}{\frac{f(\xi)}{\xi - z} \ d\xi},
\end{align*}
lo que nos dice que $f$ concide en $\Delta(a,r)$ con la integral de cauchy de la función $\varphi = \frac{1}{2\pi i}f|_{C_r}$ a lo largo de $C_r$. De aquí se sigue que $f$ es análitica en $\Delta(a,r)$, y en particular, en $a$, y así
\begin{align*}
    f(z) = \sum_{n=0}^{\infty}\left( \frac{f(\xi)}{(\xi - z)^{n+1}} \ d\xi \right)(z-a)^n,
\end{align*}
para todo $z \in \Delta(a,r)$. Además, esta serie ha de coincidir con la serie de taylor de $f$ en $a$, o sea que para $n \in \mathbb{N}$, $a_n = \frac{f^{(n)}(a)}{n!}$ no cambia de valor por mucho que cambie el valor de $r \in (o,R)$, lo que nos dice que el radio de convergencia de la serie anterior es $R$.
\end{proof}

\begin{obs}
Si $f$ es holomorfa en $\Omega$ y $\Delta(a,R) \subset \Omega$, entonces
\begin{align*}
    f^{(n)}(a) = \frac{n!}{2\pi i} \int_{|\xi -a| = r} \frac{f(\xi)}{(\xi - a)^{n+1}} \ d\xi
\end{align*}
para todo $r \in (0,R)$ y todo $z \in \Delta(a,r)$.
\end{obs}

\begin{teo}[Fórmula integral de la derivada $n$-ésima en dominios convexos]
Sea $D$ un dominio convexo y sea $f$ una función holomorfa en $D$. Sea $\gamma$ un camino cerrado en $D$. Entonces, para cada $z \in D \backslash sop(\gamma)$ se tiene que:
\begin{align*}
    f^{(n)}(z)n(\gamma,z) = \frac{n!}{2\pi i} \int_{\gamma} \frac{f(\xi)}{(\xi - z)^{n+1}} \ d\xi
\end{align*}
\end{teo}

\begin{proof}
Por la fórmula de la integral de Cauchy en dominios convexos, tenemos que
\begin{align*}
    f(z)n(\gamma,z) = \frac{1}{2\pi i} \int_{\gamma} \frac{f(\xi)}{\xi - z} \ d\xi
\end{align*}
Observamos que $F(z) = \frac{1}{2\pi i} \int_{\gamma} \frac{f(\xi)}{\xi - z} \ d\xi$ es anlítica en $D \backslash sop(\gamma)$. Derivando:
\begin{align*}
    F^{(n)}(z) = \frac{n!}{2\pi i} \int_{\gamma} \frac{f(\xi)}{(\xi - z)^{n+1}} \ d\xi
\end{align*}
para todo $z \in D \backslash sop(\gamma)$. Esto nos dice que el lado izquierdo de la igualdad también es analítico en $D \backslash sop(\gamma)$. Como $n(\gamma,z)$ es una función a trozos tenemos que la derivada $n$-ésima del lado izquierdo de la igualdad es $f^{(n)}(z)n(\gamma,z)$, y
por tanto,
\begin{align*}
    f^{(n)}(z)n(\gamma,z) = \frac{n!}{2\pi i} \int_{\gamma} \frac{f(\xi)}{(\xi - z)^{n+1}} \ d\xi
\end{align*}
\end{proof}

\begin{ejemplo}
\begin{enumerate}
    \item Sea $f(z) = \sen z$. Observamos que esta función es holomorfa en $\mathbb{D}$.
    \begin{align*}
        \frac{1}{2\pi i} \int_{|z| = 1} \frac{\sen z}{z^3} \ d\xi &= \frac{1}{2\pi i}\int_{|\xi| = 1} \frac{\sen \xi}{(\xi - 0)^3} \ d\xi = \frac{1}{2!} \cdot \frac{2!}{2\pi i}\int_{|\xi| = 1} \frac{\sen \xi}{(\xi - 0)^3} \ d\xi =  \frac{1}{2!} \cdot f''(0) = 0
    \end{align*}
    \item Sea $0 < r < 1$,
    \begin{align*}
        \frac{1}{2\pi i}\int_{|z| = 2} \frac{1}{z^2(z^2 + 4)} \ dz = \frac{1}{2\pi i}\int_{|z| = r} \frac{\frac{1}{z^2+4}}{(z-0)^2} \ dz
    \end{align*}
    Observamos que $f(z) = \frac{1}{z^2 + 4}$ es holomorfa en $\mathbb{D}$ y $f'(z) = \frac{-2z}{(z^2+4)^2}$, por tanto
    \begin{align*}
        \frac{1}{2\pi i}\int_{|z| = r} \frac{\frac{1}{z^2+4}}{(z-0)^2} \ dz = f'(0) = 0
    \end{align*}
\end{enumerate}
\end{ejemplo}

\section{Consecuencias de la analiticidad}

\begin{teo}
Sea $\Omega$ abierto de $\com$ y sea $f: \Omega \longrightarrow \com$ continua en $\Omega$ y holomorfa en $\Omega \backslash \{p\}$, siendo $p \in \Omega$. Entonces $f$ es holomorfa en $\Omega$.
\end{teo}

\begin{proof}
Basta demostrar que $f$ es holomorfa en $p$. Como $\Omega$ es abierto, existe $R > 0$ tal que $\Delta(p,R) \subset \Omega$. Por el teorema de Cauchy para dominios convexos, tenemos que $\int_{\gamma}{f(z) \ dz} = 0$ para todo camino cerrado $\gamma$ en $\Delta(p,R)$. Esto equivale a que $f$ tiene primitiva en $\Delta(a,R)$, o sea, existe $F$ holomorfa en $\Delta(p,R)$ tal que $F' = f$ en $\Delta(p,R)$. Como $F$ es holomorfa en $\Delta(p,R)$, entonces es analítica en $\Delta(p,R)$ y por tanto, $F' = f$ es holomorfa en $\Delta(p,R)$.
\end{proof}

\begin{teo}
Sea $\Omega$ abierto de $\com$ y $f: \Omega \longrightarrow \com$ holomorfa. Sean $u = \re(f)$ y $v = \im(f)$. Entonces $u$ y $v$ son armónicas en $\Omega$ y de clase $\mathscr{C}^{\infty}(\Omega)$.
\end{teo}

\begin{proof}
Como $f$ es holomorfa, entonces $f \in \mathscr{C}^{\infty}(\Omega)$, lo que nos dice que $u$ y $v$ son armónicas en $\Omega$ y por ser, $f \in \mathscr{C}^{\infty}(\Omega)$, se tiene que $u,v \in \mathscr{C}^{\infty}$ en $\Omega$.
\end{proof}

\begin{cor}
Si $u$ es armónica en un abierto $\Omega \subseteq \com$, entonces $u \in \mathscr{C}^{\infty}(\Omega)$.
\end{cor}

\begin{proof}
Como $u$ es armónica en $\Omega$, entonces $u$ es la parte real de una función holomorfa, por tanto, $u \in \mathscr{C}^{\infty}(\Omega)$.
\end{proof}

\begin{teo}[de Morera]
Sea $\Omega \subseteq \com$ abierto y $f: \Omega \longrightarrow \com$ continua en $\Omega$. Supongamos que $\int_{\gamma}{f(z) \ dz} = 0$ para todo camino cerrado $\gamma$ de $\Omega$. Entonces $f$ es holomorfa en $\Omega$.
\end{teo}

\begin{proof}
Fijamos $a \in \Omega$ y $R > 0$ tal que $\Delta(a,R) \subset \Omega$. Las hipótesis del teorema en $\Delta(a,R)$ implican que $f$ tiene primitiva en $\Delta(a,R)$, es decir, existe $F$ holomorfa en $\Delta(a,R)$ tal que $F' = f$ en $\Delta(a,R)$, por tanto, $f$ es holomorfa en $\Delta(a,R)$.
\end{proof}

\begin{teo}[de Morera para triángulos]
Sea $\Omega \subseteq \com$ abierto y $f: \Omega \longrightarrow \com$ continua en $\Omega$. Supongamos que $\int_{\partial T}{f(z) \ dz} = 0$ siempre que $T$ sea un triángulo (sólido) enn $\Omega$. Entonces $f$ es holomorfa en $\Omega$.
\end{teo}

\begin{proof}
Fijamos $a \in \Omega$ y $R > 0$ tal que $\Delta(a,R) \subset \Omega$. Definimos
\begin{align*}
    F(z) = \int_{[a,z]}{f(\xi) \ d\xi}
\end{align*}
Observamos que $F$ está bien ndefinida y, imitando la demostración del teorema de Cauchy para triángulos, tenemos que $F$ es una primitiva de $f$ en $\Delta(a,R)$. Por tanto, $F' = f$ es holomorfa en $\Delta(a,R)$.
\end{proof}

\begin{teo}[de Liouville]
Si $f$ es entera y acotada, entonces $f$ es constante.
\end{teo}

\begin{proof}
Sea $M$ tal que $|f(z)| < M$ para todo $z \in \com$. Sea $a \in \com$. Por la fórmula intergal de Cauchy, la primera derivada de $f$ en $a$ es
\begin{align*}
    f'(a) = \frac{1}{2\pi i}\int_{|z-a| = R}{\frac{f(z)}{(z-a)^2} \ dz}
\end{align*}
Tomando módulos
\begin{align*}
    |f'(a)| &= \left| \frac{1}{2\pi i}\int_{|z-a| = R}{\frac{f(z)}{(z-a)^2} \ dz} \right|  \leq \frac{1}{2\pi} long(|z-a| = R) \cdot \max_{|z-a| = R} \left| \frac{f(z)}{(z-a)^2} \right| \\
    & \leq \frac{2\pi R}{2\pi} \cdot \frac{M}{R^2} = \frac{M}{R} \xrightarrow[R \to \infty]{} 0
\end{align*}
Como $f'(a)$ no depende de $R$, se tiene entonces que $f'(a) = 0$.
\end{proof}

\begin{teo}[Teorema Fundamental del Álgebra]
Todo polinomio con coeficientes complejos no constante tiene una raíz.
\end{teo}

\begin{proof}
Sea $P$ un polinomio no constante, entoces $\lim_{z \to \infty}{|P(z)| = \infty}$. Por reducción al absurdo, supongamos que $P$ no tiene raíces, entonces podemos considerar $f(z) = \frac{1}{P(z)}$, $z \in \com$ que es una función entera y acotada ($\lim_{z \to \infty}{f(z)} = 0$). Por el teorema de Liouville, se tiene que $f$ es constante y, por tanto, $P$ es constante, lo que es una contradicción, pues suponíamos que $P$ no era constante. Luego, $P$ tiene una raíz.
\end{proof}

\begin{teo}[de Liouville]
Si $f$ es entera y no constante, entonces $f(\com)$ es denso en $\com$.
\end{teo}

\begin{proof}
Por reducción al absurdo, supongamos que $f(\com)$ no es denso en $\com$, o sea, existen $w_0 \in \com$ y $r_0 > 0$ tales que $\Delta(w_0,r_0) \cap f(\com) = \emptyset$. Esto nos dice que $|f(z) - w_0| \ge r_0$ para todo $z \in \com$, por tanto, $1 \ge \left| \frac{r_0}{f(z) - w_0} \right|$. Consideramos $g(z) = \frac{r_0}{f(z) - w_0}$, $z \in \com$, que es una función entera, acotada y nunca cero. Aplicando el teorema de Liouville, tenemos que $g$ es una constante (y no nula). Esto implica que $f(z) = \frac{r_0}{g(z)} + w_0$ es constante en $\com$, lo que es una contradicción, luego $f(\com)$ es denso en $\com$.
\end{proof}

\begin{obs}
Según este resultado, la imagen de una función entera no constante no puede omitir un disco, mucho menos un semiplano, pero ¿puede omitir una semirrecta? ¿qué pasa si $f$ es entera y $f(\com) \subset \com (-\infty,0]$? Una generalización del Teorema de Liouville nos dice que si una función entera se comporta como un polinomio en el infinito es que entonces es un polinomio.
\end{obs}

\begin{teo}[de Liouville]
Si $f$ es entera y existen $M > 0$, $\alpha \ge 0$ y $R > 0$ tales que $|f(z)| \leq M|z|^{\alpha}$ para todo $|z| > R$, entonces $f$ es un polinomio de grado a lo sumo $\alpha$.
\end{teo}

\begin{proof}
Como $f$ es entera, entonces $f$ es analítica y por tanto, $f$ es desarrollable en serie de potencias centrada en 0 y con radio de convergencia $\infty$. Sea
\begin{align*}
   f(z) = \sum_{n=0}^{\infty}{a_nz^n} 
\end{align*}
dicho desarrollo para cada $z \in \com$. Por la fórmula integral de Cauchy para la $n$-éseima derivada, nos dice que:
\begin{align*}
   \left| \frac{f^{(n)}(0)}{n!} \right| &= |a_n| = \left| \frac{1}{2\pi i}\int_{|z| = R}{\frac{f(z)}{z^{n+1}} \ dz}  \dz\right| \leq \frac{1}{2\pi}long(|z| = R) \cdot \max_{|z| = R} \left| \frac{f(z)}{z^{n+1}} \right| \\
   &\leq \frac{2\pi R}{2\pi} \cdot \frac{MR^{\alpha}}{R^{n +1}} = M \cdot R^{\alpha - n} \xrightarrow[R \to \infty]{} 0
\end{align*}
 para $n > \alpha$. O sea, $a_n = 0$ si $n > \alpha$, luego, $f(z) = \sum_{n\leq \alpha}{a_nz^n}$, que es un polinomio de grado a lo sumo $\alpha$.
\end{proof}

\begin{teo}[de Liouville]
Si $f$ es entera y existen $M > 0$, $\alpha \ge 0$ y una sucesión $\{R_k\} \subset \mathbb{R}$ creciente con $\lim_{k \to \infty}{R_k} = \infty$ y tales que $|f(z)| \leq M|z|^{\alpha}$ para $|z| = R_k$. Entonces $f$ es un polinomio de grado a lo sumo $\alpha$.
\end{teo}

\section{Sucesiones de funciones holomorfas}

\begin{defi}
Sea $\{f_n\}$ una sucesión de funciones holomorfas en un dominio $D \subseteq \com$ y sea $f: D \longrightarrow \com$ una función. Decimos que $\{f_n\}$ converge uniformemente en subconjuntos compactos de $D$ (o que converge normalmente en $D$) si para cada compacto $K \subset D$, se tiene que $f_n \xrightarrow[n \to \infty]{} f$ de manera uniforme, o sea, para $\varepsilon > 0$, existe $N_{K,\varepsilon} \in \mathbb{N}$ tal que $|f_n(z) - f(z)| < \varepsilon$ siempre que $z \in K$ y $n \ge N_{K,\varepsilon}$. 
\end{defi}

\begin{obs}
Si $\{f_n\}$ es una sucesión de funciones holomorfas que converge uniformemente a $f$ en $D$ entonces $f$ es continua en $D$.
\end{obs}

\begin{lema}
Sea $D$ un dominio en $\com$ y sean $f,f_n$ ($n \in \mathbb{N}$) funciones de $D$ en $\com$. Son equivalentes:
\begin{enumerate}
    \item Convergencia uniforme en compactos.
    \item Convergencia local uniforme. Para cada $a \in D$, existe $R > 0$ tal que $\Delta(a,R) \subset D$ y $f_n \xrightarrow[n \to \infty]{} f$ de manera uniforme en $\Delta(a,R)$
\end{enumerate}
\end{lema}

\begin{teo}[Teorema de Convergencia de Weierstrass]
Sea $\{f_n\}$ una sucesión de funciones holomorfas en un dominio $D \subseteq \com$ que converge uniformemente en compactos de $D$ a una función $f: D \longrightarrow \com$. Entonces $f$ es holomorfa en $D$. Es más, la sucesión $\{f_n^{(m)}\}$ de las derivadas $m$-ésimas converge uniformemente en compactos de $D$ a $f^{(m)}$.
\end{teo}

\begin{proof}
Es claro que $f$ es continua en $D$. Fijamos $z_0 \in D$ y $R_0$ tales que $\Delta(z_0,R_0) \subset D$. Probemos que $f$ es holomorfa en $\Delta(z_0,R_0)$, para ello, vamos a utilizar el teorema de Morera. Sea $\gamma$ un camino cerrado en $\Delta(z_0,R_0)$. Entonces:
\begin{align*}
    \int_{\gamma} f(z) \ dz = \int_{\gamma} \lim_{n \to \infty} f_n(z) \ dz = \lim_{n \to \infty} \int_{\gamma} f_n(z) \ dz = 0.
\end{align*}
El igual a 0 se debe a una aplicación directa del teorema de Cauchy para dominios convexos, ya que cada $f_n$ es holomorfa en $\Delta(z_0,R_0)$, que es convexo. Por el teorema de Morera, tenemos que $f$ es holomorfa en $\Delta(z_0,R_0)$.
\\
\newline
Fijamos $m \in \mathbb{N}$. Sea $z_0 \in D$ y sean $r_1 > r_0 > 0$ tales que
$\overline{\Delta(z_0,r_0)} \subset \overline{\Delta(z_0,r_1)} \subset D$. sea $z \in \Delta(z_0,r_0)$, por la fórmula de Cauchy para la derivada $m$-éseima, tenemos que
\begin{align*}
    \left| f_n^{(m)}(z) - f^{(m)}(z) \right| &= \left| \frac{m!}{2\pi i}  \int_{|\xi - z_0| = r_1} \frac{f_n(\xi)}{(\xi - z)^{m+1}} \ d\xi - \frac{m!}{2\pi i} \int_{|\xi - z_0| = r_1} \frac{f(\xi)}{(\xi - z)^{m+1}} \ d\xi \right| \\
    &= \left| \frac{m!}{2\pi i} \int_{|\xi - z_0| = r_1} \frac{f_n(\xi) - f(\xi)}{(\xi - z)^{m+1}} \ d\xi \right| \leq \frac{m!}{2\pi} long(|\xi - z_0| = r_1) \cdot \max_{|\xi - z_0| = r_1} \frac{|f_n(\xi) - f(\xi)|}{|\xi - z|^{m+1}} \\
    & \leq \frac{m! \cdot r_1}{(r_1 - r_0)^{m+1}} \cdot \max_{|\xi - z_0| = r_1} |f_n(\xi) - f(\xi)| 
\end{align*}
El lado derecho tiende a 0 cuando $n \to \infty$, independietemente de $z \in \overline{\Delta(z_0,r_0)}$, luego el lado izquierdo también, concluyendo que $f_n^{(m)} \to f^{(m)}$ de manera uniforme en $\overline{\Delta(z_0,r_0)}$.
\end{proof}

\section{Ramas del logaritmo y de la raíz $n$-ésima}

\begin{teo}[Recopilatorio]
Sea $D$ un dominio en $\com$ y sea $f: D \longrightarrow \com$ holomorfa y nunca nula en $D$.
\begin{enumerate}
    \item Si $g$ es una rama del $\log(f)$ en $D$, entonces cualquier otra rama del $\log(f)$ en $D$ es de la forma $g + 2\pi i$, $k \in \mathbb{Z}$.
    \item Existe una rama del $\log(f)$ en $D$ $\Longleftrightarrow$ $\frac{f'}{f}$ tiene primitiva en $D$ $\Longleftrightarrow$ Para todo camino cerrado $\gamma$ en $D$ se tiene que $\frac{1}{2\pi i}\int_{\gamma} \frac{f'(z)}{f(z)} \ dz = 0$. En este caso, si $G$ es primitiva de $\frac{f'}{f}$ en $D$, entonces existe una constante $\beta \in \com$ tal que $G + \beta$ es rama holomorfa del $\log(f)$ en $D$.
\end{enumerate}
\end{teo}

\begin{obs}
La función $\frac{f'}{f}$ recibe el nombre de \textbf{derivada logarítimica de f}, la cual tiene sentido completo siempre que $f$ sea holomorfa y nunca cero. Tenemos las siguientes reglas:
\begin{align*}
    \frac{(fg)'}{fg} = \frac{f'}{f} + \frac{g'}{g}, \ \ \ \frac{\left(  \frac{f}{g}\right)'}{\frac{f}{g}} = \frac{f'}{f} - \frac{g'}{g}, \ \ \ \frac{(f^N)'}{f^N} = N \frac{f'}{f}.
\end{align*}
\end{obs}

\begin{ejemplo}
Sea $f(z) = \frac{z+1}{z-1}$, que es holomorfa y nunca cero en $D = \com \backslash \{-1,1\}$, ¿existe rama del $\log(f)$ en $D$? Sea $\gamma$ un camino cerrado en $D$, entonces
\begin{align*}
    \frac{1}{2\pi i} \int_{\gamma} \frac{f'(z)}{f(z)} \ dz = \frac{1}{2\pi i} \int_{\gamma} \frac{1}{z+1} - \frac{1}{z-1} \ dz = n(\gamma,1) - n(\gamma,-1),
\end{align*}
que no tiene porqué ser 0, por tanto, no existe rama del $\log(f)$ en $D$.
\\
\newline
Consideramos ahora $D_1 = \com \backslash [-1,1]$, ¿existe rama del $\log(f)$ en $D$? Sea $\gamma$ camino cerrado en $D_1$, entonces -1 y 1 están en la misma componente conexa de $\com \backslash sop(\gamma)$ y 
\begin{align*}
    \frac{1}{2\pi i} \int_{\gamma} \frac{f'(z)}{f(z)} \ dz = n(\gamma,1) - n(\gamma,-1) = 0,
\end{align*}
por tanto, si existe rama del $\log(f)$ en $D_1$.
\end{ejemplo}

\begin{teo}[Recopilatorio]
Sea $n \in \mathbb{N}$, $n \ge 2$. Sea $D$ un dominio de $\com$ y $f:D \longrightarrow \com$ holomorfa y nunca cero en $D$.
\begin{enumerate}
    \item Si $g$ es rama del $\log(f)$ en $D$, entonces $h = e^{\frac{g}{n}}$ es una rama de $\sqrt[n]{f}$ en $D$, y cualquier otra rama de $\sqrt[n]{f}$ en $D$ es de la forma $\xi \cdot h$, siendo $\xi^n = 1$.
    \item Si $h$ es rama de $\sqrt[n]{f}$ en $D$, entonces $h$ es holomorfa en $D$ y $h' = \frac{f'}{nh^{n-1}}$ en $D$.
    \item Si existe una rama de $\sqrt[n]{f}$ en $D$, entonces para todo camino cerrado $\gamma$ en $D$ se tiene que 
    \begin{align*}
        \frac{1}{2\pi i} \int_{\gamma} \frac{f'(z)}{f(z)} \ dz \ \ \ \text{es un múltiplo entero de n}. 
    \end{align*}
\end{enumerate}
\end{teo}

\begin{proof}
Solo tenemos que probar $3$. Sea $h$ una rama de $\sqrt[n]{f}$ en $D$. Entonces
\begin{align*}
    \frac{f'}{f} = \frac{nh^{n-1}h'}{h^{n}} = n\frac{h'}{h}, \ \ \ \ \ z \in D.
\end{align*}
Sea $\gamma$ un camino cerrado en $D$, entonces:
\begin{align*}
    \frac{1}{2\pi i} \int_{\gamma} \frac{f'(z)}{f(z)} \ dz = n\int_{\gamma} \frac{h'(z)}{h(z)} \ dz \underset{w = h(z)}{=} n \int_{h \circ \gamma} \frac{dw}{w} = n \cdot n(h \circ \gamma, 0) \in \mathbb{Z}
\end{align*}
\end{proof}

\begin{obs}
Si $\gamma$ es un camino cerrado en $D$, entonces $h \circ \gamma$ es un camino cerrado en $h(D)$ y definimos
\begin{align*}
    \var_{\gamma}(\arg(h)) = \var_{h \circ \gamma}(\arg(z))
\end{align*}
\end{obs}

\begin{ejemplo}
Sea $f(z) = z^2 -1 = (z-1)(z+1)$, que es holomorfa y nunca cero en $D = \com \backslash \{-1,1\}$, ¿existe rama de $\sqrt[n]{f}$ en $D$? Sea $\gamma$ un camino cerrado en $D$, entonces:
\begin{align*}
    \frac{1}{2\pi i} \int_{\gamma} \frac{f'(z)}{f(z)} \ dz = \frac{1}{2\pi i} \int_{\gamma} \frac{1}{z+1} + \frac{1}{z-1} \ dz = n(\gamma,1) + n(\gamma,-1),
\end{align*}
que puede ser igual a, por ejemplo, 1 (basta tomar $\gamma$ la circunferencia de centro -1 y radio 1), por lo que no existe $\sqrt[n]{f}$ en $D$.
\\
\newline
Sea $D_1 = \com \backslash [-1,1]$, ¿existe rama de $\sqrt[n]{f}$ en $D$? Sea $\gamma$ un camino cerrado en $D_1$, entonces -1 y 1 están en la misma componente conexa de $\com \backslash sop(\gamma)$ y por tanto
\begin{align*}
    \frac{1}{2\pi i} \int_{\gamma} \frac{f'(z)}{f(z)} \ dz = \frac{1}{2\pi i} \int_{\gamma} \frac{1}{z+1} + \frac{1}{z-1} \ dz = n(\gamma,1) + n(\gamma,-1) = 2n(\gamma,1),
\end{align*}
que es un múltiplo entero de $2$. Esto nos dice que hay posibilidades de que exista rama de $\sqrt{f}$ en $D_1$. Observamos que
\begin{align*}
    f(z) = (z-1)(z+1) = (z-1)^2\frac{z+1}{z-1}
\end{align*}
Recordamos que en el anterior ejercicio hemos probado que existe $g$ rama del $\log\left( \frac{z+1}{z-1} \right)$ en $D_1$,por tanto, 
\begin{align*}
    h(z) = (z-1)e^{\frac{g(z)}{2}}
\end{align*}
es rama de $\sqrt{f}$ en $D_1$, ya que es holomorfa en $D_1$ y $h(z)^2 = f(z)$, $z \in D_1$.
\end{ejemplo}
\chapter{Estimación no paramétrica de densidades}

\noindent En la estimación de la densidad, como en la Inferencia en general, existen dos posibles vías de estudio.
\begin{itemize}
    \item \textbf{Estimación paramétrica}, en la que se asume determinada distribución de la variable y se emplean datos para la estimación de los correspondientes parámetros.
    \item \textbf{Estimación no paramétrica}, que no asume ninguna distribución, únicamente utiliza la información proporcionada por la muestra.
\end{itemize}
Tanto en la inferencia paramétrica como la no paramétrica poseen numerosos simpatizantes y detractores, pues ambas metodologías de trabajo tienen ventajas e inconvenientes que han sido ampliamente estudiados a lo largo de los años.
\\
\newline
\noindent La suposicion inicial de que la población de la que proceden los datos sigue un modelo paramétrico puede limitar mucho el ajuste del modelo. En caso deser correcta dicha suposición, el ajuste será muy bueno, pero si el modelo paramétrico es incorrecto, las conclusiones podrían ser totalmente erróneas. Por ello, es deseable considerar técnicas no paramétricas que olviden cualquier hipótesis previa y trabajen únicamente con la información que proporcionan los datos, teniendo siempre presente la aleatoriedad intrínseca a los mismos.
\\
\newline
\noindent Las principios de la estimación no paramétrica de la densidad datan de finales del siglo XIX, cuando Karl Pearson introdujo el \textbf{histograma}, que no es más que la representación de las frecuencias por clases. El histograma es un estimador discontinuo, que además depende de la elección de un punto inicial y de un parámetro ventana, con gran influencia por parte de ambos en el resultado final.
\\
\newline
\noindent Para solventar el problema de la dependencia del punto inicial, hay que esperar hasta mediados del siglo XX, cuando se desarrolló el denominado\textbf{ histograma móvil} o \textbf{estimador naive}, que sigue siendo discontinuo y dependiente de la ventana. Posteriormente, Parzen, en 1962, y Rosenblatt (1956), propusieron el estimador tipo núcleo, que sí es continuo y que, por lo tanto, en la mayor parte de las veces, se ajusta mejor a la realidad de los modelos estudiados, aunque también depende en gran medida de la eleccón de un parámetro ventana.
\\
\newline
\noindent En la literatura estadística ha sido ampliamente estudiado el papel fundamental del parámetro ventana en el estimador tipo núcleo. Dicho parámetro es el que controla el grado de suavización del estimador, y una mala elección del mismo puede derivar en un estimador tanto infra como sobresuavizado. Debido a esto, la segunda mitad del siglo XX fue muy prolífica en cuanto a métodos de seleccioón de ventana, entre los que destacan el propuesto por Silverman (1986), el meétodo de Shealter y Jones (1991) y el de Bowman (1984).

\begin{defi}
    Sea $X_1,\ldots, X_n$ una muestra de datos que proviene de una variable aleatoria $X$ continua con función de densidad desconocida, $f$. El \textbf{estimador de núcleos} de $f$ se define por
    \begin{align*}
        \hat{f}_n(x) = \frac{1}{n} \sum_{i=1}^{n} \frac{1}{h} K\left( \frac{x - X_i}{h}, \right) 
    \end{align*}
    donde $K$ es una función, denominada \textbf{función kernel, función núcleo o función peso}, que satisface ciertas condiciones de regularidad, generalmente es una función de densidad simétrica, con media 0 y varianza 1, $h > 0$ es el \textbf{parámetro de suavizado o ancho de banda}.
\end{defi}
El estimador de núcleos se puede ver como una suma de pequeñas montañas o protuberancias
situadas en las observaciones. La funcion núcleo determina la forma de estas protuberancias,
mientras que el paámetro de suavizado determina su anchura.
\\
\newline
\noindent Cada pequeña montaña o protuberancia está centrada en una observación $X_i$ y tiene una superficie de $1/n$.
\\
\newline
\noindent Si $h$ es demasiado pequeño, las montañas estarán muy separadas y observaremos muchos picos.
\\
\newline
\noindent Si $h$ es demasiado grande, observaremos una sola montaña plana.
\\
\newline
\noindent Un valor intermedio de $h$ debería ser la mejor elección.
\\
\newline
\noindent Algunas de las funciones núcleo más comunes son
\begin{itemize}
    \item \textbf{Epanechnikov}, $K(t) = \frac{3}{4}(1-t^2)$, $|t| < 1$.
    \item \textbf{Gauss}, $K(t) = \frac{1}{2\pi} e^{-\frac{t^2}{2}}$, $t \in \mathbb{R}$.
    \item \textbf{Triangular}, $K(t) = 1 - |t|$, $|t| < 1$.
    \item \textbf{Rectangular}, $K(t) = \frac{1}{2}$, $|t| < 1$.
    \item \textbf{Biweight}, $K(t) = \frac{15}{16}(1-t^2)^2$, $|t| < 1$.
    \item \textbf{Triweight}, $K(t) = \frac{35}{32}(1-t^2)^3$, $|t| < 1$.
    \item \textbf{Coseno}, $K(t) = \frac{\pi}{4} \cos \left( \frac{\pi t}{2} \right)$, $|t| < 1$.
    \item \textbf{Semicírculo}, $K(t) = \frac{2}{\pi} \sqrt{1 -  t^2}$, $|t| < 1$.
\end{itemize}

\begin{figure}[H]
    \centering
    \includegraphics[width=1.1\textwidth]{imagenes6/nucleos.png}
\end{figure}
\noindent Apliquemos todo esto a un ejemplo real. Los datos que vamos a analizar fueron analizados en Azzalini y Bowman (1990), (”Applied Smoothing Techniques for Data Analysis”) quienes registraron el tiempo (en minutos) que dura una erupcion´ del geyser Old Faithful que se encuentra en el parque nacional de Yellowstone (Wyoming, USA). Las medidas (27 erupciones en total) fueron tomadas entre el 1 y el 15 de Agosto de 1985.
\begin{figure}[H]
    \centering
    \includegraphics[width=1\textwidth]{imagenes6/nucleo1.png}
    \caption{Ajustando diferentes valores de $h$ con la función núcleo que usa Mathmatica por defecto.}
\end{figure}

\begin{figure}[H]
    \centering
    \includegraphics[width=1\textwidth]{imagenes6/nucleo3.png}
    \caption{Funciónes núcleo vistas anteriormente.}
\end{figure}
\end{document}
