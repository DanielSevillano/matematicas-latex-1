\chapter{Medidas}
\section{Introducción}

\begin{defi}
    Sea $\{f_n\}$ una sucesión de funciones tal que $f_n:[a,b] \longrightarrow \mathbb{R}$. Decimos que la sucesón $\{f_n\}$ converge puntualmente si para todo x $\in$ [a,b] existe el límite de la sucesión $\{f_n(x)\}$. En tal caso, la función  $f:[a,b]\longrightarrow\mathbb{R}$ definida por $f(x) = \displaystyle\lim_{n \to{+}\infty}{f_n(x)}$ se llama límite puntual de la sucesión $\{f_n\}$.
\end{defi}

\begin{ejemplo}
    Demos un ejemplo de una sucesión de funciones $\{f_n\}$ integrables-Riemann en [a,b] y cuyo límite puntual no sea integrable-Riemann en [a,b].
\end{ejemplo}
Sabemos que los números racionales son numerables, luego exite una sucesión de números naturales tal que $[0,1]\cap\mathbb{Q} = \{ r_n : n\in\mathbb{N}\}$.
Definimos $f_n:[0,1] \cap \mathbb{Q} \longrightarrow \mathbb{R}$ dada por:
$$
    f_n(x) = \left\{ \begin{matrix} 1 & \text{si } x \in \{r_1, ...,r_n\}     \\
               0 & \text{si } x \not\in \{r_1, ...,r_n\}
    \end{matrix}\right.
$$
que es integrable-Riemann en [0,1] para cada $n\in\mathbb{N}$. Sin embargo
$$
    f(x) = \displaystyle\lim_{n \to{+}\infty}{f_n(x)} = \left\{ \begin{matrix} 1 & \mbox{si }x\mbox{ $\in\mathbb{Q}$}
               \\ 0 & \mbox{si }x\mbox{ $\not\in\mathbb{Q}$}\end{matrix}\right.
$$
que es la función de Dirichlet, la cual no es integrable-Riemann.

\begin{teo}[Teorema de Arzelá]
    Supongamos
    \begin{enumerate}
        \item $\{f_n\}$ una sucesión de funciones que converge puntualmente a f en $[a,b]$.
        \item Para todo $n \in \mathbb{N}$, $f_n$ es integrable-Riemann en $[a,b]$.
        \item f es integrable-Riemann en $[a,b]$.
        \item Existe M tal que $|f_n(x)| \leq M$ para todo $x \in [a,b]$ y todo $n \in \mathbb{N}$.
    \end{enumerate}
    Entonces:
    $$
        \lim_{n \to{+}\infty}{\int_{a}^{b} f_n(x)\, dx} = \int_{a}^{b} \lim_{n \to{+}\infty}{f_n(x)} \, dx = \int_{a}^{b} f(x)\, dx
    $$
\end{teo}

\newpage
\section{Idea vaga de la medida y de la integral de Lebesgue}

\begin{teo}
    \label{teo:asd}
    No existe ninguna aplicación $ m_n: \mathcal{P}(\mathbb{R}^n) \longrightarrow [0, +\infty]$ tal que:
    \begin{enumerate}
        \item[(i)] $m_n([0,1) \times ... \times [0,1)) = $ $m_n([0,1)^n) = 1$.
        \item[(ii)] $m_n\left(\displaystyle\bigcup_{i=1}^{\infty}{A_i }\right) = \displaystyle\sum_{i=1}^\infty m_n(A_i)$.
        \item[(iii)] $m_n(A) = m_n(A+a)$, donde $A + a = \{x+a : x \in A\}$.
    \end{enumerate}
\end{teo}
\begin{proof}
    En primer lugar, señalamos que si $m$ cumple esas tres propiedades entonces también verifica
    \begin{enumerate}
        \item[(iv)] $A \subset B \Longrightarrow m(A) \leq m(B)$.
    \end{enumerate}
    En efecto, si aplicamos $(i)$ con $E_1 = A, E_2 = B \backslash A, E_i = \emptyset$ para $i \ge 2$, obtenemos
    \begin{align*}
        m(B) = m\left(\bigcup_{i=1}^{\infty}{E_i}\right) = \sum_{i=1}^{\infty}{m(E_i)} \ge m(E_1) = m(A)
    \end{align*}
    Una vez hecha esta observación, comenzamos la demostración propiamente dicha. Razonaremos por reducción al absurdo. Supongamos que existe $m$ con las propiedades (i), (ii) y (iii). Definimos en [0,1) la relación binaria $\sim: x \sim y$ si $x -y \in \mathbb{Q}$. Esta relación binaria es una relación de equivalencia. Veamoslo:
    \begin{enumerate}
        \item[(a)] Reflexiva.Sea $x \in [0,1)$, $x \sim x$ si $x - x \in \mathbb{Q}$ pero $x - x = 0 \in \mathbb{Q}$, luego $x \sim x$.
        \item[(b)] Simétrica. Supongamos que $x \sim y$. Veamos que $y \sim x$. Como $x \sim y$, entonces $x - y \in \mathbb{Q}$ por tanto $-(x-y) \in \mathbb{Q}$, esto es, $y - x \in \mathbb{Q}$ lo que nos dice que $y \sim x$.
        \item[(c)] Transitiva. Supongamos que $x \sim y$ e $y \sim z$. Veamos que $x \sim z$.
              \begin{enumerate}
                  \item[(*)] Como $x \sim y$ entonces $x - y \in \mathbb{Q}$.
                  \item[(**)] Como $y \sim z$ entonces $y - z \in \mathbb{Q}$.
              \end{enumerate}
    \end{enumerate}
    Sumando (*) y (**) nos dice que $x - y + y - z \in \mathbb{Q}$, es decir, $x - z \in \mathbb{Q}$, luego $x \sim z$. Entonces $\sim$ es un relación de equivalencia.

    Sea $E$ un conjunto que tenga un elemento y solamente uno de cada clase de equivalencia de $\sim$ (este conjunto se denomina conjunto de Vitali). Sea $\{r_n\}_{n \in \mathbb{N}}$ una enumeración de $\mathbb{Q} \cap [-1,1]$, es decir, $r_n \not = r_m$ si $n \not = m$ entonces para todo $r \in \mathbb{Q} \cap [-1,1]$ existe $n \in \mathbb{N}$ tal que $r = r_n$.

    Para todo $k \in \mathbb{N}$ definimos $E_k = E + r_k = \{x + r_k : x \in E \}$. Onsérvese que se cumplen las dos propiedades siguientes
    \begin{enumerate}
        \item[(I)] Si $j \not = k$ entonces $E_j \cap E_k = \emptyset$.
        \item[(II)] $[0,1) \subset \bigcup_{j=1}^{\infty}{E_j} \subset [-1,2)$.
    \end{enumerate}
    Con lo que tenemos lo siguiente
    \begin{align*}
        m([0,1)) \leq m\left( \bigcup_{j=1}^{\infty}{E_j} \right) = \sum_{j=1}^{\infty}{m(E_j)} \leq m([-1,2))
    \end{align*}
    Aplicando las propiedades (ii) y (iii), obtenemos
    \begin{align*}
        1 \leq \sum_{j=1}^{\infty}{m(E_j)} \leq 3
    \end{align*}
    De nuevo, por (iii), $m(E_j) = m(E) = \lambda \in [0,\infty]$ para todo $j$. Luego
    \begin{align*}
        1 \leq \sum_{j=1}^{\infty}{\lambda} \leq 3
    \end{align*}
    Esto es un contradicción pues ningún $\lambda \in [0,\infty]$ cumple dicha desigualdad.
\end{proof}

\section{Álgebras y $\sigma$-álgberas}
\begin{defi}
    Un álgebra $\mathcal{A}$ sobre $X$ es una familia de subconjuntos de $X$ ($\mathcal{A} \subset \mathcal{P}(X)$) tal que
    \begin{enumerate}
        \item[(i)] $X \in \mathcal{A}$.
        \item[(ii)] Si $E \in \mathcal{A}$ entonces $E^c = X \backslash E \in \mathcal{A}$.
        \item[(iii)] Si $A, B \in \mathcal{A}$ entonces $A \cup B \in \mathcal{A}$.
    \end{enumerate}
\end{defi}
Algunas propiedades inmediatas para un álgebra $\mathcal{A}$:
\begin{enumerate}
    \item[(iv)] $\emptyset \in \mathcal{A}$.
    \item[(v)] Si $A, B \in \mathcal{A}$ entonces $A \cap B \in \mathcal{A}$.
    \item[(vi)] Si $A, B \in \mathcal{A}$ entonces $A \backslash B \in \mathcal{A}$.
    \item[(vii)] Si $A_1, A_2,..., A_n \in \mathcal{A}$ entonces $\bigcup_{i=1}^{n}{A_i} \in \mathcal{A}$ y $\bigcap_{i=1}^{n}{A_i} \in \mathcal{A}$.
\end{enumerate}
\begin{defi}
    Una $\sigma$-álgebra $\mathcal{M}$ sobre $X$ es una familia de subconjuntos de $X$ ($\mathcal{M} \subset \mathcal{P}(X)$) tal que
    \begin{enumerate}
        \item[(i)] $X \in \mathcal{M}$.
        \item[(ii)] Si $E \in \mathcal{M}$, entonces $E^c = X \backslash E \in \mathcal{M}$.
        \item[(iii)] Si $\{E_i\}_{i=1}^{\infty}$ es tal que $E_i \in \mathcal{M}$ entonces $\bigcup_{i=1}^{\infty}{E_i} \in \mathcal{M}$.
    \end{enumerate}
\end{defi}
Algunas propiedades inmediatas para un $\sigma$-álgebra $\mathcal{M}$:
\begin{enumerate}
    \item[(iv)] $\mathcal{M}$ es un álgebra.
    \item[(v)] Si $\{E_i\}_{i=1}^{\infty}$ es tal que $E_i \in \mathcal{M}$ entonces $\bigcap_{i=1}^{\infty}{E_i} \in \mathcal{M}$.
\end{enumerate}
\begin{defi}
    Si $\mathcal{E} \in \mathcal{P}(X)$ es una familia de subconjuntos de $X$, la $\sigma$-álgebra generada por $\mathcal{E}$, denotada por $\mathcal{M(E)}$, es la intersección de todas las $\sigma$-álgebras que contienen a $\mathcal{E}$ (siempre hay una por lo menos ya que $\mathcal{P}(X)$ es un $\sigma$-álgebra que contiene a $\mathcal{E}$).

    Por consiguiente, si $\mathcal{E} \subset \mathcal{M}$ y $\mathcal{M}$ es un $\sigma$-álgebra entonces $\mathcal{M(E)} \subset \mathcal{M}$, dicho de otro modo, $\mathcal{M(E)}$ es la menor $\sigma$-álgebra que contiene a $\mathcal{E}$.
\end{defi}
\begin{defi}
    Sea $(X, \tau)$ un espacio topológico. La $\sigma$-álgebra de Borel $\mathcal{B}_X$ es la $\sigma$-álgebra generada por los abiertos de la topología.
\end{defi}

\newpage
\section{La $\sigma$-álgebra de Borel de $\mathbb{R}$ y de $\mathbb{R}^n$}
\begin{prop}
    La $\sigma$-álgebra de Borel $\mathcal{B}_{\mathbb{R}}$ sobre $\mathbb{R}$ está generada por cada una de las siguientes familias de intervalos siguientes:
    \begin{enumerate}
        \item[(a)] $\mathcal{E}_1 = \{(a,b) : \infty \leq a < b \leq +\infty \}$.
        \item[(b)] $\mathcal{E}_2 = \{(a,b) : a.b \in \mathbb{R} \}$.
        \item[(c)] $\mathcal{E}_3 = \{[a,b) : a,b \in \mathbb{R} \}$.
        \item[(d)] $\mathcal{E}_4 = \{(a,b] : a,b \in \mathbb{R} \}$.
        \item[(e)] $\mathcal{E}_5 = \{[a,b] : a,b \in \mathbb{R} \}$.
        \item[(f)] $\mathcal{E}_6 = \{(a,+\infty) : a \in \mathbb{R} \}$.
        \item[(g)] $\mathcal{E}_7 = \{[a,+\infty) : a \in \mathbb{R} \}$.
        \item[(h)] $\mathcal{E}_8 = \{(-\infty,b) : b \in \mathbb{R} \}$.
        \item[(i)] $\mathcal{E}_9 = \{(-\infty,b] : b \in \mathbb{R} \}$.
    \end{enumerate}
\end{prop}

\begin{prop}
    La $\sigma$-álgebra de Borel $\mathcal{B}_{\mathbb{R}^n}$ sobre $\mathbb{R}^n$ está generada por cada una de las familias siguientes de intervalos:
    \begin{enumerate}
        \item[(a)] $\mathcal{E}_2 = \{(a_1,b_1) \times ... \times (a_n,b_n) : a_i,b_i \in \mathbb{R} \}$.
        \item[(b)] $\mathcal{E}_3 = \{[a_1,b_1) \times ... \times [a_n,b_n) : a_i,b_i \in \mathbb{R} \}$.
        \item[(c)] $\mathcal{E}_4 = \{(a_1,b_1] \times ... \times (a_n,b_n] : a_i,b_i \in \mathbb{R} \}$.
        \item[(d)] $\mathcal{E}_5 = \{[a_1,b_1] \times ... \times [a_n,b_n] : a_i,b_i \in \mathbb{R} \}$.
        \item[(e)] $\mathcal{E}_6 = \{ S_i(a) : a \in \mathbb{R}, i = 1,..., n \} $, donde :
              $$S_i(a) = \{ x = (x_1, ..., x_n) : x_i > a \}.$$
        \item[(f)] $\mathcal{E}_7 = \{ S_i(a) : a \in \mathbb{R}, i = 1,..., n \} $, donde :
              $$S_i(a) = \{ x = (x_1, ..., x_n) : x_i \leq a \}.$$
        \item[(g)] $\mathcal{E}_8 = \{ S_i(a) : a \in \mathbb{R}, i = 1,..., n \} $, donde :
              $$S_i(a) = \{ x = (x_1, ..., x_n) : x_i < a \}.$$
        \item[(h)] $\mathcal{E}_9 = \{ S_i(a) : a \in \mathbb{R}, i = 1,..., n \} $, donde :
              $$S_i(a) = \{ x = (x_1, ..., x_n) : x_i \ge a \}.$$
    \end{enumerate}
\end{prop}
\section{Medidas}
\subsection{Medidas finitamente aditivas}
\begin{defi}
    Sean $X$ un conjunto y $\mathcal{A}$ un álgebra sobre X. Una medida finitamente aditiva sobre $(X, \mathcal{A})$ (o, simplemente, sobre $\mathcal{A}$) es una aplicación $\mu : \mathcal{A} \longrightarrow [0, +\infty]$ tal que
    \begin{enumerate}
        \item[(a)] $\mu (\emptyset) = 0$.
        \item[(b)] Si $A, B \in \mathcal{A}$ y $A \cap B = \emptyset$ entonces $\mu(A \cup B) = \mu(A) + \mu(B)$.
    \end{enumerate}
\end{defi}
\begin{prop}
    Sea $\mu$ una medida finitamente aditiva. Entonces $\mu$ tiene las propiedades siguientes:
    \begin{enumerate}
        \item[(a)] $\mu \left(\bigcup_{i=1}^{N}{E_i} \right) = \sum_{i=1}^{N} \mu(E_i)$ para cualquier $\{ E_i \}_{i=1}^{N}$ finita y disjunta de subconjuntos de $\mathcal{A}$.
        \item[(b)] $A, B \in \mathcal{A}$ tales que $A \subset B$ entonces $\mu(A) \leq \mu(B)$.
        \item[(c)] $A, B \in \mathcal{A}$ tales que $A \subset B$ y $\mu(A) < +\infty$ entonces $\mu(B \backslash A) = \mu(B) - \mu(A)$.
        \item[(d)] $A, B \in \mathcal{A}$ entonces $\mu(A \cup B) + \mu(A \cap B) = \mu(A) + \mu(B)$.
    \end{enumerate}
\end{prop}
\begin{proof}
    \begin{enumerate}
        \item[(a)] Se demuestra por inducción.
        \item[(b)] Es claro que $B = A \cup (B \backslash A)$ y que $A$ y $B \backslash A$ son disjuntos. Por ser $\mu$ finitamente aditiva, $\mu(B) = \mu(A) + \mu(B \backslash A)$ y puesto que $\mu(B \backslash A) \ge 0$ tenemos
              \begin{align*}
                  \mu(B) = \mu(A) + \mu(B \backslash A) \ge \mu(A)
              \end{align*}
        \item[(c)] Como antes, $\mu(B) = \mu(A) + \mu(B \backslash A)$, y, ya que $\mu(A) < +\infty$, se deduce que $\mu(B \backslash A) = \mu(B) - \mu(A)$.
        \item[(d)] Hágase como ejercicio.
    \end{enumerate}
\end{proof}

\subsection{Medidas (o medidas numerablemente aditivas)}
\begin{defi}
    Si $\mathcal{M}$ es una $\sigma$-álgebra sobre $X$, decimos que $(X, \mathcal{M})$ es un espacio medible y a los conjuntos $E \in \mathcal{M}$ se les llama conjuntos medibles.
\end{defi}
\begin{defi}
    Sea $(X, \mathcal{M})$ un espacio medibe. Una medida sobre $(X, \mathcal{M})$ es una aplicación $\mu : \mathcal{A} \longrightarrow [0, +\infty]$ tal que
    \begin{enumerate}
        \item[(a)] $\mu (\emptyset) = 0$.
        \item[(b)] $\mu \left(\bigcup_{i=1}^{\infty}{E_i} \right) = \sum_{i=1}^{\infty} \mu(E_i)$ para cualquier $\{ E_i \}_{i=1}^{\infty}$ numerable y disjunta de subconjuntos de $\mathcal{M}$.
    \end{enumerate}
    En este caso diremos que $(X, \mathcal{M}, \mu)$ es un espacio de medida y que $\mu$ es una medida sobre $X$ o sobre $\mathcal{M}$.
\end{defi}
\begin{ejemplo}
    \begin{enumerate}
        \item[(i)] Sea $X$ cualquier conjunto y $\mathcal{M} = \mathcal{P}(X)$. Definimos
              \begin{align*}
                  \mu(E) =  \left\{ \begin{array}{lcc}
                                        \#E     & si & \textit{E es finito}   \\
                                        +\infty & si & \textit{E es infinito} \\
                                    \end{array}
                  \right.
              \end{align*}
              $\mu$ es una medida que se denomina \textbf{medida contadora}.
        \item[(ii)] Sea $X$ cualquier conjunto y $\mathcal{M} = \mathcal{P}(X)$. Fijamos $a \in X$. Definimos
              \begin{align*}
                  \mu(E) =  \left\{ \begin{array}{lcc}
                                        1 & si & a \in E      \\
                                        0 & si & a \not \in E \\
                                    \end{array}
                  \right.
              \end{align*}
              $\mu$ es una medida que se denomina \textbf{delta de Dirac en el punto a} y se denota por $\delta_a$.
        \item[(iii)] Sea $X$ cualquier conjunto y
              \begin{align*}
                  \mathcal{M} =\{ E \subset X : E \textit{ es numerable o } E^c \textit{ es numerable} \} \\
                  \mu(E) =  \left\{ \begin{array}{lcc}
                                        0 & si & \textit{E es numerable}    \\
                                        1 & si & \textit{E no es numerable} \\
                                    \end{array}
                  \right.
              \end{align*}
              Se tiene que $\mu$ es una medida.
        \item[(iv)] Sea $X = \mathbb{N}$ y $\mathcal{M} = \mathcal{P}(\mathbb{N})$. Definimos
              \begin{align*}
                  \mu(E) =  \left\{ \begin{array}{lcc}
                                        0       & si & \textit{E es finito}   \\
                                        +\infty & si & \textit{E es infinito} \\
                                    \end{array}
                  \right.
              \end{align*}
              $\mu$ no es una medida pero es una medida finitamente aditiva.
    \end{enumerate}
\end{ejemplo}
\begin{prop}
    Sea $\mu$ una medida sobre $\mathcal{M}$, es decir, $(X, \mathcal{M}, \mu)$ es un espacio de medida. Se verifican las propiedades siguientes:
    \begin{enumerate}
        \item[(a)] $\mu$ es finitamente aditiva sobre $\mathcal{M}$.
        \item[(b)] Si $A, B \in \mathcal{M}$ y $A \subset B$ entonces $\mu(A) \leq \mu(B)$.
        \item[(c)] Si $A, B \in \mathcal{M}$ y $A \subset B$ y $\mu(A) < +\infty$ entonces $\mu(B \backslash A) = \mu(B) - \mu(A)$.
        \item[(d)] Si  $\{ E_i \}_{i=1}^{\infty}$ es una colecciónn numerable en $\mathcal{M}$ tal que $E_i \subset E_{i+1}$, para todo i entonces
              \begin{align*}
                  \mu \left( \bigcup_{i=1}^{\infty}{E_i} \right) = \lim_{i \to \infty}{\mu(E_i)}
              \end{align*}
    \end{enumerate}
\end{prop}
\begin{proof}

\end{proof}

\begin{defi}
    Sea $(X, \mathcal{M}, \mu)$ un espacio de medida.
    \begin{enumerate}
        \item[(a)] Decimos que el espacio de medida es finito o que $\mu$ es finita si $\mu(X) < \infty$.
        \item[(b)] Decimos que el espacio de medida es de probabilidad o que $\mu$ es una probabilidad si $\mu(X) = 1$.
        \item[(c)] Decimos que el espacio e medida es $\sigma$-finito o que $\mu$ es $\sigma$-finita si existe una sucesión $\{ X_n \}$ de conjuntos mediables ($X_n \in \mathcal{M}$) tal que  $X = \bigcup_{n=1}^{\infty}{X_n}$ y $\mu(X_n) < \infty$ para todo $n$.
    \end{enumerate}
\end{defi}
\begin{defi}
    Decimos que un espacio de medida $(X, \mathcal{M}, \mu)$ es completo si son medibles todos los subconjuntos de los conjuntos de medida cero, es decir, si $F \subset N \in \mathcal{M}$ y $\mu(N) = 0$ implica $F \in \mathcal{M}$.
\end{defi}
Los espacios de medida se pueden completar.
\begin{teo}
    \label{teo:completar}
    Sea $(X, \mathcal{M}, \mu)$ un espacio de medida. Definimos
    \begin{align*}
        \overline{\mathcal{M}} = \{ E \cup F : E \in \mathcal{M}, F \subset N \in \mathcal{M}, \mu(N) = 0\}
    \end{align*}
    y $\overline{\mu}: \overline{\mathcal{M}} \longrightarrow [0, +\infty]$ mediante
    \begin{align*}
        \overline{\mu}(A) = \mu(E) \text{, donde } A \in \overline{\mathcal{M}}
    \end{align*}
    Entonces
    \begin{enumerate}
        \item[(a)] $\mathcal{M} \subset \overline{\mathcal{M}}$, $\overline{\mathcal{M}}$ es un $\sigma$-álgebra y $\overline{\mu}$ está bien definida.
        \item[(b)] $(X, \mathcal{\overline{M}}, \overline{\mu})$ es un espacio de medida completo y $\overline{\mu}|_{\mathcal{M}} = \mu$.
        \item[(c)] Si $(X, \mathcal{N}, \mu)$ es otro espacio completo tal que $\mathcal{M} \subset \mathcal{N}$ y $\nu |_{\mathcal{M}} = \mu$ entonces $\overline{\mathcal{M}} \subset \mathcal{N}$ y $\nu |_{\overline{\mathcal{M}}} = \overline{\mu}$.
    \end{enumerate}
\end{teo}
\begin{proof}

\end{proof}

\section{Introducción a la medida de Lebesgue}
\subsection{Preliminares}
Vamos a considerar en $\mathbb{R}$ intervalos acotados con la notación habitual: $(a, b)$, $[a, b)$, $(a, b]$, $[a, b]$, $a \leq b$, $a, b \in \mathbb{R}$. Si $I$ es uno de esos intervalos la longitud de $I$ es $b - a$ y escribimos $l(I) = b - a$. Los intervalos $(a, a)$, $[a, a)$ y $(a, a]$ son el conjunto vacío.
Observamos que, en consecuencia, $l(\emptyset) = 0$ cualquiera que sea la representación elegida.

Un intervalo (acotado) $I$ en $\mathbb{R}^n$ es un producto de $n$ intervalos acotados de $\mathbb{R}$: $I = I_1 \times ... \times I_n$. El volumen de $I$ se define como $\mathcal{V}(I) = \mathcal{V}(I_1) \cdot ... \cdot \mathcal{V}(I_n)$. Como antes, $l(\emptyset) = 0$. Decimos que $I$ es un cubo si todos los intervalos $I_i$ tienen la misma longitud. Algunas propiedades de los intervalos y del volumen son las siguientes:
\begin{enumerate}
    \item[(a)] Si $\emptyset \not = I_1 \times ... \times I_n \subset J_1 \times ... \times J_n$ entonces $I_i \subset J_i$ para todo $i$.
    \item[(b)] Si $I$ y $J$ son intervalos con $I \subset J$ entonces $\mathcal{V}(I) \leq \mathcal{V}(J)$.
          \begin{enumerate}
              \item[(i)] Sea $A \not = \emptyset$ un intervalo. Entonces para todo $\varepsilon > 0$ existe un intervalo cerrado $J \subset A$ tal que $\mathcal{V}(A) - \mathcal{V}(J) < \varepsilon$.
              \item[(ii)] Sea $A$ un intervalo. Entonces para todo $\varepsilon > 0$ existe un intervalo abierto $J$ tal que $A \subset J$ y $\mathcal{V}(J) - \mathcal{V}(A) < \varepsilon$.
          \end{enumerate}
    \item[(c)] Si $A, B_1,..., B_s$ son intervalos de $\mathbb{R}^n$ y $A \subset \bigcup_{i=1}^{s}{B_i}$ entonces $\mathcal{V}(A) \leq \sum_{i=1}^{s}{\mathcal{V}(B_i)}$.
    \item[(d)] Si $A, B_1,..., B_s$ son intervalos de $\mathbb{R}^n$ y $A = \bigcup_{i=1}^{s}{B_i}$ y los intervalos $B_i$ son disjuntos entonces $\mathcal{V}(A) = \sum_{i=1}^{s}{\mathcal{V}(B_i)}$.
    \item[(e)] La $\sigma$-álgebra de Borel de $\mathbb{R}^n$ está generada por los intervalos de $\mathbb{R}^n$.
\end{enumerate}
\subsection{La medida exterior de Lebesgue}
Pretendemos definir una medida $m$ sobre una $\sigma$-álgebra $\mathcal{M}$ de $\mathbb{R}^n$ de forma que los intervalos pertenezcan a $\mathcal{M}$ y que $m(I) = \mathcal{V}(I)$.
\begin{defi}
    Definimos la medida exterior de Lebesgue $m^*: \mathcal{P}(\mathbb{R}^n) \longrightarrow [0, +\infty]$ como
    $$m^\ast(A) = \inf(H_A)$$
    donde
    \begin{align*}
        H_A = \left\{ \sum_{i=1}^{\infty}{\mathcal{V}(I_i)} : A \subset \bigcup_{i=1}^{\infty}{I_i}, I_i \textit{ es un intervalo de } \mathbb{R}^n \right\}
    \end{align*}
\end{defi}
\begin{obs}
    $m^*(A) = 0$ si y sólo si para todo $\varepsilon > 0$ existe una familia numerable de intervalos abiertos $\{ I_i\}_{i=1}^{\infty}$ tal que
    \begin{align*}
        A \subset \bigcup_{i=1}^{\infty}{I_i} \text{ y } \sum_{i=1}^{\infty}{\mathcal{V}(I_i)} < \varepsilon
    \end{align*}
\end{obs}
\begin{prop}
    Algunas propiedades de $m^*$
    \begin{enumerate}
        \item[(a)] Si $A \subset B$ entonces $m^*(A) \leq m^*(B)$.
        \item[(b)] $m^*(\{ x \}) = 0$ para todo $x \in \mathbb{R}^n$.
        \item[(c)] $m^*(\emptyset) = 0$.
        \item[(d)] La medida exterior de Lebesgue de un conjunto numerable es cero.
        \item[(e)] La medida exterior de Lebesgue de un intervalo $I$ es su volumen.
        \item[(f)] La medida exterior de Lebesgue de $\mathbb{R}^n$ es $+\infty$.
        \item[(g)] La medida exterior de Lebesgue es invariante a traslaciones: si $A \subset \mathbb{R}^n$ y $b \in \mathbb{R}^n$ entonces $m^*(A + b) = m^*(A)$.
        \item[(h)] La medida exterior de Lebesgue no es una medida sobre $\mathcal{P}(\mathbb{R}^n)$.
    \end{enumerate}
\end{prop}
\begin{proof}
    \begin{enumerate}
        \item[(a)] Se sigue fácilmente porque $A \subset B \Longrightarrow H_B \subset H_A$.
        \item[(b)] Sea $x = (x_1,...,x_n) \in \mathbb{R}^n$. Sea $\varepsilon > 0$. Sean
              \begin{align*}
                  I_1 = (x_1 - \varepsilon, x_1 + \varepsilon) \times (x_2 - \varepsilon, x_2 + \varepsilon) \times ... \times (x_n - \varepsilon, x_n + \varepsilon), I_i = \emptyset \text{ para todo } i\ge 2
              \end{align*}
              Entonces
              \begin{align*}
                  \{x\} \subset \bigcup_{i=1}^{\infty}{I_i} \text{ \ \ \ y \ \ \ } \sum_{i=1}^{\infty}{\mathcal{V}(I_i)} = (2\varepsilon)^n
              \end{align*}
              Por lo tanto, $m^*(\{x\}) \leq (2\varepsilon)^n$ para todo $\varepsilon > 0$, lo que implica $m^*(\{x\}) = 0$.
        \item[(c)] Se sigue de (a) y (b).
        \item[(d)] Escribimos $A = \{a_j\}_{j=1}^{\infty}$ y sea $\varepsilon > 0$. Por lo ya demostrado en (b), para casa $j \in \mathbb{N}$, existe un intervalo abierto $I_j$ con $a_j \in I_j$ y $\mathcal{V}(I_j) < \frac{\varepsilon}{2^j}$, de donde,
              \begin{align*}
                  m^*(A) \leq \sum_{j=1}^{\infty}{\mathcal{V}(I_j)} \leq \sum_{j=1}^{\infty}{\frac{\varepsilon}{2^j}} = \varepsilon
              \end{align*}
              y por lo tanto, $m^*(A) = 0$.
        \item[(e)] Es claro que $m^*(I) = \mathcal{V}(I)$ para intervalos abiertos $I$.

        Sea ahora $I$ un intervalo cualquiera no vacío. Dado $\varepsilon > 0$ existen dos intervalos abiertos $I_1$ e $I_2$ tales que $I_1 \subset I \subset I_2$, $\mathcal{V}(I) - \varepsilon \leq \mathcal{V}(I_1)$ y $\mathcal{V}(I_2) \leq \mathcal{V}(I) + \varepsilon$. Entonces, aplicando que $m^*$ es una medida exterior y lo ya demostrado para intervalos abiertos,
              \begin{align*}
                  \mathcal{V}(I) - \varepsilon \leq \mathcal{V}(I_1) = m^*(I_1) \leq m*(I_2) = \mathcal{V}(I_2) \leq \mathcal{V}(I) + \varepsilon
              \end{align*}
              Tomando limites cuando $\varepsilon$ tiende a cero, concluimos $m^*(I) = \mathcal{V}(I)$.
        \item[(f)] Para cada natural $k$ tenemos $I_k = (-k,k) \times ... \times (-k,k) \subset \mathbb{R}^n$. Entonces
              \begin{align*}
                  m^*(I_k) = \mathcal{V}(I_k) \leq m^*(\mathbb{R}^n)
              \end{align*}
              es decir, $(2k)^n \leq m^*(\mathbb{R})^n$ para todo $k \in \mathbb{N}$. Luego $m^*(\mathbb{R}^n) = +\infty$.
        \item[(g)] Sea $A \subset \mathbb{R}^n$ y $b \in \mathbb{R}^n$. Consideramos $A + b = \{ a + b : a \in A\}$. Sean
              \begin{align*}
                  S = \left\{  \sum_{j=1}^{\infty}{\mathcal{V}(I_j)} : A \subset \bigcup_{j=1}^{\infty}{I_j}, I_j \text{ intervalo abierto}\right\} \text{ y } \\
                  T = \left\{  \sum_{j=1}^{\infty}{\mathcal{V}(H_j)} : A + b \subset \bigcup_{j=1}^{\infty}{H_j}, H_j \text{ intervalo abierto}\right\}
              \end{align*}
              Así, $m^*(A) = \inf(S)$ y $m^*(A + b) = \inf(T)$. Bastará ver que $S = T$.

              Comenzamos probando $S \subset T$. Sea $\lambda \in S$. Entonces, existe $\{ I_j \}_{j=1}^{\infty}$ tal que $A \subset \bigcup_{j=1}^{\infty}{I_j}$, con $I_j$ intervalo abierto y $\lambda = \sum_{j=1}^{\infty}{\mathcal{V}(I_j)}$. Entonces,
              \begin{align*}
                  A + b \subset \left( \bigcup_{j=1}^{\infty}{I_j} \right) + b =  \bigcup_{j=1}^{\infty}{(I_j + b)}
              \end{align*}
              Así, si $H_j = I_j + b$ se tiene que $H_j$ es un intervalo abierto, $\mathcal{V}(H_j) = \mathcal{V}(I_j)$, $A + b \subset \bigcup_{j=1}^{\infty}{H_j}$ y
              \begin{align*}
                  \sum_{j=1}^{\infty}{\mathcal{V}(H_j)} = \sum_{j=1}^{\infty}{\mathcal{V}(I_j)} = \lambda
              \end{align*}
              por lo que $\lambda \in T$. La otra inclusión es análoga (basta restar b).
        \item[(h)] Se sigue de las propiedades (c), (e) y (g) y del teorema \ref{teo:asd}.
    \end{enumerate}
\end{proof}

\subsection{La medida de Lebesgue}
La medida exterior de Lebesgue restringida a la $\sigma$-álgebra de Borel de $\mathbb{R}^n$ es una medida. Enunciamos el teorema pero posponemos la demostración para más adelante.
\begin{teo}
    La medida exterior de Lebesgue $m^*$ restringida a $\mathcal{B}_{\mathbb{R}^n}$ es una medida, denotada por $m$, sobre la $\sigma$-álgebra de Borel de $\mathbb{R}^n$ y verifica que $m(I) = \mathcal{V}(I)$ para todo intervalo $I$. A la medida $m$ se le denomina medida Lebesgue.
\end{teo}
\begin{teo}
    La medida de Lebesgue $m = m^* |_{\mathcal{B}_{\mathbb{R}^n}}$ es la única medida $\mu$ definida sobre $\mathcal{B}_{\mathbb{R}^n}$ tal que $\mu(I) = \mathcal{V}(I)$ para todo intervalo abierto $I$ de $\mathbb{R}^n$.
\end{teo}
\begin{proof}
    Sea $\mu: \mathcal{B}_{\mathbb{R}^n} \longrightarrow [0,+\infty]$ otra medida tal que $\mu(I) = \mathcal{V}(I)$ para todo intervalo $I$ abierto de $\mathbb{R}^n$. Tenemos que probar que $m(E) = \mu(E)$ para todo $E \in B_{\mathbb{R}^n}$. Sea $E \in B_{\mathbb{R}^n}$, por definición $m(E) = \inf(H_E)$ donde
    \begin{align*}
        H_E = \left\{ \lambda = \sum_{i=1}^{\infty}{\mathcal{V}(I_i)} : A \subset \bigcup_{i=1}^{\infty}{I_i}, I_i \textit{ es un intervalo de } \mathbb{R}^n \right\}
    \end{align*}
    \begin{enumerate}
        \item[(i)] Veamos que $\mu(E) \leq m(E)$. Sea $\{I_i\}_{i=1}^{\infty}$ una familia numerable tal que $E \subset \bigcup_{i=1}^{\infty}{I_i}$. Como $\mu$ es medida
              \begin{align*}
                  \mu(E) \leq \mu \left( \bigcup_{i=1}^{\infty}{I_i}\right) \leq \sum_{i=1}^{\infty}{\mu(I_i)} = \sum_{i=1}^{\infty}{\mathcal{V}(I_i)}
              \end{align*}
              Luego $\mu(E) \leq \lambda$ para todo $\lambda \in H_E$, esto es, $\mu(E)$ es cota inferior de $H_E$, por tanto, $\mu(E) \leq \inf(H_E) = m^*(E) = m(E)$. Por tanto
              \begin{align*}
                  \mu(E) \leq m(E) \forall E \in B_{\mathbb{R}^n}
              \end{align*}
        \item[(ii)] Veamos que $m(E) \leq \mu(E)$. Nótese que $\mathbb{R}^n = \bigcup_{k=1}^{\infty}{I_k}$, donde $I_k = (-k,k) \times ... \times (-k,k) = (-k,k)^n$. Además $I_k \subset I_{k+1}$ y $E = \bigcup_{k=1}^{\infty}{E_k}$ donde $E_k = E \cap I_k$ y $E_k \subset E_{k+1}$. Por (i), sabemos que
              \begin{align*}
                  \mu(I_k \backslash E_k) \leq m(I_k \backslash E_k)
              \end{align*}
              Entonces
              \begin{align*}
                  m(E_k) + m(I_k \backslash E_k) & = m( E_k \cup (I_k \backslash E_k))                                         \\
                                                 & = m(I_k) = \mathcal{V}(I_k) = \mu(I_k)                                      \\
                                                 & = \mu ( E_k \cup (I_k \backslash E_k)) = \mu(E_k) + \mu(I_k \backslash E_k) \\
                                                 & \leq \mu(E_k) + m(I_k \backslash E_k)
              \end{align*}
              Como $\mu(I_k \backslash E_k) < +\infty$, obtenemos que
              \begin{align*}
                  m(E_k) \leq \mu(E_k)
              \end{align*}
              y, consecuentemente,
              \begin{align*}
                  m(E) = m\left( \bigcup_{i=1}^{\infty}{E_k}\right) = \lim_{k \to \infty}{m(E_k)} \leq \lim_{k \to \infty}{\mu(E_k)} = \mu \left( \bigcup_{i=1}^{\infty}{E_k}\right) = \mu(E)
              \end{align*}
              es decir
              \begin{align*}
                  m(E) \leq \mu(E) \text{, } \forall E \in \mathcal{B}_{\mathbb{R}^n}
              \end{align*}
    \end{enumerate}
    Luego $m = \mu$, como queríamos probar.
\end{proof}
\begin{teo}
    La medida de Lebesgue $m$ es la única medida sobre la $\sigma$-álgebra de Borel, invariante frente a traslaciones y tal que la medida del cubo unidad $[0,1) \times ... \times [0,1)$ es 1.
\end{teo}

\subsection{El conjunto de Cantor}
No es difícil ver que en $\mathbb{R}^2$ existen conjuntos de medida cero que no son numerables. Por ejemplo, el intervalo $\{ (x,y) : x \in [0,1], y = 0\} = [0,1] \times \{0\}$. En esta sección vamos a estudiar un conjunto muy interesante, el conjunto de Cantor que, además de otras propiedades, tiene medida cero y es no numerable.

Partimos del intervalo $I = [0,1]$ y suprimimos el tercio central. Llamemos $H_1$ a lo que suprimimos y $C_1$ a lo que queda, es decir,
\begin{align*}
    H_1 = \left( \frac{1}{3}, \frac{2}{3}\right) \text{ y } C_1 = I \backslash H_1 = \left[ 0, \frac{1}{3} \right] \cup \left[ \frac{2}{3}, 1\right]
\end{align*}
Repetimos este proceso que cada uno de los intervalos cerrados que forman $C_1$. Llamemos $H_2$ a lo que quitamos y $C_2$ a lo que queda. Así obtenemos
\begin{align*}
    H_2 = \left( \frac{1}{9}, \frac{2}{9}\right) \cup \left( \frac{7}{9}, \frac{8}{9}\right) \text{ y } C_2 = C_1 \backslash H_2
\end{align*}
Repitiendo este proceso sucesivamente, en el paso n quitamos $H_n$ que es una unión disjunta de $2^{n-1}$ intervalos abiertos de longitud $\frac{1}{3^n}$ y obtenemos $C_n$ que es una unión disjunta de $2^n$ intervalos cerrados de longitud $\frac{1}{3^n}$.

El \textbf{conjunto de Cantor} es lo que queda después de este proceso, esto es,
\begin{align*}
    C = \bigcap_{n=1}^{\infty}{C_n} = I \backslash \bigcup_{n=1}^{\infty}{H_n}
\end{align*}

Algunas propiedades del conjunto de Cantor son:
\begin{enumerate}
    \item[(1)] $0 \in C$. De hecho, pertenecen a $C$ los extremos de todos los intervalos cerrados que forman $C_n$ cualquiera que sea $n$.
    \item[(2)] $C$ es compacto.

          Es claro que es acotado y es cerrado pues es la intersección de cerrados (cada $C_n$ es cerrado).
    \item[(3)] $m(C) = 0$

          En efecto, $m(C) \leq m(C_n) = \frac{2^n}{3^n}$ para todo $n \in \mathbb{N}$. Tomando límites, $m(C) = 0$.
    \item[(4)] $C$ no contiene intervalos ([0,1] $\backslash C$ es denso en [0,1]).
    \item[(5)] $C$ no tiene puntos aislados.
    \item[(6)] $C$ no es numerable ($\#C = \#\mathbb{R}$).

          Sea $\{ 0,1 \}^{\mathbb{N}} = \{ f: \mathbb{N} \longrightarrow \{0,1\}\} = \{ \{a_n\}_{n=1}^{\infty} : a_n = 0 \text{ ó } a_n = 1\}$
          Ahora
          \begin{align*}
              \sigma: C & \longrightarrow \{ 0,1 \}^{\mathbb{N}} \\
              x \in C   & \longmapsto \{ 0,1,0,1,1,0,1,...\}
          \end{align*}
          donde 0 si $x$ está en el ntervalo de la izquierda y 1 si $x$ está en el intervalo de la derecha. $\sigma$ es  biyectiva.
          \begin{align*}
              \bar{\sigma}: \{ 0,1 \}^{\mathbb{N}} & \longrightarrow [0,1]                              \\
              \{a_n\}_{n=1}^{\infty}               & \longmapsto t = \sum_{n=1}^{\infty}\frac{a_n}{2^n}
          \end{align*}
          y $\bar{\sigma}$ es sobreyectiva.
          \begin{align*}
              \bar{\sigma} \circ{} \sigma: C \longrightarrow [0,1]
          \end{align*}
          es sobreyectiva
          \begin{align*}
              \bar{\bar{\sigma}}: C & \longrightarrow [0,1] \\
              x                     & \longmapsto x
          \end{align*}
          es inyectiva, luego $\#C = \#[0,1] = \#\mathbb{R}$.
    \item[(7)] Del punto anterior se sigue que $\#(\mathcal{P}(C)) = \#(\mathcal{P}(\mathbb{R}))$. Por otra parte, se sabe que $\#\mathcal{B}_{\mathbb{R}^n} = \#\mathbb{R}$. Luego $\#\mathcal{B}_{\mathbb{R}^n} < \#(\mathcal{P}(C))$. Esto implica que existe algún conjunto $F \subset C$ tal que $F \not \in \mathcal{B}_{\mathbb{R}^n}$ y sabemos que $C \in \mathcal{B}_{\mathbb{R}^n}$ y $m(C) = 0$. Por consiguiente, el espacio de medida $(\mathbb{R}^n, \mathcal{B}_{\mathbb{R}^n}, m)$ no es completo.
\end{enumerate}

\begin{ejemplo}
    El conjunto de Cantor generalizado. Sea $0 < \alpha < 1$. Como antes, partimos del intervalo $I = [0, 1]$ y suprimimos el intervalo central de longitud $\frac{\alpha}{3}$ y nos quedamos con un conjunto $C_{2,\alpha}$, que es unión de dos intervalos. En el segundo paso, de cada uno de los dos intervalos centrales suprimimos los intervalos centrales de longitud $\frac{\alpha}{3^2}$ y el conjunto que nos queda es $C_{2,\alpha}$. Seguimos así y consideramos el conjunto de Cantor generalizado $C_{\alpha} = \bigcap_{n=1}^{\infty}{C_{n,\alpha}}$ (si $\alpha = 1$, $C_{\alpha}$ es el conjunto de Cantor). ¿Qué propiedades tiene $C_{\alpha}$? ¿Qué medida tiene?
\end{ejemplo}

\subsection{Completación de ($\mathbb{R}^n, \mathcal{B}_{\mathbb{R}^n}, m$)}
El espacio ($\mathbb{R}^n, \mathcal{B}_{\mathbb{R}^n}, m$) no es completo. Si hacemos su completación se obtiene un nuevo espacio de medida ($\mathbb{R}^n, \overline{\mathcal{B}_{\mathbb{R}^n}}, \overline{m}$) que verifica las propiedades del Teorema \ref{teo:completar}
\begin{enumerate}
    \item[(a)] $\mathcal{B}_{\mathbb{R}^n} \subset \overline{\mathcal{B}_{\mathbb{R}^n}}$
    \item[(b)] $\overline{m}|_{\mathcal{B}_{\mathbb{R}^n}} = m = m^*|_{\mathcal{B}_{\mathbb{R}^n}}$
    \item[(c)] Si $(\mathbb{R}^n, \mathcal{N}, \mu)$ es otro espacio completo tal que $\mathcal{B}_{\mathbb{R}^n} \subset \mathcal{N}$ y $\nu |_{\mathcal{B}_{\mathbb{R}^n}} = m$ entonces $\overline{\mathcal{B}_{\mathbb{R}^n}} \subset \mathcal{N}$ y $\nu |_{\overline{\mathcal{B}_{\mathbb{R}^n}}} = \overline{m}$.

          A la $\sigma$-álgebra $\overline{\mathcal{B}_{\mathbb{R}^n}}$ se le denomina $\sigma$-álgebra de Lesbesgue u se denota $\mathcal{L}_n$, es decir,
          \begin{align*}
              \mathcal{L}_n := \overline{\mathcal{B}_{\mathbb{R}^n}}
          \end{align*}
          Recuérdese que, por definición
          \begin{align*}
              A \in \mathcal{L}_n \Longleftrightarrow \text{ existen } E, N \in \mathcal{B}_{\mathbb{R}^n} \text{ y } F \in N \text{ tales que } A = E \cup F \text{ y } m(N) = 0
          \end{align*}
          y que
          \begin{align*}
              m(A) = m(E)
          \end{align*}
          En realidad
    \item[(d)] $\overline{m}(A) = m^*(A)$ para todo $A \in \mathcal{L}_n$, es decir, $\overline{m} = m^*|_{\mathcal{L}_n}$. En efecto, supongamos que
          \begin{align*}
              A = E \cup F \text{, siendo } A,N \in \mathcal{B}_{\mathbb{R}^n} \text{ y } F \subset N \text{ y } m(N) = 0
          \end{align*}
          Entonces $A \subset E \cup N$. Por ser $\overline{m}$ una medida se sigue que
          \begin{align*}
              \overline{m}(A) = m(E) = m^*(E)
          \end{align*}
          Puesto que $E \subset A$, tenemos que $m^*(E) \leq m^*(A)$ y, por lo tanto,
          \begin{align*}
              \overline{m}(A) \leq m^*(E) \leq m^*(A)
          \end{align*}
          Por otra parte, usando de nuevo que $A \subset E \cup N$,
          \begin{align*}
              m^*(A) \leq m^*(E \cup N) = m(E \cup N) \leq m(E) + m(N) = m(E) = \overline{m}(A)
          \end{align*}
          donde la última desigualdad se ha aplicado la definición de $\overline{m}$. De ambas desigualdades se sigue que la afirmación contenida enn (d) es cierta.
          $\hfill{\square}$
\end{enumerate}
\begin{obs}
    \begin{enumerate}
        \item[(a)] $A \in \mathcal{L}_n$ si y sólo si existen $B, C \in \mathcal{B}_{\mathbb{R}^n}$ tales que $B \subset A \subset C$ y $m(C \backslash B) = 0$. A los conjuntos de $\mathcal{L}_n$ se les llama conjuntos medibles-Lebesgue.
        \item[(b)] Por lo que acabamos de enunciar, se puede decir que son casi-borelianos.
        \item[(c)] Abusando de la notación, a la medida $\overline{m} = m^*|_{\mathcal{L}_n}$ la denotaremos también por $m$ y la llamaremos medida de Lebesgue (por supuesto, deberíamos escribir $m_n$ puesto que hay una para cada dimensión).
    \end{enumerate}
\end{obs}