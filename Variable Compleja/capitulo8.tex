\chapter{Singularidades aisladas}

\section{Singularidades aisladas}

\begin{defi}
    Por una singularidad aislada de una función $f$ entendemos un punto $z_0$ de manera que $f$ está definida y es holomorfa en un entorno perforado de $z_0$, $\Delta(z_0,r) \backslash \{z_0\}$, sin ser a priori holomorfa en todo el entorno $\Delta(z_0,r)$.
\end{defi}

\begin{ejemplo}
    \begin{enumerate}
        \item Las funciones $\frac{1}{z}$, $\frac{\sen z}{z}$, $e^{1/z}$ presentan singularidades aisladas en 0, ya que están definidas y son holomorfas en un entorno perforado de 0.
        \item En principio, según la definición, si $f$ es holomorfa en $z_0$, también podría decir que $f$ presenta una singularidad aislada en $z_0$, aunque el interés es decir esto es nulo.
        \item Un primer resultado sobre singularidades aisladas ya lo vimos como consecuencia de la analiticidad de funciones holomorfas: Si $f$ es continua en un abierto $\Omega$ y $f$ es holomorfa en $\Omega \backslash \{p\}$, siendo $p \in \Omega$, entonces $f$ es holomorfa en $\Omega$. O sea, la singularidad de $p$ es evitable si $f$ es continua en $p$.
        \item Hay puntos que también podríamos llamar singularidades pero no aisladas. Por ejemplo, la función $f(z) = \frac{1}{\sen (1/z)}$ presenta singularidades aisladas en los puntos de la forma $a_n = \frac{1}{n\pi}$, $n \in \mathbb{Z} \backslash \{0\}$. El punto $a = 0$ también es ''singularidad'' de $f$, pero al ser límite de singularidades aisladas (no evitables), resulta que no es holomorfa en ningún entorno perforado de 0, por lo que 0 no puede ser singularidad aislada de $f$.
    \end{enumerate}
\end{ejemplo}

\begin{defi}
    Si $z_0$ es una singularidad de $f$, decimos que es evitable si $f$ admite una extensión holomorfa a todo un entorno de $z_0$. De esta manera, absusando de notación, la extensión holomorfa de $f$ en un entorno de una singularidad aislada evitable, también se suele denominar $f$.
\end{defi}

\begin{teo}[Teorema de Riemann sobre la singularidad evitable]
    Sea $z_0$ una singularidad aislada de $f$. Son equivalentes:
    \begin{enumerate}
        \item[(i)] $z_0$ es singularidad aislada evitable de $f$.
        \item[(ii)] $f$ admite extensión continua a $z_0$.
        \item[(iii)] Existe $\lim_{z \to z_0} f(z)$ y es finito.
        \item[(iv)] $f$ está acotada en un entorno perforado de $z_0$.
        \item[(v)] $\lim_{z \to z_0} (z-z_0)f(z) = 0$.
    \end{enumerate}
\end{teo}

\begin{proof}
    Las implicaciones $(i) \Longrightarrow (ii) \Longrightarrow (iii) \Longrightarrow (iv) \Longrightarrow (v)$ son inmediatas. Probemos $(v) \Longrightarrow (i)$.

    Supongamos que $f$ es holomorfa en $\Delta(z_0,R) \backslash \{z_0\}$ y que $\lim_{z \to z_0} (z-z_0)f(z) = 0$. Entonces la función $g: \Delta(z_0,R) \longrightarrow \com$, dada por
    \begin{align*}
        g(z) = \left\{ \begin{array}{lcc}
                           (z-z_0)f(z) & si & z \in \Delta(z_0,R) \backslash \{z_0\} \\
                           0           & si & z = z_0                                \\
                       \end{array}
        \right.
    \end{align*}
    está bien definida y es continua en $\Delta(z_0,R)$, y además es holomrfa en $\Delta(z_0,R) \backslash \{z_0\}$. Por tanto, tenemos que $g$ es holomorfa en $\Delta(z_0,R)$ y tiene un cero en $z_0$. Todo esto implica que $g$ admite una factorización del tipo $g(z) = (z-z_0)h(z)$, con $h$ holomrfa en $\Delta(z_0,R)$. Se sigue entonces que:
    \begin{align*}
        h(z) = \frac{(z-z_0)h(z)}{(z-z_0)} = \frac{g(z)}{z-z_0} = f(z), \ \ z \in \Delta(z_0,R) \backslash \{z_0\}
    \end{align*}
    probando de esta manera que $h$ es una extensión holomorfa de $f$ en $\Delta(z_0,R)$, y así, $z_0$ es singularidad evitable.
\end{proof}

\begin{defi}
    Sea $z_0$ una singularidad aislada de $f$.
    \begin{itemize}
        \item $z_0$ es evitable si existe $\lim_{z \to z_0} f(z)$ y es finito.
        \item $z_0$ es polo si existe $\lim_{z \to z_0} f(z)$ y vale $\infty$.
        \item $z_0$ es singularidad aislada esencial si no es aislada ni polo ($f$ no tiene límite en $z_0$).
    \end{itemize}
\end{defi}

\begin{ejemplo}
    \begin{enumerate}
        \item $f(z) = \frac{\sen z}{z}$ tiene una singularidad aislada evitable en $0$.
        \item $f(z) = \frac{1}{(z-z_0)^n}$ tiene un polo en $z = z_0$.
        \item $f(z) = e^{1/z}$ tiene una singularidad esencial en $z = 0$, pues el límite no existe, basta considerar:
              \begin{align*}
                   & \lim_{n \to \infty} f\left( \frac{1}{n} \right) = \lim_{n \to \infty} e^n  = \infty \\
                   & \lim_{n \to \infty} f\left(- \frac{1}{n} \right) = \lim_{n \to \infty} e^{-n}  = 0
              \end{align*}
    \end{enumerate}
\end{ejemplo}

\begin{teo}[Orden de un polo]
    Sea $z_0$ un polo de $f$. Entonces existe un primer natural $n_0$ tal que $f(z) = (z-z_0)^{n_0}g(z)$ tiene una singularidad aislada evitable en $z_0$ y, además, trás evitar la singularidad, $g$ no se anula en todo un entorno de $z_0$.

    Dicho primer natural se llama orden $z_0$ como polo de $f$.
\end{teo}

\begin{obs}
    \begin{enumerate}
        \item $f$ tiene un polo de orden $n_0$ en $z_0$ si y solo si $\frac{1}{f}$ tiene un cero de orden $n_0$ en $z_0$.
        \item Si $\Omega$ es abierto de $\com$, $z_0 \in \Omega$, y $f$ es holomorfa enn $\Omega \backslash \{z_0\}$, siendo $z_0$ polo de $f$ de orden $n_0$, entonces $g(z) = (z-z_0)^{n_0}f(z)$ es holomorfa en $\Omega$ con $g(z_0) \not = 0$.
        \item Si $z_0$ es singularidad aislada de $f$ que es evitable o polo, entonces existe $\lim_{z \to z_0} f(z)$ como valor en $\com^*$, así que definiendo $f(z_0) = \lim_{n \to z_0} f(z) \in \com^*$, obtenemos una extensión de $f$, continua con respecto a la topología de $\com^*$ en un entorno de $z_0$.
    \end{enumerate}
\end{obs}

\begin{teo}[Casorati-Weierstrass]
    Si $D$ es un dominio en $\com$ y $f$ es holomorfa en $D \backslash \{z_0\}$, siendo $z_0 \in D$ una singularidad aislada esencial de $f$, entonces para $r>0$ tal que $\Delta(z_0,r) \subset D$, se tiene que $f(\Delta(z_0,r) \backslash \{z_0\})$ es denso en $\com$.
\end{teo}

\section{Desarrollos de Laurent}
Sabemos que si $f$ es holomorfa en $z_0$, entonces $f$ es desarrollable en serie de potencias alrededor de $z_0$:
\begin{align*}
    f(z) = \sum_{n=0}^{\infty}{a_n(z-z_0)^n}
\end{align*}
Cuando $z_0$ es una singularidad aislada evitable o un polo de $f$, también obtenemos un desarrollo en serie de potencias de $(z-z_0)$ de la siguiente forma:
\begin{itemize}
    \item Si $z_0$ es singularidad evitable de $f$, la extensión holomorfa de $f$ en $z_0$, que la llamaremos nuevamente $f$, se encarga de proporcionarnos un desarrollo en serie de potencias ''no negativas'' de $(z-z_0)$:
          \begin{align*}
              f(z) = \sum_{n=0}^{\infty}{a_n(z-z_0)^n}
          \end{align*}
    \item Si $z_0$ es un polo de orden $n_0$ de $f$, entonces $(z-z_0)^nf(z)$ tiene una singularidad evitable en $z_0$, y $n_0$ es el primer natural con esta propiedad. Tras evitar la singularidad enn $z_0$, obtenemos
          \begin{align*}
              (z-z_0)^{n_0}f(z) = \sum_{n=0}^{\infty}{b_n(z-z_0)^n}
          \end{align*}
          De aquí se sigue que
          \begin{align*}
              f(z) & = \frac{1}{(z-z_0)^{n_0}}\sum_{n=0}^{\infty}{b_n(z-z_0)^n} = \sum_{n=0}^{\infty}{b_n(z-z_0)^{n-n_0}} \underset{k = n-n_0}{=} \sum_{k=-n_0}^{\infty}{a_k(z-z_0)^k} \\
                   & = \frac{a_{-n_0}}{(z-z_0)^{n_0}} + \frac{a_{-n_0 +1}}{(z-z_0)^{n_0-1}} + ... + \frac{a_{-1}}{(z-z_0)} + a_0 + a_1(z-z_0)+ ....
          \end{align*}
\end{itemize}
Cuando $z_0$ sea una singularidad aislada esencial de $f$, veremos aparecer infinitas potencias negativas de $(z-z_0)$.

\begin{teo}[Desarrollos de Laurent]
    Sea $a \in \com$, $0 \leq R_1 < R_2 \leq \infty$, y $f$ holomorfa en el anillo $A = A(a;R_1,R_2) = \{ z \in \com : R_1 < |z-a| < R_2\}$. Entonces $f$ admite un desarrollo, llamado desarrollo de Laurent de $f$ en $A$, de la forma:
    \begin{align*}
        f(z) = \sum_{n = -\infty}^{\infty}{a_n(z-a)^n}, \ \ z \in A
    \end{align*}
    siendo la convergencia de la serie absoluta y uniforme en cada compacto de $A$. Además, los coeficientes $a_n$ vienen dados por la fórmula:
    \begin{align*}
        a_n = \frac{1}{2\pi i} \int_{\gamma} {\frac{f(\xi)}{(\xi -a)^{n+1}} \ d \xi}
    \end{align*}
    donde $\gamma$ es cualquier ciclo de $A$ con $n(\gamma,a) = 1$.
\end{teo}

\begin{obs}
    Si $\{a_n\}_{n = -\infty}^{\infty}$ es una sucesión, decimos que la serie $\sum_{n=-\infty}^{\infty} a_n$ converge si las dos series $\sum_{n=0}^{\infty} a_n$, $\sum_{n=1}^{\infty} a_{-n}$ convergen. En tal caso, escribimos $\sum_{n=-\infty}^{\infty} a_n = \sum_{n=0}^{\infty} a_n  +\sum_{n=1}^{\infty} a_{-n}$. Decimos que la serie $\sum_{n=-\infty}^{\infty} a_n$ converge absolutamente si $\sum_{n=-\infty}^{\infty} |a_n|$ es convergente.

    Si $S$ es un conjunto, y para cada $n \in \mathbb{Z}$, $f_n$ es una función de $S$ en $\com$, decimos que la serie $\sum_{n=-\infty}^{\infty} f_n$ converge uniformemente en $S$ si las dos series  funcionales $\sum_{n=0}^{\infty} f_n$, $\sum_{n=1}^{\infty} f_{-n}$ convergen uniformemente en $S$. En tal caso, escribimos $\sum_{n=-\infty}^{\infty} f_n = \sum_{n=0}^{\infty} f_n  +\sum_{n=1}^{\infty} f_{-n}$.
\end{obs}

\begin{obs}
    \begin{enumerate}
        \item Una consecuencia de la demostración del teorema sobre desarrollos de Laurent, es que $f$ es descompone como $f = f_1 + f_2$, $f_1(z) = \sum_{n=-\infty}^{-1}a_n(z-a)^n$ y $f_2(z) = \sum_{n=0}^{\infty} a_n (z-a)^n$, siendo $f_1$ holomorfa en $\{ z \in \com : |z-a| > R_1 \}$ y $f_2$ holomorfa en $\{ z \in \com : |z-a| < R_2 \}$.
        \item Si $f$ tiene una singularidad aislada en $a$, entonces $f$ es holomorfa en un anillo de la forma $A(a;0,R)$, para algún $R > 0$, y admite un desarrollo de Laurent alrededor de $a$:
              \begin{align*}
                  f(z) = \sum_{n=-\infty}^{\infty}{a_n(z-a)^n}, \ \ z \in A(a;0,R)
              \end{align*}
              La serie de potencias negativas, $f_1 = \sum_{n=-\infty}^{-1}a_n(z-a)^n = P_{f,a}(z)$ se llama \textbf{parte principal del desarrollo de Laurent de $f$ en $a$}. Según el aspecto de esta parte prinicipal, obtenemos la siguiente caracterización de singularidades aisladas:
              \begin{enumerate}
                  \item $a$ es singularidad evitable de $f$ si y solo si $f_1 \equiv 0$.
                  \item $a$ es polo de $f$ de orden $n_0$ si y solo si $f_1$ es un polinomio de grado $n_0$ en la variable $\frac{1}{z-a}$.
                  \item $a$ es singlaridad aislada esencial de $f$ si y solo $f_1$ tiene infinitos sumandos.
              \end{enumerate}
        \item Si $f$ tiene una singularidad aislada en $a$, $f$ es holomorfa en $A(a;0,R)$, para algún $R > 0$, y su desarrollo de Laurent es de la forma
              \begin{align*}
                  f(z) = \sum_{n=-\infty}^{\infty}{a_n(z-a)^n}, \ \ z \in A(a;0,R)
              \end{align*}
              Si ahora $\gamma$ es un ciclo en $A(a;0,R)$ tal que $n(\gamma,a) = 1$, entonces
              \begin{align*}
                  a_{-1} = \frac{1}{2\pi i} \int_{\gamma} f(\xi) \ d \xi
              \end{align*}
              O sea, hablando coloquialmente, $a_{-1}$ es el único coeficiente del desarrollo de Laurent de $f$ en $a$ que sobrevive al integrar $f$ a lo largo de $\gamma$. Recibe el nombre de \textbf{residuo de $f$ en $a$}
              \begin{align*}
                  \boxed{
                      Res(f,a) = a_{-1} = \frac{1}{2\pi i} \int_{\gamma} f(\xi) \ d \xi
                  }
              \end{align*}
              cualquiera que sea el ciclo $\gamma$ en $A(a,0,R)$ con $n(\gamma,a) = 1$.
              \begin{enumerate}
                  \item Si $a$ es singularidad evitable de $f$, entonces $Res(f,a) = 0$.
                  \item Si $a$ es un polo de $f$ de orden $n_0$, entonces
                        \begin{align*}
                            f(z) = \sum_{n = -n_0}^{\infty}{a_n(z-a)^n}
                        \end{align*}
                        con lo que
                        \begin{align*}
                            (z-a)^{n_0}f(z) = \sum_{k = 0}^{\infty}{a_{k-n_0}(z-a)^k}
                        \end{align*}
                        De donde deducimos que
                        \begin{align*}
                            \boxed{
                            Res(f,a) = a_{-1} = \frac{1}{(n_0-1)!} \frac{d^{n_0-1}}{dz^{n_0 - 1}}\Big|_{z=a} \left[ (z-a)^{n_0}f(z)\right]
                            }
                        \end{align*}
              \end{enumerate}
    \end{enumerate}
\end{obs}

\section{El infinito}

\begin{defi}
    Sea $f$ una función holomorfa definida en un entorno de $\infty$, esto es, existe $R>0$ tal que $f$ está definida en $\{z \in \com : |z| > R \}$. Decimos que $f$ es holomorfa en $\infty$ si $f(1/z)$ tiene una singularidad evitable en $0$, o sea, si $f(1/z)$ es holomorfa en $0$. Decimos que $\infty$ es una singularidad aislada (evitable, polo, esencial) de $f$ si $0$ es singularidad aislada (evitable, polo, esencial) de $f(1/z)$.
\end{defi}

\begin{obs}
    \begin{enumerate}
        \item Si $f$ tiene una singularidad aislada en $\infty$, entonces $g(\xi) = f(1/\xi)$ tiene una singularidad aislada en $0$, con desarrollo de Laurent en $0$ del tipo:
              \begin{align*}
                  g(\xi) = \sum_{n=-\infty}^{\infty}{b_n \xi^n} = \sum_{n=-\infty}^{\infty}{b_{-n} \left(\frac{1}{\xi}\right)^n}
              \end{align*}
              y parte prinicipal
              \begin{align*}
                  g_1(\xi) = \sum_{n=-\infty}^{1}{b_n \xi^n} = \sum_{n=1}^{\infty}{b_{-n} \left(\frac{1}{\xi}\right)^n}
              \end{align*}
              De esta manera, podemos decir que el desarrollo de Laurent de $f$ en $\infty$ es como sigue
              \begin{align*}
                  f(z) = g(1/z) = \sum_{n=-\infty}^{\infty}{b_{-n}z^n} = \sum_{n=-\infty}^{\infty}{a_nz^n}
              \end{align*}
              y su parte prinicipal es
              \begin{align*}
                  f_1(z) = g_1(1/z) = \sum_{n=1}^{\infty}{a_nz^n}
              \end{align*}
              o sea, es una serie de potencias centrada en 0 con radio de convergencia $\infty$ (porque tiene que estar definida y ser holomorfa en un entorno ''perforado'' de $\infty$). En consecuencia:
              \begin{enumerate}
                  \item $\infty$ es singualaridad evitable si y solo si $f_1 \equiv 0$.
                  \item $\infty$ es polo de orden $n_0$ de $f$ si y solo si $f_1(z) = \sum_{n=1}^{n_0}{a_nz^n}$ es un polinomio de grado $n_0$.
                  \item $\infty$ es singularidad aislada esencial de $f$ si y solo si $f_1(z) = \sum_{n=1}^{\infty}{a_nz^n}$ es una serie de potencias alrededor de 0, con infinitos sumandos, y radio de convergencia $\infty$. En otras palabras, $\infty$ es singularidad aislada esencial de $f$ si y solo si $f_1$ es una función enntera distinta de un polinomio.
              \end{enumerate}
        \item En virtud de lo anterior:
              \begin{enumerate}
                  \item Todo polimomio de grado $N$ tiene un polo de orden $N$ en $\infty$.
                  \item $\frac{1}{z^N}$ es holomorfa en $\infty$ y tiene un cero de orden $N$ es $\infty$.
                  \item $e^z = \sum_{n=0}^{\infty}{\frac{z^n}{n!}}$ tiene una singularidad esencial en $\infty$, pues su parte principal es $\sum_{n=1}^{\infty}{\frac{z^n}{n!}} = e^z -1$, que es una función entera que no es un polinomio.
                  \item $e^{1/z} = \sum_{n=0}^{\infty}{\frac{1}{n!z^n}} = \sum_{n=-\infty}^{0}{\frac{z^n}{(-n)!}}$ tiene una singularidad aislada en $\infty$, con parte principal igual a $0$. Luego, $e^{1/z}$ es holomorfa en $\infty$ (y en $\infty$ vale $1$).
              \end{enumerate}
    \end{enumerate}
\end{obs}

\begin{defi}
    Si $f$ tiene una sigularidad aislada en $\infty$, y $R$ es tal que $f$ es holomorfa en $\{z \in \com : |z| > R\}$, definimos el residuo de $f$ en $\infty$ como
    \begin{align*}
        Res(f,\infty) = - \frac{1}{2\pi i}\int_{|\xi| = r}{f(\xi) \ d\xi}
    \end{align*}
    cualquiera que sea $r > 0$.
\end{defi}

\begin{obs}
    Una forma rápida para calcular el residuo de $f$ en $\infty$ es la siguiente:
    \begin{align*}
        Res(f,\infty) = - \frac{1}{2\pi i} \int_{|\xi| = r}{f(\xi) \ d\xi} \underset{\xi = 1/w}{=} \frac{1}{2\pi i}\int_{|w| = 1/r}{f\left(\frac{1}{w}\right)\left(-\frac{1}{w^2}\right) \ dw} = Res\left( -\frac{1}{w}f\left(\frac{1}{w} \right),0 \right)
    \end{align*}
\end{obs}

\begin{defi}
    Sea $\Omega$ abierto de $\com^*$ y sea $f: \Omega \longrightarrow \com^*$. Decimos que $f$ es meromorfa en $\Omega$ si $f$ es holomorfa en $\Omega$ salvo por polos (y singularidades evitables, como tales, trás evitarlas, dejan de ser singularidades)
\end{defi}

\begin{obs}
    Si $f$ es meromorfa en el abierto $\Omega$ de $\com^*$, entonces el conjunto de polos de $f$ no tiene puntos acumulación en $\Omega$. De lo contrario, $f$ tendría ''singularidades no aisladas'' en $\Omega$. Se sigue entonces que el conjunto de polos de $f$, además de no tener puntos de acumulación en $\Omega$, es a lo sumo numerable, y sus putos de acumulación están en $\partial_{\infty} \Omega$.
\end{obs}

\begin{teo}
    \begin{itemize}
        \item Si $f$ es holomorfa en $\com^*$, entonces $f$ es constante.
        \item Si $f$ es meromorfa en $\com^*$, entonces $f$ es una función racional.
    \end{itemize}
\end{teo}

\section{El teorema de los residuos}

\begin{prop}
    Sea $R$ una función racional. Entonces la suma de los residuos de $R$ es $0$.
\end{prop}

\begin{teo}[Teorema de los residuos]
    Si $f$ es holomorfa en un abierto $\Omega \subseteq \com$ excepto en $S \subset \Omega$, conjunto de singularidades aisladas (finitas) de $f$ (puede haber singularidades aisladas esenciales), entonces
    \begin{align*}
        \frac{1}{2\pi i}\int_{\gamma}{f(z) \ dz} = \sum_{a \in S}{n(\gamma,a)Res(f,a)}
    \end{align*}
    para todo ciclo $\gamma$ en $\Omega \backslash S$, homólogo a 0 módulo $\Omega$.
\end{teo}

\begin{obs}
    El teorema de los residuos engloba todos los resultados importantes vistos hasta ahora.
    \begin{enumerate}
        \item \underline{Teorema de Cauchy}: Si $f$ es holomorfa en el abierto $\Omega \subseteq \com$ y $\gamma$ es un ciclo en $\Omega$ homólogo a 0 módulo $\Omega$, entoces $\int_{\gamma}{f(\xi) \ d\xi} = 0$, pues $f$ no tiene singularidades aisladas.
        \item \underline{Fórmula integral de Cauchy}: Si $f$ es holomorfa en el abierto $\Omega \subseteq \com$, $z_0 \in \Omega$ y $\gamma$ es un ciclo en $\Omega \backslash \{z_0\}$ homólogo a 0 módulo $\Omega$, entonces el conjunto de singularidades de $g(z) = \frac{f(z)}{z-z_0}$, $z \in \Omega \backslash \{z_0\}$ es $S = \{z_0\}$, y observamos que $z_0$ es polo simple de $g$, por lo que $Res(g,z_0) = \lim_{z \to z_0} g(z)(z-z_0) = f(z_0)$. De ahí que
              \begin{align*}
                  \frac{1}{2\pi i}\int_{\gamma} \frac{f(\xi)}{\xi - z_0} \ d\xi = \frac{1}{2\pi i} \int_{\gamma} g(\xi) \ d\xi = n(\gamma,z_0)Res(g,z_0) = n(\gamma,z_0)f(z_0)
              \end{align*}
        \item \underline{Fórmula integral de Cauchy para la $n$-ésima derivada}: Si $f$ es holomorfa en el abierto $\Omega \subseteq \com$, $z_0 \in \Omega$, $n \in \mathbb{N}$, y $\gamma$ es un ciclo en $\Omega \backslash \{z_0\}$ homólogo a 0 módulo $\Omega$, entonces el conjunto de singularidades de $g(z) = \frac{f(z)}{(z-z_0)^{n+1}}$, $z \in \Omega \backslash \{z_0\}$ es $S = \{z_0\}$ y observamos que $z_0$ es un polo de $g$ de orden $n+1$. Teniendo en cuenta que el desarrollo de Taylor de $f$ en $z_0$, $f(z) = \sum_{k=0}^{\infty}{a_k (z-z_0)^k}$, nso da el desarrollo de Laurent de $g$ en $z_0$ :
              \begin{align*}
                  g(z) = (z-z_0)^{-n-1}f(z) = \sum_{k=0}^{\infty}{a_k(z-z_0)^{-n-1+k}}
              \end{align*}
              de donde, obtenemos que
              \begin{align*}
                  Res(g,z_0) = a_n = \frac{f^{(n)}(z_0)}{n!}
              \end{align*}
              De aquí se sigue que
              \begin{align*}
                  \frac{n!}{2\pi i} \int_{\gamma} \frac{f(\xi)}{(\xi - z_0)^{n+1}} \ d\xi = \frac{n!}{2\pi i} \int_{\gamma} g(\xi) \ d\xi = n!n(\gamma,z_0)Res(g,z_0) = n(\gamma,z_0)f^{(n)}(z_0)
              \end{align*}
    \end{enumerate}
\end{obs}

\section{Principio del argumento}

\begin{teo}[De la curva de Jordan]
    Sea $J$ el soporte de una curva de Jordan en $\com$. Entonces $\com \backslash J$ tiene exactamente 2 componentes conexas y $J$ es la frontera de ambas.
    \begin{itemize}
        \item A la componente acotada de $J$ se le llama \textbf{dominio interior de $J$}, y se denota por $I(J)$.
        \item A la componente no acotada de $J$ se le llama \textbf{dominio exterior de $J$}, y se denota por $E(J)$.
    \end{itemize}
\end{teo}

Otrs resultados que aceptaremos como válidos (pero que no demostraremos) son los siguientes:
\begin{enumerate}
    \item Si $\gamma$ es un camino de Jordan y $J = sop(\gamma)$, entonces $n(\gamma,z) = 0$ para todo $z \in E(J)$, (tambien escribiremos $n(\gamma,z) = n(J,z)$), mientras que $n(\gamma,z) = 1$ para todo $z \in I(J)$.
          \begin{itemize}
              \item Si $n(\gamma,z) = 1$ para todo $I(J)$, decimos que $J$ está orientado positivamente.
              \item Si $n(\gamma,z) = -1$ para todo $I(J)$, decimos que $J$ está orientado negativamente.
          \end{itemize}
    \item Cuando $J$ es el soporte de un camino de Jordan positivamente orientado, entonces $I(J)$ recibe el nombre de \textbf{dominio de Jordan}. Dicho de otra forma, un dominio $D \subseteq \com$ se dice que es un \textbf{dominio de Jordan}, si $\partial D$ es el soporte de un camino de Jordan positivamente orientado.
    \item Existen curvas de Jordan con área positiva.
\end{enumerate}

\begin{teo}[Principio del argumento]
    Sea $J$ el soporte de un camino de Jordan $\gamma$ positivamente orientado. Sea $D$ un dominio simplemete conexo que contiene a $I(J) \cup J$. Sea $f$ meromorfa en $D$ sin ceros ni polos en $J$, entonces
    \begin{align*}
        \frac{1}{2\pi i} \int_{\gamma} \frac{f'(\xi)}{f(\xi)} \ d\xi = (*) - (**)
    \end{align*}
    siendo
    \begin{enumerate}
        \item[(*)] el número de ceros de $f$ en $I(J)$ contando multiplicidades.
        \item[(**)] el número de polos de $f$ en $I(J)$ contando multiplicidades.
    \end{enumerate}
\end{teo}

\begin{obs}
    Bajo las hipótesis del teorema de los residuos, tenemos que
    \begin{align*}
        n(f \circ \gamma,0) = \frac{1}{2\pi i}\int_{f \circ \gamma} \frac{1}{w} \ dw = \frac{1}{2\pi} Var_{f \circ \gamma} (\arg(w)) = \frac{1}{2\pi} Var_{\gamma}(\arg(f))
    \end{align*}
\end{obs}

\begin{teo}[Propiedad recubridora local de las funciones holomorfas]
    Sea $f$ una función holomorfa en $z_0 \in \com$. Sea $w_0 = f(z_0)$ y sea $N \in \mathbb{N}$ el orden de $z_0$ como cero de $f-w_0$. Entonces $f$ es una aplicación $N \longleftrightarrow 1$ en un entorno de $z_0$, queriendo esto decir que existe $R > 0$, tal que si $f$ es holomorfa en $\Delta(z_0,R)$, y tal que para todo $r \in (o,R)$, existe $\delta > 0$ con la propiedad de que si $w \in A(w_0,\delta) \backslash \{w_0\}$, entonces existen $N$ puntos distintos $z_1(w),...,z_N(w) \in \Delta(z_0,r)$ con $f(z_j(w)) = w$, $j = 1,...,N$.
\end{teo}

\begin{cor}[Teoerma de la aplicación abierta]
    Si $f$ es meromorfa y no constante, entonces $f$ es una aplicación abierta (en la topologia de $\com^*$).
\end{cor}

\begin{obs}
    \begin{enumerate}
        \item Si $f$ es meromorfa y no constante en el abierto $\Omega$, entoncnes $f(\Omega)$ es abierto.
        \item Si $f$ es meromorfa y no constante en el dominio $D$, entonces  $f(D)$ es dominio.
    \end{enumerate}
\end{obs}

\begin{teo}[Teorema de Rouché]
    Sea $J$ el soporte de un camino de Jordan $\gamma$. Sea $D$ un dominio simplemente conexo en $\com$, satisfaciendo que $I(J) \cup J \subset D$. Sean $f,g$ holomorfas en $D$ y tales que
    \begin{align*}
        |f(z) - g(z)| < |g(z)|, \ \ \text{para todo } z \in J.
    \end{align*}
    Entonces $f$ y $g$ tienen el mismo números de ceros en $I(J)$ (por supuesto, contando multiplicidad).
\end{teo}

\begin{teo}[De Hurwitz (I)]
    Supongamos que $\{f_n\}$ es una sucesión de funciones holomorfas en un dominio $D$, que converge normalmente a una función $f$ (que sabemos que es holomorfa en $D$). Entonces, o bien $f \equiv 0$ en $D$, o bien cada vez que $z_0 \in D$ sea un cero de orden $N \ge 0$, existen $r_0 > 0$ y $n_0 \in \mathbb{N}$ con la propiedad que, para todo $n \ge n_0$ $f_n$ tiene exactamente $N$ ceros en $\Delta(z_0,r_0)$ contando multiplicidades. Es más, estos ceros convergenn a  $z_0$ en medida que $n \to \infty$.

    En particular, si cada $f_n$ carece de ceros y $f$ no es identicamente cero, entonces $f$ también carece de ceros.
\end{teo}

\begin{teo}[De Hurwitz (II)]
    Supongamos que $\{f_n\}$ es ua sucesión de funciones holomorfas e inyectivas en un dominio $D$, que converge normalmente a una función $f$ (que sabemos que es holomorfa en $D$). Entonces, obien $f$ es constante en $D$, o bien $f$ es inyectiva en $D$.
\end{teo}