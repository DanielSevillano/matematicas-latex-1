\chapter{Series en $\com$. Series de potencias}

Establezcamos cierta notación.
\begin{itemize}
    \item Para $a \in \mathbb{Z}$, $\mathbb{N}_a = [a,+\infty) \cap \mathbb{Z}$.
    \item Si $a,b \in \mathbb{R}$, $a \leq b$, entonces $[[a,b]] = [a,b] \cap \mathbb{Z}$.
    \item Si $a,b \in \mathbb{R}$, $a < b$, entonces $((a,b)) = (a,b) \cap \mathbb{Z}$.
\end{itemize}

\begin{defi}
    Sea $\{z_n\}_{n \ge 0} \subset \com$ una sucesión.
    \begin{itemize}
        \item La sucesión de sumas parciales asociada a $\{z_n\}$ es $\{S_n\}$ dada por $S_n = \sum_{k=0}^{n}{z_k}$.
        \item La serie asociada $\{z_n\}$ es la sucesión de sumas parciales asociada a $\{z_n\}$ denotada por $\sum_{n=0}^{\infty}{z_n}$.
        \item Decimos que la serie $\sum_{n=0}^{\infty}{z_n}$ converge, si la sucesón de sumas parciales converge. En tal caso, diremos que la sucesión es sumable y que su suma es
              \begin{align*}
                  \sum_{n=0}^{\infty}{z_n} = \lim_{n \to \infty}{S_n} = \lim_{n \to \infty}\sum_{k=0}^{n}{z_k}
              \end{align*}
    \end{itemize}
\end{defi}

\begin{obs}
    \begin{enumerate}
        \item El hecho de que la sucesión $\{z_n\}$ tenga primer término en 0, 1 ó $n_0$ es irrelevante, es decir, no afecta al carácter de la serie (pero sí a su suma).
        \item Si $\sum_{n=0}^{\infty}{z_k}$ converge a $Z \in \com$, entonces para cada $n \in \mathbb{N}_0$, la serie $\sum_{k = n+1}^{\infty}{z_k}$ también converge y la sucesión de sumas parciales es
              \begin{align*}
                  S_{n,N} = \sum_{k = n+1}^{N}{z_k} = S_N - S_n, \ \ N > n
              \end{align*}
              Observamos que
              \begin{align*}
                  \sum_{k = n+1}^{\infty}{z_k} = \lim_{N \to \infty}{S_{n,N}} = \lim_{N \to \infty}{S_N - S_n} = Z - S_n \equiv Z_n
              \end{align*}
              Si $n \to \infty$
              \begin{align*}
                  \lim_{n \to \infty}{Z_n} = \lim_{n \to \infty}{Z - S_n} = 0
              \end{align*}
        \item $\sum_{k=0}^{n}{z_k} = \sum_{k=0}^{n}{\re(z_k)} + i\sum_{k=0}^{n}{\im(z_k)}$.
        \item \underline{Criterio necesario}: Si $\sum_{k=0}^{\infty}{z_k}$ converge, entonces $\{z_n\} \to 0$.
        \item \underline{Criterio de Cauchy} $\sum_{k=0}^{\infty}{z_k}$ converge si y solo si $S_n$ converge si y solo si $S_n$ es de Cauchy.
        \item \underline{Linealidad}:
              \begin{align*}
                  \mathcal{S} = \{ \{z_n\}_{n \ge 0} : z_n \in \com \ \forall \ n \}
              \end{align*}
              es un espacio vectorial complejo. Es más
              \begin{align*}
                  S : \mathcal{S} & \longrightarrow \com                              \\
                  \{z_n\}         & \longmapsto S(\{z_n\}) = \sum_{n=0}^{\infty}{z_n}
              \end{align*}
              es una función lineal, es decir,
              \begin{itemize}
                  \item $\sum_{n=0}^{\infty}{(z_n + w_n)} = \sum_{n=0}^{\infty}{z_n} + \sum_{n=0}^{\infty}{w_n}$.
                  \item $\sum_{n=0}^{\infty}{\lambda z_n} = \lambda \sum_{n=0}^{\infty}{z_n}$.
              \end{itemize}
    \end{enumerate}
\end{obs}

\begin{defi}
    Sea $\{z_n\}_{n \ge 0}$ una sucesión en $\com$. Decimos que la serie $\sum_{n=0}^{\infty}{z_n}$ converge absolutamente si $\sum_{n=0}^{\infty}{|z_n|}$ es convergente.
\end{defi}

\begin{obs}
    \begin{enumerate}
        \item Si $\sum_{n=0}^{\infty}{z_n}$ converge absolutamente, entonces también converge y
              \begin{align*}
                  \left| \sum_{n=0}^{\infty}{z_n} \right| \leq \sum_{n=0}^{\infty}{|z_n|}
              \end{align*}
    \end{enumerate}
\end{obs}

\section{Convergencia puntual y uniforme de sucesiones de funciones}

\begin{defi}
    Sea $S$ un conjunto y sean $\{f_n\}$ una sucesión de funciones de $S$ en $\com$.
    \begin{itemize}
        \item Decimos que $f: S \longrightarrow \com$ es el límite puntual de $\{f_n\}$ en $S$, o que $\{f_n\}$ converge puntualemente a $f$ en $S$, si $\lim_{n \to \infty}{f_n} = f(x)$ para todo $x \in S$.
        \item Decimos que $f: S \longrightarrow \com$ es el límite uniforme de $\{f_n\}$ en $S$, o que $\{f_n\}$ converge uniformemente a $f$ en $S$ si, para cada $\varepsilon > 0$, existe $n_0 \in \mathbb{N}$ tal que si $n \ge n_0$ entonces $|f_n(x) - f(x)| < \varepsilon$ para todo $x \in S$.
    \end{itemize}
\end{defi}

\begin{obs}
    \begin{enumerate}
        \item Si $\{f_n\}$ converge uniformemente en $S$, entonces $\{f_n\}$ converge puntualmente en $S$.
        \item La convergencia puntual no implica (en general) la convergencia uniforme.
              \begin{ejemplo}
                  Sea $f_n : [0,1] \longrightarrow \com$, dadas por $f_n(x) = x^n$. Es claro que, fijado $x \in [0,1]$ se tiene que
                  \begin{align*}
                      \lim_{n \to \infty}{f_n(x)} = \lim_{n \to \infty}{x^n} = \left\{ \begin{array}{lcc}
                                                                                           0 & si & x \in [0,1) \\
                                                                                           1 & si & x = 1       \\
                                                                                       \end{array}
                      \right. = f(x)
                  \end{align*}
                  Por tanto, $\{f_n\}$ converge puntualmente a  $f$ en $[0,1]$, pero la convergencia no es uniforme (porque $f_n$ son todas continuas en $[0,1]$ y $f$ no es continua en $[0,1]$).
              \end{ejemplo}
        \item Como $\com$ es completo
              \begin{itemize}
                  \item $\{f_n\}$ converge puntualmente en $S$ si y solo si $\{f_n\}$ es puntualmente de Cauchy en $S$.
                  \item $\{f_n\}$ converge uniformemente a $f$ en $S$ si y solo si $\{f_n\}$ es uniformemente de Cauchy en $S$ si y solo para cada $\varepsilon > 0$ existe $n_0 \in \mathbb{N}$ tal que si $n,m \in \mathbb{N}$ con $n > m  \ge n_0$, entonces $|f_n(x) - f_m(x)| < \varepsilon$ para todo $x \in S$.
              \end{itemize}
    \end{enumerate}
\end{obs}

\begin{defi}
    Si $\sum_{n=0}^{\infty}{z_n}$ converge absolutamente, decimos que converge incondicionalmente si se satisfacen las siguientes propiedades
    \begin{enumerate}
        \item Si $\sigma : \mathbb{N}_0 \longrightarrow \mathbb{N}_0$ es una biyección, entonces $\sum_{n=0}^{\infty}{z_{\sigma(n)}}$ converge absolutamente y $\sum_{n=0}^{\infty}{z_n} = \sum_{n=0}^{\infty}{z_{\sigma(n)}}$.
        \item Para toda partición $\{A_n\}_{n \ge 1}$ de $\mathbb{N}_0$ se tiene que para cada $n$ tal que $A_n$ es infinito, $\sum_{k\in A_n}{z_n}$ converge absolutamente y además
              \begin{align*}
                  \sum_{n=0}^{\infty}{z_n} = \sum_{n \ge 1}\sum_{k\in A_n}{z_n}
              \end{align*}
    \end{enumerate}
\end{defi}

\begin{prop}
    Si $\sum_{n=0}^{\infty}{a_n}$ y $\sum_{n=0}^{\infty}{b_n}$ son absolutamente convergentes, entonces la serie "producto de Cauchy"
    \begin{align*}
        \sum_{n=0}^{\infty}{c_n}, \text{ siendo } c_n = \sum_{k=0}^{n}{a_kb_{n-k}}
    \end{align*}
    es también absolutamente convergente y
    \begin{align*}
        \sum_{n=0}^{\infty}{c_n} = \left( \sum_{n=0}^{\infty}{a_n}\right) \cdot \left( \sum_{n=0}^{\infty}{b_n}\right)
    \end{align*}
\end{prop}

\begin{prop}
    Sea $X$ un espacio topológico y sea $\{f_n\}$ una sucesión de funciones continuas de $X$ en $\com$. Supongamos que $f: X \longrightarrow \com$ es el límite uniforme de $\{f_n\}$ en $X$. Entonces $f$ es continua en $X$.
\end{prop}

\section{Series funcionales. Series de potencias}

\begin{defi}
    Sea $\{f_n\}_{n \ge 0}$ una sucesión de funciones complejas definidas sobre un conjunto $S$.
    \begin{itemize}
        \item La serie funcional asociada a $\{f_n\}$, y denotada $\sum_{n}^{\infty}{f_n}$, se define como la sucesión de las sumas parciales asociadas a $\{f_n\}$
              \begin{align*}
                  S_n = \sum_{k=0}^{n}{f_k(x)}, \ \ x \in S
              \end{align*}
        \item Decimos que $\sum_{n}^{\infty}{f_n}$ converge uniformemente (respectivamente puntualmente convergente) en $S$ si así lo hace la correspondiente sucesión de sumas parciales.
        \item Decimos que $\sum_{n}^{\infty}{f_n}$ converge absolutamente y uniformemente (respectivamente puntualmente) en $S$ si la serie asociada a $\{|f_n|\}$ es uniformemente (respectivamente puntualmente) convergente en $S$.
    \end{itemize}
\end{defi}

\begin{teo}[Criterio de Cauchy]
    $\sum_{n=0}^{\infty}{f_n}$ es uniformemente convergente en $S$ si y solo $\sum_{n=0}^{\infty}{f_n}$ es uniformemente de Cauchy en $S$ si y solo si para cada $\varepsilon > 0$, existe $n_0 \in \mathbb{N}$ tal que si $n,m \in \mathbb{N}$ con $n > m \ge n_0$, entonces
    \begin{align*}
        \sup_{x \in S}{\left| \sum_{k=m+1}^{n}{f_k(x)}\right|} < \varepsilon
    \end{align*}
\end{teo}

\begin{prop}[Criterio Mayorante de Weierstrass]
    Sea $S$ un conjunto y sea $\{f_n\}_{n \ge 0}$ una sucesión defunciones complejas definidas en $S$. Supongamos que
    \begin{enumerate}
        \item Existe una sucesión $\{M_n\}_{n \ge 0} \subset \mathbb{R}^+$ con $\sum_{n = 0}^{\infty}{M_n}$ convergente.
        \item Existe $N \in \mathbb{N}$ tal que $|f_n(x)| \leq M_n$ para todo $x \in S$ y todo $n \ge N$.
    \end{enumerate}
    Entonces $\sum_{n=0}^{\infty}{f_n}$ converge absoluta y uniformemente en $S$.
\end{prop}

\begin{defi}[Series de potencias]
    Una serie de potencias centrada en $a \in \com$ es una serie funcional del tipo
    \begin{align*}
        \sum_{n=0}^{\infty}{a_n(z-a)^n} = a_0 + a_1(z-a) + a_2(z-a)^2 + ...
    \end{align*}
\end{defi}

\begin{obs}
    En la definición anterior, en el caso de que $z = a$, consideramos que $0^0 = 1$ (por convenio). Además, si $z = a$, entonces $\sum_{n=0}^{\infty}{a_n(z-a)^n}$ converge y su valor es $a_0$.
\end{obs}

\begin{ejemplo}
    \underline{La serie geométrica}: $\sum_{k=0}^{\infty}{z^n}$.
    \begin{align*}
        \text{Sumas parciales}: S_n(z) = \sum_{k=0}^{n}{z^k} = \left\{ \begin{array}{lcc}
                                                                           \frac{1-z^{n+1}}{1-z} & si & z \not = 1 \\
                                                                           n+1                   & si & z = 1      \\
                                                                       \end{array}
        \right.
    \end{align*}
    Observamos que si $|z| < 1$, $S_n(z) \xrightarrow[n \to \infty]{} \frac{1}{1-z}$ y si $|z| > 1$, entonces $S_n(z)$ no converge. Por tanto, $\{S_n\}$ es puntualmente convergente a $\frac{1}{1-z}$ en $\mathbb{D} = \{|z| < 1\}$.

    Veamos que $\sum_{k=0}^{\infty}{z^n}$ converge absolutamente y uniformemente en cualquier compacto $K$ de $\mathbb{D}$.
    \begin{proof}
        Sea $K \subset \mathbb{D}$ compacto. Entonces exite $r \in (0,1)$ tal que $K \subset \Delta(0,r)$. Ahora, si $z \in K$ y $n \in \mathbb{N}$, observamos que $|z^n| = |z|^n \leq r^n$ y que $\sum_{k=0}^{\infty}{r^n} = \frac{1}{1-r}$ converge. Por tanto, por el Criterio Mayorante de Weierstrass, $\sum_{k=0}^{\infty}{z^n}$ converge absoluta y uniformemente en $K$.
    \end{proof}
\end{ejemplo}

\begin{defi}
    Sea $\{x_n\} \subset \mathbb{R}$.
    \begin{itemize}
        \item $\limsup_{n \to \infty}{\{x_n\}} = \lim_{n \to \infty}{\sup_{k}{\{x_k : k \ge n \}}}$
        \item $\liminf_{n \to \infty}{\{x_n\}} = \lim_{n \to \infty}{\inf_{k}{\{x_k : k \ge n \}}}$
    \end{itemize}
\end{defi}

\begin{teo}
    Sea $\sum_{n=0}^{\infty}{a_n(z-a)^n}$ una serie de potencias centrada en $a \in \com$. Definimos
    \begin{align*}
        R = \frac{1}{\limsup_{n \to \infty}{\sqrt[n]{|a_n|}}}
    \end{align*}
    Entonces
    \begin{enumerate}
        \item[a)] Si $z \in \Delta(a,R)$ entonces $\sum_{n=0}^{\infty}{a_n(z-a)^n}$ converge absolutamente.
        \item[b)] Si $|z-a| > R$, la serie de potencias no converge
        \item[c)] $\sum_{n=0}^{\infty}{a_n(z-a)^n}$ converge absoluta y uniformemente en cada subconjunto compacto de $\Delta(a,R)$.
    \end{enumerate}
\end{teo}

\begin{proof}
    $b)$ Tiene que ser $R < \infty$. Ahora, si $|z-a| > R$, entonces existe $r > 0$ tal que
    \begin{align*}
        R < r < |z-a| \Longleftrightarrow \frac{1}{|z-a|} < \frac{1}{r} < \frac{1}{R} = \limsup_{n \to \infty}{\sqrt[n]{|a_n|}}
    \end{align*}
    Entonces existe $\varphi : \mathbb{N} \longrightarrow \mathbb{N}$ creciente tal que
    \begin{align*}
        \sqrt[\varphi(n)]{\left|a_{\varphi(n)}\right|} > \frac{1}{r} \Longleftrightarrow \left|a_{\varphi(n)}\right| > \frac{1}{r^{\varphi(n)}}, \ \ \ \forall n \in \mathbb{N}
    \end{align*}
    Por tanto
    \begin{align*}
        \left| a_{\varphi(n)}(z-a)^{\varphi(n)} \right| = \left| a_{\varphi(n)}\right|\left| z-a \right|^{\varphi(n)} > \frac{r^{\varphi(n)}}{r^{\varphi(n)}} = 1
    \end{align*}
    Luego, $\sum_{n=0}^{\infty}{a_n(z-a)^n}$ no converge.

    $a)$ \text{ y } $c)$ Tiene que ser $R > 0$. Sea $K$ un compacto en $\Delta(a,R)$. Entonces existe $r \in (0,R)$ tal que $K \subset \Delta(a,r)$. Sea $\rho \in (r,R)$. Observamos que
    \begin{align*}
        r < \rho < R \Longleftrightarrow \frac{1}{R} < \frac{1}{\rho} < \frac{1}{r}
    \end{align*}
    y que
    \begin{align*}
        \frac{1}{R} = \limsup_{n \to \infty}{\sqrt[n]{|a_n|}}
    \end{align*}
    Entonces existe $n_0 \in \mathbb{N}$ tal que si $n \ge n_0$
    \begin{align*}
        \sqrt[n]{|a_n|} < \frac{1}{\rho} \Longleftrightarrow |a_n| < \frac{1}{\rho^n}
    \end{align*}
    Ahora, si $n \ge n_0$
    \begin{align*}
        \left| a_n(z-a)^n \right| = \left| a_n \right| \left| z-a \right|^n \leq \frac{r^n}{\rho^n} = \left( \frac{r}{\rho} \right)^n
    \end{align*}
    y $\sum_{n=0}^{\infty}{\left( \frac{r}{\rho} \right)^n}$ converge, pues $0 <  \frac{r}{\rho} < 1$. Así que, por el Criterio Mayorante de Weierstrass, la serie  $\sum_{n=0}^{\infty}{a_n(z-a)^n}$ converge absoluta y uniformemente en $K$.
\end{proof}

\begin{obs}
    \begin{enumerate}
        \item Sea $\sum_{n=0}^{\infty}{a_n(z-a)^n}$ una serie de potencias. Si $R > 0$ entonces dicha serie define una función continua en $\Delta(a,R)$.
        \item \underline{Otra fórmula para $R$}: Sea
              \begin{align*}
                  A = \{ r \ge 0 : \{|a_n|r^n\} \text{ es acotada}\}
              \end{align*}
              Entonces $R = \sup(A)$.
              \begin{proof}
                  Sea $R_0 = \sup(A)$. Veamos que $R \ge R_0$. Supongamos que $R_0 > 0$ (si $R_0 = 0$ no hay que probar nada). Sea $\rho \in (0,R_0)$. Entonces $\{|a_n|\rho^n\}$ es acotada, digamos por $M$, por tanto
                  \begin{align*}
                      |a_n|^{\frac{1}{n}} \leq \frac{M^{\frac{1}{n}}}{\rho}, \ \ n \in \mathbb{N}
                  \end{align*}
                  Luego
                  \begin{align*}
                      \limsup{\sqrt[n]{|a_n|}} \leq \frac{1}{\rho} \Longrightarrow \rho \leq \frac{1}{\limsup{\sqrt[n]{|a_n|}}} = R
                  \end{align*}
                  Como $\rho < R_0$ es aribitrario, entonces $R_0 \leq R$.

                  Veamos que $R \leq R_0$. Supongamos que $R > 0$ (Si $R = 0$ no hay que probar nada). Sea $\rho \in (0,R)$ Entonces
                  \begin{align*}
                      \frac{1}{R} = \limsup{\sqrt[n]{|a_n|}} < \frac{1}{\rho}
                  \end{align*}
                  Entonces, existe $n_0 \in \mathbb{N}$ tal que $\sqrt[n]{|a_n|} < \frac{1}{\rho}$ para todo $n \ge n_0$, lo que nos dice que si $n \ge n_0$ entonces
                  \begin{align*}
                      |a_n| < \frac{1}{\rho^n} \Longrightarrow |a_n|\rho^n < 1 \Longrightarrow \{|a_n|\rho^n \} \text{ es acotada}
                  \end{align*}
                  Por tanto, $\rho \in A$ y $\rho \leq \sup(A) = R_0$. Como $\rho < R$ es arbitrario, concluimos que $R \leq R_0$.
              \end{proof}
        \item \underline{Otra fórmula para $R$}:
              \begin{align*}
                  \text{Si} \ \ \lim_{n \to \infty}{\frac{|a_{n+1}|}{|a_n|}} = \rho, \ \ \text{entonces } \ \ R = \frac{1}{\rho}
              \end{align*}
              \begin{proof}
                  Observamos que
                  \begin{itemize}
                      \item Podemos escribir
                            \begin{align*}
                                |a_n|^{\frac{1}{n}} = e^{\frac{\log|a_n|}{n}}
                            \end{align*}
                      \item $\{n\}$ es una sucesión creciente hacia $\infty$.
                      \item
                            \begin{align*}
                                \frac{\log|a_{n+1}| - \log|a_n|}{(n+1) -n} = \log\left| \frac{a_{n+1}}{a_n} \right| \xrightarrow[n \to \infty]{} \log(\rho)
                            \end{align*}
                  \end{itemize}
                  Luego
                  \begin{align*}
                      \lim_{n \to \infty}{e^{\frac{\log|a_n|}{n}}} = \lim_{n \to \infty}{e^{\rho}} = \rho
                  \end{align*}
                  Por tanto, $R = \frac{1}{\rho}$.
              \end{proof}
    \end{enumerate}
\end{obs}

\begin{prop}
    Sean
    \begin{itemize}
        \item $f(z) = \sum_{n=0}^{\infty}{a_n(z - a)^n}$ con $R_f \ge R$.
        \item $g(z) = \sum_{n=0}^{\infty}{b_n(z - a)^n}$ con $R_g \ge R$.
    \end{itemize}
    Entonces
    \begin{itemize}
        \item $(f + g)(z) = \sum_{n=0}^{\infty}{(a_n + b_n)(z - a)^n}$ es serie de pontencias con $R_{(f+g)} \ge R$.
        \item $(f \cdot g)(z) = \sum_{n=0}^{\infty}\left( \sum_{k=0}^{n} \binom{n}{k}a_kb_{n-k} \right)(z-a)^n$ es una serie de potencias con $R_{(f \cdot g)} \ge R$.
    \end{itemize}
\end{prop}

\begin{prop}
    Sea $f(z) = \sum_{n=0}^{\infty}{a_n(z - a)^n}$ una serie de potencias con radio de convergencia $R_a > 0$. Sea $b \in \Delta(a,R_a)$ y sea $r_b = R_a - |b-a|$. Entonces existe una serie de potencias $g(z) = \sum_{n=0}^{\infty}{b_n(z - b)^n}$, centrada en $b$, con radio de convergencia $R_b \ge r_b$ tal que $f(z) = g(z)$ para cada $z \in \Delta(b,r_b)$.
\end{prop}

\begin{proof}
    Sea $z \in \Delta(b,r_b)$. Entonces
    \begin{align*}
        f(z) & = \sum_{n=0}^{\infty}{a_n(z - a)^n} = \sum_{n=0}^{\infty}{a_n[(z-b) + (b-a)]^n} = \sum_{n = 0}^{\infty}\left( a_n \sum_{k=0}^{n} \binom{n}{k}(z-b)^k(b-a)^{n-k}\right) \\
             & = \sum_{n=0}^{\infty} \sum_{k=0}^{n} {a_n\binom{n}{k}(z-b)^k(b-a)^{n-k}} = \sum_{k=0}^{\infty} \left( \sum_{n=k}^{\infty} {a_n\binom{n}{k}(b-a)^{n-k}}\right)(z-b)^k
    \end{align*}
    Definiendo $b_k = \sum_{n=k}^{\infty} {b_k\binom{n}{k}(b-a)^{n-k}}$, tenemos que $g(z) = \sum_{k=0}^{\infty}{b_k(z-b)^k}$ es serie de potencias centrada en $b$ con radio de convergencia $R_b \ge r_b$.
\end{proof}

\begin{cor}
    De hecho, $f(z) = g(z)$ para cada $z \in \Delta(a,R_a) \cap \Delta(b,R_b)$.
\end{cor}

\begin{defi}
    Sea $\Omega \subseteq \com$ abierto. Decimos que $f: \Omega \longrightarrow \com$ es análitica en $\Omega$, si para cada $a \in \Omega$, $f$ se puede expresar en forma de serie de potencias alrededor de $a$.
\end{defi}

\begin{obs}
    Toda serie de potencias es análitica en su disco de convergencia.
\end{obs}

\begin{teo}[Diferenciabilidad de las series de potencias]
    Sea $f(z) = \sum_{n=0}^{\infty}{a_n(z-a)^n}$ una serie de potencias con radio de convergencia $R > 0$. Entonces
    \begin{enumerate}
        \item[a)] Para cada $k \in \mathbb{N}$, la serie de potencias
              \begin{align*}
                  \sum_{n=k}^{\infty}{n(n-1)...(n-k+1)a_n(z-a)^{n-k}}
              \end{align*}
              es una serie de potencias centrada en $a$ de radio de convergencia $R$.
        \item[b)] $f$ es infinitamente derivable en $\Delta(a,R)$ y para cada $k \in \mathbb{N}_0$ y para cada $z \in \Delta(a,R)$
              \begin{align*}
                  f^{(k)}(z) = \sum_{n=k}^{\infty}{n(n-1)...(n-k+1)a_n(z-a)^{n-k}}
              \end{align*}
        \item[c)] Para cada $k \in \mathbb{N}_0$, $f^{(k)}(a) = k! a_k$.
    \end{enumerate}
\end{teo}

\begin{defi}
    Si $f$ es infinitamente derivable en $a$, entonces su serie de Taylor centrada en a es
    \begin{align*}
        \sum_{n=0}^{\infty}{\frac{f^{(n)}(a)}{n!}(z-a)^n}
    \end{align*}
\end{defi}

\begin{proof}
    $c)$ Se tiene de forma directa de $a)$ y $b)$.

    $a)$ y $b)$ Basta probarlo para $k = 1$ (el resto se haría por inducción sobre $k$)
    \begin{align*}
        (a_1) \ \sum_{n=1}^{\infty}{na_n(z-a)^{n-1}} = \sum_{n=0}^{\infty}{(n+1)a_{n+1}(z-a)^n}
    \end{align*}
    Observamos que
    \begin{align*}
        \limsup_{n \to \infty}{\sqrt[n]{|(n+1)a_{n+1}|}} = \limsup_{n \to \infty}{(n+1)^{\frac{1}{n}} \cdot \left( |a_{n+1}|^{\frac{1}{n+1}}\right)^{\frac{n+1}{n}}} = 1 \cdot \frac{1}{R} = \frac{1}{R}
    \end{align*}
    $(b_1)$ $g(z) = \sum_{n=1}^{\infty}{na_n(z-a)^{n-1}}$ converge absoluta y uniformemente en cada compacto contenido en $\Delta(a,R)$. Veamos que $f'(z) = g(z)$ para cada $z \in \Delta(a,R)$. Sea $z_0 \in \Delta(a,R)$, tenemos que probar que
    \begin{align*}
        \lim_{n \to z_0}{\frac{f(z) - f(z_0)}{z-z_0} - g(z_0) = 0}
    \end{align*}
    Sea $\varepsilon > 0$. Como $z_0 \in \Delta(a,R)$, entonces existe $r > 0$ tal que $z_0 \in \overline{\Delta(a,r)} \subset \Delta(a,R)$. También existe $\rho > 0$ ($f = r - |z_0 -a|$) tal que
    \begin{align*}
        z_0 \in \overline{\Delta(a_0,\rho)} \subset \Delta(a,r) \subset \Delta(a,R)
    \end{align*}
    Ahora, para $z \in \Delta(z_0,\rho)$, $z \not = z_0$
    \begin{align*}
        \frac{f(z) - f(z_0)}{z-z_0} - g(z_0) & = \frac{1}{z-z_0}\left( \sum_{n=1}^{\infty}{a_n((z-a)^n - (z_0 - a)^n)}\right) - \sum_{n=1}^{\infty}{na_n(z_0 -a)^{n-1}}                           \\
                                             & = \sum_{n=1}^{\infty}{a_n \sum_{k=0}^{n-1}{(z-a)^{n-1-k}(z_0 -a)^k}} - \sum_{n=1}^{\infty}{na_n(z_0-a)^{n-1}}                                      \\
                                             & = \left( \sum_{n=1}^{N} + \sum_{n=N+1}^{\infty}\right)a_n\left( \frac{(z-a)^{n} - (z_0-a)^n}{z-z_0} -n(z_0-a)^{n-1}\right) \equiv I_N(z) + II_N(z)
    \end{align*}
    donde lo de dentro del paréntesis es
    \begin{align*}
        \sum_{n=1}^{N}{a_n\left( \frac{(z-a)^n - (z_0-a)^n}{z-z_0} - n(z_0-a)^{n-1}\right)} + \sum_{ n= N+1}^{\infty}{a_n\left( \frac{(z-a)^n - (z_0-a)^n}{z-z_0} -n(z_0-a)^{n-1}\right)}
    \end{align*}
    (hemos usado que $x^n - y^n = (x-y)(x^{n-1} + +x^{n-2}y + ... + xy^{n-1}  y^{-1})$).

    \underline{Empezamos con $II_N$}: Teemos convergencia absoluta
    \begin{align*}
        \left| II_N(z) \right| & \leq \sum_{n = N +1}^{\infty}{|a_n| \cdot \left| \frac{(z-a)^n - (z_0-a)^n}{z-z_0} \right| + \left| n(z_0-a)^{n-1}\right|} \\\
                               & \leq \sum_{n = N +1}^{\infty}{|a_n|\left[ \sum_{k=0}^{n-1}{|z-a|^{n-1-k} \cdot |z_0-a|^{k} + n|z_0-a|^{n-1}}\right]}
    \end{align*}
    Observamos que
    \begin{itemize}
        \item $|z-a|^{n-1-k} \leq \rho{n-1-k} \leq r^{n-1-k}$.
        \item $|z_0-a|^k \leq r^k$.
    \end{itemize}
    Luego
    \begin{align*}
        \left| II_N(z) \right|  \leq \sum_{n=N+1}^{\infty}{n|a_n|\left( r^{n-1} + r^{n-1}\right)} \leq \sum_{n = N+1}^{\infty}{2n|a_n|r^{n-1}}
    \end{align*}
    Por el Criterio Mayorante de Weierstrass, $|II_N(z)|$ converge uniformemente en $\Delta(z_n,\rho)\backslash \{z_0\}$ porque $\sum_{n = N+1}^{\infty}{2n|a_n|r^{n-1}}$ converge. Así,existe $N_0 \in mathbb{N}$ tal que
    \begin{align*}
        |II_{N_0}(z)| < \frac{\varepsilon}{2}
    \end{align*}
    \underline{Pasamos a estimar $I_{N_0}$}:
    \begin{align*}
        I_{N_0}(z) = \sum_{n=1}^{N_0}{a_n\left( \frac{(z-a)^n - (z_0-a)^n}{z-z_0} - n(z_0-a)^{n-1}\right)} \xrightarrow[z \to z_0]{} 0
    \end{align*}
    Por tanto, existe $\delta > 0$, que podemos suponnenr $\delta < \rho$ tal que
    \begin{align*}
        \left| I_{N_0}(z) \right| < \frac{\varepsilon}{2}, \ \ \forall z \in \Delta(z_0,\delta)\backslash \{z_0\}
    \end{align*}
    De esta manera, si $z \in \Delta(z_0,\delta)\backslash \{z_0\}$
    \begin{align*}
        \left| \frac{f(z) - f(z_0)}{z-z_0} - g(z_0) \right| \leq \left| I_{N_0}(z) \right| + \left| II_{N_0}(z) \right| < \frac{\varepsilon}{2} + \frac{\varepsilon}{2} = \varepsilon
    \end{align*}
\end{proof}

\begin{obs}
    \begin{itemize}
        \item Si $f$ es análitica, entonces $f \in \mathcal{C}^{\infty}$.
        \item En $\mathbb{R}$. Si $f \in \mathcal{C}^{\infty}$, entonces $f$ no tiene por qué ser analítica. Por ejemplo
              \begin{align*}
                  f(x) = \left\{ \begin{array}{lcc}
                                     e^{-1/x^2} & si & x \not = 0 \\
                                     0          & si & x = 0      \\
                                 \end{array}
                  \right.
              \end{align*}
              Resulta que $f \in \mathcal{C}^{\infty}$, $f^{(n)}(0) = 0$. Su serie de Taylor es
              \begin{align*}
                  \sum_{n=0}^{\infty}{\frac{f^{(n)}(0)}{n!}x^n = 0}
              \end{align*}
              que no es $f$ en un en entorno del 0. Sin embargo, en $\com$, ser $\mathcal{C}^{\infty}$ si implica ser analítica.
    \end{itemize}
\end{obs}

\begin{obs}
    La función
    \begin{align*}
        f(z) = \left\{ \begin{array}{lcc}
                           e^{-1/z^2} & si & z \not = 0 \\
                           0          & si & z = 0      \\
                       \end{array}
        \right.
    \end{align*}
    no es ni siquiera continua en 0.
\end{obs}

\begin{ejemplo}
    Sea $f(z) = \frac{1}{z}$, $z \in \comz$, que es holomorfa en $\comz$. ¿Serie de potencias de $f$ centrada en 1? Si existe, es la serie de Taylor de $f$ en 1, es decir,
    \begin{align*}
        \sum_{n=0}^{\infty}{\frac{f^{(n)}(1)}{n!}(z-1)^n}
    \end{align*}
    ¿Como calcular $f^{(n)}(1)$ rápido?
    \begin{align*}
        \frac{1}{z} = \frac{1}{1+z-1} = \frac{1}{1+(z-1)} = \sum_{n=0}^{\infty}{(-(z-1))^n} = \sum_{n=0}^{\infty}{(-1)^n(z-1)^n}
    \end{align*}
    que converge si $z \in \Delta(1,1)$. Por tanto $f^{(n)}(1) = n!(-1)^n$.
\end{ejemplo}

\section{Las funciones trigonométricas}
En analogía con el caso real
\begin{align*}
     & \sen(z) = \sum_{n=0}^{\infty}{(-1)^n\frac{z^{2n+1}}{(2n+1)!}}, \ \ z \in \com \\
     & \cos(z) = \sum_{n=0}^{\infty}{(-1)^n\frac{z^{2n}}{(2n)!}}, \ \ z \in \com
\end{align*}
Con esto, es fácil ver que
\begin{align*}
     & \cos z = \frac{e^{iz}+e^{-iz}}{2} = \cosh z            \\
     & \sen z = \frac{e^{iz}-e^{-iz}}{2i} = \frac{\senh z}{i}
\end{align*}
\underline{Propiedades}:
\begin{enumerate}
    \item Es fácil comprobar que para cada $z \in \com$,
          \begin{itemize}
              \item $\sen'(z) = \cos(z)$.
              \item $\cos'(z) = -\sen(z)$.
          \end{itemize}
    \item
          \begin{align*}
              \cos(z) + i\sen(z) & = \sum_{n=0}^{\infty}{(-1)^n\frac{x^{2n}}{(2n)!}} + i\sum_{n=0}^{\infty}{(-1)^n\frac{x^{2n+1}}{(2n+1)!}}  \\
                                 & = \sum_{n=0}^{\infty}{i^{2n}\frac{x^{2n}}{(2n)!}} + \sum_{n=0}^{\infty}{i^{2n+1}\frac{x^{2n+1}}{(2n+1)!}} \\
                                 & = \sum_{k=0}^{\infty}{\frac{(iz)^k}{k!}} = e^{iz}
          \end{align*}
    \item $\cos^2(z) + \sen^2(z) = 1$, para cada $z \in \com$.
    \item Es fácil comprobar que para cada $z \in \com$,
          \begin{itemize}
              \item $\cos(-z) = \cos(z)$ y $\sen(-z) = -\sen(z)$.
              \item $\cos(z + 2\pi) = \cos(z)$ y $\sen(z +2\pi) = \sen(z)$.
          \end{itemize}
    \item Para cada $z,w \in \com$
          \begin{align*}
              \cos(z+w) & = \frac{e^{i(z+w)} + e^{-i(z+w)}}{2} = \frac{1}{2}\left( e^{iz}e^{iw} + e^{-iz}e^{-iw}\right)        \\
                        & = \frac{1}{2}\left( (\cos z + \sen z)(\cos w + i\sen w) + (\cos z -i\sen z)(\cos w - i\sen w)\right) \\
                        & = ... = \cos z \cos w - \sen z \sen w                                                                \\
              \sen(z+w) & = ... = \sen z \cos w + \cos z \sen w
          \end{align*}
    \item $\sen$ y $\cos$ no son acotadas en $\com$. Sea $x + iy \in \com$, entonces
          \begin{align*}
              \sen(x+iy)   & = \sen x \cos(iy) + \cos x \sen(iy) = \sen x \frac{e^{iiy} + e^{-iiy}}{2} + \cos x\frac{e^{iiy}-e^{-iiy}}{2i} \\
                           & = \sen x \cosh y + i \cos x \senh y                                                                           \\
              \cos(x + iy) & = ... = \cos x \cosh y - i\sen x \senh y
          \end{align*}
          Con esto, tenemos que
          \begin{align*}
              |\sen(x+iy)|^2 & = \sen^2 x \cosh^2 y + \cos^2 x \senh^2 y = \sen^2 x (1 + \senh^2 y) + \cos^2 x \senh^2 y \\
                             & = \sen^2 x + \senh^2 y                                                                    \\
              |\cos(x+iy)|^2 & = \cos^2 x + \senh^2 y
          \end{align*}
    \item ¿Ceros de $\sen$ y $\cos$? Sea $x +iy \in \com$, entonces
          \begin{align*}
              \sen(x+iy) = 0   & \Longleftrightarrow \sen^2 x + \senh^2 y = 0 \Longleftrightarrow \left\{ \begin{array}{lcc}
                                                                                                              \sen x = 0  \\
                                                                                                              \senh y = 0 \\
                                                                                                          \end{array}
              \right. \Longleftrightarrow \left\{ \begin{array}{lcc}
                                                      x = k\pi, & k \in \mathbb{Z} \\
                                                      y = 0                        \\
                                                  \end{array}
              \right.                                                                                                       \\
                               & \Longleftrightarrow z = k\pi, \ k \in \mathbb{Z}                                           \\
              \cos(x + iy) = 0 & \Longleftrightarrow  z = \frac{\pi}{2} + k\pi, \ k \in \mathbb{Z}
          \end{align*}
    \item \underline{Visulización de la función $\sen$} : S
          \begin{itemize}
              \item Comportamiento de las líneas horizontales ($y = y_0$)
                    \begin{align*}
                         & \underline{y_0 = 0} \ \ \ \sen(z) = \sen(t), \text{ se reccore el segmento } [-1,1]                                       \\
                         & \underline{y_0 \not = 0} \ \ \ \sen(z) = \sen(t + iy_0) = \sen t \cosh y_0 + i\cos t \senh y_0, \text{ que es una elipse}
                    \end{align*}
              \item Comportamiento de las líneas verticales ($x = x_0$)
                    \begin{align*}
                         & \underline{x_0 = 0} \ \ \                                                                        \\
                         & \underline{x_0 \not = 0} \ \ \ \sen(z) = \sen(x_0 + ty_0) = \sen x_0 \cosh t + i\cos x_0 \senh t
                    \end{align*}
                    El caso de que $x_0 \not = 0$, tenemos que se describe una rama de la hipérbola de centro 0, vértices princales $\sen x_0 $ y $-\sen x_0$ y asíntocas de pendientes $\frac{\cos x_0}{\sen x_0}$ y $-\frac{\cos x_0}{\sen x_0}$
          \end{itemize}
    \item \underline{Inyectividad de la función $\sen$}: Sean $z,w \in \com$,, entonces
          \begin{align*}
              \sen z = \sen w & \Longleftrightarrow \frac{e^{iz}-e^{-iz}}{2i} = \frac{e^{iw}-e^{-iw}}{2i} \Longleftrightarrow e^{iz}-e^{-iz} = e^{iw}-e^{-iw} \\
                              & \Longleftrightarrow e^{iz}(1 - e^{i(w-z)}) = -e^{iw}(-e^{i(w-z)} + 1)                                                         \\
                              & \Longleftrightarrow e^{i(z+w)}(1 - e^{i(w-z)}) = -(-e^{i(w-z)} + 1)                                                           \\
                              & \Longleftrightarrow (1 + e^{i(w+z)})(1 - e^{i(w-z)}) = 0                                                                      \\
                              & \Longleftrightarrow \left\{ \begin{array}{lcc}
                                                                e^{i(w+z)} = -1 \\
                                                                \text{ó}        \\
                                                                e^{i(w-z)} = 1  \\
                                                            \end{array}
              \right. \Longleftrightarrow \left\{ \begin{array}{lcc}
                                                      w + z = \pi + 2k\pi, \ k \in \mathbb{Z} \textit{ (Simetría)} \\
                                                      \text{ó}                                                     \\
                                                      w - z = 2k\pi, \ k \in \mathbb{Z} \textit{ (Periodicidad)}   \\
                                                  \end{array}
              \right.                                                                                                                                         \\
                              & \Longleftrightarrow w \text{ y } z \text{ son simétricos respecto  } \frac{2k +1}{2}\pi
          \end{align*}

          \underline{Dominio de inyectividad de la función $\sen$}
          \begin{itemize}
              \item \textit{Periodicidad}: Nos restringimos a una  banda vertical de ancho máximo $2\pi$.
              \item \textit{Simetría}: Nos restringimos a
                    \begin{align*}
                        S = \left\{ z \in \com : -\frac{\pi}{2} < z < \frac{\pi}{2} \right\}
                    \end{align*}
          \end{itemize}
\end{enumerate}