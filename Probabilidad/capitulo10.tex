\chapter{Vectores aleatorios}

\section{Vectores aleatorios}

Sea $(\Omega, \mathcal{A}, P)$ un espacio de probabilidad y consideremos la función \textbf{\textit{X}} definida de la forma
\begin{align*}
    \textbf{\textit{X}} : (\Omega, \mathcal{A}, P) \longrightarrow (\mathbb{R}^n, \mathbb{B}_n, P_{\textbf{\textit{X}}})
\end{align*}
donde $\textbf{\textit{X}}(\omega) = (X_1(\omega), X_2(\omega),..., X_n(\omega))$ y $\mathbb{B}_n$ es la $\sigma$-álgebra de Borel en $\mathbb{R}^n$, es decir, la mínima $\sigma$-álgebra generada por los conjuntos
\begin{align*}
    C_{\textbf{\textit{X}}} = \{ \textbf{y} \in \mathbb{R}^n : y_1 \leq x_1, ..., y_n \leq x_n\}
\end{align*}
con $\textbf{\textit{x}} = (x_1,x_2,...,x_n) \in \mathbb{R}^n$.

\begin{defi}
Sea $\textbf{\textit{X}} = (X_1, X_2,..., X_n)$ un vectores de funciones definido en $(\Omega, \mathcal{A}, P)$ a $\mathbb{R}^n$, donde para cada $\omega \in \Omega$
\begin{align*}
    \textbf{\textit{X}}(\omega) = (X_1(\omega), X_2(\omega),..., X_n(\omega))
\end{align*}
Diremos que \textbf{\textit{X}} es un vector aleatorio o una variable aleatoria n-dimensional si, para cada vector de números reales, $\textbf{a} = (a_1,a_2,...,a_n)$, la imagen inversa del intervalo n-dimensional $I = \{ (x_1,...,x_n) : x_i \leq a_i, i = 1,2,...,n \}$ pertenece a la $\sigma$-álgebra $\mathcal{A}$, es decir,
\begin{align*}
    \textbf{\textit{X}}^{-1}(I) = \{ \omega \in \Omega : X_1(\omega) \leq a_1, ..., X_n(\omega) \leq a_n \} \in \mathcal{A}.
\end{align*}
\end{defi}

\begin{teo}
Sean $X_1,X_2,...,X_n$, n variables  aleatorias definidas sobre $(\Omega, \mathcal{A}, P)$. Entonces  $\textbf{\textit{X}} = (X_1, X_2,..., X_n)$ es una variable aleatoria n-dimensional sobre $(\Omega, \mathcal{A}, P)$.
\end{teo}

\begin{proof}
Sea $I = \{ (x_1,...,x_n) : x_i \leq a_i, a_i \in \mathbb{R}, i = 1,2,...,n \}$. Entonces
\begin{align*}
    (X_1,X_2,...,X_n)^{-1}(I) &= \{ \omega \in \Omega : X_1(\omega) \leq a_1, ..., X_n(\omega) \leq a_n \} \\
    &= \bigcap_{i=1}^{n}{\{ \omega \in \Omega : X_i(\omega) \leq a_i \}}\\
    &= \bigcap_{i=1}^{n}{X_i^{-1}(-\infty,a_i]}
\end{align*}
Por ser $X_i$ variable aleatoria, se tiene que $X_i^{-1}(-\infty,a_i] \in \mathcal{A}$, por lo que $(X_1,X_2,...,X_n)^{-1}(I) \in \mathcal{A}$.
\end{proof}

\section{Distribución de probabilidad de un vector aleatorio}

Dado un vector aleatorio \textbf{\textit{X}}, se denomina distribución de probabilidad inducida por \textbf{\textit{X}} en $(\mathbb{R}^n, \mathbb{B}_n)$, a la función
\begin{align*}
    P_{\textbf{\textit{X}}} : \mathbb{B}_n \longrightarrow [0,1]
\end{align*}
definida para cada $B \in \mathbb{B}_n$ de la forma
\begin{align*}
    P_{\textbf{\textit{X}}}(B) = P\{ \textbf{\textit{X}}^{-1}(B) \} = P\{\textbf{\textit{X}} \in B \}.
\end{align*}
$P_{\textbf{\textit{X}}}$ es una medida de probabilidad, por lo tanto, el vector aleatorio \textbf{\textit{X}} transforma el espacio de probabilidad $(\Omega, \mathcal{A}, P)$ en un nuevo espacio de probabilidad $(\mathbb{R}^n, \mathbb{B}_n, P_{\textbf{\textit{X}}})$.

\section{Función de distribucón conjunta}

Dado un vector aleatorio \textbf{\textit{X}}, se define la función de distribución conjunta como la función $F_{\textbf{\textit{X}}} : \mathbb{R}^n \longrightarrow [0,1]$ definida de la forma
\begin{align*}
    F_{\textbf{\textit{X}}}(x_1,x_2,...,x_n) = P(X_1 \leq x_1, X_2 \leq x_2, ..., X_n \leq x_n).
\end{align*}

\begin{teo}
La función de distribución de probabilidad de un vector aleatorio satisface las siguientes propiedades:
\begin{enumerate}
    \item[(P1)] $F(+\infty,+\infty,...,+\infty) = 1$.
    \item[(P2)] $F(x_1,...,x_{i-1},-\infty,x_{i+1},...,x_n) = 0$.
    \item[(P3)] La función de distribución conjunta es creciente para cada componente.
    \item[(P4)] La función de distribución conjunta es continua por la derecha para cada componente.
\end{enumerate}
\end{teo}

\subsection{Vectores bidimensionales}

Estudiaremos con más detenimiento el caso de que $n = 2$. Por lo tanto consideramos un vector aleatorio bidiensional $\textbf{\textit{X}} = (X,Y)$, con función de distribución $F_{(X,Y)}(x,y) := F(x,y)$. Las propiedades quedan ahora de la forma
\begin{teo}
\begin{enumerate}
    \item[(P1)] $F(+\infty,+\infty) = 1$.
    \item[(P2)] $F(x,-\infty) = 0$ y $F(-\infty,y) = 0$.
    \item[(P3)] La función de distribución conjunta es creciente para cada componente.
    \item[(P4)] La función de distribución conjunta es continua por la derecha para cada componente.
\end{enumerate}
\end{teo}

\begin{proof}
\begin{enumerate}
    \item[(P1)] $F(n,n) = P(X \leq n, Y \leq n) = P_{(X,Y)}(C_{(n,n)})$ donde
    \begin{align*}
        C_{(n,n)} = \{ (x,y) \in \mathbb{R}^2 : x \leq n, y \leq n\}.
    \end{align*}
    Nótese que $\{C_{(n,n)} \} \uparrow \mathbb{R}^2$, por tanto, la probabilidad y el límite pueden conmutar y tenemos
    \begin{align*}
        \lim_{n \to \infty}{F(n,n)} = \lim_{n \to \infty}{ P_{(X,Y)}(C_{(n,n)})} = P_{(X,Y)}(\lim_{n \to \infty}{C_{(n,n)}}) = P_{(X,Y)}(\mathbb{R}^2) = 1.
    \end{align*}
    \item[(P2)] $F(0,-n) = P_{(X,Y)(C_{(x,-n)})}$. Nótese que $\{C_{(x,-n)}\} \downarrow \emptyset$, por tanto, la probabilidad y el límite pueden conmutar y tenemos
    \begin{align*}
        \lim_{n \to \infty}{F(x,-n)} = \lim_{n \to \infty}{P_{(X,Y)(C_{(x,-n)})}} = P_{(X,Y)}{(\lim_{n \to \infty}{C_{(x,-n)}})} = P_{(X,Y)}{(\emptyset)} = 0.
    \end{align*}
    El argumento es análolgo para $F(-\infty,y)$.
    \item[(P3)] Veamos que $F$ es creciente en la primera componente. Sean $x_1, x_2 \in \mathbb{R}$ con $x_1 < x_2$. Nótese que $C_{(x_1,y)} \subset C_{(x_2,y)}$. Entonces
    \begin{align*}
        F(x_1,y) = P_{(X,Y)}(C_{(x_1,y)}) \leq P_{(X,Y)}(C_{(x_2,y)}) = F(x_2,y)
    \end{align*}
    El argumento es análogo para la segunda componente.
\end{enumerate}

\end{proof}

\begin{obs}
Estas cuatro propiedades no caracterizan a la función de distribución de un vector aleatorio, en el sentido de que dada una funcón $F: \mathbb{R}^2 \longrightarrow [0,1]$ que satisfaga estas cuatro propiedades no asegura que $F$ sea la función de distribución de un vector aleatorio. Para poder asegurarlo, tenemos que añadir una propiedad adicional.
\\
\newline
Si $F: \mathbb{R} \longrightarrow [0,1]$ verifica $P1$, $P2$, $P3$ y $P4$, $F$ será función de distribución si además verifica
\begin{align*}
    P_{(X,Y)}((a,b] \times (c,d]) \in [0,1] \Longleftrightarrow F(b,d) - F(b,c) - F(a,d) + F(a,c) \in [0,1].
\end{align*}
\end{obs}

\subsection{Cálculo de probabilidad de conjuntos de $\mathbb{R}^2$ a partir de la función de distribución conjunta}

\begin{enumerate}
    \item[(1)] $P_{\textbf{\textit{X}}}((-\infty,x) \times (-\infty,y]) = P(X < x, Y \leq y) = F(x^-,y)$.
    \item[(2)] $P_{\textbf{\textit{X}}}(\{x_1\} \times (-\infty,y]) = F(x_1,y) - F(x_1^-,y)$.
    \item[(3)] $P_{\textbf{\textit{X}}}((-\infty,x] \times \{y\}) = F(x,y) - F(x,y^-)$.
    \item[(4)] $P_{\textbf{\textit{X}}}((-\infty,x] \times (-\infty,y]) = F(x,y)$.
\end{enumerate}
\section{Vectores aleatorios discretos}

Son vectores de la forma $\textbf{\textit{X}} = (X,Y)$ donde $X$ e $Y$ son variables aleatorias discretas. Además
\begin{itemize}
    \item $(X,Y) = \cup_{(i,j) \in I \times J}{(x_i,y_j)}$.
    \item $P(X = x_i, Y = y_j) \in [0,1]$.
    \item $\sum_{i \in I}{\sum_{j \in I}{P(X = x_i, Y = y_j)}} = 1$.
    \item Si $D \subset \mathbb{R}^2$ entonces $P_{(X,,Y)}(D) = \sum_{(x_i,y_j) \in D}{P(X = x_i, Y = y_j)}$.
    \item $P_{ij} = P(X = x_i, Y = y_j)$.
\end{itemize}

\section{Vectores aleatorios absolutamente continuos}

\begin{defi}
La distribución de probabilidad de P en $(\mathbb{R}^2, \mathbb{B}_2)$ y su función de distribución $F(x,y)$ se denominan absolutamente continuas si existe una función no negativa $f: \mathbb{R}^2 \longrightarrow \mathbb{R}$ tal que para todo rectángulo $I \subset \mathbb{R}^2$
\begin{align*}
    P(I) = \int_{I}{f(x,y) \ dxdy}.
\end{align*}
\end{defi}
No hay problema en extender esta integral a rectángulos no acotados mediante el paso al límite, ya que $F \ge 0$ por lo que $\int_{I}{f}$ crece al crecer $I$ y además está acotada por 1. En consecuencia
\begin{align*}
    F(x,y) = \int_{-\infty}^{x}{\int_{-\infty}^{y}{f(s,t) \ dsdt}}.
\end{align*}
donde $F$ es la función de distribución conjunta de la variable aleatoria bidimensional $P_{\textbf{\textit{X}}} = (X,Y)$. A esta función $f$ se le llama \textit{función de densidad} de la variable aleatoria bidimensional $P_{\textbf{\textit{X}}} = (X,Y)$.
\\
\newline
Además la función de distribución conjunta de la variable aleatoria bidimensional debe verificar que
\begin{align*}
    F(+\infty, +\infty) = \lim_{x,y \to +\infty}\int_{-\infty}^{x}{\int_{-\infty}^{y}{f(s,t) \ dsdt}} = {\int_{-\infty}^{+\infty}{\int_{-\infty}^{+\infty}{f(s,t) \ dsdt}}} = 1.
\end{align*}

\begin{prop}
Una función $f: \mathbb{R} \longrightarrow \mathbb{R}$ no negativa e integrable-Riemann en cualquier rectángulo de $\mathbb{R}^2$, es función de densidad de alguna distribución bidimensional si y solo si
\begin{align*}
    {\int_{-\infty}^{+\infty}{\int_{-\infty}^{+\infty}{f(s,t) \ dsdt}}} = 1.
\end{align*}
En tal caso, f es la función de densidad de la función de distribución absolutamente continua F definida como
\begin{align*}
    F(x,y) = \int_{-\infty}^{x}{\int_{-\infty}^{y}{f(s,t) \ dsdt}}.
\end{align*}
\end{prop}

\begin{teo}
Sea $\textbf{\textit{X}} = (X,Y)$ una variable aleatoria bidimmensional continua con función de densidad f y función de distribución F. Entonces:
\begin{enumerate}
    \item[(a)] F es continua para todo $(x,y) \in \mathbb{R}^2$.
    \item[(b)] Si f es continua en un entorno del punto $(x_0,y_0)$, entonces F admite en dicho entorno derivada segunda respecto a $(x,y)$ siendo para todo $(x_1,y_1)$ de este entorno
    \begin{align*}
        \frac{\partial F(x_1,y_1)}{\partial x \partial y} = \frac{\partial F(x_1, y_1)}{\partial y \partial x} = f(x_1,y_1).
    \end{align*}
    \item[(c)] El conjunto de valores de la variable aleatoria bidimensional con probabilidad positiva es el conjunto vacío.
    \item[(d)] Para todo $B \in \mathbb{B}_2$
    \begin{align*}
        P(B) = \int_{B}{f(x,y) \ dxdy}.
    \end{align*}
\end{enumerate}
\end{teo}

\section{Distribuciones marginales}

\subsection{Caso discreto}
\begin{itemize}
    \item $(X,Y)$ variable aleatoria bidimensional discreta.
    \item $(X,Y) = \cup_{i \in I, j \in J}{\{(x_i,y_j)\}}$.
    \item \textit{Distribución marginal de X}. Sea $X$ variable aleatoria discreta, $X = \cup_{i \in I}{\{x_i\}}$, entonces
    \begin{align*}
        P(X = x_i) = \sum_{j \in J}{P(X = x_i, Y = y_j)}.
    \end{align*}
    \item \textit{Distribución marginal de Y}. Sea $Y$ variable aleatoria discreta, $Y = \cup_{j \in J}{\{x_j\}}$, entonces
    \begin{align*}
        P(Y = y_j) = \sum_{i \in I}{P(X = x_i, Y = y_j)}.
    \end{align*}
\end{itemize}

\subsection{Caso continuo}
\begin{defi}
Llamaremos función de distribución marginal de X a una función real de variable real $F_X : \mathbb{R} \longrightarrow [0,1]$ definida por
\begin{align*}
    F_X(x) = F(x,+\infty) = \lim_{y \to +\infty}{F(x,y)} = \int_{-\infty}^{x}{\left(\int_{-\infty}^{+\infty}{f(s,t) \ dt }\right) ds}.
\end{align*}
\end{defi}
A la función $f_X(x) = \int_{-\infty}^{+\infty}{f(x,y) \ dy}$ se le llama \textit{densidad marginal de X} y en los puntos en los que $F_X$ sea derivable la densidad marginal coincide con esta derivada.

\begin{defi}
Llamaremos función de distribución marginal de Y a una función real de variable real $F_X : \mathbb{R} \longrightarrow [0,1]$ definida por
\begin{align*}
    F_X(y) = F(-\infty,,y) = \lim_{x \to -\infty}{F(x,y)} = \int_{-\infty}^{y}{\left(\int_{-\infty}^{+\infty}{f(s,t) \ ds }\right) dt}.
\end{align*}
\end{defi}
A la función $f_Y(x) = \int_{-\infty}^{+\infty}{f(x,y) \ dx}$ se le llama \textit{densidad marginal de Y}.

\section{Distribuciones condicionadas}

\subsection{Caso discreto}
\begin{itemize}
    \item $(X,Y)$ variable aleatoria bidimensional discreta.
    \item $(X,Y) = \cup_{i \in I, j \in J}{\{(x_i,y_j)\}}$.
    \item $\sum_{i \in I}{\sum_{j \in J}{P_{ij}}} = 1$.
    \item $X | Y = y_j$.
    \begin{align*}
    P(X = x_i | y = y_j) = \frac{P(X = x_i, Y = y_j)}{P(Y = y_j)} = \frac{P_{ij}}{P(Y = y_j)}.
    \end{align*}
    \item $Y | X = x_i$.
    \begin{align*}
        P(Y = y_j, X = x_i) = \frac{P(X = x_i, Y = y_j)}{P(X = x_i)} = \frac{P_{ij}}{P(X = x_i)}
    \end{align*}
\end{itemize}
\subsection{Caso continuo}
Supongamos $A = (-\infty,x]$ y $B = (-\infty,y_0]$, en tal caso
\begin{align*}
    P[X \leq x | Y \leq y_0] &= \frac{P[X \leq x , Y \leq y_0] }{P[Y \leq y_0]} = \frac{\int_{-\infty}^{x}{\int_{-\infty}^{y_0}{f(u,v) \ dudv}}}{\int_{-\infty}^{y_0}\int_{-\infty}^{+\infty}{f(u,v) \ dudv}} \\
    &= \frac{\int_{-\infty}^{x}{\int_{-\infty}^{y_0}{f(u,v) \ dudv}}}{\int_{-\infty}^{y_0}{f_Y(y) \ dy}} = \frac{F(x,y_0)}{F_Y(y_0)}.
\end{align*}

\begin{defi}
Llamaremos función de distribución de la variable aleatoria X condicionada a que $Y = y_0$ al límite
\begin{align*}
    \lim_{h \to 0^+}{P[X \leq x | y_0 - h < Y \leq y_0 + h]}
\end{align*}
en caso de que exista y sea un función de distribución en $\mathbb{R}$ se se notará $F_{X | y_0}(x)$ o bien, $F(x | Y = y_0)$.
\end{defi}

\begin{defi}
Llamaremos función de densidad de la variable aleatoria X condicionada a que $Y = y_0$ y se representará por $f_{X | y_0}(x)$ a una función de $\mathbb{R}$ en $\mathbb{R}$ no negativa, de manera que para cada $y_0 \in \mathbb{R}$ fijo
\begin{align*}
    \int_{\mathbb{R}}{f_{X | y_0}(x) \ dx} = 1
\end{align*}
y tal que
\begin{align*}
    F_{X | y_0}(x) = \int_{-\infty}^{x}{f_{X | y_0}(u) \ du}, \text{para todo } x \in \mathbb{R}.
\end{align*}
\end{defi}

\begin{prop}
En los puntos donde f sea continua y $f_Y$ sea positiva y continua,, la función de densidad de la variable aleatoria X condicionada a que $Y = y$, existe y puede ser expresada como
\begin{align*}
    f_{X | y}(x) = \frac{f(x,y)}{f_Y(y)}.
\end{align*}
\end{prop}

\section{Independencia de variables aleatorias continuas}

Sabemos que dos sucesos $A$ y $B$ son \textit{estocásticamente independientes} si se verifica que
\begin{align*}
    P(A|B) = P(A) \ \ \ \text{o} \ \ \ P(B|A) = P(B),
\end{align*}
o de manera equivalente, si
\begin{align*}
    P(A \cap B) = P(A) \cdot P(B).
\end{align*}

\begin{defi}
X es independiente de Y si y solo si para cualesquiera borelianos $B_1$ y $B_2$ los sucesos $X^{-1}(B_1)$ e $Y^{-1}(B_2)$ de $\mathcal{A}$ son independientes en la probabilidad $P$. Es decir:
\begin{align*}
    P[\{ \omega : X(\omega) \in B_1\} \cap \{ \omega : Y(\omega) \in B_2\}] = P[\{\omega : X(\omega) \in B_1\}] \cdot P[\{\omega : Y(\omega) \in B_2\}].
\end{align*}
\end{defi}

Si consideramos $x,y \in \mathbb{R}$, $B_1 = \{ \omega : X(\omega) \leq x\}$ y $B_2 = \{ \omega : Y(\omega) \leq y\}$, entonces $X$ e $Y$ son independientes se tiene que
\begin{align*}
    P[\{ \omega : X(\omega) \leq x\} \cap \{ \omega : Y(\omega) \leq y\}] = P[\{\omega : X(\omega) \leq x\}] \cdot P[\{\omega : Y(\omega) \leq y\}].
\end{align*}
Si $F$ es la función de distribución conjunta y $F_X$ y $F_Y$ las funciones de distribución marginales de $X$ e $Y$ entonces
\begin{align*}
    F(x,y) = F_X(x)F_Y(y).
\end{align*}

\begin{defi}
Diremos que dos variables aleatorias X e Y son independientes si para todo $x,y \in \mathbb{R}$
\begin{align*}
    F(x,y) = F_X(x)F_Y(y),
\end{align*}
siendo $F$ la función de distribución conjunta y $F_X$ y $F_Y$ las funciones de distribución marginales de $X$ e $Y$ respectivamente.
\end{defi}

Si $F$ es derivable respecto de $x$ y respecto de $y$ y $F_X$ y $F_Y$ son derivables, tenemos que
\begin{align*}
    \frac{\partial^2 F(x,y)}{\partial x \partial y} = \frac{\partial F_x(x)}{\partial x} \cdot \frac{\partial F_Y(y)}{\partial y},
\end{align*}
es decir,
\begin{align*}
    f(x,y) = f_X(x) \cdot f_Y(y)
\end{align*}
donde $f$, $f_X$ y $f_Y$ son, respectivamente, la función de densidad conjunta y las funciones de densidad marginales de $X$ e $Y$.

\section{Cambio de variable}

\begin{prop}
Sea \textbf{\textit{X}} un vector aleatorio bidimensional absolutamente continuo con densidad conjunta $f(x,y)$ definida sobre $R \subset \mathbb{R}^2$, abierto o cerrado no degenerado donde $f > 0$. Si $g : \mathbb{R}^2 \longrightarrow \mathbb{R}^2$ es una función inyectiva y diferenciable en $R$, cuyo Jacobiano $J_g$ no se anula en ningún punto de R, entonces $(U,V) = g(X,Y)$ tiene una distribución absolutamente continua en $g(R)$ con densidad
\begin{align*}
    f^*(u,v) = f(g^{-1}(u,v))|J_g(g^{-1}(u,v))|^{-1}.
\end{align*}
\end{prop}
Si $h = g^{-1}$, entonces
\begin{align*}
    f^*(u,v) = f(h(u,v))|J_h(h(u,v))|.
\end{align*}

\section{Momentos conjuntos de un vector (X,Y)}
\begin{defi}[Momento ordinario de orden $r,s \in \mathbb{Z}^+$]
\begin{align*}
    m_{r,s} &= E[X^r \cdot Y^s] \\
    &= \sum_{i\in I}{\sum_{j \in J}{x_i^ry_j^sP(X = x_i, Y = y_j)}} \\
    &= \int_{-\infty}^{+\infty}{\int_{-\infty}^{+\infty}{x^ry^sf(x,y) \ dxdy}}
\end{align*}
\end{defi}
Algunos momentos ordinarios destacados son
\begin{itemize}
    \item $m_{0,0} = 1$.
    \item $m_{1,1} = E[X \cdot Y]$.
    \item $m_{r,0} = E[X^r]$, $m_{1,0} = E[X]$, $m_{2,0} = E[X^2]$.
    \item $m_{0,s} = E[Y^s]$, $m_{0,1} = E[Y]$, $m_{0,2} = E[Y^2]$.
\end{itemize}

\begin{defi}[Momento central de orden $r,s \in \mathbb{Z}^+$]
\begin{align*}
    M_{r,s} &= E[(X - E(X))^r \cdot (Y - E(Y))^s] \\
    &= \sum_{i\in I}{\sum_{j \in J}{(x_i - E(X))^r \cdot (y_j - E(Y))^sP(X = x_i, Y = y_j)}} \\
    &= \int_{-\infty}^{+\infty}{\int_{-\infty}^{+\infty}{(x - E(X))^r \cdot (y - E(Y))^sf(x,y) \ dxdy}}
\end{align*}
\end{defi}

Algunos momentos centrales destacados son
\begin{itemize}
    \item $M_{1,1} = E[(X - E(X)) \cdot (Y - E(Y))] = Cov(X,Y)$.
    \item $M_{1,0} = E[X - E(X)] = 0$ y $M_{0,1} = E[Y - E(Y)] = 0$.
    \item $M_{2,0} = V(X)$ y $M_{0,2} = V(Y)$.
\end{itemize}

\begin{obs}
\begin{itemize}
    \item $V(X+Y) = V(X) + V(Y) + 2Cov(X,Y)$.
    \item $V(X-Y) = V(X) + V(Y) - 2Cov(X,Y)$.
\end{itemize}
\end{obs}
Si $X$ e $Y$ son independientes
\begin{itemize}
    \item $m_{r,s} = m_{r,0} \cdot m_{0,s}$.
    \item $M_{r,s} = M_{r,0} \cdot M_{0,s}$.
    \item $Cov(X,Y) = 0$.
    \item $V(X + Y) = V(X) + V(Y) = V(X - Y)$.
\end{itemize}

\section{Distribución del máximo y del mínimo}

Sea \textit{\textbf{X}} = $(X_1,...,X_n)$ un vector aleatorio con función de distribución $F_{\textit{\textbf{X}}}$ y sean las variables aleatorias $M = \max(X_1,...,X_n)$ y $N = \min(X_1,...,X_n)$ definidas de la forma
\begin{align*}
    M(\omega) &= \max(X_1(\omega),...,X_n(\omega))\\
    N(\omega) &= \min(X_1(\omega),...,X_n(\omega))
\end{align*}
Tanto $M$ como $N$ son variables aleatorias y
\begin{align*}
    F_M(x) &= P[M \leq x] = P[X_1 \leq x,..., X_n \leq x] = F_X(x,...,x)\\
    F_N(x) &= P[N \leq x] = 1 - P[N > x] = 1 - P[X_1 > x,...,X_n > x].
\end{align*}

\begin{defi}
Sea $(X,Y)$ un vector aleatorio, se definen las curvas generales de regresión como las curvas
\begin{itemize}
    \item $C_{x;y} = \{ (E(X | y = y_0), y) : y_0 \in \mathbb{R} \}$.
    \item $C_{y;x} = \{ (x_0, E(Y | x = x_0)) : x_0 \in \mathbb{R} \}$.
\end{itemize}
\end{defi}