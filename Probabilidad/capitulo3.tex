\chapter{Función de distribución de probabilidad}

\section{Función de distribución asociada a una medida de probabilidad}

\begin{defi}
Una función $F: \mathbb{R} \longrightarrow [0,1]$ decimos que es una función de distribución si verifica
\begin{enumerate}
    \item[(i)] F es creciente.
    \item[(iii)] F es continua por la derecha, es decir,
    \begin{align*}
        \lim_{x \to a^*}{F(x)} = F(a) \ para \ todo \ a \in \mathbb{R}
    \end{align*}
    \item[(iii)]
    \begin{align*}
        \lim_{x \to -\infty}{F(x) = 0} \ \ \ y \ \ \ \lim_{x \to +\infty}{F(x) = 1}.
    \end{align*}
\end{enumerate}
\end{defi}

\begin{prop}
Sea P una medida de probabilidad definida en $(\mathbb{R}, \mathbb{B}_1)$. La función $F: \mathbb{R} \longrightarrow [0,1]$ dada por $F(x) = P((-\infty,x])$ para todo $x \in \mathbb{R}$, es una distribución de probabilidad.
\end{prop}

\begin{proof}
$(i)$ F es creciente. Sean $x_1, x_2 \in \mathbb{R}$ tales que $x_1 < x_2$. Entonces
\begin{align*}
    (-\infty, x_1] \subset (-\infty, x_2] \Longrightarrow P((-\infty, x_1]) \leq P((-\infty, x_2]) \Longrightarrow F(x1) \leq F(x2).
\end{align*}
$(ii)$ F es continua por la derecha.
\begin{align*}
    \lim_{n \to \infty}{F\left( a + \frac{1}{n}\right)} &= \lim_{n \to \infty}{P\left(\left( -\infty, a + \frac{1}{n}\right]\right)} = P\left( \lim_{n \to \infty}{\left\{ \left( -\infty, a + \frac{1}{n} \right]\right\}} \right)\\
    &= P((-\infty, a)) = F(a)
\end{align*}
para todo $a \in \mathbb{R}$.
\\
\newline
Probemos $(iii)$.
\begin{align*}
    \lim_{n \to \infty}{F(n)} &= \lim_{n \to \infty}{P((-\infty, n])} = P\left(\lim_{n \to \infty}{\{(-\infty, n) \}}\right) =P(\mathbb{R}) = 1\\
    \lim_{n \to -\infty}{F(n)} &= \lim_{n \to -\infty}{P((-\infty, n])} = P\left(\lim_{n \to -\infty}{\{(-\infty, -n) \}}\right) =P(\mathbb{\emptyset}) = 0.
\end{align*}
\end{proof}

\begin{obs}
Sean $P: \mathbb{B}_1 \Longrightarrow [0,1]$ una medida de probabilidad y $F(x) = P((-\infty,x])$ función de distribución asociada a $P$. Sean $a, b \in \mathbb{R}$, $a < b$. Entonces
\begin{itemize}
    \item $P((a,b]) = F(b) - F(a)$.
    \begin{align*}
        (-\infty,b] = (-\infty,a] \cup (a,b] &\Longrightarrow P((-\infty, b]) = P((-\infty,a]) + P((a,b]) \\
        & \Longleftrightarrow P((a,b]) = F(b) - F(a).
    \end{align*}
    \item $P(\{a\}) = F(a) - F(a^-)$.
    \item $P((a, +\infty)) = 1 = P((-\infty,]) = 1 - F(a)$.
    \item $P((a,b)) = F(b) - F(a) - P(\{b\})$.
    \item $P([a,b)) = F(b) - F(a) - P(\{b\}) + P(\{a\})$.
    \item $P([a,b]) = F(b) - F(a) + P(\{a\})$.
\end{itemize}
\end{obs}

\section{Aleatorización de la recta real}
Sea $X$ una variable aleatoria definida sobre $(\Omega, \mathcal{A}, P)$ y sea $P_X$ la probabilidad inducida por $X$ en $(\mathbb{R}, \mathbb{B}_1)$. Entonces $F: \mathbb{R} \longrightarrow [0,1]$ dada por $F(x) = P_X((-\infty,x])$ para todo $x \in \mathbb{R}$, es una función de distribución de probabilidad.

\begin{defi}
Sea X variable aleatoria definida sobre $(\Omega, \mathcal{A}, P)$, sea $P_X$ la probabilidad inducida por $X$ en $(\mathbb{R}, \mathbb{B}_1)$ y sea $F_X$ la función de distribución de probabilidad asociada a $P_X$. Definimos
\begin{itemize}
    \item Conjunto de puntos de salto de $F_X$ como
    \begin{align*}
        D_X = \{ X \in \mathbb{R} : P_X(\{x\}) > 0 \}.
    \end{align*}
    \item Soporte de X (o soporte de la distribución) como
    \begin{align*}
        S_X = \{x \in \mathbb{R} : F(x+\varepsilon) - F(x-\varepsilon) > 0 \}
    \end{align*}
    para todo $\varepsilon > 0$.
\end{itemize}
\end{defi}

\begin{teo}
El conjunto $D_X$ es a lo sumo infinito numerable.
\end{teo}

\begin{proof}
Nótese que
\begin{align*}
    D_X = \{ X \in \mathbb{R} : P_X(\{x\}) > 0 \} = \bigcup_{r \in \mathbb{N}}{\left\{ X \in \mathbb{R} : P_X(\{x\}) > \frac{1}{r} \right\}}
\end{align*}
Sea $D_r = \left\{ X \in \mathbb{R} : P_X(\{x\}) > \frac{1}{r} \right\}$. Veamos que $D_r$ es finito.
\begin{align*}
    D_1 &= \left\{ X \in \mathbb{R} : P_X(\{x\}) > 1 \right\} \text{ tiene a lo sumo 1 elemento}\\
    D_2 &= \left\{ X \in \mathbb{R} : P_X(\{x\}) > \frac{1}{2} \right\} \text{ tiene a lo sumo 2 elementos}\\
    \vdots &  \\
    D_r &= \left\{ X \in \mathbb{R} : P_X(\{x\}) > \frac{1}{r} \right\} \text{ tiene a lo sumo r elementos}
\end{align*}
Por tanto, $D_X$ es a lo sumo infinito numerable.
\end{proof}

\begin{teo}[Teorema de Descomposición]
Dada cualquier función de distribución $F: \mathbb{R} \longrightarrow [0,1]$ siempre existe una única descomposición
\begin{align*}
    F = F_d + F_c
\end{align*}
en la que $F_d,F_c: \mathb{R} \longrightarrow [0,1]$ verifican
\begin{enumerate}
    \item[(1)] $F_d(-\infty) = F_c(-\infty) = 0$.
    \item[(2)] $F_d$ es continua por la derecha para todo $x \in \mathb{R}$ y varía a saltos.
    \item[(3)] $F_c$ es continua para todo $x \in \mathbb{R}$.
    \item[(4)] $F_d$, $F_c$ son crecientes.
    \item[(5)] Las normalizaciones
    \begin{align*}
        F_1 = \frac{F_d}{F_d(+\infty)} \ \ \ y \ \ \ F_2 = \frac{F_c}{F_c(+\infty)}
    \end{align*}
    son funciones de distribución tales que
    \begin{align*}
        F(x) = \alpha F_1(x) + (1 - \alpha)F_2(x)
    \end{align*}
    donde $\alpha \in (0,1)$ y $\alpha = F_d(+\infty)$ (dicho $\alpha$ se conoce como mixtura de la distribución).
\end{enumerate}
\end{teo}
