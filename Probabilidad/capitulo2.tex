\chapter{Variables aleatorias}

\section{Variable aleatoria. Función indicadora. Variables aleatorias simples}

\begin{defi}
    Sea $(\Omega, \mathcal{A})$ espacio probabilizable. Una aplicación $X: \Omega \longrightarrow \mathbb{R}$ diremos que es variable aleatoria si $X^{-1}(B) \in \mathcal{A}$ para todo $B \in \mathbb{B}_1$.
\end{defi}

\begin{defi}
    Sea $(\Omega, \mathcal{A})$ espacio probabilizable. Una aplicación $X: \Omega \longrightarrow \mathbb{R}$ diremos que es variable aleatoria si $X^{-1}((-\infty,a]) \in \mathcal{A}$ para todo $a \in \mathbb{R}$.
\end{defi}

\begin{obs}
    Estas dos definiciones son equivalentes.
\end{obs}

\begin{defi}
    Sea $(\Omega, \mathcal{A})$ espacio probabilizable y sea $A \in \mathcal{A}$. La función $I_A: \Omega \longrightarrow \mathbb{R}$ definida por
    \begin{align*}
        I_A(\omega) = \left\{ \begin{array}{lcc}
                                  1 & si & \omega \in \mathcal{A}      \\
                                  0 & si & \omega \not \in \mathcal{A} \\
                              \end{array}
        \right.
    \end{align*}
    es una variable aleatoria y se denomina función indicadora o variable indicadora.
\end{defi}

\begin{defi}
    Sea $(\Omega, \mathcal{A})$ espacio probabilizable y sea $\{A_i\}_{i=1}^{n}$ una partición de $\Omega$, $A_i \in \mathcal{A}$. Sean $x_1,...,x_n \in \mathbb{R}$. La aplicación $X: \Omega \longrightarrow \mathbb{R}$ definida por
    \begin{align*}
        X(\omega) = \left\{ \begin{array}{lcc}
                                x_1    & si     & \omega \in A_1  \\
                                x_2    & si     & \omega  \in A_2 \\
                                \vdots & \vdots & \vdots          \\
                                x_n    & si     & \omega  \in A_n
                            \end{array}
        \right.
    \end{align*}
    es una variable aleatoria y se denomina variable aleatoria simple. Nótese que $X(\omega) = \sum_{i=1}^{n}{x_iI_{A_i}(\omega)}$.
\end{defi}

\section{Propiedades y operaciones algebraicas con variables aleatorias}

\begin{teo}
    Sean X e Y variables aleatorias definidas sobre $(\Omega, \mathcal{A})$. Entonces
    \begin{enumerate}
        \item[(i)] $A = \{ \omega \in \Omega : X(\omega) < Y(\omega) \} \in \mathcal{A}$.
        \item[(ii)] $B = \{ \omega \in \Omega : X(\omega) \leq Y(\omega) \} \in \mathcal{A}$.
        \item[(iii)] $C = \{ \omega \in \Omega : X(\omega) = Y(\omega) \} \in \mathcal{A}$.
    \end{enumerate}
\end{teo}

\begin{proof}
    Probemos $(i)$. $A = \{ \omega \in \Omega : X(\omega) < Y(\omega) \}$. Sea $\omega_0 \in A$, entonces $X(\omega_0) < Y(\omega_0)$. Ambos son números reales, por tanto, existe $r_0 \in \mathbb{Q}$ tal que $X(\omega_0) < r_0 < Y(\omega_0)$. Por tanto
    \begin{align*}
        A = \{ \omega \in \Omega : X(\omega) < Y(\omega) \} & = \bigcup_{r \in \mathbb{Q}}{\{ \omega \in \Omega : X(\omega) < r < Y(\omega) \}}                                                \\
                                                            & = \bigcup_{r \in \mathbb{Q}}{\left(\{ \omega \in \Omega : X(\omega) < r \} \cap \{ \omega \in \Omega :  r < Y(\omega) \}\right)} \\
                                                            & = \bigcup_{r \in \mathbb{Q}}{\left( X^{-1}((-\infty, r)) \cap Y^{-1}((r,+\infty))\right)}
    \end{align*}
    Como $X$ e $Y$ son variables aleatorias se tiene que $ X^{-1}((-\infty, r)) \in \mathcal{A}$ e $Y^{-1}((r,+\infty)) \in \mathcal{A}$. Como $A$ es unión numerable de elementos de $\mathcal{A}$ se tiene que $A \in \mathcal{A}$.

    Para probar $(ii)$ basta ver que
    \begin{align*}
        B = \{ \omega \in \Omega : X(\omega) \leq Y(\omega) \} = (A^*)^c
    \end{align*}
    donde $A^* = \{ \omega \in \Omega : X(\omega) > Y(\omega) \}$. Como $A^* \in \mathcal{A}$ entonces $(A^*)^c = B \in \mathcal{A}$.

    Para probar $(iii)$ basta ver que
    \begin{align*}
        C = \{ \omega \in \Omega : X(\omega) = Y(\omega) \} = B \backslash A = B \cap A^c \in \mathcal{A}.
    \end{align*}
\end{proof}

\begin{prop}
    Sean X e Y variables aleatorias definidas sobre $(\Omega, \mathcal{A})$. Entonces
    \begin{enumerate}
        \item[(1)] Dado $a \in \mathbb{R}$ se tiene que Z = aX es variable aleatoria.
        \item[(2)] X + Y es variable aleatoria.
        \item[(3)] $X^2$ es variable aleatoria.
        \item[(4)] $|X|$ es variable aleatoria.
        \item[(5)] $X \cdot Y$ es variable aleatoria.
    \end{enumerate}
\end{prop}

\begin{teo}
    Sea X variable aleatoria definida sobre  $(\Omega, \mathcal{A})$. Sea $g: \mathbb{R} \longrightarrow \mathbb{R}$ una función tal que $g^{-1}(B) \in \mathbb{B}_1$ para todo $B \in \mathbb{B}_1$. Entonces $Y = g(X)$ es también una variable aleatoria definida sobre  $(\Omega, \mathcal{A})$.
\end{teo}

\begin{proof}
    \begin{align*}
        \Omega \xrightarrow[]{X} \mathbb{R} \xrightarrow[]{g} \mathbb{R}
    \end{align*}
    Seea $B \in \mathbb{B}_1$. Entonces
    \begin{align*}
        Y^{-1}(B) = X^{-1}(g^{-1}(B)),
    \end{align*}
    $g^{-1}(B) \in \mathbb{B}_1$ por hipótesis y como $X$ es variable aleatoria, entonces $X^{-1}(g^{-1}(B)) = Y^{-1}(B) \in \mathbb{B}_1$.
\end{proof}

\section{Probabilidad inducida en $\mathbb{R}$ por una variable aleatoria}
\begin{prop}
    Sea $(\Omega, \mathcal{A}), P$ un espacio de probabilidad y sea X una variable aleatoria definida en $\Omega$, esto es, $X: \Omega \longrightarrow \mathbb{R}$. Entonces X induce sobre $(\mathbb{R}, \mathbb{B}_1)$ una medida de probabilidad $P_X : \mathbb{B}_1 \longrightarrow [0,+\infty)$ dada por
    \begin{align*}
        P_X(B) = P[\{\omega \in \Omega : X(\omega) \in B\}] = P(X^{-1}(B))
    \end{align*}
    para todo $B \in \mathbb{B}_1$.
\end{prop}