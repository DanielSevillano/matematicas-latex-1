\chapter{Distribuciones discretas más usuales}


\section{Distribución degenerada}

\begin{defi}
Se dice que una variable aleatoria $X \sim D(a)$ si su distribución de probabilidad viene dada por
\begin{align*}
    P(X = x) = \left\{ \begin{array}{lcc}
             1 &  si  & x = a\\
             0 &  si  & x \not = a\\
             \end{array}
        \right.
\end{align*}
\end{defi}

\begin{obs}
Si $X \sim D(a)$ entonces
\begin{enumerate}
    \item[(i)] $D_X = \{a\}$, $E[X] = a$ y $V[X] = 0$.
    \item[(ii)] Su función de distribución es
    \begin{align*}
        F_X(x) = \left\{ \begin{array}{lcc}
             0 &  si  & x < a\\
             1 &  si  & x \ge a\\
             \end{array}
        \right.
    \end{align*}
    \item[(iii)] $G_X(s) = E[s^X] = s^aP(X = a) = s^a$ para todo $s \in \mathbb{R}$.
    \item[(iv)] $M_X(t) = e^{ta}$.
\end{enumerate}
\end{obs}

\section{Distribución de Bernoulli}

\begin{defi}
Se dice que una variable aleatoria $X \sim Ber(p)$ si
\begin{itemize}
    \item $D_X = \{0, 1\}$.
    \item $P(X = 0) = 1 - p$ y $P(X = 1) = 1$, $0 < p < 1$.
\end{itemize}
\end{defi}

\begin{obs}
Si $X \sim Ber(p)$ entonces
\begin{enumerate}
    \item[(i)] $E[X] = p$.
    \item[(ii)] $V[X] = p(1 - p)$.
    \item[(iii)] $G_X(s) = E[s^X] = ps + q$ para todo $s \in \mathbb{R}$.
    \item[(iv)] $M_X(t) = pe^t + q$.
\end{enumerate}
\end{obs}

\begin{defi}
Sean $X_1,...,X_n$ variables aleatorias definidas sobre el mismo espacio de probabilidad. Decimos que son independientes si dados $B_1,...,B_n \in \mathbb{B}_1$ se verifica
\begin{align*}
    P(X_1 \in B_1, X_2 \in B_2, ..., X_n \in B_n) = \prod_{i=1}^{n}{P(X_i \in B_i)}.
\end{align*}
\end{defi}

\begin{obs}
\begin{enumerate}
    \item[(1)] Si $X_,...,X_n$ son variables aleatorias discretas, la condición de independencia se traduce como
    \begin{align*}
        &P(X_1 = x_1, ..., X_n = x_n) = \prod_{i=1}^{n}{P(X_i = x_i)} \\
        &P(X_1 \leq x_1,...,X_n \leq x_n) = \prod_{i=1}^{n}{P(X_i \leq x_i)} = \prod_{i=1}^{n}{F_{X_i}(x_i)}.
    \end{align*}
    \item[(2)] Si $X_1,...,X_n$ son variables aleatorias absolutamente continuas, la condición de independencia se traduce como
    \begin{align*}
        &f(x_1,...,x_n) = \prod_{i=1}^{n}{f_i(x_i)} \\
        &P(X_1 \leq x_1,...,X_n \leq x_n) = \prod_{i=1}^{n}{F_{X_i}(x_i)}.
    \end{align*}
    \item[(3)] Si $X_1,...,X_n$ son variables aleatorias discretas independientes y si $T = \sum_{i=1}^{n}{X_i}$. Entonces $T$ es variable aleatoria y 
    \begin{align*}
        G_T(s) = \prod_{i=1}^{n}{G_{X_i}(s)}.
    \end{align*}
    \item[(4)] Si $X_1,...,X_n$ son variables aleatorias discretas independientes y si $T = \sum_{i=1}^{n}{X_i}$. Entonces $T$ es variable aleatoria y 
    \begin{align*}
        M_T(t) = \prod_{i=1}^{n}{M_{X_i}(t)}.
    \end{align*}
\end{enumerate}
\end{obs}

\section{Distribución Binomial}

\begin{defi}
Sea X = número de éxitos al realizar n pruebas independientes de Bernoulli. Entonces $X \sim Bi(n,p)$ y 
\begin{itemize}
    \item $D_X = \{0,1,...,n\}$.
    \item $P(X = i) = \binom{n}{i}p^i(1-p)^{n-i}$, $i \in D_X$.
\end{itemize}
\end{defi}

\begin{obs}
Si $X \sim Bi(n,p)$ entonces
\begin{enumerate}
    \item[(i)] $G_X(s) = (ps + q)^n$.
    \item[(ii)] $M_X(t) = (pe^t + q)^n = m(t)$.
    \item[(iii)] $E[X] = m'(0) = np$.
    \item[(iv)] $V[X] = np(1 - p)$.
\end{enumerate}
\end{obs}

\section{Distribución Geométrica}

\begin{defi}
Sea X = número de fracasos hasta obtener el primer éxito al realizar pruebas independientes con $P(éxito = p)$. Entonces $X \sim Ge(p)$ y 
\begin{itemize}
    \item $D_X = \{0,1,...\}$.
    \item $P(X = n) = (1 - p)^np$, $n \in D_X$.
\end{itemize}
\end{defi}

\begin{obs}
Si $X \sim Ge(p)$ entonces
\begin{enumerate}
    \item[(i)] = $G_X(s) = \sum_{n=0}^{\infty}{s^nP(X = n)} = \sum_{n=0}^{\infty}{s^n(1-p)^np} = p\sum_{n=0}^{\infty}{(sq)^n} = \frac{p}{1 - sq}$ para todo $|s| \leq \frac{1}{q}$.
    \item[(ii)] $M_X(t) = \frac{p}{q - e^tq}$.
    \item[(iii)] $E[X] = \frac{q}{p}$.
    \item[(iv)] $V[X] = \frac{q}{p^2}$.
\end{enumerate}
\end{obs}

\section{Distribución Binomial Negativa}

\begin{defi}
Sea X = número de fracasos hasta conseguir el r-ésimo éxito al realizar pruebas independientes con $P(éxito) = p$. Entonces $X \sim BN(r,p)$ y 
\begin{itemize}
    \item $D_X = \{0,1,...\}$.
    \item $P(X = n) = \binom{n + r - 1}{n}(1 - p)^np^r$, $r \in D_X$.
\end{itemize}
\end{defi}

\begin{obs}
Si $X \sim BN(r,p)$ entonces
\begin{itemize}
    \item $X = \sum_{i=1}^{r}{X_i}$, donde $X_i \sim Ge(p)$ son variables aleatorias independientes.
    \item $G_X(s) = \left(\frac{p}{1 - s(1 - p)}\right)^r$.
    \item $m(t) = \left(\frac{p}{1 - e^t(1 - p)}\right)^r$.
    \item $E[X] = \frac{r(1 - p)}{p}$.
    \item $V[X] = \frac{r(1 - p)}{p^2}$.
\end{itemize}
\end{obs}

\section{Distribución de Poisson}

\begin{defi}
$X \sim Po(\lambda)$, $\lambda > 0$, si
\begin{itemize}
    \item $D_X = \{0,1,...\}$.
    \item $P(X = n) = e^{-\lambda}\frac{\lambda^n}{n!}$, $n \in D_X$.
\end{itemize}
\end{defi}

\begin{obs}
Si $X \sim Po(\lambda)$, $\lambda > 0$, entonces
\begin{itemize}
    \item $G_X(s) = e^{-\lambda(1 - s)}$.
    \item $M_X(t) = e^{-\lambda(1 - e^t)}$.
    \item $E[X] = V[X] = \lambda$.
\end{itemize}
\end{obs}

\section{Distribución Hipergeométrica}

\begin{defi}
Consideremos una población de $N$ elementos, donde hay $D$ elementos de la clase $A$ y $N - D$ elementos de la clase $B$. Sea X = número de elementos de la clase A al tomar una muestra (sin reemplazamiento) de tamaño n. Entonces $X \sim HG(N,D,n)$ y
\begin{itemize}
    \item $D_X = \{0,1,...,n\}$.
    \item 
    \begin{align*}
        P(X = i) = \frac{\binom{D}{i}\binom{N - D}{n - i}}{\binom{N}{n}}, \ i \in D_X.
    \end{align*}
\end{itemize}
\end{defi}

